\label{sec:glosario}


	Este capítulo describe los términos usados a lo largo del documento que tienen un significado singular en SREP y que se considera necesario definirlos para evitar ambigüedades o malos entendidos.\\

	Para fines de este documento la siguiente lista de términos se deben interpretar como se describen en este capítulo.

%====================================================================
\section{Términos técnicos}
\label{gls:terminosTecnicos}

	En esta sección se definen los términos técnicos que se utilizan para describir el comportamiento del sistema.
	
	\begin{description}
		\BRterm{gls:srep}{SREP} Sistema de Regstro y Evaluación de protocolos.
	
	    \BRterm{gls:archivoDigital}{Archivo digital} Es una unidad de datos o información almacenada 
	            en algún medio que puede ser utilizada por aplicaciones de la computadora. 
	    
		\BRterm{gls:atributo}{Atributo} Son las características que definen o identifican a una entidad en un conjunto de entidades.
	
		\BRterm{gls:booleano}{Booleano} Es un \cdtRef{gls:tipoDato}{tipo de dato} que puede tomar los siguientes valores verdadero o falso (1 ó 0).
		
		\BRterm{gls:cadena}{Cadena} Es el \cdtRef{gls:tipoDato}{tipo de dato} definido por cualquier valor que se compone de una secuencia de caracteres, con o sin acentos, espacios, dígitos y signos de puntuación. Existen tres tipos de cadenas: Palabra, frase y párrafo.
		
		\BRterm{gls:decimal}{Decimal} Es un \cdtRef{gls:tipoDato}{tipo de dato} \cdtRef{gls:numerico}{numérico}. Los números decimales son valores que denotan números racionales e irracionales, es decir que los números decimales son la expresión de números no enteros, que a diferencia de los números fraccionarios, no se escriben como el cociente de dos números enteros sino como una aproximación de tal valor.
		
		\BRterm{gls:entero}{Entero} Es el \cdtRef{gls:tipoDato}{tipo de dato} \cdtRef{gls:numerico}{numérico} definido por todos los valores numéricos enteros, tanto positivos como negativos.

		\BRterm{gls:entidad}{Entidad} Término genérico que se utiliza para determinar un ente, el cual puede ser concreto, abstracto o conceptual por ejemplo: coach, persona, etc. La entidades se caracterizan por los atributos que las componen.
		\BRterm{gls:fecha}{Fecha} Es un \cdtRef{gls:tipoDato}{tipo de dato} que indica un día único en referencia al calendario gregoriano. Los tipos de fecha utilizados son: \cdtRef{gls:fechaCorta}{fecha corta} y \cdtRef{gls:fechaLarga}{fecha larga}. %con formato DD/MM/YYYY, por ejemplo: 24/02/2013.

		\BRterm{gls:fechaCorta}{Fecha corta} Es la representación del \cdtRef{gls:tipoDato}{tipo de dato} \cdtRef{gls:fecha}{fecha} en la forma \textbf{Abbreviated Month DD, YYYY}, por ejemplo: Jun 30, 2009.

		\BRterm{gls:fechaLarga}{Fecha larga} Es la representación del \cdtRef{gls:tipoDato}{tipo de dato} \cdtRef{gls:fecha}{fecha} en la forma \textbf{Month DD, YYYY}, por ejemplo: June 30, 2009.

		\BRterm{gls:frase}{Frase} Es un \cdtRef{gls:tipoDato}{tipo de dato}  conformado por \cdtRef{gls:palabra}{palabras} y espacios.
		
		\BRterm{gls:imagen}{Imagen} Es un \cdtRef{gls:archivoDigital}{Archivo digital} que específicamente tiene
		        un formato de imagen, puede ser JPG, JPEG, PNG, etcétera.						
		\BRterm{gls:numerico}{Numérico} Es un \cdtRef{gls:tipoDato}{tipo de dato} que se compone de la combinación de los símbolos \textit{0, 1, 2, 3, 4, 5, 6, 7, 8, 9, . } y \textit{-}  que representan una cantidad.		
		\BRterm{gls:opcional}{Opcional} Es un elemento que el actor puede o no porporcionar en el formulario o la pantalla, su decisión no afectará la ejecución de la operación solicitada.

		\BRterm{gls:palabra}{Palabra} Es un \cdtRef{gls:tipoDato}{tipo de dato} \cdtRef{gls:cadena}{cadena} conformado por letras y se caracteriza por no tener espacios.

		\BRterm{gls:parrafo}{Párrafo} Es un \cdtRef{gls:tipoDato}{tipo de dato}  conformado por \cdtRef{gls:frase}{frases}.

		\BRterm{gls:requerido}{Requerido} Es un tipo de dato que debe proporcionarse de manera obligatoria, el no ingresar la información requerida puede deterner alguna operación del sistema.

		\BRterm{gls:tipoDato}{Tipo de dato} Es el dominio o conjunto de valores que puede tomar un atributo de una \cdtRef{gls:entidad}{entidad} en el modelo de información. Los tipos de datos utilizados son: \cdtRef{gls:palabra}{palabra}, \cdtRef{gls:frase}{frase}, \cdtRef{gls:parrafo}{párrafo}, \cdtRef{gls:numerico}{numérico}, \cdtRef{gls:decimal}{decimal}, \cdtRef{gls:fecha}{fecha} y \cdtRef{gls:booleano}{booleano}.

\end{description}


%====================================================================
\section{Términos del negocio}
\label{gls:terminosNegocio}

	En esta sección se definen los términos del negocio que se utilizan para comprender el comportamiento del sistema.

%\begin{description}
%    \BRterm{gls:Agreement}{Agreement} La formalización dela impartición de clases se realiza a través de un contrato al cual se le denominará Agreement, este es un archivo digital en formato pdf.
%    
%    \BRterm{gls:educationalLevel}{Educational Level} Se refiere al nivel más alto de estudios que un individuo ha alcanzado, es un \cdtRef{gls:tipoDato}{tipo de dato} para el sistema que puede tomar los siguientes valores:	Early childhood Education, Primary education, Lower secondary education, Upper secondary education,
%    Post-secondary non-tertiary education, Short-cycle tertiary education, 
%    Bachelor or equivalent, Master or equivalent o 
%    Doctoral or equivalent.
%    
%    \BRterm{gls:coachStatus}{Coach Status} Situación en la  cual se halla el coach, es un \cdtRef{gls:tipoDato}{tipo de dato} para el sistema que puede tomar los siguientes valores: Initial, Active o Inactive.
%    
%    \BRterm{gls:eventType}{Event Type} Es el tipo de evento que se tiene programado con el cliente potencial, es un \cdtRef{gls:tipoDato}{tipo de dato} para el sistema que puede tomar los siguientes valores: Call, Visit, Email o Other.
%    
% 	\BRterm{gls:gender}{Gender} Variable biológica y genética que divide a los seres humanos en dos: hombre o mujer, es un \cdtRef{gls:tipoDato}{tipo de dato} para el sistema que puede tomar los siguientes valores: Male o Female.
%
%    \BRterm{gls:language}{Language} Conjunto de palabras y expresiones utilizadas y entendidas por un determinado grupo de gente, es un \cdtRef{gls:tipoDato}{tipo de dato} para el sistema.
%    
%    \BRterm{gls:searchBox}{Seach Box} Se utiliza para buscar direcciones en algún mapa.
% 	
% 	\BRterm{gls:methodOfContactCategory}{Method of Contact Category} Se utiliza para indicar la categoría 
% 	del medio de contacto, 
% 	es un \cdtRef{gls:tipoDato}{tipo de dato} para el sistema que puede tomar los siguientes valores: Personal, 
% 	Work o Emergency.
% 	
%    \BRterm{gls:methodOfContactType}{Method of Contact Type} Se utiliza para indicar el tipo de medio de contacto, 
% 	es un \cdtRef{gls:tipoDato}{tipo de dato} para el sistema que puede tomar los siguientes valores: Telephone, 
% 	Mobile, Email Address, Fax, Linkedin, Facebook o Twitter.
% 	
% 	\BRterm{gls:groupStatus}{Group Status}: Estado en el que se encuentra el grupo, es una \tdPalabra y puede tomar los siguientes valores: active o inactive. 		  
%\end{description}