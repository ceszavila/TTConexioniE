\documentclass[10pt]{book}
\usepackage{cdt/cdtBusiness}
\usepackage{paear}
\usepackage{subfigure}
\usepackage{cite}
\usepackage[spanish]{babel}


%%%%%%%%%%%%%%%%%%%%%%%%%%%%%%%%%%%%%%%%%%%%%%%%%%%%%%%%%%%%%%%%
% Datos del proyecto

\organizacion[]{Escuela Superior de Cómputo}
\autor[]{Instituto Politécnico Nacional}
\sistema[AMCEE]{Aplicación Móvil para la Comunidad Estudiantil de la ESCOM}
\proyecto[2017-A108]{Conexión iE}
\documento{C2--DA}{Documento de Análisis Etapa 1}{\RELEASE{1.0}}%{\DRAFT{\today}} 

\entregable{}{\Large{Aplicación Móvil para la Comunidad Estudiantil de la ESCOM}}

% Descomentar y establecer la fecha cuando se desee congelar la fecha del documento.
%\fecha{12 de Abril de 2013}

%%%%%%%%%%%%%%%%%%%%%%%%%%%%%%%%%%%%%%%%%%%%%%%%%%%%%%%%%%%%%%%%
% Datos para revisión
\elaboro[Alumnos de la Licenciatura en Ingeniería en Sistemas Computacionales IPN - ESCOM]{Cesar Raúl Avila Padilla\vspace{0.4cm} Ivo Sebastián Sam Álvarez-Tostado\vspace{0.4cm}} % Responsable del contenido (IPN) Lic. Ulises Vélez Saldaña
\superviso[Alumnos de la Licenciatura en Ingeniería en Sistemas Computacionales IPN - ESCOM]{Cesar Raúl Avila Padilla\vspace{0.4cm} Ivo Sebastián Sam Álvarez-Tostado\vspace{0.4cm}}
\aprobo[Directores del Trabajo Terminal 2017-A108]{M. en C. Ulises Velez Saldaña\\ M. en C. José David Ortega Pacheco} % Responsable Técnico (Contraparte)

%\title{\varProyecto}
%\subtitle{\varCveDocumento--\varDocumento}


%%%%%%%%%%%%%%%%%%%%%%%%%%%%%%%%%%%%%%%%%%%%%%%%%%%%%%%%%%%%%%%%
% Documentos relacionados con el documento actual

% TODO: Escriba los documentos en los que está basado este documento.

%%%%%%%%%%%%%%%%%%%%%%%%%%%%%%%%%%%%%%%%%%%%%%%%%%%%%%%%%%%%%%%%
% Elementos contenidos en el documento

% TODO: Al finalizar el análisis resuma aquí todos los elementos del componente: RN, CU, IU, MSG.
%\elemRefs{
%
%	%Glosario de términos
%   \elemItem{Glosario de términos}{1.0}{Descripción de los terminos técnicos y de negocio utilizados}
%%---------------------------------------------------------------------------------------------------------------------------------------
%
%	%Modelo de información
%   \elemItem{Modelo Registro de escuelas}{1.0}{Descripción del modelo de información del Registro de escuelas}
%
%}

%%%%%%%%%%%%%%%%%%%%%%%%%%%%%%%%%%%%%%%%%%%%%%%%%%%%%%%%%%%%%%%%
\begin{document}

%=========================================================
% Portada
\ThisLRCornerWallPaper{1}{cdt/theme/agua.png}
\thispagestyle{empty}

\maketitle
 
%=========================================================
% Hoja de revisión
\makeDocInfo
\vspace{0.5cm}
\makeElemRefs
\makeDocRefs
\makeObservaciones[3cm]
\vspace{0.5cm}
\makeFirmas

%=========================================================
% Indices del documento
\frontmatter
 \LRCornerWallPaper{1}{cdt/theme/pleca.png}
\tableofcontents
\listoffigures
%\listoftables
\mainmatter

% Para esconder la información del documentador se descomenta el \hideControlVersion
 \hideControlVersion
 
%=========================================================
\chapter{Introducción}\label{chp:introduccion}
    \cfinput{introduccion}
   
%=========================================================
\chapter{Marco Teórico}\label{chp:marcoTeorico}
    \cfinput{marcoteorico}
    
%=========================================================
%\chapter{Glosario de términos}\label{chp:glosarioTerminos}
%    \cfinput{ModeloNegocios/glosario}
\chapter{Requerimientos}\label{chp:requerimientos}
% \cfinput{req}

%=========================================================
%\chapter{Modelo de negocio}\label{chp:modeloNegocios}
%    \cfinput{ModeloNegocios/modelo}
%%    \cfinput{ModeloNegocios/estructura}
%%    \cfinput{ModeloNegocios/registro}
%%    \cfinput{ModeloNegocios/informacion-base}
%%    \cfinput{ModeloNegocios/plan-accion}
%%    \cfinput{ModeloNegocios/seguimiento}
%%    \cfinput{ModeloNegocios/reglas}
%%    \cfinput{ModeloNegocios/indicadores-agua}
%%    \cfinput{ModeloNegocios/indicadores-residuos}
%%    \cfinput{ModeloNegocios/indicadores-energia}
%%    \cfinput{ModeloNegocios/indicadores-biodiversidad}
%%    \cfinput{ModeloNegocios/indicadores-ambienteEscolar}
%%    \cfinput{ModeloNegocios/indicadores-consumoResponsable}
%    
%===========================================================
\chapter{Modelo de comportamiento}\label{chp:modeloComportamiento}
  \cfinput{ModeloComportamiento/comportamiento}

%===========================================================
\chapter{Modelo de comportamiento del módulo: Consulta de Salones \label{chp:modeloComportamientoProfesores}}
En este capítulo se describen los casos de uso referentes a la consulta de espacios con la finalidad de brindarles la ubicación de áreas en la Escuela Superior de Cómputo a los alumnos y visitantes. \bigskip

     \begin{objetivos}[Elementos de un caso de uso]
	\item {\bf Resumen:} Descripción textual del caso de uso.
	\item {\bf Actores:} Lista de los actores que intervienen en el caso de uso.
	\item {\bf Propósito:} Una breve descripción del objetivo que busca el actor al ejecutar el caso de uso.
	\item {\bf Entradas:} Lista de los datos de entrada requeridos durante la ejecución del caso de uso.
	\item {\bf Salidas:} Lista de los datos de salida que presenta el sistema durante la ejecución del caso de uso.
	\item {\bf Precondiciones:} Descripción de las operaciones o condiciones que se deben cumplir previamente para que el caso de uso pueda ejecutarse correctamente.
	\item {\bf Postcondiciones:} Lista de los cambios que ocurrirán en el sistema después de la ejecución del caso de uso y de las consecuencias en el sistema.
	\item {\bf Reglas de negocio:} Lista de las reglas que describen, limitan o controlan algún aspecto del negocio del caso de uso.
	\item {\bf Errores:} Lista de los posibles errores que pueden surgir durante la ejecución del caso de uso.
	\item {\bf Trayectorias:} Secuencia de los pasos que ejecutará el caso de uso.
    \end{objetivos}

	\cfinput{ModuloRegistro/cur1/uc}
	\cfinput{ModuloRegistro/cur2/uc}

%===========================================================
\chapter{Modelo de comportamiento del módulo: Consultar Salones asignados a grupo \label{chp:modeloComportamientoInformacionBase}}
     
  En este capítulo se describen los casos de uso referentes al registro, modificación y eliminación de información base para los indicadores. \bigskip

     \begin{objetivos}[Elementos de un caso de uso]
	\item {\bf Resumen:} Descripción textual del caso de uso.
	\item {\bf Actores:} Lista de los actores que intervienen en el caso de uso.
	\item {\bf Propósito:} Una breve descripción del objetivo que busca el actor al ejecutar el caso de uso.
	\item {\bf Entradas:} Lista de los datos de entrada requeridos durante la ejecución del caso de uso.
	\item {\bf Salidas:} Lista de los datos de salida que presenta el sistema durante la ejecución del caso de uso.
	\item {\bf Precondiciones:} Descripción de las operaciones o condiciones que se deben cumplir previamente para que el caso de uso pueda ejecutarse correctamente.
	\item {\bf Postcondiciones:} Lista de los cambios que ocurrirán en el sistema después de la ejecución del caso de uso y de las consecuencias en el sistema.
	\item {\bf Reglas de negocio:} Lista de las reglas que describen, limitan o controlan algún aspecto del negocio del caso de uso.
	\item {\bf Errores:} Lista de los posibles errores que pueden surgir durante la ejecución del caso de uso.
	\item {\bf Trayectorias:} Secuencia de los pasos que ejecutará el caso de uso.
    \end{objetivos}

  \cfinput{ModuloInformacionBase/cuib1/uc}
% 	\cfinput{ModuloInformacionBase/cuib2/uc}
% 	\cfinput{ModuloInformacionBase/cuib3/uc}
 

%===========================================================
%\chapter{Modelo de comportamiento del subsistema: Plan de acción} \label{chp:modeloComportamientoInformacionBase}
%     
%     En este capítulo se describen los casos de uso referentes al registro, modificación y eliminación de objetivos, metas y acciones del plan de acción. \bigskip
%
%     \begin{objetivos}[Elementos de un caso de uso]
%	\item {\bf Resumen:} Descripción textual del caso de uso.
%	\item {\bf Actores:} Lista de los actores que intervienen en el caso de uso.
%	\item {\bf Propósito:} Una breve descripción del objetivo que busca el actor al ejecutar el caso de uso.
%	\item {\bf Entradas:} Lista de los datos de entrada requeridos durante la ejecución del caso de uso.
%	\item {\bf Salidas:} Lista de los datos de salida que presenta el sistema durante la ejecución del caso de uso.
%	\item {\bf Precondiciones:} Descripción de las operaciones o condiciones que se deben cumplir previamente para que el caso de uso pueda ejecutarse correctamente.
%	\item {\bf Postcondiciones:} Lista de los cambios que ocurrirán en el sistema después de la ejecución del caso de uso y de las consecuencias en el sistema.
%	\item {\bf Reglas de negocio:} Lista de las reglas que describen, limitan o controlan algún aspecto del negocio del caso de uso.
%	\item {\bf Errores:} Lista de los posibles errores que pueden surgir durante la ejecución del caso de uso.
%	\item {\bf Trayectorias:} Secuencia de los pasos que ejecutará el caso de uso.
%    \end{objetivos}
%
%	\cfinput{ModuloPlanAccion/cup1/uc}
%
%
%   	
%%===========================================================
%\chapter{Modelo de comportamiento del subsistema: Seguimiento y acreditación \label{chp:modeloSeguimientoYAcreditacion}}
%     
%     En este capítulo se describen los casos de uso referentes al registro de avance de metas y acciones, así como también de actualización de las diferentes líneas de acción. \bigskip
%
%     \begin{objetivos}[Elementos de un caso de uso]
%  \item {\bf Resumen:} Descripción textual del caso de uso.
%  \item {\bf Actores:} Lista de los actores que intervienen en el caso de uso.
%  \item {\bf Propósito:} Una breve descripción del objetivo que busca el actor al ejecutar el caso de uso.
%  \item {\bf Entradas:} Lista de los datos de entrada requeridos durante la ejecución del caso de uso.
%  \item {\bf Salidas:} Lista de los datos de salida que presenta el sistema durante la ejecución del caso de uso.
%  \item {\bf Precondiciones:} Descripción de las operaciones o condiciones que se deben cumplir previamente para que el caso de uso pueda ejecutarse correctamente.
%  \item {\bf Postcondiciones:} Lista de los cambios que ocurrirán en el sistema después de la ejecución del caso de uso y de las consecuencias en el sistema.
%  \item {\bf Reglas de negocio:} Lista de las reglas que describen, limitan o controlan algún aspecto del negocio del caso de uso.
%  \item {\bf Errores:} Lista de los posibles errores que pueden surgir durante la ejecución del caso de uso.
%  \item {\bf Trayectorias:} Secuencia de los pasos que ejecutará el caso de uso.
%    \end{objetivos}     
%    
%  \cfinput{ModuloSeguimiento/cus1/uc}
%
%%---------------------------------------------------------------------
%
%%===========================================================
%\chapter{Modelo de comportamiento del subsistema: Indicadores \label{chp:indicadores}}
%     
%     En este capítulo se describen los casos de uso referentes a la consulta y al calculo de indicadores. \bigskip
%
%     \begin{objetivos}[Elementos de un caso de uso]
%  \item {\bf Resumen:} Descripción textual del caso de uso.
%  \item {\bf Actores:} Lista de los actores que intervienen en el caso de uso.
%  \item {\bf Propósito:} Una breve descripción del objetivo que busca el actor al ejecutar el caso de uso.
%  \item {\bf Entradas:} Lista de los datos de entrada requeridos durante la ejecución del caso de uso.
%  \item {\bf Salidas:} Lista de los datos de salida que presenta el sistema durante la ejecución del caso de uso.
%  \item {\bf Precondiciones:} Descripción de las operaciones o condiciones que se deben cumplir previamente para que el caso de uso pueda ejecutarse correctamente.
%  \item {\bf Postcondiciones:} Lista de los cambios que ocurrirán en el sistema después de la ejecución del caso de uso y de las consecuencias en el sistema.
%  \item {\bf Reglas de negocio:} Lista de las reglas que describen, limitan o controlan algún aspecto del negocio del caso de uso.
%  \item {\bf Errores:} Lista de los posibles errores que pueden surgir durante la ejecución del caso de uso.
%  \item {\bf Trayectorias:} Secuencia de los pasos que ejecutará el caso de uso.
%    \end{objetivos}     
%    
%  \cfinput{ModuloIndicadores/cui1/uc}
% 
%
%%---------------------------------------------------------------------
%
%%===========================================================
%\chapter{Modelo de comportamiento del subsistema: Escuela Libre de Derecho \label{chp:ELD}}
%
%En este capítulo se describen los casos de uso referentes a la consulta y al calculo de indicadores. \bigskip
%
%\begin{objetivos}[Elementos de un caso de uso]
%	\item {\bf Resumen:} Descripción textual del caso de uso.
%	\item {\bf Actores:} Lista de los actores que intervienen en el caso de uso.
%	\item {\bf Propósito:} Una breve descripción del objetivo que busca el actor al ejecutar el caso de uso.
%	\item {\bf Entradas:} Lista de los datos de entrada requeridos durante la ejecución del caso de uso.
%	\item {\bf Salidas:} Lista de los datos de salida que presenta el sistema durante la ejecución del caso de uso.
%	\item {\bf Precondiciones:} Descripción de las operaciones o condiciones que se deben cumplir previamente para que el caso de uso pueda ejecutarse correctamente.
%	\item {\bf Postcondiciones:} Lista de los cambios que ocurrirán en el sistema después de la ejecución del caso de uso y de las consecuencias en el sistema.
%	\item {\bf Reglas de negocio:} Lista de las reglas que describen, limitan o controlan algún aspecto del negocio del caso de uso.
%	\item {\bf Errores:} Lista de los posibles errores que pueden surgir durante la ejecución del caso de uso.
%	\item {\bf Trayectorias:} Secuencia de los pasos que ejecutará el caso de uso.
%\end{objetivos}     
%
%	%\cfinput{Carpeta/cuID/uc}
%	%\cfinput{Carpeta/cuID/uc}
%	%\cfinput{Carpeta/cuID/uc}
%	%\cfinput{Carpeta/cuID/uc}
%	%\cfinput{Carpeta/cuID/uc}
%	%\cfinput{Carpeta/cuID/uc}
%	%\cfinput{Carpeta/cuID/uc}
%	%\cfinput{Carpeta/cuID/uc}
%	%\cfinput{Carpeta/cuID/uc}
%	%\cfinput{Carpeta/cuID/uc}
%	%\cfinput{Carpeta/cuID/uc}
%	%\cfinput{Carpeta/cuID/uc}
%	
%	
%	\cfinput{seleccionEstudiantes/cuse1/uc}
%
%	%\cfinput{seleccionEstudiantes/cuse4/uc}
%
%%---------------------------------------------------------------------

    	
\chapter{Modelo de interacción con el usuario}\label{chp:modeloInteraccionUsuario}
\cfinput{ModeloInteraccion/interaccion}

%---------------------------------------------------------------------
\section{Interfaces del subsistema: Registro de escuelas}
      \cfinput{ModuloRegistro/cur1/ui} 


%---------------------------------------------------------------------
\section{Interfaces del subsistema: Información base para los indicadores}
%       \cfinput{ModuloInformacionBase/cuib1/ui}
%       \cfinput{ModuloInformacionBase/cuib2/ui}
      \cfinput{ModuloInformacionBase/cuiba1/ui}
 %---------------------------------------------------------------------
\section{Interfaces del subsistema: Plan de acción}
	\cfinput{ModuloPlanAccion/cup1/ui}



%---------------------------------------------------------------------

\section{Interfaces del subsistema: Seguimiento y acreditación}
      \cfinput{ModuloSeguimiento/cus1/ui}

%---------------------------------------------------------------------

\section{Interfaces del subsistema: Indicadores}
      \cfinput{ModuloIndicadores/cui2/ui}

%---------------------------------------------------------------------


%---------------------------------------------------------------------

\section{Diseño de mensajes}
 \cfinput{ModeloInteraccion/mensajes}

% Bibliografía
% \bibliographystyle{plain}
% \bibliography{bibSEIPA}
%\addcontentsline{toc}{chapter}{Bibliografía}

\clossing
\end{document}
