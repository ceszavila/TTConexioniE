\subsection{IUUM-5.4 Consultar Unidades de Aprendizaje}

\subsubsection{Objetivo}

% Explicar el objetivo para el que se construyo la interfaz, generalmente es la descripción de la actividad a desarrollar, como Seleccionar grupos para inscribir materias de un alumno, controlar el acceso al sistema mediante la solicitud de un login y password de los usuarios, etc.
	
    Esta pantalla permite al \cdtRef{actor:CIEAlumno}{Alumno} visualizar el listado de las unidades de aprendizaje que estarán disponibles para un periodo escolar vigente.
\subsubsection{Diseño}

% Presente la figura de la interfaz y explique paso a paso ``a manera de manual de usuario'' como se debe utilizar la interfaz. No olvide detallar en la redacción los datos de entradas y salidas. Explique como utilizar cada botón y control de la pantalla, para que sirven y lo que hacen. Si el Botón lleva a otra pantalla, solo indique la pantalla y explique lo que pasará cuando se cierre dicha pantalla (la explicación sobre el funcionamiento de la otra pantalla estará en su archivo correspondiente).

    En la figura \ref{IUUM-5.4} se muestra la pantalla ``Consultar Unidades de Aprendizaje'', la cual contiene una tabla en donde se muestra un listado con los nombres de las unidades de aprendizaje, dicha tabla deberá estar ordena por orden alfabético.
    \IUfig[.3]{/ModuloUnidades/IUUM5_4}{IUUM-5.4}{Consultar Unidades de Aprendizaje}



\subsubsection{Comandos}
    \begin{itemize}

	\item \textbf{Nombre de la unidad de aprendizaje}, dirige a la pantalla \cdtIdRef{IUUM-5.4.1}{Consultar Detalle de Unidad de Aprendizaje}.
	\item \cdtButton{Atrás}, dirige a la pantalla \cdtIdRef{IUPP}{Pantalla Principal}
    \end{itemize}
