\subsection{IUPA Pantalla Principal Administrador}

\subsubsection{Objetivo}

% Explicar el objetivo para el que se construyo la interfaz, generalmente es la descripción de la actividad a desarrollar, como Seleccionar grupos para inscribir materias de un alumno, controlar el acceso al sistema mediante la solicitud de un login y password de los usuarios, etc.
	
 	Esta pantalla permite ingresar a los módulos de la aplicación, es la pantalla donde se inicia cada uno de los módulos mencionados en la sección \textbf{Trabajo Realizado}.
\subsubsection{Diseño}

% Presente la figura de la interfaz y explique paso a paso ``a manera de manual de usuario'' como se debe utilizar la interfaz. No olvide detallar en la redacción los datos de entradas y salidas. Explique como utilizar cada botón y control de la pantalla, para que sirven y lo que hacen. Si el Botón lleva a otra pantalla, solo indique la pantalla y explique lo que pasará cuando se cierre dicha pantalla (la explicación sobre el funcionamiento de la otra pantalla estará en su archivo correspondiente).

    En la figura \ref{IUPA} se muestra la pantalla 'Pantalla Principal Administrador', en ella se encuentran los iconos de los diversos módulos de la aplicación \textbf{Conexión iE}.
    \IUfig[.3]{/ModuloUnidades/IUPA5A}{IUPA}{Pantalla Principal Administrador}



\subsubsection{Comandos}
\begin{itemize}
	
	\item Botón \botSalones, dirige a la pantalla \cdtIdRef{IUAS-3}{Gestionar Salones}
	\item Botón \botProfesores, dirige a la pantalla \cdtIdRef{IUAP-4}{Consultar Profesor}
	\item Botón \botUnidades, dirige a la pantalla \cdtIdRef{IUAU-5}{Gestionar Unidades de Aprendizaje}
\end{itemize}

