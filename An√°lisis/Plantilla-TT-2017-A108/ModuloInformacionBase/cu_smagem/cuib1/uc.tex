%!TEX encoding = UTF-8 Unicode

\begin{UseCase}{CUIB 1}{Administrar registro de información base para escuelas}
    {
	La información base de una escuela para las diferentes líneas de acción permite conocer el estado en que se encuentra la escuela con respecto a agua, energía eléctrica, biodiversidad, consumo responsable, ambiente escolar y residuos sólidos . Este caso de uso sirve como punto de acceso para acceder a la consulta de la información base registrada para las diferentes escuelas dentro del sistema.
    }
    
    \UCitem{Versión}{1.0}
    \UCccsection{Administración de Requerimientos}
    \UCitem{Autor}{Jessica Stephany Becerril Delgado}
    \UCccitem{Evaluador}{}
    \UCitem{Operación}{Administrar}
    \UCccitem{Prioridad}{Media}
    \UCccitem{Complejidad}{Media}
    \UCccitem{Volatilidad}{Alta}
    \UCccitem{Madurez}{Media}
    \UCitem{Estatus}{Terminado}
    \UCitem{Fecha del último estatus}{25 de Noviembre del 2014}

    
%% Copie y pegue este bloque tantas veces como revisiones tenga el caso de uso.
%% Esta sección la debe llenar solo el Revisor
% %--------------------------------------------------------
% 	\UCccsection{Revisión Versión XX} % Anote la versión que se revisó.
% 	% FECHA: Anote la fecha en que se terminó la revisión.
% 	\UCccitem{Fecha}{Fecha en que se termino la revisión} 
% 	% EVALUADOR: Coloque el nombre completo de quien realizó la revisión.
% 	\UCccitem{Evaluador}{Nombre de quien revisó}
% 	% RESULTADO: Coloque la palabra que mas se apegue al tipo de acción que el analista debe realizar.
% 	\UCccitem{Resultado}{Corregir, Desechar, Rehacer todo, terminar.}
% 	% OBSERVACIONES: Liste los cambios que debe realizar el Analista.
% 	\UCccitem{Observaciones}{
% 		\begin{UClist}
% 			% PC: Petición de Cambio, describa el trabajo a realizar, si es posible indique la causa de la PC. Opcionalmente especifique la fecha en que considera razonable que se deba terminar la PC. No olvide que la numeración no se debe reiniciar en una segunda o tercera revisión.
% 			\RCitem{PC1}{\TODO{Descripción del pendiente}}{Fecha de entrega}
% 			\RCitem{PC2}{\TODO{Descripción del pendiente}}{Fecha de entrega}
% 			\RCitem{PC3}{\TODO{Descripción del pendiente}}{Fecha de entrega}
% 		\end{UClist}		
% 	}
% %--------------------------------------------------------

    \UCsection{Atributos}
    \UCitem{Actor}{\cdtRef{actor:usuarioSMAGEM}{Director del programa}}
    \UCitem{Propósito}{Consultar la información base registrada para las diferentes escuelas registradas en el sistema.}
    \UCitem{Entradas}{
	\begin{UClist}
	    \UCli Ninguna.
	\end{UClist}
    }

    \UCitem{Salidas}{
	\begin{UClist} 
	    \UCli \cdtRef{escuela}{Escuelas}: \ioTabla{\cdtRef{escuela:cct}{Clave de centro de trabajo}, \cdtRef{escuela:nombreEscuela}{Escuela} y  \cdtRef{escuela:municipio}{Municipio}}{que estén en el sistema}.
	\end{UClist}
    }
    \UCitem{Precondiciones}{
	\begin{UClist}
	    \UCli {\bf Interna:} Que la información base para indicadores se encuentre en estado \cdtRef{estado:aprobarInfoBase}{Por aprobar}.
	    \UCli {\bf Interna:} Que la escuela cuente con información base registrada para cada una de sus líneas de acción.
	\end{UClist}
    }
    
    \UCitem{Postcondiciones}{
	\begin{UClist}
	    \UCli {\bf Interna:} Se podrá acceder a la aprobación de la información base de cada uno de los registros de escuelas en el sistema por medio del caso de uso \cdtIdRef{CUIB 2}{Aprobar la información base para indicadores}.	    
	\end{UClist}
    }
    \UCitem{Reglas de negocio}{
    	\begin{UClist}
	    \UCli Ninguna.
	\end{UClist}
    }
    
    \UCitem{Errores}{
	\begin{UClist}
	    \UCli \cdtIdRef{MSG2}{No existe información registrada por el momento}: Se muestra en la pantalla \cdtIdRef{IUIB 1}{Administrar registro de información base para escuelas} indicando al actor que no existen registros de escuelas en el sistema por el momento.
	    \UCli \cdtIdRef{MSG28}{Operación no permitida por estado de la entidad}: Se muestra en la pantalla \cdtIdRef{IUIB 1}{Administrar registro de información base para escuelas} indicando al actor que no se puede administrar el registro de información base para escuelas debido al estado en que se encuentra la información base para indicadores.
	\end{UClist}
    }

    \UCitem{Tipo}{Primario.}

    %FUENTE: Especifique la fuente de información principal para la especificación del Caso de Uso: un documento, sistema existente, persona, minuta. Referencie dicho documento, sistema o persona.
%    \UCitem{Fuente}{
%	\begin{UClist}
%	    \UCli Minuta de la reunión \cdtIdRef{M-17RT}{Reunión de trabajo}.
%	\end{UClist}
%    }

\end{UseCase}

 \begin{UCtrayectoria}
    \UCpaso[\UCactor] Solicita administrar la información base para las escuelas, seleccionando en el menú \cdtIdRef{MN1}{Menú del Director del programa} la opción ``Información base para indicadores'' y posteriormente la opción ``Aprobar información base''. 
    \UCpaso[\UCsist] Verifica que existan registros de escuelas en el sistema. \refTray{A}.
    \UCpaso[\UCsist] Verifica que la  información base para indicadores se encuentre en estado ``Por aprobar''. \refTray{B}.
    \UCpaso[\UCsist] Muestra la pantalla \cdtIdRef{IUIB 1}{Administrar registro de información base para escuelas}.
    \UCpaso[\UCactor] Administra la información base de las escuelas a través del botón \botOk. \label{cuib1:Aprobar}
 \end{UCtrayectoria}
 
   \begin{UCtrayectoriaA}[Fin del caso de uso]{A}{La información base para indicadores no se encuentra en un estado que permita administrarla.}
    \UCpaso[\UCsist] Muestra el mensaje \cdtIdRef{MSG28}{Operación no permitida por estado de la entidad} indicando al actor que no puede administrar la información base para indicadores debido a que esta no se encuentra en estado ``Por aprobar''. 
 \end{UCtrayectoriaA}
 
   \begin{UCtrayectoriaA}[Fin del caso de uso]{B}{No hay registros de escuelas para mostrar.}
    \UCpaso[\UCsist] Muestra el mensaje \cdtIdRef{MSG2}{No existe información registrada por el momento} en la pantalla \cdtIdRef{IUIB 1}{Administrar registro de información base para escuelas} indicando al actor que aún no hay escuelas registradas. 
 \end{UCtrayectoriaA}
 
\subsection{Puntos de extensión}

\UCExtensionPoint
{El actor desea aprobar la información base de una escuela.}
{ Paso \ref{cuib1:Aprobar} de la trayectoria principal}
{\cdtRef{CUIB 2}{Aprobar información base para indicadores}}
