\subsection{IUIBA 1 Administrar información base para indicadores de agua}

\subsubsection{Objetivo}
	
    En esta pantalla el \cdtRef{actor:usuarioEscuela}{Coordinador del programa} puede acceder al registro o modificación de la información base para indicadores de agua.

\subsubsection{Diseño}

    En la figura~\ref{IUIBA 1} se muestra la pantalla ``Administrar información base para indicadores de agua'', por medio de la cual se podrá acceder al registro de la información base para indicadores de agua. El actor tendrá la facultad de registrar o modificar la información que proporcione una visión general del estado en que se encuentra la escuela respecto al consumo y ahorro de agua  a través del botón \botEdit.  

    \IUfig[.9]{pantallas/InformacionBase/cuiba1/IUIBA1AdministrarInformacion.png}{IUIBA 1}{Administrar información base para indicadores de agua}


\subsubsection{Comandos}
    \begin{itemize}
	\item \botEdit [Modificar información base]: Permite al actor registrar la información base para indicadores de agua o modificarla cuando existe información previamente registrada , dirige a la pantalla \cdtIdRef{IUIBA 2}{Registrar información base para indicadores de agua}. 
    \end{itemize}

\subsubsection{Mensajes}

    \begin{description}
	\item [\cdtIdRef{MSG28}{Operación no permitida por estado de la entidad}:] Se muestra en la pantalla \cdtIdRef{IUIBA 1}{Administrar información base para indicadores de agua} indicando al actor que no se puede realizar la operación debido al estado en que se encuentra la escuela.
	
	\item [\cdtIdRef{MSG41}{Acción fuera del periodo}:] Se muestra en la pantalla \cdtIdRef{IUIBA 1}{Administrar información base para indicadores de agua} para indicarle al actor que no puede realizar la operación debido a que la fecha actual se encuentra fuera del periodo definido por la SMAGEM para realizarla.
    \end{description}
