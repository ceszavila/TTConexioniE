\subsection{IUIBB 2 Registrar información base para indicadores de biodiversidad}

\subsubsection{Objetivo}

      En esta pantalla el \cdtRef{actor:usuarioEscuela}{Coordinador del programa} puede registrar o modificar la información base para indicadores de biodiversidad que permita conocer el nivel de conocimiento del concepto de biodiversidad en el ámbito escolar, así como también los ecosistemas y áreas verdes cercanos a la escuela.

\subsubsection{Diseño}

    En la figura~\ref{IUIBB 2} se muestra la pantalla ``Registrar información base para indicadores de biodiversidad'', por medio de la cual se podrá registrar la información base que permitirá conocer la cantidad de personas que conocen el concepto de biodiversidad, además de los ecosistemas y áreas verdes que se localizan cerca de la escuela, la superficie del predio total, construido y de áreas verdes . El actor deberá ingresar la información solicitada en la pantalla, en el caso de que se trate de una modificación de la información aparecerán los datos previamente ingresados en los campos respectivos de la pantalla.\\
    
    En el caso de que se seleccione alguna de las opciones en la pregunta ``Tipo de ecosistemas'' se mostrarán los campos ``Distancia desde la escuela'' y ``Ubicación'' por cada tipo de ecosistema seleccionado, como se muestra en la figura~\ref{IUIBB 2.2}. \\
    
    Una vez que se haya ingresado toda la información solicitada para el registro de la información deberá oprimir el botón \cdtButton{Aceptar}, si se trata de una modificación se mostrará el mensaje \cdtIdRef{MSG30}{Confirmar la modificación de un registro} en una pantalla emergente como se muestra en la figura~\ref{IUIBB 2.1} para indicar al actor que la información previa a la modificación se perderá. El sistema validará y registrará la información sólo si se han cumplido todas las reglas de negocio establecidas.\\
    
    Finalmente se mostrará el mensaje \cdtIdRef{MSG1}{Operación realizada exitosamente} en la pantalla \cdtIdRef{IUIBB 1}{ Administrar información base para indicadores de biodiversidad}, para indicar que la información base se ha registrado o modificado exitosamente.
      
    \IUfig[.9]{pantallas/InformacionBase/cuibb2/IUIBB2RegistrarInformacion.png}{IUIBB 2}{Registrar información base para indicadores de biodiversidad}
    \IUfig[.7]{pantallas/InformacionBase/cuibb2/IUIBB2Modificar.png}{IUIBB 2.1}{Registrar información base para indicadores de biodiversidad: Confirmar modificación}
    \IUfig[.9]{pantallas/InformacionBase/cuibb2/IUIBB2RegistrarInformacion2.png}{IUIBB 2.2}{Registrar información base para indicadores de biodiversidad: Tipos de ecosistemas}    
    


\subsubsection{Comandos}
    \begin{itemize}
	\item \cdtButton{Aceptar}: Permite al actor confirmar el registro o modificación de la información base para indicadores de biodiversidad, dirige a la pantalla \cdtIdRef{IUIBB 1}{ Administrar información base para indicadores de biodiversidad}.
	\item \cdtButton{Cancelar}: Permite al actor cancelar el registro o modificación de la información base para indicadores de biodiversidad, dirige a la pantalla \cdtIdRef{IUIBB 1}{ Administrar información base para indicadores de biodiversidad}.
    \end{itemize}

\subsubsection{Mensajes}

    \begin{description}
      
	    \item [\cdtIdRef{MSG1}{Operación realizada exitosamente}:] Se muestra en la pantalla \cdtIdRef{IUIBB 1}{Administrar información base para indicadores de biodiversidad} cuando el registro de la información base para indicadores de biodiversidad se ha realizado correctamente.	    
	    
	    \item [\cdtIdRef{MSG5}{Falta un dato requerido para efectuar la operación solicitada}:] Se muestra en la pantalla \cdtIdRef{IUIBB 2}{Registrar información base para indicadores de biodiversidad} cuando no se ha ingresado un dato marcado como requerido.
	    
	    \item [\cdtIdRef{MSG6}{Formato incorrecto}:] Se muestra en la pantalla \cdtIdRef{IUIBB 2}{Registrar información base para indicadores de biodiversidad} cuando el tipo de dato ingresado no cumple con el tipo de dato solicitado en el campo.
	    
	    \item [\cdtIdRef{MSG7}{Se ha excedido la longitud máxima del campo}:] Se muestra en la pantalla \cdtIdRef{IUIBB 2}{Registrar información base para indicadores de biodiversidad} cuando se ha excedido la longitud de alguno de los campos.
	    
	    \item[\cdtIdRef{MSG28}{Operación no permitida por estado de la entidad}:] Se muestra en la pantalla \cdtIdRef{IUIBB 1}{Administrar información base para indicadores de biodiversidad} indicando al actor que no se puede realizar la operación debido al estado en que se encuentra la escuela.
	    
	    \item [\cdtIdRef{MSG30}{Confirmar la modificación de un registro}:] Se muestra en la pantalla emergente \cdtIdRef{IUIBA 2.1}{Registrar información base para indicadores de agua: Mensaje de confirmación} para indicar al actor que al guardar los cambios realizados en la información base la información previa se perderá.  
	    
	    \item [\cdtIdRef{MSG41}{Acción fuera del periodo}:] Se muestra sobre la pantalla en la pantalla \cdtIdRef{IUIBB 1}{Administrar información base para indicadores de biodiversidad} para indicarle al actor que no puede realizar la operación debido a que la fecha actual se encuentra fuera del periodo definido por la SMAGEM para realizarla.
    \end{description}
