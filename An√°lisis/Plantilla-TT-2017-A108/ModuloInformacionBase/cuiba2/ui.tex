\subsection{IUIBA 2 Registrar información base para indicadores de agua}

\subsubsection{Objetivo}

      En esta pantalla el \cdtRef{actor:usuarioEscuela}{Coordinador del programa} puede registrar o modificar la información base para indicadores de agua que permita conocer el estado en que se encuentra una escuela con respecto al consumo y ahorro de agua.

\subsubsection{Diseño}

    En la figura~\ref{IUIBA 2} se muestra la pantalla ``Registrar información base para indicadores de agua'', por medio de la cual se podrá registrar la información base que permitirá conocer el estado en que se encuentra una escuela con respecto al consumo y ahorro de agua. El actor deberá ingresar la información solicitada en la pantalla, en el caso de que se trate de una modificación de la información aparecerán los datos previamente ingresados en la pantalla.\\
    
    Si el actor selecciona la opción ``Sin acceso al agua'' en la pregunta ``¿Con qué tipos de abastecimiento de agua cuenta la escuela?'' el sistema únicamente mostrará los botones \cdtButton{Aceptar} y \cdtButton{Cancelar} sin solicitar mas información como se muestra en la figura~\ref{IUIBA 2.2}.\\
    
    Si se selecciona la opción ``Si'' en la pregunta ``¿Cuenta con recibos sobre el consumo de agua en la escuela?'' se solicitará la frecuencia con la que recibe los recibos de cobro por concepto de agua como se muestra en la figura~\ref{IUIBA 2.3}. Si elige la opción ``Anual'' se mostrarán dos campos de texto para el ``Consumo anual'' en metros cúbicos y el ``Importe anual'' en pesos respectivamente.\\
    
    En el caso de que seleccione la opción ``Bimestral'', deberá seleccionar el bimestre al que corresponde cada uno de los recibos que registrará e ingresar el consumo de agua por bimestre en metros cúbicos y el importe bimestral en pesos. Los registros por cada uno de los recibos bimestrales se mostrarán en una tabla, el consumo e importe por cada bimestre se sumará para mostrar los totales de ambos datos al final de la tabla como se muestra en la figura~\ref{IUIBA 2.4}. Lo mismo ocurrirá para las opciones ``Mensual'' y ``Semestral''. \\
    
    Si se selecciona la opción ``No'' en la pregunta ``¿Cuenta con recibos sobre el consumo de agua en la escuela?'' se solicitará únicamente el consumo de agua e importe anual promedio, como se muestra en la figura~\ref{IUIBA 2.5}.\\
    
    Una vez que se haya ingresado toda la información solicitada para el registro de la información deberá oprimir el botón \cdtButton{Aceptar}, si se trata de una modificación se mostrará el mensaje \cdtIdRef{MSG30}{Confirmar la modificación de un registro} en una pantalla emergente como se muestra en la figura~\ref{IUIBA 2.1} para indicar al actor que la información previa a la modificación se perderá. El sistema validará y registrará la información sólo si se han cumplido todas las reglas de negocio establecidas.\\
    
    Finalmente se mostrará el mensaje \cdtIdRef{MSG1}{Operación realizada exitosamente} en la pantalla \cdtIdRef{IUIBA 2}{Registrar información base para indicadores de agua}, para indicar que la información base se ha registrado o modificado exitosamente.
      
    \IUfig[.9]{pantallas/InformacionBase/cuiba2/IUIBA2RegistrarInformacion.png}{IUIBA 2}{Registrar información base para indicadores de agua}
    \IUfig[.7]{pantallas/InformacionBase/cuiba2/IUIBA2ModificarInformacion.png}{IUIBA 2.1}{Registrar información base para indicadores de agua: Mensaje de confirmación}
    \IUfig[.9]{pantallas/InformacionBase/cuiba2/IUIBA2RegistrarInformacion1.png}{IUIBA 2.2}{Registrar información base para indicadores de agua: Sin acceso al agua}
    \IUfig[.9]{pantallas/InformacionBase/cuiba2/IUIBA2RegistrarInformacionRecibos.png}{IUIBA 2.3}{Registrar información base para indicadores de agua: Recibos de agua}
    \IUfig[.9]{pantallas/InformacionBase/cuiba2/IUIBA2RegistrarInformacionBimestre.png}{IUIBA 2.4}{Registrar información base para indicadores de agua: Recibos de agua bimestrales}
    \IUfig[.9]{pantallas/InformacionBase/cuiba2/IUIBA2RegistrarInformacionSinRecibos.png}{IUIBA 2.5}{Registrar información base para indicadores de agua: Sin recibos de agua}
    
    


\subsubsection{Comandos}
    \begin{itemize}
	\item \cdtButton{Aceptar}: Permite al actor registrar o modificar la información base para indicadores de agua, dirige a la pantalla \cdtIdRef{IUIBA 1}{ Administrar información base para indicadores de agua}.
	\item \cdtButton{Cancelar}: Permite al actor cancelar el registro o modificación de la información base para indicadores de agua, dirige a la pantalla \cdtIdRef{IUIBA 1}{ Administrar información base para indicadores de agua}.
    \end{itemize}

\subsubsection{Mensajes}

    \begin{description}
	  \item [\cdtIdRef{MSG1}{Operación realizada exitosamente}:] Se muestra en la pantalla \cdtIdRef{IUIBA 1}{Administrar información base para indicadores de agua} cuando el registro de la información base para indicadores de agua se ha realizado correctamente.
	    
	  \item [\cdtIdRef{MSG5}{Falta un dato requerido para efectuar la operación solicitada}:] Se muestra en la pantalla \cdtIdRef{IUIBA 2}{Registrar información base para indicadores de agua} cuando no se ha ingresado un dato marcado como requerido.
	    
	  \item  [\cdtIdRef{MSG6}{Formato incorrecto}:] Se muestra en la pantalla \cdtIdRef{IUIBA 2}{Registrar información base para indicadores de agua} cuando el tipo de dato ingresado no cumple con el tipo de dato solicitado en el campo.
	    
	  \item [\cdtIdRef{MSG7}{Se ha excedido la longitud máxima del campo}:] Se muestra en la pantalla \cdtIdRef{IUIBA 2}{Registrar información base para indicadores de agua} cuando se ha excedido la longitud de alguno de los campos.	    
	    
	   \item  [\cdtIdRef{MSG30}{Confirmar la modificación de un registro}:] Se muestra en la pantalla emergente \cdtIdRef{IUIBA 2.1}{Registrar información base para indicadores de agua: Mensaje de confirmación} para indicar al actor que al guardar los cambios realizados en la información base la información previa se perderá.  
	   
	   \item[\cdtIdRef{MSG28}{Operación no permitida por estado de la entidad}:] Se muestra en la pantalla \cdtIdRef{IUIBA 1}{Administrar información base para indicadores de agua} indicando al actor que no se puede realizar la operación debido al estado en que se encuentra la escuela.
	   
	   \item [\cdtIdRef{MSG41}{Acción fuera del periodo}:] Se muestra sobre la pantalla en la pantalla \cdtIdRef{IUIBA 1}{Administrar información base para indicadores de agua} para indicarle al actor que no puede realizar la operación debido a que la fecha actual se encuentra fuera del periodo definido por la SMAGEM para realizarla.
    \end{description}
