%!TEX encoding = UTF-8 Unicode

\begin{UseCase}{CUIBAE 2}{Registrar información base para indicadores de ambiente escolar}
    {
	  Este caso de uso permite al actor el registro o modificación de informacón base que contribuya como referencia para los indicadores de ambiente escolar, proporcionando una visión general del estado en que se encuentra una escuela con respecto a sus instalaciones, espacios de estudio y recreación. Una vez que los datos solicitados para el registro de la información base para indicadores de ambiente escolar han sido ingresados, el sistema valida la información y esta queda registrada.
    }
    
    \UCitem{Versión}{1.0}
    \UCccsection{Administración de Requerimientos}
    \UCitem{Autor}{Jessica Stephany Becerril Delgado}
    \UCccitem{Evaluador}{}
    \UCitem{Operación}{Registro}
    \UCccitem{Prioridad}{Media}
    \UCccitem{Complejidad}{Media}
    \UCccitem{Volatilidad}{Media}
    \UCccitem{Madurez}{Media}
    \UCitem{Estatus}{Terminado}
    \UCitem{Fecha del último estatus}{20 de Noviembre del 2014}

%% Copie y pegue este bloque tantas veces como revisiones tenga el caso de uso.
%% Esta sección la debe llenar solo el Revisor
% %--------------------------------------------------------
 	\UCccsection{Revisión Versión XX} % Anote la versión que se revisó.
% 	% FECHA: Anote la fecha en que se terminó la revisión.
 	\UCccitem{Fecha}{Fecha en que se termino la revisión} 
% 	% EVALUADOR: Coloque el nombre completo de quien realizó la revisión.
 	\UCccitem{Evaluador}{Nombre de quien revisó}
% 	% RESULTADO: Coloque la palabra que mas se apegue al tipo de acción que el analista debe realizar.
 	\UCccitem{Resultado}{Corregir, Desechar, Rehacer todo, terminar.}
% 	% OBSERVACIONES: Liste los cambios que debe realizar el Analista.
 	\UCccitem{Observaciones}{
 		\begin{UClist}
 			% PC: Petición de Cambio, describa el trabajo a realizar, si es posible indique la causa de la PC. Opcionalmente especifique la fecha en que considera razonable que se deba terminar la PC. No olvide que la numeración no se debe reiniciar en una segunda o tercera revisión.
 			\RCitem{PC1}{\TOCHK{Precondiciones, tomar en cuenta el estado de EDICIÓN}}{Fecha de entrega}
 			\RCitem{PC1}{\TOCHK{Trayectoria principal, validar el estado de EDICIÓN}}{Fecha de entrega}
 			\RCitem{PC2}{\TOCHK{Resumen, cambiar la palabra áreas por espacios}}{Fecha de entrega}

 		\end{UClist}		
 	}
% %--------------------------------------------------------

    \UCsection{Atributos}
    \UCitem{Actor}{\cdtRef{actor:usuarioEscuela}{Coordinador del programa}}
    \UCitem{Propósito}{Registrar o modificar la información base que permita conocer el estado en que se encuentra una escuela en cuestión de sus instalaciones, espacios de estudio y recreación.}
    \UCitem{Entradas}{
	\begin{UClist}
	    \UCli Información de los espacios con los que cuenta la escuela:
	    \begin{itemize}
	     \item \cdtRef{ambiente:tipoEspacio}{Tipo de espacios}: \ioCheckBox.
	     \item \cdtRef{ambiente:usoJardin}{Uso de jardín}: \ioEscribir. 
	     \item \cdtRef{ambiente:usoBiblioteca}{Uso de biblioteca}: \ioEscribir. 
	     \item \cdtRef{ambiente:usoPatio}{Uso de patio}: \ioEscribir. 
	     \item \cdtRef{ambiente:usoPeriodicoMural}{Uso del área para periódico mural}: \ioEscribir. 
	     \item \cdtRef{ambiente:usoComedor}{Uso de comedor}: \ioEscribir. 
	     \item \cdtRef{ambiente:usoSalonMusica}{Uso de salón de música}: \ioEscribir. 
	     \item \cdtRef{ambiente:usoSalonComputo}{Uso de salón de cómputo}: \ioEscribir. 
	     \item \cdtRef{ambiente:usoSalonAudiovisual}{Uso de salón audiovisual}: \ioEscribir. 
	     \item \cdtRef{ambiente:usoDeportivas}{Uso de instalaciones deportivas}: \ioEscribir. 
	     \item \cdtRef{ambiente:usoSalaJuntas}{Uso de sala de juntas}: \ioEscribir. 
	     \item \cdtRef{ambiente:usoAulas}{Uso de aulas}: \ioEscribir. 
	     \item \cdtRef{ambiente:usoAdministrativa}{Uso de  administrativas}: \ioEscribir. 
	     \item \cdtRef{ambiente:usoSanitarios}{Uso de sanitarios}: \ioEscribir. 
	    \end{itemize}

	\end{UClist}
    }

    \UCitem{Salidas}{
	\begin{UClist} 
	    \UCli Información de los espacios con los que cuenta la escuela:
	    \begin{itemize}
	     \item \cdtRef{ambiente:tipoEspacio}{Tipo de espacios}: \ioObtener.
	     \item \cdtRef{ambiente:usoJardin}{Uso de jardín}: \ioObtener. 
	     \item \cdtRef{ambiente:usoBiblioteca}{Uso de biblioteca}: \ioObtener. 
	     \item \cdtRef{ambiente:usoPatio}{Uso de patio}: \ioObtener. 
	     \item \cdtRef{ambiente:usoPeriodicoMural}{Uso del área para periódico mural}: \ioObtener. 
	     \item \cdtRef{ambiente:usoComedor}{Uso de comedor}: \ioObtener. 
	     \item \cdtRef{ambiente:usoSalonMusica}{Uso de salón de música}: \ioObtener. 
	     \item \cdtRef{ambiente:usoSalonComputo}{Uso de salón de cómputo}: \ioObtener. 
	     \item \cdtRef{ambiente:usoSalonAudiovisual}{Uso de salón audiovisual}: \ioObtener. 
	     \item \cdtRef{ambiente:usoDeportivas}{Uso de instalaciones deportivas}: \ioObtener. 
	     \item \cdtRef{ambiente:usoSalaJuntas}{Uso de sala de juntas}: \ioObtener. 
	     \item \cdtRef{ambiente:usoAulas}{Uso de aulas}: \ioObtener. 
	     \item \cdtRef{ambiente:usoAdministrativa}{Uso de  administrativas}: \ioObtener. 
	     \item \cdtRef{ambiente:usoSanitarios}{Uso de sanitarios}: \ioObtener. 
	    \end{itemize}
	    
	    \UCli \cdtIdRef{MSG1}{Operación realizada exitosamente:} Se muestra en la pantalla \cdtIdRef{IUIBAE 1}{Administrar información base para indicadores de ambiente escolar} cuando el registro de la información base para indicadores de ambiente escolar se ha realizado correctamente.
	    
	    \UCli \cdtIdRef{MSG30}{Confirmar la modificación de un registro}: Se muestra en la pantalla emergente \cdtIdRef{IUIBAE 2.1}{Registrar información base para indicadores de ambiente escolar: Mensaje de confirmación} para indicar al actor que al guardar los cambios realizados la información previa se perderá.  
	\end{UClist}
    }

    \UCitem{Precondiciones}{
	\begin{UClist}
	    \UCli {\bf Interna:} Que la escuela se encuentre en estado \cdtRef{estado:infoBaseEdicion}{Información base en edición}.
	    \UCli {\bf Interna:} Que el periodo de registro de información base se encuentre vigente. 
	\end{UClist}
    }
    
    \UCitem{Postcondiciones}{
	\begin{UClist}
	    \UCli {\bf Interna:} El actor podrá modificar la información base definida para los indicadores de ambiente escolar a través del caso de uso \cdtIdRef{CUIBAE 2}{Registrar información base para indicadores de ambiente escolar}.
	\end{UClist}
    }
    
    \UCitem{Reglas de negocio}{
    	\begin{UClist}
	    \UCli \cdtIdRef{RN-S1}{Información correcta}: Verifica que la información introducida sea correcta.
	\end{UClist}
    }
    
    \UCitem{Errores}{
	\begin{UClist}
	    
	    \UCli \cdtIdRef{MSG5}{Falta un dato requerido para efectuar la operación solicitada}: Se muestra en la pantalla \cdtIdRef{IUIBAE 2}{Registrar información base para indicadores de ambiente escolar} cuando no se ha ingresado un dato marcado como requerido.
	    
	     \UCli \cdtIdRef{MSG6}{Formato incorrecto}: Se muestra en la pantalla \cdtIdRef{IUIBAE 2}{Registrar información base para indicadores de ambiente escolar} cuando el tipo de dato ingresado no cumple con el tipo de dato solicitado en el campo.
	    
	    \UCli \cdtIdRef{MSG7}{Se ha excedido la longitud máxima del campo}: Se muestra en la pantalla \cdtIdRef{IUIBAE 2}{Registrar información base para indicadores de ambiente escolar} cuando se ha excedido la longitud de alguno de los campos.	
	    
	    \UCli \cdtIdRef{MSG28}{Operación no permitida por estado de la entidad}: Se muestra en la pantalla \cdtIdRef{IUIBAE 1}{Administrar información base para indicadores de ambiente escolar} indicando al actor que no se puede realizar la operación debido al estado en que se encuentra la escuela.
	    
	    \UCli \cdtIdRef{MSG41}{Acción fuera del periodo}: Se muestra sobre la pantalla en la pantalla \cdtIdRef{IUIBAE 1}{Administrar información base para indicadores de ambiente escolar} para indicarle al actor que no puede realizar la operación debido a que la fecha actual se encuentra fuera del periodo definido por la SMAGEM para realizarla.
	\end{UClist}
    }

    \UCitem{Tipo}{Secundario, extiende del caso de uso \cdtIdRef{CUIBAE 1}{Administrar información base para indicadores de ambiente escolar}.}

%    \UCitem{Fuente}{
%	\begin{UClist}
%	    \UCli Minuta de la reunión \cdtIdRef{M-17RT}{Reunión de trabajo}.
%	\end{UClist}
%    }
\end{UseCase}

\begin{UCtrayectoria}
    \UCpaso[\UCactor] Solicita registrar la información base para indicadores de ambiente escolar oprimiendo el botón \botEdit de la pantalla \cdtIdRef{IUIBAE 1}{Administrar información base para indicadores de ambiente escolar}.
    \UCpaso[\UCsist] Verifica que la escuela se encuentre en estado ``Información base en edición''. \refTray{A}.
    \UCpaso[\UCsist] Verifica que la fecha actual se encuentre dentro del periodo definido por la SMAGEM para realizar la operación. \refTray{B}.
    \UCpaso[\UCsist] Busca la información previamente registrada referente a la información base para indicadores de ambiente escolar.
    \UCpaso[\UCsist] Muestra la pantalla \cdtIdRef{IUIBAE 2}{Registrar información base para indicadores de ambiente escolar} por medio de la cual se realizará el registro de información base para indicadores de ambiente escolar.
    \UCpaso[\UCactor] Ingresa la información referente a la información base para indicadores de ambiente escolar en la pantalla \cdtIdRef{IUIBAE 2}{Registrar información base para indicadores de ambiente escolar}. \label{cuibae2:IngresarDatos}
    \UCpaso[\UCactor] Oprime el botón \cdtButton{Aceptar} en la pantalla \cdtIdRef{IUIBAE 2}{Registrar información base para indicadores de ambiente escolar} para confirmar el registro de la información base para indicadores de ambiente escolar. \refTray{C}. \refTray{E}. 
    \UCpaso[\UCsist] Verifica que la escuela se encuentre en estado ``Información base en edición''. \refTray{A}. \label{cuibae2:VerificarRegistro}
    \UCpaso[\UCsist] Verifica que la fecha actual se encuentre dentro del periodo definido por la SMAGEM para realizar la operación. \refTray{B}.
    \UCpaso[\UCsist] Verifica que los datos ingresados proporcionen información correcta con base en la regla de negocio \cdtIdRef{RN-S1}{Información correcta}. \refTray{F}. \refTray{G}. \refTray{H}.     
    \UCpaso[\UCsist] Registra la información base para indicadores de ambiente escolar.
    \UCpaso[\UCsist] Muestra el mensaje \cdtIdRef{MSG1}{Operación realizada exitosamente} en la pantalla \cdtIdRef{IUIBAE 1}{Administrar información base para indicadores de ambiente escolar} para indicar al actor que el registro de la información se ha realizado exitosamente.    
 \end{UCtrayectoria}
 
    \begin{UCtrayectoriaA}[Fin del caso de uso]{A}{La escuela no se encuentra en un estado que permita realizar la operación.}
    \UCpaso[\UCsist] Muestra el mensaje \cdtIdRef{MSG28}{Operación no permitida por estado de la entidad} en la pantalla \cdtIdRef{IUIBAE 1}{Administrar información base para indicadores de ambiente escolar} indicando al actor que no puede realizar la operación debido a que la escuela no se encuentra en estado ``Información base en edición''. 
 \end{UCtrayectoriaA}

   \begin{UCtrayectoriaA}[Fin del caso de uso]{B}{La fecha actual se encuentra fuera del periodo definido por la SMAGEM para realizar la operación.}
    \UCpaso[\UCsist] Muestra el mensaje \cdtIdRef{MSG41}{Acción fuera del periodo} sobre la pantalla \cdtIdRef{IUIBAE 1}{Administrar información base para indicadores de ambiente escolar} indicando al actor que no puede realizar la operación debido a que la fecha actual se encuentra fuera del periodo definido por la SMAGEM para realizarla. 
 \end{UCtrayectoriaA}
 
  \begin{UCtrayectoriaA}{C}{El actor desea modificar la información base para indicadores de ambiente escolar.}
    \UCpaso[\UCsist] Muestra el mensaje \cdtIdRef{MSG30}{Confirmar la modificación de un registro} en la pantalla emergente \cdtIdRef{IUIBAE 2.1}{Registrar información base para indicadores de ambiente escolar: Mensaje de confirmación} para que el actor confirme la modificación de la información base para indicadores de ambiente escolar.
    \UCpaso[\UCactor] Oprime el botón \cdtButton{Aceptar} de la pantalla emergente \cdtIdRef{IUIBAE 2.1}{Registrar información base para indicadores de ambiente escolar: Mensaje de confirmación} confirmando la modificación de la información. \refTray{D}.
    \UCpaso[] Continúa con el paso \ref{cuibae2:VerificarRegistro} de la trayectoria principal.    
 \end{UCtrayectoriaA}
 
   \begin{UCtrayectoriaA}{D}{El actor desea cancelar la modificación de la información base para indicadores de ambiente escolar.}
    \UCpaso[\UCactor] Solicita cancelar la modificación de la información oprimiendo el botón \cdtButton{Cancelar} de la pantalla emergente \cdtIdRef{IUIBAE 2.1}{Administrar información base para indicadores de ambiente escolar: Mensaje de confirmación}.
    \UCpaso[] Continúa con el paso \ref{cuibae2:IngresarDatos} de la trayectoria principal.    
 \end{UCtrayectoriaA}
 
    \begin{UCtrayectoriaA}[Fin del caso de uso]{E}{El actor desea cancelar la operación.}
    \UCpaso[\UCactor] Solicita cancelar la operación oprimiendo el botón \cdtButton{Cancelar} en la pantalla \cdtIdRef{IUIBAE 2}{Registrar información base para indicadores de ambiente escolar}.
    \UCpaso[] Regresa a la pantalla \cdtIdRef{IUIBAE 1}{Administrar información base para indicadores de ambiente escolar}. 
    \end{UCtrayectoriaA}
  
    \begin{UCtrayectoriaA}{F}{El actor no ingresó un dato marcado como requerido.}    
    \UCpaso[\UCsist] Muestra el mensaje \cdtIdRef{MSG5}{Falta un dato requerido para efectuar la operación solicitada} en la pantalla \cdtIdRef{IUIBAE 2}{Registrar información base para indicadores de ambiente escolar} indicando que el registro de información base para indicadores de ambiente escolar no puede realizarse debido a la falta de información requerida.
    \UCpaso[] Continúa con el paso \ref{cuibae2:IngresarDatos} de la trayectoria principal.     
    \end{UCtrayectoriaA}
 
        \begin{UCtrayectoriaA}{G}{El actor ingresó un tipo de dato incorrecto.}    
    \UCpaso[\UCsist] Muestra el mensaje \cdtIdRef{MSG6}{Formato incorrecto} en la pantalla \cdtIdRef{IUIBAE 2}{Registrar información base para indicadores de ambiente escolar} indicando que el registro de información base para indicadores de ambiente escolar no puede realizarse debido a que la información ingresada no es correcta.
    \UCpaso[] Continúa con el paso \ref{cuibae2:IngresarDatos} de la trayectoria principal.     
    \end{UCtrayectoriaA}
    
            \begin{UCtrayectoriaA}{H}{El actor ingresó un dato que excede la longitud máxima.}    
    \UCpaso[\UCsist] Muestra el mensaje \cdtIdRef{MSG7}{Se ha excedido la longitud máxima del campo} en la pantalla \cdtIdRef{IUIBAE 2}{Registrar información base para indicadores de ambiente escolar} indicando que el registro de información base para indicadores de ambiente escolar no puede realizarse debido a que la longitud del campo excede la longitud máxima definida.
    \UCpaso[] Continúa con el paso \ref{cuibae2:IngresarDatos} de la trayectoria principal.     
    \end{UCtrayectoriaA}