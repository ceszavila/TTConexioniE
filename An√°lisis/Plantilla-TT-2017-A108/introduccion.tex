%%%%%   INTRODUCCIÓN   %%%%%
El presente Trabajo tiene por objetivo evaluar las principales problemáticas de comunicación entre los alumnos de nivel superior dentro del Instituto Politécnico Nacional y las diferentes instancias con las que éste interctúa. \\
Se presenta un marco teórico que nos permite entrar en contexto con la solución propuesta para atacar las problemáticas identificadas. Tal solución comprende el análisis y diseño de prototipos como trabajo realizado, en específico de los dos primeros módulos propuestos. Se presentan las dificultades y tecnologías utilizadas para el desarrollo, el avance realizado y el trabajo que realizaremos a futuro para la siguiente parte de Trabajo Terminal.

	
%---------------------------------------------------------
%\section{Objetivo General}
    
%    En la primera etapa del \varSistema{} se desarrollará un sistema de información geográfico (SIG-SEIPA) que permita poner a disposición de la SMAGEM, a través de la Internet, la información sustantiva que sirve para satisfacer las peticiones de información del Gobierno del Estado de México en materia de Medio Ambiente.\\
%    El desarrollo del sistema se realizará con base en el documento {\bf C1-AP Componente 1: Análisis del proceso para la gestión de Áreas Naturales Protegidas y Superficies Forestales del Estado de México}; así como el documento {\bf C3-BD Modelo Entidad - Relación y los scripts de construcción de la base de datos del SIG}.\\
%    Estos documentos tienen como finalidad presentar las decisiones de análisis y diseño referentes al sistema, determinar su comportamiento final y la forma de recibir el software, parcial y final.
	
%	Desarrollar una aplicación móvil enfocada en ayudar a los estudiantes del Instituto Politécnico Nacional y visitantes en general ofreciendo una alternativa de consulta para dar a conocer las diversas actividades, espacios e información pública del personal docente.
	
	 
%	En el presente documento \varCveDocumento\ se muestra a detalle los requerimientos funcionales y no funcionales, modelos de información, reglas de negocio, modelos de comportamiento, casos de uso e interfaces correspondientes a cada uno de los módulos propuestos para esta etapa.



%\section{Objetivos Particulares}	


%=============================================
%
%\begin{UClist}
%	\UCli Identificar las áreas de la Escuela con el fin de mostrar su información en la aplicación.
%	\UCli Ayudar al \cdtRef{actor:CIEAlumno}{Alumno} de nuevo ingreso a ubicar sus salones mediante los grupos asignados en los primeros días de clase.
%	\UCli Ayudar al \cdtRef{actor:CIEAlumno}{Alumno} a ubicar salones, salas o cubículos dentro de la Escuela.
%	\UCli Ofrecer información sobre los cursos, certificaciones y trámites que se pueden realizar en la Escuela.
%	\UCli Ofrecer información básica sobre las unidades de aprendizaje que se imparten en la Escuela.
%	\UCli Ofrecer un espacio para material didáctico de la autoría de los profesores o repositorios gratuitos con la finalidad de tener información sobre materiales de apoyo para las clases.
%\end{UClist}

%El presente documento tiene como objetivo mostrar en detalle los términos, reglas de negocio, casos de uso, interfaces y mensajes correspondientes a cada uno de los módulos propuestos para el sistema, que deberán ser revisados y aprobados por ambas partes antes de iniciar su desarrollo.
%
%	%El presente documento tiene como objetivo mostrar en detalle los términos, reglas de negocio, casos de uso, interfaces y mensajes correspondientes al módulo de Registro de escuelas, que deberán ser revisados y aprobados por ambas partes antes de iniciar su desarrollo.
%	
%	%En este documento se presentan: términos, reglas de negocio, casos de uso, interfaces y mensajes, que deberán ser revisados y aprobados por ambas partes antes de iniciar el desarrollo de dicho sistema informático.
%
%\section{Alcance y acuerdo de conformidad}
%
%   El presente componente \varCveDocumento\ está conformado por los casos de uso de los módulos de  Registro de escuelas, Información base para indicadores, Plan de acción, Seguimiento y acreditación e Indicadores, los cuales se describen en el capítulo \ref{chp:modeloComportamiento}. Dichos casos de uso corresponden a lo definido por parte del programa respecto a la participación de las escuelas. También se incluye el glosario de términos, reglas de negocio, modelo de comportamiento y de interacción que complementan la especificación de dichos casos de uso.\\
%
%    Así mismo, este documento contiene una {\bf hoja resumen} que lista los artefactos de análisis contenidos en este componente. Una vez que el presente componente fue leído, firmado y rubricado, en cada una de sus hojas, por ambas partes es necesario indicar qué artefactos son aprobados en la {\bf hoja resumen}, es importante destacar que los artefactos aprobados pasarán a la etapa de desarrollo. \\
%
%    El IPN y la SMAGEM asumirán los costos en tiempo y alcance del sistema que se deriven de cambios a los artefactos ya aprobados, estos cambios deberán solicitarse por escrito y rubricados por el responsable del proyecto correspondiente y serán evaluados por el IPN para verificar si proceden o están dentro del alcance del proyecto.
%
%%---------------------------------------------------------
%\section{Estructura del documento}
%
%    El contenido del documento se encuentra estructurado de la siguiente forma:
%
%    \begin{Citemize}
%	\item El capítulo 2 presenta el glosario de términos técnicos y de negocio.
%	\item El capítulo 3 muestra el modelo de negocio, el cual consiste en el modelo de información y reglas de negocio.
%	\item El capítulo 4 describe los módulos que conforman el sistema y se detallan los perfiles de los usuario participantes.
%	\item El capítulo 5 describe los casos de uso que conforman el módulo: {\bf Registro de escuelas}.
%	\item El capítulo 6 describe los casos de uso que conforman el módulo: {\bf Información base para indicadores}.
%	\item El capítulo 7 describe los casos de uso que conforman el módulo: {\bf Plan de acción}.
%	\item El capítulo 8 describe los casos de uso que conforman el módulo: {\bf Seguimiento y Acreditación}.
%	\item El capítulo 9 describe los casos de uso que conforman el módulo: {\bf Indicadores}.	
%	\item El capítulo 10 presenta el modelo de interacción con el usuario mediante las interfaces de usuario y los mensajes del sistema.
%	\item El capítulo 11 presenta el modelo de persistencia del \saear.
%    \end{Citemize}
%
%%---------------------------------------------------------
%\section{Notación utilizada en el documento}
%
%    El presente documento utiliza una serie de abreviaciones e identificadores que se utilizan por conveniencia del lenguaje y otros para identificar diferentes elementos referenciados por otros.
%
%    \subsection{Abreviaciones}
%
%	A continuación se presentan las abreviaciones utilizadas a lo largo del presente documento y el significado con el que se utilizan.
%
%	\begin{description}
%	    %\BRterm{nom:anp}{ANP:} Área Natural Protegida.
%	    \BRterm{nom:bpmn}{BPMN:} Notación para el Modelado del Proceso de Negocio, BPMN por sus siglas en inglés.
%	    \BRterm{nom:cdt}{CDT:} Coordinación de Desarrollo Tecnológico.
%	    \BRterm{nom:dgair}{DGAIR:} Dirección General de Acreditación, Incorporación y Revalidación.
%	    \BRterm{nom:sep}{SEP:} Secretaría de Educación Pública.
%	    %\BRterm{nom:conservacion}{Conservación Ecológica:} Coordinación General de Conservación Ecológica.
%	    %\BRterm{nom:eventobpmn}{``Evento'':} Con comillas (``''). Palabra usada para referirse a los \cdtRef{gls:eventobpmn}{``eventos''} en el ambiente de BPMN.
%	    %\BRterm{nom:evento}{Evento:} Palabra usada para referirse a los \cdtRef{gls:evento}{eventos} en el ambiente del negocio de la SMAGEM.
%	    \BRterm{nom:escom}{ESCOM:} Escuela Superior de Cómputo.
%	    %\BRterm{nom:escom}{IGECEM:} Instituto de Información e Investigación Geográfica, Estadística y Catastral del Estado de México. 
%	    \BRterm{nom:ipn}{IPN:} Instituto Politécnico Nacional.
%	    \BRterm{nom:paear}{PAEAR:} Programa de Acreditación de Escuelas Ambientalmente Responsables.
%	    \BRterm{nom:saear}{SAEAR:} Sistema de Acreditación de Escuelas Ambientalmente Responsables.
%	    \BRterm{nom:smagem}{SMAGEM:} Secretaría del Medio Ambiente del Gobierno del Estado de México.
%	\end{description}
%
%    \subsection{Nomenclatura para identificadores}
%
%	Otras abreviaciones y claves que se usan en el presente documento tienen la finalidad de identificar los elementos presentados. Las claves utilizadas son generalmente seguidas de un número. Las X’s que se muestran en la presente lista se usan para indicar que la abreviación es acompañada por un número.
%
%	\begin{description}
%
%	    %\BRterm{nom:cuaX}{CU:} Caso de uso. Se utiliza para nombrar los casos de uso que pertenecen a este subsistema, así cada caso de uso se identifica por las letras ``CU'' más una cadena que hace referencia al subsistema al que pertenece, por ejemplo CUA refiere a un caso de uso de ``Administración'', CUE refiere a uno de ``Eventos''.
%
%		%%%% Casos de uso %%%%
%	    \BRterm{nom:curX}{CUR X:} Casos de uso del módulo ``Registro de escuelas''. Se utiliza para nombrar los casos de uso que pertenecen a este módulo, así cada caso de uso se identifica por las letras ``CUR'' más un espacio en blanco seguido por el número de caso de uso ``X''. Por ejemplo: CUR 1.
%
%	    \BRterm{nom:cuibX}{CUIB X:} Casos de uso del módulo ``Información base para indicadores''. Se utiliza para nombrar los casos de uso que pertenecen a este módulo, cada caso de uso se identifica por las letras ``CUIB'' más un espacio en blanco seguido por el número de caso de uso ``X''. Por ejemplo: CUIB 1. Para especificar los casos de uso referentes únicamente a una línea de acción, en seguida de la letra ``B'' se agrega una o dos letras que identifican la línea de acción a la que corresponden, ``A'' para ``Agua'', ``RS'' para ``Residuos sólidos'', ``E'' para ``Energía'',  ``B'' para ``Biodiversidad'',  ``AE'' para ``Ambiente escolar'' y  ``CR'' para ``Consumo responsable''.
%
%	    \BRterm{nom:cupX}{CUP X:} Caso de uso del módulo ``Plan de acción''. Se utiliza para nombrar los casos de uso que pertenecen a este módulo, así cada caso de uso se identifica por las letras ``CUP'' más un espacio en blanco seguido por el número de caso de uso ``X''. Por ejemplo: CUP 1. Para especificar los casos de uso referentes únicamente a una línea de acción, en seguida de la letra ``P'' se agrega una o dos letras que identifican la línea de acción a la que corresponden,  ``RS'' para ``Residuos sólidos'', ``B'' para ``Biodiversidad'',  ``AE'' para ``Ambiente escolar'', ``CR'' para ``Consumo responsable'' y ``L'' para las líneas de acción de ``Agua'' y ``Energía''.
%
%	    \BRterm{nom:cusX}{CUS X:} Caso de uso del módulo ``Seguimiento y acreditación''. Se utiliza para nombrar los casos de uso que pertenecen a este módulo, así cada caso de uso se identifica por las letras ``CUS'' más un espacio en blanco seguido por el número de caso de uso ``X''. Por ejemplo: CUS 1.
%
%	    \BRterm{nom:cuiX}{CUI X:} Caso de uso del módulo ``Indicadores''. Se utiliza para nombrar los casos de uso que pertenecen a este módulo, así cada caso de uso se identifica por las letras ``CUI'' más un espacio en blanco seguido por el número de caso de uso ``X''. Por ejemplo: CUI 1.
%
%%	    \BRterm{nom:cuaX}{CUA X:} Caso de uso del módulo de ``Administración de usuarios''. Se utiliza para nombrar los casos de uso que pertenecen a este módulo, así cada caso de uso se identifica por las letras ``CUA'' más un espacio en blanco seguido por el número de caso de uso ``X''. Por ejemplo: CUA 1.
%
%		%%%% Pantalla %%%%
%	    \BRterm{nom:iurX}{IUR X:} Interfaz del caso de uso del módulo de ``Registro de escuelas''. Se utiliza para describir las pantallas que acompañan los casos de uso correspondientes al registro de una escuela, se identifica por las letras ``IUR'' más el número de caso de uso ``X''. Por ejemplo: IUR 1.
%
%	    \BRterm{nom:iuibX}{IUIB X:} Interfaz del caso de uso del módulo de ``Información base para indicadores''. Se utiliza para describir las pantallas que acompañan los casos de uso correspondientes a la información base de cada escuela, se identifica por las letras ``IUIB'' más el número de caso de uso ``X''. Por ejemplo: IUP 1. Para especificar las interfaces referentes únicamente a una línea de acción, en seguida de la letra ``B'' se agrega una o dos letras que identifican la línea de acción a la que corresponden, ``A'' para ``Agua'', ``RS'' para ``Residuos sólidos'', ``E'' para ``Energía'',  ``B'' para ``Biodiversidad'',  ``AE'' para ``Ambiente escolar'' y  ``CR'' para ``Consumo responsable''.
%	    
%	    \BRterm{nom:iupX}{IUP X:} Interfaz del caso de uso del módulo de ``Plan de acción''. Se utiliza para describir las pantallas que acompañan el caso de uso correspondiente al plan de acción de cada escuela, se identifica por las letras ``IUP'' más el número de caso de uso ``X''. Por ejemplo: IUP 1. Para especificar las interfaces referentes únicamente a una línea de acción, en seguida de la letra ``P'' se agrega una o dos letras que identifican la línea de acción a la que corresponden,  ``RS'' para ``Residuos sólidos'', ``B'' para ``Biodiversidad'',  ``AE'' para ``Ambiente escolar'', ``CR'' para ``Consumo responsable'' y ``L'' para las líneas de acción de ``Agua'' y ``Energía''.
%
%	    \BRterm{nom:iusX}{IUS X:} Interfaz del caso de uso del módulo de ``Seguimiento y acreditación''. Se utiliza para describir las pantallas que acompañan el caso de uso correspondiente al Seguimiento y acreditación ambiental de cada escuela, se identifica por las letras ``IUS'' más el número de caso de uso ``X''. Por ejemplo: IUS 1.
%
%	    \BRterm{nom:iuiX}{IUI X:} Interfaz del caso de uso del módulo de ``Indicadores''. Se utiliza para describir las pantallas que acompañan el caso de uso correspondiente a los indicadores ambientales de cada escuela, se identifica por las letras ``IUI'' más el número de caso de uso ``X''. Por ejemplo: IUI 1.
%
%%	    \BRterm{nom:iuaX}{IUA X:} Interfaz del caso de uso del módulo ``Administración de usuarios''. Se utiliza para describir las pantallas que acompañan el caso de uso correspondiente a la Administración de usuarios, se identifica por las letras ``IUA'' más el número de caso de uso ``X''. Por ejemplo: IUA 1.
%
%		%%%% MENSAJES %%%%
%	    \BRterm{nom:msgx}{MSGX:} Mensaje. Se utiliza para nombrar los mensajes utilizados en el sistema para informar errores o notificar operaciones, identificados por las letras ``MSG'' seguido por el número de mensaje ``X''. Por ejemplo: MSG1.
%
%		%%%% REGLAS DE NEGOCIO %%%%
%	    \BRterm{nom:rnNx}{RN-NX:} Regla de Negocio. Se utiliza para nombrar a las reglas de negocio identificadas, así cada regla de negocio se identifica con las letras ``RN-N'' seguida del número de regla de negocio ``X''. Por ejemplo: RN-N1.
%
%	    \BRterm{nom:rnSx}{RN-SX:} Regla de Negocio del sistema. Se utiliza para nombrar a las reglas de negocio identificadas, así cada regla de negocio se identifica con las letras ``RN-S'' seguida del número de regla de negocio ``X''. Por ejemplo: RN-S1.	

%	\end{description}
