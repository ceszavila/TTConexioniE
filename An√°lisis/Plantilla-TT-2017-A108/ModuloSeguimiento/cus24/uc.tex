%!TEX encoding = UTF-8 Unicode

\begin{UseCase}{CUS 24}{Registrar avance de meta de energía}
    {
	Las metas permiten conocer el resultado que se espera obtener al realizar el conjunto de acciones definidas para esta. Este caso de uso permite visualizar la información de la meta de energía y especificar el avance cuantificable que esta tiene.
    }
    
    \UCitem{Versión}{1.0}
    \UCccsection{Administración de Requerimientos}
    \UCitem{Autor}{Jessica Stephany Becerril Delgado}
    \UCccitem{Evaluador}{}
    \UCitem{Operación}{Registrar}
    \UCccitem{Prioridad}{Media}
    \UCccitem{Complejidad}{Media}
    \UCccitem{Volatilidad}{Media}
    \UCccitem{Madurez}{Media}
    \UCitem{Estatus}{Terminado}
    \UCitem{Fecha del último estatus}{8 de Diciembre del 2014}

    
%% Esta sección la debe llenar solo el Revisor
% %--------------------------------------------------------
 	\UCccsection{Revisión Versión 0.1} % Anote la versión que se revisó.
% 	% FECHA: Anote la fecha en que se terminó la revisión.
 	\UCccitem{Fecha}{9-Dic} 
% 	% EVALUADOR: Coloque el nombre completo de quien realizó la revisión.
 	\UCccitem{Evaluador}{Nayeli Vega}
% 	% RESULTADO: Coloque la palabra que mas se apegue al tipo de acción que el analista debe realizar.
 	\UCccitem{Resultado}{Corregir}
% 	% OBSERVACIONES: Liste los cambios que debe realizar el Analista.
 	\UCccitem{Observaciones}{
 		\begin{UClist}
% 			% PC: Petición de Cambio, describa el trabajo a realizar, si es posible indique la causa de la PC. Opcionalmente especifique la fecha en que considera razonable que se deba terminar la PC. No olvide que la numeración no se debe reiniciar en una segunda o tercera revisión.
 			\RCitem{PC1}{\TOCHK{Agregar precondiciones, mensajes, pasos a la trayectoria y trayectorias alternativas para verificación de estado y de fecha}}{Fecha de entrega}
 			\RCitem{PC3}{\TOCHK{Entradas: La etiqueta de Avance del consumo de energía no creo que deba ser colocada así, porque no hace referencia unicamente al consumo de energía, hace eferencia al valor asociado a la meta que puede ser consumo de energía u otra cosa }}{Fecha de entrega} 			
 			\RCitem{PC2}{\TOCHK{Verificar la sección de salidas, no todas son salidas}}{Fecha de entrega}
 			\RCitem{PC2}{\TOCHK{Revisar trayectoria principal, no se calcula el avance acumulado, lo mismo que para el registro de avance de meta de agua}}{Fecha de entrega} 			

 		\end{UClist}		
 	}
% %--------------------------------------------------------

    \UCsection{Atributos}
    \UCitem{Actor}{\cdtRef{actor:usuarioEscuela}{Coordinador del programa}}
    \UCitem{Propósito}{Registrar los avances de la meta de energía.}
    \UCitem{Entradas}{
	\begin{UClist}
	    \UCli  Avance del valor asociado a la meta:
	    \begin{itemize}
	     \item \cdtRef{gls:avanceAccion}{Avance de actividades}: \ioEscribir.
	     \item \cdtRef{gls:avanceAccion}{Avance de alumnos}: \ioEscribir.
	     \item \cdtRef{gls:avanceAccion}{Avance de personal docente y administrativo}: \ioEscribir.
	     \item \cdtRef{gls:avanceAccion}{Avance de personal directivo}: \ioEscribir.
	     \item \cdtRef{gls:avanceAccion}{Avance de padres de familia}: \ioEscribir.	     
	     \item \cdtRef{gls:avanceAccion}{Avance del consumo de energía}: \ioEscribir.
	    \end{itemize}
	    
	    \UCli Evidencias:
	    \begin{itemize}
	     \item Archivo: \ioSeleccionar.
	    \end{itemize}


	\end{UClist}
    }
    \UCitem{Salidas}{
	\begin{UClist} 
	    \UCli Información de la meta:
	    \begin{itemize}
		\item Periodo: \ioObtener.
		\item \cdtRef{meta:problematica}{Problemática}: \ioObtener.
		\item \cdtRef{meta:meta}{Meta}: \ioObtener.
		\item \cdtRef{meta:fechaInicio}{Fecha de inicio de ejecución}: \ioObtener.
		\item \cdtRef{meta:fechaTermino}{Fecha de término de ejecución}: \ioObtener.	    
	    \end{itemize}
	    
	    \UCli Avance de meta
	    \begin{itemize}
	     \item Avance de meta: \ioCalcular \cdtIdRef{RN-N17}{Calcular avance de la meta}.
	    \end{itemize}

	    \UCli Avance del valor asociado a la meta:
	    \begin{itemize}
	     \item \cdtRef{avanceCapacitacion:avanceActividades}{Avance acumulado de actividades}: \ioObtener.
	     \item \cdtRef{capacitacionSensibilizacion:numActividades}{Número de actividades a alcanzar}: \ioObtener.
	     
	     \item \cdtRef{avanceAgua:avance}{Avance acumulado del valor asociado a la meta}: \ioObtener.
	     \item \cdtRef{meta:valorAlcanzar}{Valor del valor asociado a la meta a alcanzar}: \ioObtener.
	     
	     \item \cdtRef{avanceCapacitacion:avanceAlumnos}{Avance acumulado de alumnos}: \ioObtener.
	     \item \cdtRef{capacitacionSensibilizacion:numAlumnos}{Número de alumnos a alcanzar}: \ioObtener.
	     
	     \item \cdtRef{avanceCapacitacion:avanceDocenteAdministrativo}{Avance acumulado de personal docente y administrativo}: \ioObtener.
	     \item \cdtRef{capacitacionSensibilizacion:numDocenteAdministrativo}{Número de personal docente y administrativo a alcanzar}: \ioObtener.
	     
	     \item \cdtRef{avanceCapacitacion:avanceDirectivos}{Avance acumulado de personal directivo}: \ioObtener.
	     \item \cdtRef{capacitacionSensibilizacion:numDirectivos}{Número de personal directivo a alcanzar}: \ioObtener.
	     
	     \item \cdtRef{avanceCapacitacion:avancePadres}{Avance acumulado de padres de familia}: \ioObtener.
	     \item \cdtRef{capacitacionSensibilizacion:numPadres}{Número de padres de familia a alcanzar}: \ioObtener.
	    \end{itemize}

	    
	    \UCli \cdtIdRef{MSG1}{Operación realizada exitosamente:} Se muestra en la pantalla \cdtIdRef{IUS 22}{Administrar avances de energía} cuando el registro del avance para la acción se ha realizado correctamente.
	\end{UClist}
    }
    
    \UCitem{Precondiciones}{
	\begin{UClist}
	    \UCli {\bf Interna:} Que la escuela se encuentre en estado \cdtRef{estado:avanceEdicion}{Avance en edición}.
	    \UCli {\bf Interna:} Que el periodo de registros de avances se encuentre vigente.
	\end{UClist}
    }
    
    \UCitem{Postcondiciones}{
	\begin{UClist}
	    \UCli {\bf Interna:} La meta tendrá avances registrados.
	\end{UClist}
    }

    \UCitem{Reglas de negocio}{
    	\begin{UClist}
	    \UCli \cdtIdRef{RN-S1}{Información correcta}: Verifica que la información introducida sea correcta.
    	    \UCli \cdtIdRef{RN-N16}{Calcular avance acumulado}: Calcula el avance acumulado para cada uno de los valores asociados a la meta de energía.
    	    \UCli \cdtIdRef{RN-N17}{Calcular avance de la meta}: Calcula el avance de la meta.
	\end{UClist}
    }

    \UCitem{Errores}{
	\begin{UClist}
	    
	    \UCli \cdtIdRef{MSG6}{Formato incorrecto}: Se muestra en la pantalla \cdtIdRef{IUS 24}{Registrar avance de meta de energía} cuando el tipo de dato ingresado no cumple con el tipo de dato solicitado en el campo.
	    
	    \UCli \cdtIdRef{MSG7}{Se ha excedido la longitud máxima del campo}: Se muestra en la pantalla \cdtIdRef{IUS 24}{Registrar avance de meta de energía} cuando se ha excedido la longitud de alguno de los campos.	
	    
	    \UCli  \cdtIdRef{MSG28}{Operación no permitida por estado de la entidad}: Se muestra en la pantalla \cdtIdRef{IUS 22}{Administrar avances de energía} indicando al actor que no se puede realizar la operación debido al estado en que se encuentra la escuela.
	    
	    \UCli \cdtIdRef{MSG41}{Acción fuera del periodo}: Se muestra en la pantalla \cdtIdRef{IUS 22}{Administrar avances de energía} para indicarle al actor que no puede realizar la operación debido a que la fecha actual se encuentra fuera del periodo definido por la SMAGEM para realizarla.
	\end{UClist}
    }

    
    \UCitem{Tipo}{Secundario, extiende del caso de uso \cdtIdRef{CUS 22}{Administrar avances de energía}.}

%    \UCitem{Fuente}{
%	\begin{UClist}
%	    \UCli Minuta de la reunión \cdtIdRef{M-17RT}{Reunión de trabajo}.
%	\end{UClist}
%    }

\end{UseCase}


 \begin{UCtrayectoria}
    \UCpaso[\UCactor] Solicita registrar un avance en una meta de energía oprimiendo el botón \botMetas del registro correspondiente en la pantalla \cdtIdRef{IUS 22}{Administrar avances de energía}.
    \UCpaso[\UCsist] Verifica que la escuela se encuentre en estado ``Avance en edición''. \refTray{A}.
    \UCpaso[\UCsist] Verifica que la fecha actual se encuentre dentro del periodo definido por la SMAGEM para realizar la operación. \refTray{B}.
    \UCpaso[\UCsist] Calcula el avance de la meta de energía con base en la regla de negocio \cdtIdRef{RN-N17}{Calcular avance de la meta}.
    \UCpaso[\UCsist] Muestra la pantalla \cdtIdRef{IUS 24}{Registrar avance de meta de energía} con la información referente a la meta de energía. 
    \UCpaso[\UCactor] Ingresa los datos requeridos en la pantalla \cdtIdRef{IUS 24}{Registrar avance de meta de energía}. \label{cus24:RegAvanceEnergia}
    \UCpaso[\UCactor] Oprime el botón \cdtButton{Aceptar} en la pantalla \cdtIdRef{IUS 24}{Registrar avance de meta de energía}. \refTray{C}. 
    \UCpaso[\UCsist] Verifica que la escuela se encuentre en estado ``Avance en edición''. \refTray{A}.
    \UCpaso[\UCsist] Verifica que la fecha actual se encuentre dentro del periodo definido por la SMAGEM para realizar la operación. \refTray{B}.
    \UCpaso[\UCsist] Verifica que los datos ingresados proporcionen información correcta con base en la regla de negocio \cdtIdRef{RN-S1}{Información correcta}.  \refTray{D}. \refTray{E}.
    \UCpaso[\UCsist] Calcula el avance acumulado para los valores asociados a la meta de energía con base en la regla de negocio \cdtIdRef{RN-N16}{Calcular avance acumulado}.
    \UCpaso[\UCsist] Registra los avances en la meta de energía.
    \UCpaso[\UCsist] Muestra el mensaje \cdtIdRef{MSG1}{Operación realizada exitosamente} en la pantalla \cdtIdRef{IUS 22}{Administrar avances de energía} indicando al actor que el avance se ha registrado exitosamente.
 \end{UCtrayectoria}
 
        \begin{UCtrayectoriaA}[Fin del caso de uso]{A}{La escuela no se encuentra en un estado que permita realizar la operación.}
	\UCpaso[\UCsist] Muestra el mensaje \cdtIdRef{MSG28}{Operación no permitida por estado de la entidad} en la pantalla \cdtIdRef{IUS 22}{Administrar avances de energía} indicando al actor que no puede realizar la operación debido a que la escuela no se encuentra en estado ``Avance en edición''. 
    \end{UCtrayectoriaA}
 
    \begin{UCtrayectoriaA}[Fin del caso de uso]{B}{La fecha actual se encuentra fuera del periodo definido por la SMAGEM para realizar la operación.}
      \UCpaso[\UCsist] Muestra el mensaje \cdtIdRef{MSG41}{Acción fuera del periodo} en la pantalla \cdtIdRef{IUS 22}{Administrar avances de energía} indicando al actor que no puede realizar la operación debido a que la fecha actual se encuentra fuera del periodo definido por la SMAGEM para realizarla. 
  \end{UCtrayectoriaA}
 
    \begin{UCtrayectoriaA}[Fin del caso de uso]{C}{El actor desea cancelar la operación.}
      \UCpaso[\UCactor] Solicita cancelar la operación oprimiendo el botón \cdtButton{Cancelar} en la pantalla \cdtIdRef{IUS 24}{Registrar avance de meta de energía}.
      \UCpaso[] Regresa a la pantalla \cdtIdRef{IUS 22}{Administrar avances de energía}. 
    \end{UCtrayectoriaA}
 
    \begin{UCtrayectoriaA}{D}{El actor ingresó un tipo de dato incorrecto.}    
	\UCpaso[\UCsist] Muestra el mensaje \cdtIdRef{MSG6}{Formato incorrecto} en la pantalla \cdtIdRef{IUS 24}{Registrar avance de meta de energía} indicando que el registro del avance para la meta no puede realizarse debido a que la información ingresada no es correcta.
	\UCpaso[] Continúa con el paso \ref{cus24:RegAvanceEnergia} de la trayectoria principal.     
    \end{UCtrayectoriaA}
    
    \begin{UCtrayectoriaA}{E}{El actor ingresó un dato que excede la longitud máxima.}    
	\UCpaso[\UCsist] Muestra el mensaje \cdtIdRef{MSG7}{Se ha excedido la longitud máxima del campo} en la pantalla \cdtIdRef{IUS 24}{Registrar avance de meta de energía} indicando que el registro del avance para la meta de energía no puede realizarse debido a que la longitud del campo excede la longitud máxima definida.
	\UCpaso[] Continúa con el paso \ref{cus24:RegAvanceEnergia} de la trayectoria principal.     
    \end{UCtrayectoriaA}
 
     