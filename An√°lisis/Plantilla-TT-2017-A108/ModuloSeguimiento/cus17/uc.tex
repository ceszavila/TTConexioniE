%!TEX encoding = UTF-8 Unicode

\begin{UseCase}{CUS 17}{Eliminar flora}
    {
    Después de registrar una especie vegetal el actor puede haber detectado algún error en la información ingresada o puede ocurrir que la especie vegetal ya no sea requerida, en ese caso el actor podrá eliminarla a través de este caso de uso.
    }
    
    \UCitem{Versión}{1.0}
    \UCccsection{Administración de Requerimientos}
    \UCitem{Autor}{Francisco Javier Ponce Cruz}
    \UCccitem{Evaluador}{}
    \UCitem{Operación}{Eliminar}
    \UCccitem{Prioridad}{Media}
    \UCccitem{Complejidad}{Media}
    \UCccitem{Volatilidad}{Alta}
    \UCccitem{Madurez}{Media}
    \UCitem{Estatus}{Terminado}
    \UCitem{Fecha del último estatus}{9 diciembre 2014}
    
%% Copie y pegue este bloque tantas veces como revisiones tenga el caso de uso.
%% Esta sección la debe llenar solo el Revisor
% %--------------------------------------------------------
%   \UCccsection{Revisión Versión XX} % Anote la versión que se revisó.
%   % FECHA: Anote la fecha en que se terminó la revisión.
%   \UCccitem{Fecha}{Fecha en que se termino la revisión} 
%   % EVALUADOR: Coloque el nombre completo de quien realizó la revisión.
%   \UCccitem{Evaluador}{Nombre de quien revisó}
%   % RESULTADO: Coloque la palabra que mas se apegue al tipo de acción que el analista debe realizar.
%   \UCccitem{Resultado}{Corregir, Desechar, Rehacer todo, terminar.}
%   % OBSERVACIONES: Liste los cambios que debe realizar el Analista.
%   \UCccitem{Observaciones}{
%       \begin{UClist}
%           % PC: Petición de Cambio, describa el trabajo a realizar, si es posible indique la causa de la PC. Opcionalmente especifique la fecha en que considera razonable que se deba terminar la PC. No olvide que la numeración no se debe reiniciar en una segunda o tercera revisión.
%           \RCitem{PC1}{\TODO{Descripción del pendiente}}{Fecha de entrega}
%           \RCitem{PC2}{\TODO{Descripción del pendiente}}{Fecha de entrega}
%           \RCitem{PC3}{\TODO{Descripción del pendiente}}{Fecha de entrega}
%       \end{UClist}        
%   }
% %--------------------------------------------------------

    \UCsection{Atributos}
    \UCitem{Actor}{\cdtRef{actor:usuarioEscuela}{Coordinador del programa}}
    \UCitem{Propósito}{Eliminar una especie vegetal que fue registrada por error, que después de una revisión se determinó que es innecesaria, o que ya no existe durante el seguimiento del plan de acción.}
    \UCitem{Entradas}{
    \begin{UClist}
       \UCli Ninguna.
    \end{UClist}
    }
    \UCitem{Salidas}{
    \begin{UClist}
        \UCli \cdtIdRef{MSG31}{Confirmar eliminación:} Se muestra en una pantalla emergente para que el actor confirme la eliminación de la especie vegetal. 
    \end{UClist}
    }

    \UCitem{Precondiciones}{
    \begin{UClist}
        \UCli {\bf Interna:} Que la escuela se encuentre en estado \cdtRef{estado:avanceEdicion}{Avance en edición}.
        \UCli {\bf Interna:} Que el periodo de registros de avances se encuentre vigente.
        \end{UClist}
    }
    
    \UCitem{Postcondiciones}{
    \begin{UClist}
        \UCli {\bf Interna:} El registro de la especie vegetal desaparecerá del sistema.
    \end{UClist}
    }
    
    \UCitem{Reglas de negocio}{
        \begin{UClist}
        \UCli Ninguna.
    \end{UClist}
    }
    
    \UCitem{Errores}{
    \begin{UClist}
            
        \UCli \cdtIdRef{MSG28}{Operación no permitida por estado de la entidad}: Se muestra en la pantalla \cdtIdRef{IUS 15}{Administrar inventario de flora} indicando al actor que no se puede eliminar la especie vegetal debido al estado en que se encuentra la escuela.
        
         \UCli \cdtIdRef{MSG41}{Acción fuera del periodo}: Se muestra sobre la pantalla en la pantalla \cdtIdRef{IUS 15}{Administrar inventario de flora} para indicarle al actor que no puede eliminar la especie vegetal debido a que la fecha actual se encuentra fuera del periodo definido por la SMAGEM para realizar la acción.
        
    \end{UClist}
    }

    \UCitem{Tipo}{Secundario, extiende del caso de uso \cdtIdRef{CUS 15}{Administrar inventario de flora}.}

%    \UCitem{Fuente}{
%   \begin{UClist}
%       \UCli Minuta de la reunión \cdtIdRef{M-17RT}{Reunión de trabajo}.
%   \end{UClist}
 %   }
\end{UseCase}

 \begin{UCtrayectoria}
    \UCpaso[\UCactor] Solicita eliminar una especie vegetal oprimiendo el botón \botKo del registro correspondiente en la pantalla \cdtIdRef{IUS 15}{Administrar inventario de flora}.
    \UCpaso[\UCsist] Verifica que la escuela se encuentre en estado ``Avance en edición''. \refTray{A}.
    \UCpaso[\UCsist] Verifica que la fecha actual se encuentre dentro del periodo definido por la SMAGEM para eliminar la especie vegetal. \refTray{B}.
    \UCpaso[\UCsist] Muestra el mensaje \cdtIdRef{MSG31}{Confirmar eliminación} en una pantalla emergente para que el actor confirme la eliminación de la especie vegetal.
    \UCpaso[\UCactor] Oprime el botón \cdtButton{Aceptar} de la pantalla emergente. \refTray{C}.
    \UCpaso[\UCsist] Verifica que la escuela se encuentre en estado ``Avance en edición''. \refTray{A}.
    \UCpaso[\UCsist] Verifica que la fecha actual se encuentre dentro del periodo definido por la SMAGEM para eliminar la especie vegetal. \refTray{B}.
    \UCpaso[\UCsist] Elimina la información de la especie vegetal del sistema.
    \UCpaso[\UCsist] Muestra el mensaje \cdtIdRef{MSG1}{Operación realizada exitosamente} en la pantalla \cdtIdRef{IUS 15}{Administrar inventario de flora} para indicar al actor que el registro de la flora se ha eliminado exitosamente. 
    
 \end{UCtrayectoria}

    \begin{UCtrayectoriaA}[Fin del caso de uso]{A}{La escuela no se encuentra en un estado que permita eliminar la especie vegetal.}
    \UCpaso[\UCsist] Muestra el mensaje \cdtIdRef{MSG28}{Operación no permitida por estado de la entidad} en la pantalla \cdtIdRef{IUS 15}{Administrar inventario de flora} indicando al actor que no puede eliminar la flora debido a que la escuela no se encuentra en estado ``Avance en edición''. 
 \end{UCtrayectoriaA}

   \begin{UCtrayectoriaA}[Fin del caso de uso]{B}{La fecha actual se encuentra fuera del periodo definido por la SMAGEM para la eliminación de una especie vegetal.}
    \UCpaso[\UCsist] Muestra el mensaje \cdtIdRef{MSG41}{Acción fuera del periodo} en la pantalla \cdtIdRef{IUS 15}{Administrar inventario de flora} indicando al actor que no puede eliminar la especie vegetal debido a que la fecha actual se encuentra fuera del periodo definido por la SMAGEM para realizar la acción. 
 \end{UCtrayectoriaA}
 
    \begin{UCtrayectoriaA}[Fin del caso de uso]{C}{El actor desea cancelar la eliminación del registro.}
    \UCpaso[\UCactor] Solicita cancelar eliminación del registro de la especie vegetal oprimiendo el botón \cdtButton{Cancelar} de la pantalla emergente.
    \UCpaso[] Regresa a la pantalla \cdtIdRef{IUS 15}{Administrar inventario de flora}.    
 \end{UCtrayectoriaA}
