\subsection{IUS 28 Registrar avance de meta de residuos sólidos}

\subsubsection{Objetivo}

    Esta pantalla permite al \cdtRef{actor:usuarioEscuela}{Coordinador del programa} visualizar la información de la meta de residuos sólidos y especificar el avance de los valores asociados a esta
    
\subsubsection{Diseño}

        En la figura~\ref{IUS 28} se muestra la pantalla ``Registrar avance de meta de residuos sólidos'', por medio de la cual se podrá consultar la información de la meta de residuos sólidos. El actor tendrá la facultad de registrar avances para los valores asociados a esta.\\
        
        En el caso de que se trate de una meta cuantificable se mostrará únicamente el campo para el registro del avance en el valor asociado a la meta, si se trata de una meta no cuantificable se mostrarán campos para que el actor registre el avance referente al número de participantes como se muestra en la figura~\ref{IUS 28.1}.\\
        
        Una vez que se haya ingresado toda la información solicitada para el registro de los avances deberá oprimir el botón \cdtButton{Aceptar}, el sistema validará y registrará la información sólo si se han cumplido todas las reglas de negocio establecidas.\\
    
      Finalmente se mostrará el mensaje \cdtIdRef{MSG1}{Operación realizada exitosamente} en la pantalla \cdtIdRef{IUS 26}{Administrar avances de residuos sólidos}, para indicar que la información de los avances para la meta se ha registrado exitosamente.
        
      \IUfig[.9]{pantallas/seguimiento/cus28/IUS28RegistrarMetaResiduos.png}{IUS 28}{Registrar avance de meta de residuos sólidos}
      \IUfig[.9]{pantallas/seguimiento/cus28/IUS28RegistrarMetaResiduos1.png}{IUS 28.1}{Registrar avance de meta de residuos sólidos: Meta cuantificable}

\subsubsection{Comandos}
    \begin{itemize}	
	\item \cdtButton{Aceptar}: Permite confirmar el registro de los avances para la meta de residuos sólidos, dirige a la pantalla \cdtIdRef{IUS 26}{Administrar avances de residuos sólidos}.
	\item \cdtButton{Cancelar}: Permite cancelar el registro de los avances para la meta de residuos sólidos, dirige a la pantalla \cdtIdRef{IUS 26}{Administrar avances de residuos sólidos}.
    \end{itemize}

\subsubsection{Mensajes}

    \begin{description}
    
	    \item [\cdtIdRef{MSG1}{Operación realizada exitosamente}:] Se muestra en la pantalla \cdtIdRef{IUS 26}{Administrar avances de residuos sólidos} cuando el registro del avance para la acción se ha realizado correctamente.
	    
	    \item [\cdtIdRef{MSG6}{Formato incorrecto}:] Se muestra en la pantalla \cdtIdRef{IUS 28}{Registrar avance de meta de residuos sólidos} cuando el tipo de dato ingresado no cumple con el tipo de dato solicitado en el campo.
	    
	    \item [\cdtIdRef{MSG7}{Se ha excedido la longitud máxima del campo}:] Se muestra en la pantalla \cdtIdRef{IUS 28}{Registrar avance de meta de residuos sólidos} cuando se ha excedido la longitud de alguno de los campos.	
	    
	    \item [\cdtIdRef{MSG28}{Operación no permitida por estado de la entidad}:] Se muestra en la pantalla \cdtIdRef{IUS 26}{Administrar avances de residuos sólidos} indicando al actor que no se puede realizar la operación debido al estado en que se encuentra la escuela.
	    
	    \item [\cdtIdRef{MSG41}{Acción fuera del periodo}:] Se muestra en la pantalla \cdtIdRef{IUS 26}{Administrar avances de residuos sólidos} para indicarle al actor que no puede realizar la operación debido a que la fecha actual se encuentra fuera del periodo definido por la SMAGEM para realizarla.
    \end{description}
