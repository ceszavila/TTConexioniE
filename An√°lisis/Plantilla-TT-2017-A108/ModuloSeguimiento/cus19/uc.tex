%!TEX encoding = UTF-8 Unicode

\begin{UseCase}{CUS 19}{Registrar avance de acciones de agua}
    {
	Las acciones son actividades que se requieren completar para alcanzar una meta. Este caso de uso permite visualizar las acciones registradas para una meta de agua y especificar si estas han finalizado o, si se encuentran en proceso, registrar el avance cuantificable de ellas.
    }
    
    \UCitem{Versión}{1.0}
    \UCccsection{Administración de Requerimientos}
    \UCitem{Autor}{Jessica Stephany Becerril Delgado}
    \UCccitem{Evaluador}{}
    \UCitem{Operación}{Registrar}
    \UCccitem{Prioridad}{Media}
    \UCccitem{Complejidad}{Media}
    \UCccitem{Volatilidad}{Media}
    \UCccitem{Madurez}{Media}
    \UCitem{Estatus}{Terminado}
    \UCitem{Fecha del último estatus}{4 de Diciembre del 2014}
    
%% Copie y pegue este bloque tantas veces como revisiones tenga el caso de uso.
%% Esta sección la debe llenar solo el Revisor
% %--------------------------------------------------------
 %	\UCccsection{Revisión Versión 0.1} % Anote la versión que se revisó.
% 	% FECHA: Anote la fecha en que se terminó la revisión.
 %	\UCccitem{Fecha}{9-Dic} 
% 	% EVALUADOR: Coloque el nombre completo de quien realizó la revisión.
 %	\UCccitem{Evaluador}{Nayeli Vega}
% 	% RESULTADO: Coloque la palabra que mas se apegue al tipo de acción que el analista debe realizar.
 %	\UCccitem{Resultado}{Corregir}
% 	% OBSERVACIONES: Liste los cambios que debe realizar el Analista.
 %	\UCccitem{Observaciones}{
 %		\begin{UClist}
% 			% PC: Petición de Cambio, describa el trabajo a realizar, si es posible indique la causa de la PC. Opcionalmente especifique la fecha en que considera razonable que se deba terminar la PC. No olvide que la numeración no se debe reiniciar en una segunda o tercera revisión.
 %			\RCitem{PC0}{\TOCHK{Agregar precondiciones, mensajes y rutas alternativas para las dos condiciones}}{Fecha de entrega}
 %			\RCitem{PC1}{\TOCH{Revisar redacción de Resumen: ... si se encuenrtan en proceso, así como registrar el avance cuantificable de ellas.}}{Fecha de entrega}
 %			\RCitem{PC2}{\TOCHK{Porpósito: revisar redacción, acciones de una meta de agua, así como está da a entender que sólo hay una meta y pueden ser varias}}{Fecha de entrega}
% 			\RCitem{PC3}{\TODO{Descripción del pendiente}}{Fecha de entrega}
 %		\end{UClist}		
 %	}
% %--------------------------------------------------------

    \UCsection{Atributos}
    \UCitem{Actor}{\cdtRef{actor:usuarioEscuela}{Coordinador del programa}}
    \UCitem{Propósito}{Registrar los avances de las acciones de una meta de agua.}
    \UCitem{Entradas}{
	\begin{UClist}
	    \UCli  \cdtRef{gls:avanceAccion}{Avance}: \ioEscribir.
	    \UCli \cdtRef{accion:finalizada}{Acción finalizada}: \ioRadioBoton.
	\end{UClist}
    }
    \UCitem{Salidas}{
	\begin{UClist} 
	    \UCli \cdtRef{accion:descripcion}{Acción}: \ioObtener.
	    \UCli \cdtRef{accion:recursos}{Recursos materiales y financieros}: \ioObtener.
	    \UCli \cdtRef{accion:cuantificable}{Cuantificable}: \ioObtener.
	    \UCli \cdtRef{accion:avanceAcumulado}{Avance acumulado}: \ioObtener.
	    \UCli \cdtRef{accion:valorAlcanzar}{Valor a alcanzar}: \ioObtener.
	    \UCli \cdtRef{accion:unidad}{Unidad}: \ioObtener.
	    
	    \UCli \cdtIdRef{MSG1}{Operación realizada exitosamente:} Se muestra en la pantalla \cdtIdRef{IUS 18}{Administrar avances de agua} cuando el registro del avance para las acciones se ha realizado correctamente.
	\end{UClist}
    }
    
    
    \UCitem{Precondiciones}{
	\begin{UClist}
	    \UCli {\bf Interna:} Que la escuela se encuentre en estado \cdtRef{estado:avanceEdicion}{Avance en edición}.
	    \UCli {\bf Interna:} Que el periodo de registros de avances se encuentre vigente.
	\end{UClist}
    }
    
    \UCitem{Postcondiciones}{
	\begin{UClist}
	    \UCli {\bf Interna:} La acción tendrá un avance registrado.
	\end{UClist}
    }

    \UCitem{Reglas de negocio}{
    	\begin{UClist}
	    \UCli \cdtIdRef{RN-S1}{Información correcta}: Verifica que la información introducida sea correcta.
	    \UCli \cdtIdRef{RN-N16}{Calcular avance acumulado}: Calcula el avance acumulado para cada uno de los valores asociados a la meta de agua.
	\end{UClist}
    }

    \UCitem{Errores}{
	\begin{UClist}
	    
	    \UCli \cdtIdRef{MSG6}{Formato incorrecto}: Se muestra en la pantalla \cdtIdRef{IUS 19}{Registrar avance de acciones de agua} cuando el tipo de dato ingresado no cumple con el tipo de dato solicitado en el campo.
	    
	    \UCli \cdtIdRef{MSG7}{Se ha excedido la longitud máxima del campo}: Se muestra en la pantalla \cdtIdRef{IUS 19}{Registrar avance de acciones de agua} cuando se ha excedido la longitud de alguno de los campos.	
	    
	    \UCli  \cdtIdRef{MSG28}{Operación no permitida por estado de la entidad}: Se muestra en la pantalla \cdtIdRef{IUS 18}{Administrar avances de agua} indicando al actor que no se puede realizar la operación debido al estado en que se encuentra la escuela.
	    
	    \UCli \cdtIdRef{MSG41}{Acción fuera del periodo}: Se muestra en la pantalla \cdtIdRef{IUS 18}{Administrar avances de agua} para indicarle al actor que no puede realizar la operación debido a que la fecha actual se encuentra fuera del periodo definido por la SMAGEM para realizarla.
	\end{UClist}
    }

    
    \UCitem{Tipo}{Secundario, extiende del caso de uso \cdtIdRef{CUS 18}{Administrar avances de agua}.}

%    \UCitem{Fuente}{
%	\begin{UClist}
%	    \UCli Minuta de la reunión \cdtIdRef{M-17RT}{Reunión de trabajo}.
%	\end{UClist}
%    }

\end{UseCase}


 \begin{UCtrayectoria}
    \UCpaso[\UCactor] Solicita registrar un avance en las acciones registradas para una meta de agua oprimiendo el botón \botAutoAjus del registro correspondiente en la pantalla \cdtIdRef{IUS 18}{Administrar avances de agua}.
    \UCpaso[\UCsist] Verifica que la escuela se encuentre en estado ``Avance en edición''. \refTray{A}.
    \UCpaso[\UCsist] Verifica que la fecha actual se encuentre dentro del periodo definido por la SMAGEM para realizar la operación. \refTray{B}.
    %\UCpaso[\UCsist] Calcula el avance acumulado para cada una de las acciones de la meta de agua con base en la regla de negocio \cdtIdRef{RN-N16}{Calcular avance acumulado}.
    \UCpaso[\UCsist] Muestra la pantalla \cdtIdRef{IUS 19}{Registrar avance de acciones de agua} con la información referente a las acciones de la meta de agua. 
    \UCpaso[\UCactor] Ingresa los datos requeridos en la pantalla \cdtIdRef{IUS 19}{Registrar avance de acciones de agua}. \label{cus19:RegAvanceAgua}
    \UCpaso[\UCactor] Oprime el botón \cdtButton{Aceptar} en la pantalla \cdtIdRef{IUS 19}{Registrar avance de acciones de agua}. \refTray{C}.
    \UCpaso[\UCsist] Verifica que la escuela se encuentre en estado ``Avance en edición''. \refTray{A}.
    \UCpaso[\UCsist] Verifica que la fecha actual se encuentre dentro del periodo definido por la SMAGEM para realizar la operación. \refTray{B}.
    \UCpaso[\UCsist] Verifica que los datos ingresados proporcionen información correcta con base en la regla de negocio \cdtIdRef{RN-S1}{Información correcta}.  \refTray{D}. \refTray{E}.    
    \UCpaso[\UCsist] Calcula el avance acumulado para cada una de las acciones de la meta de agua con base en la regla de negocio \cdtIdRef{RN-N16}{Calcular avance acumulado}.
    \UCpaso[\UCsist] Registra los avances en las acciones de la meta de agua.
    \UCpaso[\UCsist] Muestra el mensaje \cdtIdRef{MSG1}{Operación realizada exitosamente} en la pantalla \cdtIdRef{IUS 18}{Administrar avances de agua} indicando al actor que el avance se ha registrado exitosamente.
 \end{UCtrayectoria}
 
    \begin{UCtrayectoriaA}[Fin del caso de uso]{A}{La escuela no se encuentra en un estado que permita realizar la operación.}
	\UCpaso[\UCsist] Muestra el mensaje \cdtIdRef{MSG28}{Operación no permitida por estado de la entidad} en la pantalla \cdtIdRef{IUS 18}{Administrar avances de agua} indicando al actor que no puede realizar la operación debido a que la escuela no se encuentra en estado ``Avance en edición''. 
    \end{UCtrayectoriaA}
 
 \begin{UCtrayectoriaA}[Fin del caso de uso]{B}{La fecha actual se encuentra fuera del periodo definido por la SMAGEM para realizar la operación.}
    \UCpaso[\UCsist] Muestra el mensaje \cdtIdRef{MSG41}{Acción fuera del periodo} en la pantalla \cdtIdRef{IUS 18}{Administrar avances de agua} indicando al actor que no puede realizar la operación debido a que la fecha actual se encuentra fuera del periodo definido por la SMAGEM para realizarla. 
    \end{UCtrayectoriaA}

    \begin{UCtrayectoriaA}[Fin del caso de uso]{C}{El actor desea cancelar la operación.}
      \UCpaso[\UCactor] Solicita cancelar la operación oprimiendo el botón \cdtButton{Cancelar} en la pantalla \cdtIdRef{IUS 19}{Registrar avance de acciones de agua}.
      \UCpaso[] Regresa a la pantalla \cdtIdRef{IUS 18}{Administrar avances de agua}. 
    \end{UCtrayectoriaA}
        
    \begin{UCtrayectoriaA}{D}{El actor ingresó un dato que excede la longitud máxima.}    
	\UCpaso[\UCsist] Muestra el mensaje \cdtIdRef{MSG7}{Se ha excedido la longitud máxima del campo} en la pantalla \cdtIdRef{IUS 19}{Registrar avance de acciones de agua} indicando que el registro del avance de la acción de la meta de agua no puede realizarse debido a que la longitud del campo excede la longitud máxima definida.
	\UCpaso[] Continúa con el paso \ref{cus19:RegAvanceAgua} de la trayectoria principal.     
    \end{UCtrayectoriaA}
 
    \begin{UCtrayectoriaA}{E}{El actor ingresó un tipo de dato incorrecto.}    
	\UCpaso[\UCsist] Muestra el mensaje \cdtIdRef{MSG6}{Formato incorrecto} en la pantalla \cdtIdRef{IUS 19}{Registrar avance de acciones de agua} indicando que el registro del avance para la acción no puede realizarse debido a que la información ingresada no es correcta.
	\UCpaso[] Continúa con el paso \ref{cus19:RegAvanceAgua} de la trayectoria principal.     
    \end{UCtrayectoriaA}