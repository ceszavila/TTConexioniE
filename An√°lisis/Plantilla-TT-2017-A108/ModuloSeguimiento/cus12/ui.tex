\subsection{IUS 12 Administrar inventario de fauna}

\subsubsection{Objetivo}

    En esta pantalla el \cdtRef{actor:usuarioEscuela}{Coordinador del programa} puede conocer las especies animales que se encuentran ubicadas en los ecosistemas y áreas verdes cercanas a la escuela, que fueron registradas en la información base, el número total de especies y el número total de especies endémicas, además puede acceder a la eliminación y registro de especies.

\subsubsection{Diseño}

    En la figura~\ref{IUS 12} se muestra la pantalla ``Administrar inventario de fauna'', por medio de la cual se podrá acceder al registro, consulta y eliminación de registros de especies animales que fueron registrados en la información base.\\
    
    El actor tendrá la facultad de consultar las especies animales que se encuentran registrados, registrar nuevas especies animales o eliminarlas, esto a través de los botones \cdtButton{Registrar} y \botKo.\\
    
    En el caso de que no existan registros de especies animales en el sistema los campos de ``Total de especies'' y ``Total de especies endémicas'' aparecerán con el número ``0''. Además se mostrará el mensaje \cdtIdRef{MSG2}{No existe información registrada por el momento} indicando que no se encuentran registros de especies animales en el sistema.

  \IUfig[.9]{pantallas/seguimiento/cus12/ius12}{IUS 12}{Administrar inventario de fauna}


\subsubsection{Comandos}
    \begin{itemize}
    \item \cdtButton{Registrar}: Permite al actor registrar especies animales en el sistema, dirige a la pantalla \cdtIdRef{IUS 13}{Registrar fauna}.
    
    \item \botKo[Eliminar]: Permite al actor eliminar especies animales del sistema, dirige a la pantalla emergente \cdtIdRef{IIUS 14}{Eliminar fauna}.
    
    \item \cdtButton{Regresar}: Permite al actor cancelar la administración de inventario de fauna, dirige a la pantalla \cdtIdRef{IUS 11}{Actualizar inventarios de flora y fauna}.
    \end{itemize}

\subsubsection{Mensajes}

    \begin{description}
    \item [\cdtIdRef{MSG2}{No existe información registrada por el momento}:] Se muestra en la pantalla \cdtIdRef{IUS 12}{Administrar inventario de fauna} indicando al actor que no existen registros de especies animales en el sistema por el momento.
    
    \item[\cdtIdRef{MSG28}{Operación no permitida por estado de la entidad}:] Se muestra en la pantalla \cdtIdRef{IUS 11}{Actualizar inventarios de flora y fauna} indicando al actor que no puede administrar el inventario de fauna debido al estado en que se encuentra la escuela.
    
    \item [\cdtIdRef{MSG41}{Acción fuera del periodo}:] Se muestra sobre la pantalla en la pantalla \cdtIdRef{IUS 11}{Actualizar inventarios de flora y fauna} para indicarle al actor que no puede administrar el inventario de fauna debido a que la fecha actual se encuentra fuera del periodo definido por la SMAGEM para realizar la acción.
    \end{description}