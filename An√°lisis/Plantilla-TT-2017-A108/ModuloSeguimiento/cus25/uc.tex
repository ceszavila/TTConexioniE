%!TEX encoding = UTF-8 Unicode

\begin{UseCase}{CUS 25}{Actualizar información del consumo de energía}
    {
	  Este caso de uso permite al actor actualizar la información referente al consumo de energía. Una vez que los datos solicitados para la actualización de la información del consumo de energía han sido ingresados, el sistema valida la información y esta queda registrada.
    }
    
    \UCitem{Versión}{1.0}
    \UCccsection{Administración de Requerimientos}
    \UCitem{Autor}{Jessica Stephany Becerril Delgado}
    \UCccitem{Evaluador}{}
    \UCitem{Operación}{Registro}
    \UCccitem{Prioridad}{Media}
    \UCccitem{Complejidad}{Media}
    \UCccitem{Volatilidad}{Media}
    \UCccitem{Madurez}{Media}
    \UCitem{Estatus}{Terminado}
    \UCitem{Fecha del último estatus}{8 de Diciembre del 2014}


    \UCsection{Atributos}
    \UCitem{Actor}{\cdtRef{actor:usuarioEscuela}{Coordinador del programa}}
    \UCitem{Propósito}{Actualizar la información referente al consumo de energía.}
    \UCitem{Entradas}{
	\begin{UClist}
	    \UCli Información del consumo y abastecimiento de energía:
	    \begin{itemize}
	    \item \cdtRef{energia:cuentaConServicio}{Cuenta con servicio de energía}: \ioRadioBoton.
	    \item \cdtRef{energia:cuentaConRecibos}{Recibos de energía}: \ioRadioBoton.
	    \item \cdtRef{energia:tipoDePeriodo}{Tipo de periodo}: \ioSeleccionar.	    
	    \item \cdtRef{mensual:mes}{Mes}: \ioSeleccionar.
	    \item \cdtRef{bimestral:bimestre}{Bimestre}: \ioSeleccionar.
	    \item \cdtRef{semestral:semestre}{Semestre}: \ioSeleccionar.
	    \item \cdtRef{periodo-consumo:Anio}{Año}: \ioSeleccionar.	    
	    \item \cdtRef{periodo-consumo:consumo}{Consumo por tipo de periodo}: \ioEscribir.
	    \item \cdtRef{periodo-consumo:importe}{Importe por tipo de periodo}: \ioEscribir.
	    
	    \item Consumo total: \ioCalcular \cdtIdRef{RN-N12}{Calcular consumo total}.
	    \item Importe total: \ioCalcular \cdtIdRef{RN-N13}{Calcular importe total}.
	    \end{itemize}
	\end{UClist}
    }

    \UCitem{Salidas}{
	\begin{UClist}
	    \UCli Información del consumo y abastecimiento de energía eléctrica:
	    \begin{itemize}
	    \item Consumo total: \ioCalcular \cdtIdRef{RN-N12}{Calcular consumo total}.
	    \item Importe total: \ioCalcular \cdtIdRef{RN-N13}{Calcular importe total}.
	    \end{itemize}

	    \UCli \cdtIdRef{MSG1}{Operación realizada exitosamente:} Se muestra en la pantalla \cdtIdRef{IUS 22}{Administrar avances de energía} cuando la actualización de la información del consumo de energía se ha realizado correctamente.
	\end{UClist}
    }

    \UCitem{Precondiciones}{
	\begin{UClist}
	    \UCli {\bf Interna:} Que la escuela se encuentre en estado \cdtRef{estado:avanceEdicion}{Avance en edición}.
	    \UCli {\bf Interna:} Que el periodo de registros de avances se encuentre vigente.
	\end{UClist}
    }
    
    \UCitem{Postcondiciones}{
	\begin{UClist}
	    \UCli {\bf Interna:} La información del consumo de energía estará actualizada.
	\end{UClist}
    }
    
    \UCitem{Reglas de negocio}{
    	\begin{UClist}
	    \UCli \cdtIdRef{RN-S1}{Información correcta}: Verifica que la información introducida sea correcta.
	    \UCli \cdtIdRef{RN-N12}{Calcular consumo total}: Calcula el consumo total de energía con base en la información ingresada de los recibos.
	    \UCli \cdtIdRef{RN-N13}{Calcular importe total}: Calcula el importe total de energía con base en la información ingresada de los recibos.
	\end{UClist}
    }
    
    \UCitem{Errores}{
	\begin{UClist}
	    \UCli \cdtIdRef{MSG4}{No se encontró información sustantiva}: Se muestra en la pantalla \cdtIdRef{IUS 22}{Administrar avances de energía} cuando el sistema no cuenta con información en los catálogos de tipo de periodo, mes, bimestre, semestre o año.
	    
	    \UCli \cdtIdRef{MSG5}{Falta un dato requerido para efectuar la operación solicitada}: Se muestra en la pantalla \cdtIdRef{IUS 25}{Actualizar información del consumo de energía} cuando no se ha ingresado un dato marcado como requerido.
	    
	    \UCli \cdtIdRef{MSG6}{Formato incorrecto}: Se muestra en la pantalla \cdtIdRef{IUS 25}{Actualizar información del consumo de energía} cuando el tipo de dato ingresado no cumple con el tipo de dato solicitado en el campo.
	    
	    \UCli \cdtIdRef{MSG7}{Se ha excedido la longitud máxima del campo}: Se muestra en la pantalla \cdtIdRef{IUS 25}{Actualizar información del consumo de energía} cuando se ha excedido la longitud de alguno de los campos.	    
	    
	    \UCli  \cdtIdRef{MSG28}{Operación no permitida por estado de la entidad}: Se muestra en la pantalla \cdtIdRef{IUS 22}{Administrar avances de energía} indicando al actor que no se puede realizar la operación debido al estado en que se encuentra la escuela.
	    
	    \UCli \cdtIdRef{MSG41}{Acción fuera del periodo}: Se muestra en la pantalla \cdtIdRef{IUS 22}{Administrar avances de energía} para indicarle al actor que no puede realizar la operación debido a que la fecha actual se encuentra fuera del periodo definido por la SMAGEM para realizarla.
	    
	\end{UClist}
    }

    \UCitem{Tipo}{Secundario, extiende del caso de uso \cdtIdRef{CUS 22}{Administrar avances de energía}.}

%    \UCitem{Fuente}{
%	\begin{UClist}
%	    \UCli Minuta de la reunión \cdtIdRef{M-17RT}{Reunión de trabajo}.
%	\end{UClist}
%    }
\end{UseCase}

\begin{UCtrayectoria}
    \UCpaso[\UCactor] Solicita actualizar la información referente al consumo de energía oprimiendo el botón \cdtButton{Actualizar} de la pantalla \cdtIdRef{IUS 22}{Administrar avances de energía}.
    \UCpaso[\UCsist] Verifica que la escuela se encuentre en estado ``Avance en edición''. \refTray{A}.
    \UCpaso[\UCsist] Verifica que la fecha actual se encuentre dentro del periodo definido por la SMAGEM para realizar la operación. \refTray{B}.
    \UCpaso[\UCsist] Busca la información referente a los catálogos tipo de periodo, mes, bimestre, semestre y año. \refTray{C}.
    \UCpaso[\UCsist] Busca la información previamente registrada referente a la información del consumo de energía.
    \UCpaso[\UCsist] Muestra la pantalla \cdtIdRef{IUS 25}{Actualizar información del consumo de energía} por medio de la cual se realizará la actualización de información del consumo de energía.
    \UCpaso[\UCactor] Ingresa los datos referentes a la actualización de información del consumo de energía en la pantalla \cdtIdRef{IUS 25}{Actualizar información del consumo de energía}. \label{cus25:IngresarDatos}
    \UCpaso[\UCactor] Oprime el botón \cdtButton{Aceptar} en la pantalla \cdtIdRef{IUS 25}{Actualizar información del consumo de energía} para confirmar la actualización de la información de consumo de energía. \refTray{D}. 
    \UCpaso[\UCsist] Verifica que la escuela se encuentre en estado ``Avance en edición''. \refTray{A}.
    \UCpaso[\UCsist] Verifica que la fecha actual se encuentre dentro del periodo definido por la SMAGEM para realizar la operación. \refTray{B}.
    \UCpaso[\UCsist] Verifica que los datos ingresados proporcionen información correcta con base en la regla de negocio \cdtIdRef{RN-S1}{Información correcta}. \refTray{E}. \refTray{F}. \refTray{G}.
    \UCpaso[\UCsist] Registra la actualización de información referente al consumo de energía.
    \UCpaso[\UCsist] Muestra el mensaje \cdtIdRef{MSG1}{Operación realizada exitosamente} en la pantalla \cdtIdRef{IUS 22}{Administrar avances de energía} para indicar al actor que la actualización de la información del consumo de energía se ha realizado exitosamente.    
 \end{UCtrayectoria}
 
      \begin{UCtrayectoriaA}[Fin del caso de uso]{A}{La escuela no se encuentra en un estado que permita realizar la operación.}
	\UCpaso[\UCsist] Muestra el mensaje \cdtIdRef{MSG28}{Operación no permitida por estado de la entidad} en la pantalla \cdtIdRef{IUS 22}{Administrar avances de energía} indicando al actor que no puede realizar la operación debido a que la escuela no se encuentra en estado ``Avance en edición''. 
    \end{UCtrayectoriaA}
 
 \begin{UCtrayectoriaA}[Fin del caso de uso]{B}{La fecha actual se encuentra fuera del periodo definido por la SMAGEM para realizar la operación.}
    \UCpaso[\UCsist] Muestra el mensaje \cdtIdRef{MSG41}{Acción fuera del periodo} en la pantalla \cdtIdRef{IUS 22}{Administrar avances de energía} indicando al actor que no puede realizar la operación debido a que la fecha actual se encuentra fuera del periodo definido por la SMAGEM para realizarla. 
 \end{UCtrayectoriaA}
 
  \begin{UCtrayectoriaA}[Fin del caso de uso]{C}{No existe información base en los catálogos de tipo de periodo, mes, bimestre, semestre o año.}
    \UCpaso[\UCsist] Muestra el mensaje \cdtIdRef{MSG4}{No se encontró información sustantiva} en la pantalla \cdtIdRef{IUS 22}{Administrar avances de energía} indicando al actor que no puede actualizar la información del consumo de energía debido a que no se cuenta con información sustantiva para los catálogos de tipo de periodo, mes, bimestre, semestre o año.
 \end{UCtrayectoriaA}
 
    \begin{UCtrayectoriaA}[Fin del caso de uso]{D}{El actor desea cancelar la operación.}
    \UCpaso[\UCactor] Solicita cancelar la operación oprimiendo el botón \cdtButton{Cancelar} en la pantalla \cdtIdRef{IUS 25}{Actualizar información del consumo de energía}.
    \UCpaso[] Regresa a la pantalla \cdtIdRef{IUS 22}{Administrar avances de energía}. 
    \end{UCtrayectoriaA}
  
    \begin{UCtrayectoriaA}{E}{El actor no ingresó un dato marcado como requerido.}    
    \UCpaso[\UCsist] Muestra el mensaje \cdtIdRef{MSG5}{Falta un dato requerido para efectuar la operación solicitada} en la pantalla \cdtIdRef{IUS 25}{Actualizar información del consumo de energía} indicando que la actualización de información del consumo de energía no puede realizarse debido a la falta de información requerida.
    \UCpaso[] Continúa con el paso \ref{cus25:IngresarDatos} de la trayectoria principal.     
    \end{UCtrayectoriaA}
 
        \begin{UCtrayectoriaA}{F}{El actor ingresó un tipo de dato incorrecto.}    
    \UCpaso[\UCsist] Muestra el mensaje \cdtIdRef{MSG6}{Formato incorrecto} en la pantalla \cdtIdRef{IUS 25}{Actualizar información del consumo de energía} indicando que la actualización de información referente al consumo de energía no puede realizarse debido a que la información ingresada no es correcta.
    \UCpaso[] Continúa con el paso \ref{cus25:IngresarDatos} de la trayectoria principal.     
    \end{UCtrayectoriaA}
    
    \begin{UCtrayectoriaA}{G}{El actor ingresó un dato que excede la longitud máxima.}    
    \UCpaso[\UCsist] Muestra el mensaje \cdtIdRef{MSG7}{Se ha excedido la longitud máxima del campo} en la pantalla \cdtIdRef{IUS 25}{Actualizar información del consumo de energía} indicando que el registro de información referente al consumo de energía no puede realizarse debido a que la longitud del campo excede la longitud máxima definida.
    \UCpaso[] Continúa con el paso \ref{cus25:IngresarDatos} de la trayectoria principal.     
    \end{UCtrayectoriaA}