%!TEX encoding = UTF-8 Unicode

\begin{UseCase}{CUS 32}{Eliminar residuo sólido}
    {
	Después de registrar un residuo sólido el actor puede haber detectado algún error en la información ingresada o puede ocurrir que el residuo sólido ya no sea producido por la escuela, en ese caso el actor podrá eliminar el registro a través de este caso de uso.
    }
    
    \UCitem{Versión}{1.0}
    \UCccsection{Administración de Requerimientos}
    \UCitem{Autor}{Jessica Stephany Becerril Delgado}
    \UCccitem{Evaluador}{}
    \UCitem{Operación}{Eliminar}
    \UCccitem{Prioridad}{Media}
    \UCccitem{Complejidad}{Media}
    \UCccitem{Volatilidad}{Alta}
    \UCccitem{Madurez}{Media}
    \UCitem{Estatus}{Terminado}
    \UCitem{Fecha del último estatus}{8 de Diciembre del 2014}
    

    \UCsection{Atributos}
    \UCitem{Actor}{\cdtRef{actor:usuarioEscuela}{Coordinador del programa}}
    \UCitem{Propósito}{Eliminar un residuo sólido que fue registrada por error o que después de una revisión se determinó que es innecesario.}
    \UCitem{Entradas}{
	\begin{UClist}
	   \UCli Ninguna.
	\end{UClist}
    }
    \UCitem{Salidas}{
	\begin{UClist}
	    \UCli \cdtIdRef{MSG31}{Confirmar eliminación:} Se muestra en la pantalla emergente \cdtIdRef{IUS 32}{Eliminar residuo sólido} para que el actor confirme la eliminación del residuo sólido. 
	\end{UClist}
    }

    \UCitem{Precondiciones}{
	\begin{UClist}
	    	     \UCli {\bf Interna:} Que la escuela se encuentre en estado \cdtRef{estado:avanceEdicion}{Avance en edición}.
	    \UCli {\bf Interna:} Que el periodo de registros de avances se encuentre vigente.	
	\end{UClist}
    }
    
    \UCitem{Postcondiciones}{
	\begin{UClist}
	    \UCli {\bf Interna:} El registro del residuo sólido desaparecerá del sistema.
	\end{UClist}
    }
    
    \UCitem{Reglas de negocio}{
    	\begin{UClist}
	    \UCli Ninguna.
	\end{UClist}
    }
    
    \UCitem{Errores}{
	\begin{UClist}
		    
	    \UCli \cdtIdRef{MSG28}{Operación no permitida por estado de la entidad}: Se muestra en la pantalla \cdtIdRef{IUS 26}{Administrar avances de residuos sólidos} indicando al actor que no se puede realizar la operación debido al estado en que se encuentra la escuela.
	    
	    \UCli \cdtIdRef{MSG41}{Acción fuera del periodo}: Se muestra en la pantalla \cdtIdRef{IUS 26}{Administrar avances de residuos sólidos} para indicarle al actor que no puede realizar la operación debido a que la fecha actual se encuentra fuera del periodo definido por la SMAGEM para realizarla.
	    
	\end{UClist}
    }

    \UCitem{Tipo}{Secundario, extiende del caso de uso \cdtIdRef{CUS 29}{Actualizar información de residuos sólidos}.}

%    \UCitem{Fuente}{
%	\begin{UClist}
%	    \UCli Minuta de la reunión \cdtIdRef{M-17RT}{Reunión de trabajo}.
%	\end{UClist}
 %   }
\end{UseCase}

 \begin{UCtrayectoria}
    \UCpaso[\UCactor] Solicita eliminar un residuo sólido oprimiendo el botón \botKo del registro correspondiente en la pantalla \cdtIdRef{IUS 29}{Actualizar información de residuos sólidos}.
    \UCpaso[\UCsist] Verifica que la escuela se encuentre en estado ``Avance en edición''. \refTray{A}.
    \UCpaso[\UCsist] Verifica que la fecha actual se encuentre dentro del periodo definido por la SMAGEM para realizar la operación. \refTray{B}.
    \UCpaso[\UCsist] Muestra el mensaje \cdtIdRef{MSG31}{Confirmar eliminación} en una pantalla emergente para que el actor confirme la eliminación del residuo sólido.
    \UCpaso[\UCactor] Oprime el botón \cdtButton{Aceptar} de la pantalla emergente. \refTray{C}.
    \UCpaso[\UCsist] Verifica que la escuela se encuentre en estado ``Avance en edición''. \refTray{A}.
    \UCpaso[\UCsist] Verifica que la fecha actual se encuentre dentro del periodo definido por la SMAGEM para realizar la operación. \refTray{B}.
    \UCpaso[\UCsist] Elimina la información del residuo sólido del sistema.
    \UCpaso[\UCsist] Muestra el mensaje \cdtIdRef{MSG1}{Operación realizada exitosamente} en la pantalla \cdtIdRef{IUS 29}{Actualizar información de residuos sólidos} para indicar al actor que el registro del residuo sólido se ha eliminado exitosamente. 
    
 \end{UCtrayectoria}
 
    \begin{UCtrayectoriaA}[Fin del caso de uso]{A}{La escuela no se encuentra en un estado que permita realizar la operación.}
    \UCpaso[\UCsist] Muestra el mensaje \cdtIdRef{MSG28}{Operación no permitida por estado de la entidad} en la pantalla \cdtIdRef{IUS 26}{Administrar avances de residuos sólidos} indicando al actor que no puede realizar la operación debido a que la escuela no se encuentra en estado ``Avance en edición''. 
 \end{UCtrayectoriaA}
 
    \begin{UCtrayectoriaA}[Fin del caso de uso]{B}{La fecha actual se encuentra fuera del periodo definido por la SMAGEM para realizar la operación.}
    \UCpaso[\UCsist] Muestra el mensaje \cdtIdRef{MSG41}{Acción fuera del periodo} en la pantalla \cdtIdRef{IUS 26}{Administrar avances de residuos sólidos} indicando al actor que no puede realizar la operación  debido a que la fecha actual se encuentra fuera del periodo definido por la SMAGEM para realizarla. 
 \end{UCtrayectoriaA}
 
    \begin{UCtrayectoriaA}[Fin del caso de uso]{C}{El actor desea cancelar la eliminación del registro.}
    \UCpaso[\UCactor] Solicita cancelar eliminación del registro del residuo sólido oprimiendo el botón \cdtButton{Cancelar} de la pantalla emergente.
    \UCpaso[] Regresa a la pantalla \cdtIdRef{IUS 29}{Actualizar información de residuos sólidos}.    
 \end{UCtrayectoriaA}
