%!TEX encoding = UTF-8 Unicode

\begin{UseCase}{CUS 11}{Actualizar inventarios de flora y fauna}
{
    A lo largo del seguimiento se reportan los avances obtenidos de la aplicación del plan de acción. Este caso de uso permite acceder al registro o modificación de las especies de flora y fauna que hasta el momento cuenta la escuela, actualizando la información que ha sido proporcionada en el registro de la información base.  
}
    
    \UCitem{Versión}{1.0}
    \UCccsection{Administración de Requerimientos}
    \UCitem{Autor}{Francisco Javier Ponce Cruz}
    \UCccitem{Evaluador}{}
    \UCitem{Operación}{Administración}
    \UCccitem{Prioridad}{Media}
    \UCccitem{Complejidad}{Media}
    \UCccitem{Volatilidad}{Alta}
    \UCccitem{Madurez}{Media}
    \UCitem{Estatus}{Terminado}
    \UCitem{Fecha del último estatus}{9 diciembre 2014}
    
%% Copie y pegue este bloque tantas veces como revisiones tenga el caso de uso.
%% Esta sección la debe llenar solo el Revisor
% %--------------------------------------------------------
%   \UCccsection{Revisión Versión XX} % Anote la versión que se revisó.
%   % FECHA: Anote la fecha en que se terminó la revisión.
%   \UCccitem{Fecha}{Fecha en que se termino la revisión} 
%   % EVALUADOR: Coloque el nombre completo de quien realizó la revisión.
%   \UCccitem{Evaluador}{Nombre de quien revisó}
%   % RESULTADO: Coloque la palabra que mas se apegue al tipo de acción que el analista debe realizar.
%   \UCccitem{Resultado}{Corregir, Desechar, Rehacer todo, terminar.}
%   % OBSERVACIONES: Liste los cambios que debe realizar el Analista.
%   \UCccitem{Observaciones}{
%       \begin{UClist}
%           % PC: Petición de Cambio, describa el trabajo a realizar, si es posible indique la causa de la PC. Opcionalmente especifique la fecha en que considera razonable que se deba terminar la PC. No olvide que la numeración no se debe reiniciar en una segunda o tercera revisión.
%           \RCitem{PC1}{\TODO{Descripción del pendiente}}{Fecha de entrega}
%           \RCitem{PC2}{\TODO{Descripción del pendiente}}{Fecha de entrega}
%           \RCitem{PC3}{\TODO{Descripción del pendiente}}{Fecha de entrega}
%       \end{UClist}        
%   }
% %--------------------------------------------------------

    \UCsection{Atributos}
    \UCitem{Actor}{\cdtRef{actor:usuarioEscuela}{Coordinador del programa}}
    \UCitem{Propósito}{Actualizar la información base para indicadores de biodiversidad y  los inventarios de flora y fauna.}
    \UCitem{Entradas}{
    \begin{UClist}
        \UCli Ninguna.
    \end{UClist}
    }
    \UCitem{Salidas}{
    \begin{UClist} 
        \UCli Ninguna.
    \end{UClist}
    }

    \UCitem{Precondiciones}{
    \begin{UClist}
        \UCli {\bf Interna:} Que la escuela se encuentre en estado \cdtRef{estado:avanceEdicion}{Avance en edición}.
        \UCli {\bf Interna:} Que el periodo de registros de avances se encuentre vigente. 
    \end{UClist}
    }
    
    \UCitem{Postcondiciones}{
    \begin{UClist}
        \UCli {\bf Interna:} Se podrá actualizar el inventario de fauna a través del caso de uso \cdtIdRef{CUS 12}{Administrar inventario de fauna}.
        \UCli {\bf Interna:} Se podrá actualizar el inventario de flora a través del caso de uso \cdtIdRef{CUS 15}{Administrar inventario de flora}.
    \end{UClist}
    }
    
    \UCitem{Reglas de negocio}{
        \begin{UClist}
        \UCli Ninguna.
    \end{UClist}
    }
    
    \UCitem{Errores}{
    \begin{UClist}
        \UCli \cdtIdRef{MSG28}{Operación no permitida por estado de la entidad}: Se muestra en la pantalla \cdtIdRef{IUS 8}{Administrar avances de biodiversidad} indicando al actor que no se pueden actualizar los inventarios debido al estado en que se encuentra la escuela.
        
        \UCli \cdtIdRef{MSG41}{Acción fuera del periodo}: Se muestra en la pantalla \cdtIdRef{IUS 8}{Administrar avances de biodiversidad} para indicarle al actor que no pueden actualizar los inventarios debido a que la fecha actual se encuentra fuera del periodo definido por la SMAGEM para realizar la acción.
    \end{UClist}
    }

    \UCitem{Tipo}{Secudario, extiende del caso de uso \cdtIdRef{CUS 8}{Administrar avances de biodiversidad}.}

%   \UCitem{Fuente}{
%   \begin{UClist}
%       \UCli Minuta de la reunión \cdtIdRef{M-17RT}{Reunión de trabajo}.
%   \end{UClist}
 %  }
\end{UseCase}

 \begin{UCtrayectoria}
    \UCpaso[\UCactor] Solicita actualizar los inventarios de flora y fauna, oprimiendo el botón \cdtButton{Actualizar} en la pantalla \cdtIdRef{IUS 8}{Administrar avances de biodiversidad}.
    \UCpaso[\UCsist] Verifica que la escuela se encuentre en estado ``Avance en edición''. \refTray{A}.
    \UCpaso[\UCsist] Verifica que la fecha actual se encuentre dentro del periodo definido por la SMAGEM para administrar los avances. \refTray{B}.
    \UCpaso[\UCsist] Muestra la pantalla \cdtIdRef{IUS 11}{Actualizar inventarios de flora y fauna}.
    \UCpaso[\UCactor] Administra los inventarios de flora y fauna, a través del botón \botEdit. \label{cus11:RegistrarIA}
 \end{UCtrayectoria}
 
 \begin{UCtrayectoriaA}[Fin del caso de uso]{A}{La escuela no se encuentra en un estado que permita administrar avances.}
    \UCpaso[\UCsist] Muestra el mensaje \cdtIdRef{MSG28}{Operación no permitida por estado de la entidad} indicando al actor que no puede actualizar los inventarios debido a que la escuela no se encuentra en estado ``Avance en edición''.
 \end{UCtrayectoriaA}

\begin{UCtrayectoriaA}[Fin del caso de uso]{B}{La fecha actual se encuentra fuera del periodo definido por la SMAGEM para administrar avances.}
    \UCpaso[\UCsist] Muestra el mensaje \cdtIdRef{MSG41}{Acción fuera del periodo} en la pantalla \cdtIdRef{IUS 8}{Administrar avances de biodiversidad} indicando al actor que no puede actualizar los inventarios debido a que la fecha actual se encuentra fuera del periodo definido por la SMAGEM para realizar la acción. 
 \end{UCtrayectoriaA}

\subsection{Puntos de extensión}

\UCExtensionPoint
{El actor desea actualizar el inventario de fauna}
{Paso \ref{cus11:RegistrarIA} de la trayectoria principal}
{\cdtIdRef{CUS 12}{Administrar inventario de fauna}}

\UCExtensionPoint
{El actor desea administrar el inventario de flora}
{ Paso \ref{cus11:RegistrarIA} de la trayectoria principal}
{\cdtIdRef{CUS 15}{Administrar inventario de flora}}
