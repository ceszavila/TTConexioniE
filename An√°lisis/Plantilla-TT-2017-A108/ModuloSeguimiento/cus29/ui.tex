\subsection{IUS 29 Actualizar información de residuos sólidos}

\subsubsection{Objetivo}

    En esta pantalla el \cdtRef{actor:usuarioEscuela}{Coordinador del programa} puede conocer los residuos sólidos que se generan principalmente en la escuela, además podrá actualizar esta información a través del registro, modificación y eliminación de registros de residuos sólidos.

\subsubsection{Diseño}

    En la figura~\ref{IUS 29} se muestra la pantalla ``Actualizar información de residuos sólidos'', por medio de la cual se podrá actualizar la información referente a los residuos sólidos que se producen en la escuela. El actor tendrá la facultad de consultar los residuos sólidos que se encuentran en el sistema, registrar nuevos residuos sólidos, modificarlos o eliminarlos, esto a través de los botones \cdtButton{Registrar}, \botEdit y \botKo respectivamente.\\
    
    \IUfig[.9]{pantallas/seguimiento/cus29/IUS29ActualizarInformacion.png}{IUS 29}{Actualizar información de residuos sólidos}
    %\IUfig[.7]{pantallas/InformacionBase/cuibr1/IUIBR1EliminarResiduo.png}{IUIBR 4}{Eliminar residuo sólido}


\subsubsection{Comandos}
    \begin{itemize}
	\item \cdtButton{Registrar}: Permite al actor registrar residuos sólidos en el sistema, dirige a la pantalla \cdtIdRef{IUS 30}{Registrar residuo sólido}.
	
	\item \botEdit [Modificar]: Permite al actor modificar residuos sólidos en el sistema, dirige a la pantalla \cdtIdRef{IUS 31}{Modificar residuo sólido}.
	
	\item \botKo [Eliminar]: Permite al actor eliminar residuos sólidos del sistema, dirige a la pantalla emergente \cdtIdRef{IUS 32}{Eliminar residuo sólido}.
	
    \end{itemize}

\subsubsection{Mensajes}

    \begin{description}
	\item [\cdtIdRef{MSG2}{No existe información registrada por el momento}:] Se muestra en la pantalla \cdtIdRef{IUS 29}{Actualizar información de residuos sólidos} indicando al actor que no existen registros de residuos sólidos en el sistema por el momento.
	
	\item [\cdtIdRef{MSG28}{Operación no permitida por estado de la entidad}:] Se muestra en la pantalla \cdtIdRef{IUS 26}{Administrar avances de residuos sólidos} indicando al actor que no se puede realizar la operación debido al estado en que se encuentra la escuela.
	    
	    \item [\cdtIdRef{MSG41}{Acción fuera del periodo}:] Se muestra en la pantalla \cdtIdRef{IUS 26}{Administrar avances de residuos sólidos} para indicarle al actor que no puede realizar la operación debido a que la fecha actual se encuentra fuera del periodo definido por la SMAGEM para realizarla.
    \end{description}