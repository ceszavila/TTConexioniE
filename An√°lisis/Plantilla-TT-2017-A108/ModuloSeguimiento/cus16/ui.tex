\subsection{IUS 16 Registrar flora}

\subsubsection{Objetivo}

En esta pantalla el \cdtRef{actor:usuarioEscuela}{Coordinador del programa} puede registrar en el sistema las especies vegetales que se encuentran ubicadas en los ecosistemas y áreas verdes cercanas a la escuela durante el seguimiento del plan de acción.

\subsubsection{Diseño}

    En la figura~\ref{IUS 16} se muestra la pantalla ``Registrar flora'', por medio de la cual se podrán registrar las especies vegetales que se localizan en los ecosistemas y áreas verdes cercanos a la escuela durante el seguimiento del plan de acción. El actor deberá ingresar la información solicitada para el registro de la especie vegetal en la pantalla .\\
        
    Una vez que se haya ingresado toda la información solicitada para el registro de la especie vegetal deberá oprimir el botón \cdtButton{Aceptar}, el sistema validará y registrará la información sólo si se han cumplido todas las reglas de negocio establecidas.\\
    
    Finalmente se mostrará el mensaje \cdtIdRef{MSG1}{Operación realizada exitosamente} en la pantalla \cdtIdRef{IUS 15}{ Administrar inventario de flora}, para indicar que la información de la especie vegetal se ha registrado exitosamente.
      
    \IUfig[.9]{pantallas/seguimiento/cus16/ius16}{IUS 16}{Registrar flora}

\subsubsection{Comandos}
    \begin{itemize}
    \item \cdtButton{Aceptar}: Permite al actor confirmar el registro de la especie vegetal, dirige a la pantalla \cdtIdRef{IUS 15}{Administrar inventario de flora}.
    \item \cdtButton{Cancelar}: Permite al actor cancelar el registro de la especie vegetal, dirige a la pantalla \cdtIdRef{IUS 15}{Administrar inventario de flora}.
    \end{itemize}

\subsubsection{Mensajes}

    \begin{description}
      
        \item [\cdtIdRef{MSG1}{Operación realizada exitosamente}:] Se muestra en la pantalla \cdtIdRef{IUS 15}{Administrar inventario de flora} cuando el registro de la especie vegetal se ha realizado correctamente.        
        
        \item [\cdtIdRef{MSG4}{No se encontró información sustantiva}:] Se muestra en la pantalla \cdtIdRef{IUS 15}{Administrar inventario de flora} cuando el sistema no cuenta con información en el catálogo de categoría.
        
        \item [\cdtIdRef{MSG5}{Falta un dato requerido para efectuar la operación solicitada}:] Se muestra en la pantalla \cdtIdRef{IUS 16}{Registrar flora} cuando no se ha ingresado un dato marcado como requerido.
        
         \item [\cdtIdRef{MSG6}{Formato incorrecto}:] Se muestra en la pantalla \cdtIdRef{IUS 16}{Registrar flora} cuando el tipo de dato ingresado no cumple con el tipo de dato solicitado en el campo.
        
        \item [\cdtIdRef{MSG7}{Se ha excedido la longitud máxima del campo}:] Se muestra en la pantalla \cdtIdRef{IUS 16}{Registrar flora} cuando se ha excedido la longitud de alguno de los campos.
        
        \item [\cdtIdRef{MSG8}{Registro repetido}:] Se muestra en la pantalla \cdtIdRef{IUS 16}{Registrar flora} cuando el actor proporcionó un nombre científico  para la especie vegetal que ya se encuentra registrado en el sistema.
         
        \item[\cdtIdRef{MSG28}{Operación no permitida por estado de la entidad}:] Se muestra en la pantalla \cdtIdRef{IUS 15}{Administrar inventario de flora} indicando al actor que no se puede registrar una especie vegetal debido al estado en que se encuentra la escuela.
       
       \item [\cdtIdRef{MSG41}{Acción fuera del periodo}:] Se muestra sobre la pantalla en la pantalla \cdtIdRef{IUS 15}{Administrar inventario de flora} para indicarle al actor que no puede registrar una especie vegetal debido a que la fecha actual se encuentra fuera del periodo definido por la SMAGEM para realizar el registro.
        
    \end{description}
