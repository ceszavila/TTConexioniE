%!TEX encoding = UTF-8 Unicode

\begin{UseCase}{CUS 30}{Registrar residuo sólido}
    {
	Los residuos sólidos son aquellos desechos procedentes de materiales utilizados en la fabricación, transformación o utilización de bienes de consumo, este caso de uso permite al actor registrar los residuos sólidos que se producen en la escuela.
    }
    
    \UCitem{Versión}{1.0}
    \UCccsection{Administración de Requerimientos}
    \UCitem{Autor}{Jessica Stephany Becerril Delgado}
    \UCccitem{Evaluador}{}
    \UCitem{Operación}{Registro}
    \UCccitem{Prioridad}{Media}
    \UCccitem{Complejidad}{Media}
    \UCccitem{Volatilidad}{Alta}
    \UCccitem{Madurez}{Media}
    \UCitem{Estatus}{Terminado}
    \UCitem{Fecha del último estatus}{8 de Diciembre del 2014}
    

    \UCsection{Atributos}
    \UCitem{Actor}{\cdtRef{actor:usuarioEscuela}{Coordinador del programa}}
    \UCitem{Propósito}{Registra la información referente a los residuos sólidos que se producen en la escuela.}
    \UCitem{Entradas}{
	\begin{UClist}
	   \UCli {\bf Información del residuo sólido:}
	   \begin{itemize}
	    \item \cdtRef{residuoSolido:tipoDeResiduo}{Tipo}: \ioSeleccionar.
	    \item \cdtRef{residuoSolido:residuo}{Residuo}: \ioSeleccionar.
	    \item \cdtRef{residuoSolido:cantidadSemanal}{Cantidad generada a la semana}: \ioEscribir.
	    \item \cdtRef{residuoSolido:cantidadReciclaje}{Cantidad separada para reciclaje}: \ioEscribir.
	    \end{itemize}
	\end{UClist}
    }
    \UCitem{Salidas}{
	\begin{UClist}
	    \UCli \cdtIdRef{MSG1}{Operación realizada exitosamente:} Se muestra en la pantalla \cdtIdRef{IUS 29}{Actualizar información de residuos sólidos} cuando el registro del residuo sólido se ha realizado correctamente.
	\end{UClist}
    }

    \UCitem{Precondiciones}{
	\begin{UClist}
	     \UCli {\bf Interna:} Que la escuela se encuentre en estado \cdtRef{estado:avanceEdicion}{Avance en edición}.
	    \UCli {\bf Interna:} Que el periodo de registros de avances se encuentre vigente.	
	\end{UClist}
    }
    
    \UCitem{Postcondiciones}{
	\begin{UClist}	    
	    \UCli {\bf Interna:} Se podrá modificar el registro del residuo sólido a través del caso de uso \cdtIdRef{CUS 31}{Modificar residuo sólido}.
	    \UCli {\bf Interna:} Se podrá eliminar el registro del residuo sólido a través del caso de uso \cdtIdRef{CUS 32}{Eliminar residuo sólido}.
	\end{UClist}
    }
    
    \UCitem{Reglas de negocio}{
    	\begin{UClist}
	    \UCli \cdtIdRef{RN-S1}{Información correcta}: Verifica que la información introducida sea correcta.
	\end{UClist}
    }
    
    \UCitem{Errores}{
	\begin{UClist}
	
	    \UCli \cdtIdRef{MSG4}{No se encontró información sustantiva}: Se muestra en la pantalla \cdtIdRef{IUS 29}{Actualizar información de residuos sólidos} cuando el sistema no cuenta con información en los catálogos de origen y tipo.
	    
	    \UCli \cdtIdRef{MSG5}{Falta un dato requerido para efectuar la operación solicitada}: Se muestra en la pantalla \cdtIdRef{IUS 30}{Registrar residuo sólido} cuando no se ha ingresado un dato marcado como requerido.
	    
	     \UCli \cdtIdRef{MSG6}{Formato incorrecto}: Se muestra en la pantalla \cdtIdRef{IUS 30}{Registrar residuo sólido} cuando el tipo de dato ingresado no cumple con el tipo de dato solicitado en el campo.
	    
	    \UCli \cdtIdRef{MSG7}{Se ha excedido la longitud máxima del campo}: Se muestra en la pantalla \cdtIdRef{IUS 30}{Registrar residuo sólido} cuando se ha excedido la longitud de alguno de los campos.
	    
	    \UCli \cdtIdRef{MSG28}{Operación no permitida por estado de la entidad}: Se muestra en la pantalla \cdtIdRef{IUS 26}{Administrar avances de residuos sólidos} indicando al actor que no se puede realizar la operación debido al estado en que se encuentra la escuela.
	    
	    \UCli \cdtIdRef{MSG41}{Acción fuera del periodo}: Se muestra en la pantalla \cdtIdRef{IUS 26}{Administrar avances de residuos sólidos} para indicarle al actor que no puede realizar la operación debido a que la fecha actual se encuentra fuera del periodo definido por la SMAGEM para realizarla.
	\end{UClist}
    }

    \UCitem{Tipo}{Secundario, extiende del caso de uso \cdtIdRef{CUS 29}{Actualizar información de residuos sólidos}.}

%    \UCitem{Fuente}{
%	\begin{UClist}
%	    \UCli Minuta de la reunión \cdtIdRef{M-17RT}{Reunión de trabajo}.
%	\end{UClist}
 %   }
\end{UseCase}

 \begin{UCtrayectoria}
    \UCpaso[\UCactor] Solicita registrar un residuo sólido oprimiendo el botón \cdtButton{Registrar} en la pantalla \cdtIdRef{IUS 29}{Actualizar información de residuos sólidos}.
    \UCpaso[\UCsist] Verifica que la escuela se encuentre en estado ``Avance en edición''. \refTray{A}.
    \UCpaso[\UCsist] Verifica que la fecha actual se encuentre dentro del periodo definido por la SMAGEM para realizar la operación. \refTray{B}.
    \UCpaso[\UCsist] Busca la información de origen y tipo registrada en el sistema. \refTray{C}.
    \UCpaso[\UCsist] Muestra la pantalla \cdtIdRef{IUS 30}{Registrar residuo sólido}.
    \UCpaso[\UCactor] Ingresa los datos del residuo sólido en la pantalla \cdtIdRef{IUS 30}{Registrar residuo sólido}. \label{cus30:Registrar}
    \UCpaso[\UCactor] Solicita guardar la información del residuo sólido oprimiendo el botón \cdtButton{Aceptar} en la pantalla \cdtIdRef{IUS 30}{Registrar residuo sólido}. \refTray{D}.
    \UCpaso[\UCsist] Verifica que la escuela se encuentre en estado ``Avance en edición''. \refTray{A}.
    \UCpaso[\UCsist] Verifica que la fecha actual se encuentre dentro del periodo definido por la SMAGEM para realizar la operación. \refTray{B}.
    \UCpaso[\UCsist] Verifica que los datos requeridos sean proporcionados correctamente como se especifica en la regla de negocio \cdtIdRef{RN-S1}{Información correcta}. \refTray{E}. \refTray{F}.
    \UCpaso[\UCsist] Registra la información del residuo sólido en el sistema.
    \UCpaso[\UCsist] Muestra el mensaje \cdtIdRef{MSG1}{Operación realizada exitosamente} en la pantalla \cdtIdRef{IUS 29}{Actualizar información de residuos sólidos} para indicar al actor que el registro del residuo sólido se ha realizado exitosamente. 
    
 \end{UCtrayectoria}
 
   \begin{UCtrayectoriaA}[Fin del caso de uso]{A}{La escuela no se encuentra en un estado que permita realizar la operación.}
    \UCpaso[\UCsist] Muestra el mensaje \cdtIdRef{MSG28}{Operación no permitida por estado de la entidad} en la pantalla \cdtIdRef{IUS 26}{Administrar avances de residuos sólidos} indicando al actor que no puede realizar la operación debido a que la escuela no se encuentra en estado ``Avance en edición''. 
 \end{UCtrayectoriaA}
 
    \begin{UCtrayectoriaA}[Fin del caso de uso]{B}{La fecha actual se encuentra fuera del periodo definido por la SMAGEM para realizar la operación.}
    \UCpaso[\UCsist] Muestra el mensaje \cdtIdRef{MSG41}{Acción fuera del periodo} en la pantalla \cdtIdRef{IUS 26}{Administrar avances de residuos sólidos} indicando al actor que no puede realizar la operación debido a que la fecha actual se encuentra fuera del periodo definido por la SMAGEM para realizarla. 
 \end{UCtrayectoriaA}
 
 \begin{UCtrayectoriaA}[Fin del caso de uso]{C}{No existe información base en el catálogo de origen o tipo.}
    \UCpaso[\UCsist] Muestra el mensaje \cdtIdRef{MSG4}{No se encontró información sustantiva} en la pantalla \cdtIdRef{IUS 29}{Actualizar información de residuos sólidos} indicando al actor que no puede registrar la información del residuo sólido debido a que no se cuenta con información sustantiva para el catálogo de origen o tipo.
 \end{UCtrayectoriaA}
 
    \begin{UCtrayectoriaA}[Fin del caso de uso]{D}{El actor desea cancelar la operación.}
    \UCpaso[\UCactor] Solicita cancelar la operación oprimiendo el botón \cdtButton{Cancelar} en la pantalla \cdtIdRef{IUS 30}{Registrar residuo sólido}.
    \UCpaso[\UCsist] Regresa a la pantalla \cdtIdRef{IUS 29}{Actualizar información de residuos sólidos}. 
    \end{UCtrayectoriaA}
 
    \begin{UCtrayectoriaA}{E}{El actor no ingresó un dato marcado como requerido.}    
    \UCpaso[\UCsist] Muestra el mensaje \cdtIdRef{MSG5}{Falta un dato requerido para efectuar la operación solicitada} en la pantalla \cdtIdRef{IUS 30}{Registrar residuo sólido} indicando que el registro del residuo sólido no puede realizarse debido a la falta de información requerida.
    \UCpaso[] Continúa con el paso \ref{cus30:Registrar} de la trayectoria principal.     
    \end{UCtrayectoriaA}
 
        \begin{UCtrayectoriaA}{F}{El actor ingresó un tipo de dato incorrecto.}    
    \UCpaso[\UCsist] Muestra el mensaje \cdtIdRef{MSG6}{Formato incorrecto} en la pantalla \cdtIdRef{IUS 30}{Registrar residuo sólido} indicando que el registro del residuo sólido no puede realizarse debido a que la información ingresada no es correcta.
    \UCpaso[] Continúa con el paso \ref{cus30:Registrar} de la trayectoria principal.     
    \end{UCtrayectoriaA}
    
            \begin{UCtrayectoriaA}{E}{El actor ingresó un dato que excede la longitud máxima.}    
    \UCpaso[\UCsist] Muestra el mensaje \cdtIdRef{MSG7}{Se ha excedido la longitud máxima del campo} en la pantalla \cdtIdRef{IUS 30}{Registrar residuo sólido} indicando que el registro del residuo sólido no puede realizarse debido a que la longitud del campo excede la longitud máxima definida.
    \UCpaso[] Continúa con el paso \ref{cus30:Registrar} de la trayectoria principal.     
    \end{UCtrayectoriaA}
