\begin{UseCase}{CUS 9}{Registrar avance de acciones de biodiversidad}
    {
    Las acciones son actividades que se requieren completar para alcanzar una meta. Este caso de uso permite visualizar las acciones registradas para una meta de biodiversidad, visualizar si se encuentran finalizadas, así como el registro de sus avances. 
    }
    
    \UCitem{Versión}{1.0}
    \UCccsection{Administración de Requerimientos}
    \UCitem{Autor}{Francisco Javier Ponce Cruz}
    \UCccitem{Evaluador}{}
    \UCitem{Operación}{Registrar}
    \UCccitem{Prioridad}{Media}
    \UCccitem{Complejidad}{Media}
    \UCccitem{Volatilidad}{Media}
    \UCccitem{Madurez}{Media}
    \UCitem{Estatus}{Terminado}
    \UCitem{Fecha del último estatus}{8 de Diciembre del 2014}

    
%% Copie y pegue este bloque tantas veces como revisiones tenga el caso de uso.
%% Esta sección la debe llenar solo el Revisor
% %--------------------------------------------------------
%   \UCccsection{Revisión Versión XX} % Anote la versión que se revisó.
%   % FECHA: Anote la fecha en que se terminó la revisión.
%   \UCccitem{Fecha}{Fecha en que se termino la revisión} 
%   % EVALUADOR: Coloque el nombre completo de quien realizó la revisión.
%   \UCccitem{Evaluador}{Nombre de quien revisó}
%   % RESULTADO: Coloque la palabra que mas se apegue al tipo de acción que el analista debe realizar.
%   \UCccitem{Resultado}{Corregir, Desechar, Rehacer todo, terminar.}
%   % OBSERVACIONES: Liste los cambios que debe realizar el Analista.
%   \UCccitem{Observaciones}{
%       \begin{UClist}
%           % PC: Petición de Cambio, describa el trabajo a realizar, si es posible indique la causa de la PC. Opcionalmente especifique la fecha en que considera razonable que se deba terminar la PC. No olvide que la numeración no se debe reiniciar en una segunda o tercera revisión.
%           \RCitem{PC1}{\TODO{Descripción del pendiente}}{Fecha de entrega}
%           \RCitem{PC2}{\TODO{Descripción del pendiente}}{Fecha de entrega}
%           \RCitem{PC3}{\TODO{Descripción del pendiente}}{Fecha de entrega}
%       \end{UClist}        
%   }
% %--------------------------------------------------------

    \UCsection{Atributos}
    \UCitem{Actor}{\cdtRef{actor:usuarioEscuela}{Coordinador del programa}}
    \UCitem{Propósito}{Registrar los avances correspondientes a las acciones de una meta de biodiversidad.}
    \UCitem{Entradas}{
    \begin{UClist}
        \UCli \cdtRef{gls:avanceAccion}{Avance}: \ioEscribir.
        \UCli \cdtRef{accion:finalizada}{Acción finalizada}: \ioRadioBoton.
    \end{UClist}
    }
    \UCitem{Salidas}{
    \begin{UClist} 
        \UCli \cdtRef{accion:descripcion}{Acción}: \ioObtener.
        \UCli \cdtRef{accion:recursos}{Recursos materiales y financieros}: \ioObtener.
        \UCli \cdtRef{accion:cuantificable}{Cuantificable}: \ioObtener.
        \UCli \cdtRef{accion:avanceAcumulado}{Avance acumulado}: \ioCalcular \cdtIdRef{RN-N16}{Calcular avance acumulado}.
        \UCli \cdtRef{accion:valorAlcanzar}{Valor a alcanzar}: \ioObtener.
        \UCli \cdtRef{accion:unidad}{Unidad}: \ioObtener.
        
        \UCli \cdtIdRef{MSG1}{Operación realizada exitosamente:} Se muestra en la pantalla \cdtIdRef{IUS 8}{Administrar avances de biodiversidad} cuando el registro del avance para la acción se ha realizado correctamente.
    \end{UClist}
    }
    
    \UCitem{Precondiciones}{
    \begin{UClist}
        \UCli {\bf Interna:} Que la escuela se encuentre en estado \cdtRef{estado:avanceEdicion}{Avance en edición}.
        \UCli {\bf Interna:} Que el periodo de registros de avances se encuentre vigente.
    \end{UClist}
    }
    
    \UCitem{Postcondiciones}{
    \begin{UClist}
        \UCli {\bf Interna:} La acción tendrá un avance registrado.
    \end{UClist}
    }

    \UCitem{Reglas de negocio}{
        \begin{UClist}
        \UCli \cdtIdRef{RN-N16}{Calcular avance acumulado}: Calcula el avance acumulado para cada uno de los valores asociados a la meta de biodiversidad.
    \end{UClist}
    }

    \UCitem{Errores}{
    \begin{UClist}
        
        \UCli \cdtIdRef{MSG6}{Formato incorrecto}: Se muestra en la pantalla \cdtIdRef{IUS 9}{Registrar avance de acciones de biodiversidad} cuando el tipo de dato ingresado no cumple con el tipo de dato solicitado en el campo.
        
        \UCli \cdtIdRef{MSG7}{Se ha excedido la longitud máxima del campo}: Se muestra en la pantalla \cdtIdRef{IUS 9}{Registrar avance de acciones de biodiversidad} cuando se ha excedido la longitud de alguno de los campos.    
        
        \UCli \cdtIdRef{MSG28}{Operación no permitida por estado de la entidad}: Se muestra en la pantalla \cdtIdRef{IUS 8}{Administrar avances de biodiversidad} indicando al actor que no se pueden registrar avances en las acciones de la meta de biodiversidad debido al estado en que se encuentra el seguimiento.
    \end{UClist}
    }

    
    \UCitem{Tipo}{Secundario, extiende del caso de uso \cdtIdRef{CUS 8}{Administrar avances de biodiversidad}.}

%    \UCitem{Fuente}{
%   \begin{UClist}
%       \UCli Minuta de la reunión \cdtIdRef{M-17RT}{Reunión de trabajo}.
%   \end{UClist}
%    }

\end{UseCase}


\begin{UCtrayectoria}
    \UCpaso[\UCactor] Solicita registrar un avance en las acciones registradas para una meta oprimiendo el botón \botAutoAjus del registro correspondiente en la pantalla \cdtIdRef{IUS 8}{Administrar avances de biodiversidad}.
    \UCpaso[\UCsist] Verifica que la escuela se encuentre en estado ``Avance en edición''. \refTray{A}.
    \UCpaso[\UCsist] Verifica que la fecha actual se encuentre dentro del periodo definido por la SMAGEM para registrar avances. \refTray{E}.
    %\UCpaso[\UCsist] Calcula el avance acumulado para cada una de las acciones de la meta de biodiversidad con base en la regla de negocio \cdtIdRef{RN-N16}{Calcular avance acumulado}.
    \UCpaso[\UCsist] Muestra la pantalla \cdtIdRef{IUS 9}{Registrar avance de acciones de biodiversidad} con la información referente a las acciones de la meta. 
    \UCpaso[\UCactor] Ingresa los datos requeridos en la pantalla \cdtIdRef{IUS 9}{Registrar avance de acciones de biodiversidad}. \label{cus9:RegAvance}
    \UCpaso[\UCactor] Oprime el botón \cdtButton{Aceptar} en la pantalla \cdtIdRef{IUS 9}{Registrar avance de acciones de biodiversidad}. \refTray{D}.
    \UCpaso[\UCsist] Verifica que la escuela se encuentre en estado ``Avance en edición''. \refTray{A}.
    \UCpaso[\UCsist] Verifica que la fecha actual se encuentre dentro del periodo definido por la SMAGEM para registrar avances. \refTray{E}.
    \UCpaso[\UCsist] Verifica que los datos ingresados proporcionen información correcta con base en la regla de negocio \cdtIdRef{RN-S1}{Información correcta}. \refTray{B}. \refTray{C}.  
    \UCpaso[\UCsist] Muestra el mensaje \cdtIdRef{MSG1}{Operación realizada exitosamente} en la pantalla \cdtIdRef{IUS 8}{Administrar avances de biodiversidad} indicando al actor que el avance se ha registrado exitosamente.
\end{UCtrayectoria}
 
\begin{UCtrayectoriaA}[Fin del caso de uso]{A}{El seguimiento no se encuentra en un estado que permita registrar avance en las acciones.}
    \UCpaso[\UCsist] Muestra el mensaje \cdtIdRef{MSG28}{Operación no permitida por estado de la entidad} en la pantalla \cdtIdRef{IUS 8}{Administrar avances de biodiversidad} indicando al actor que no puede registrar avance para la acción debido a que el seguimiento no se encuentra en estado ``Edición''. 
\end{UCtrayectoriaA}
 
    \begin{UCtrayectoriaA}{B}{El actor ingresó un tipo de dato incorrecto.}    
    \UCpaso[\UCsist] Muestra el mensaje \cdtIdRef{MSG6}{Formato incorrecto} en la pantalla \cdtIdRef{IUS 9}{Registrar avance de acciones de biodiversidad} indicando que el registro del avance para la acción no puede realizarse debido a que la información ingresada no es correcta.
    \UCpaso[] Continúa con el paso \ref{cus9:RegAvance} de la trayectoria principal.     
    \end{UCtrayectoriaA}
    
    \begin{UCtrayectoriaA}{C}{El actor ingresó un dato que excede la longitud máxima.}    
    \UCpaso[\UCsist] Muestra el mensaje \cdtIdRef{MSG7}{Se ha excedido la longitud máxima del campo} en la pantalla \cdtIdRef{IUS 9}{Registrar avance de acciones de biodiversidad} indicando que el registro de los avances no puede realizarse debido a que la longitud del campo excede la longitud máxima definida.
    \UCpaso[] Continúa con el paso \ref{cus9:RegAvance} de la trayectoria principal.     
    \end{UCtrayectoriaA}
 
    \begin{UCtrayectoriaA}{D}{El actor desea cancelar la operación.}
      \UCpaso[\UCactor] Solicita cancelar la operación oprimiendo el botón \cdtButton{Cancelar} en la pantalla \cdtIdRef{IUS 9}{Registrar avance de acciones de biodiversidad}.
      \UCpaso[\UCsist] Regresa a la pantalla \cdtIdRef{IUS 8}{Administrar avances de biodiversidad}. 
    \end{UCtrayectoriaA}

    \begin{UCtrayectoriaA}[Fin del caso de uso]{E}{La fecha actual se encuentra fuera del periodo definido por la SMAGEM para el registro de avances.}
    \UCpaso[\UCsist] Muestra el mensaje \cdtIdRef{MSG41}{Acción fuera del periodo} en la pantalla \cdtIdRef{IUS 8}{Administrar avances de biodiversidad} indicando al actor que no puede registrar avances debido a que la fecha actual se encuentra fuera del periodo definido por la SMAGEM para realizar la acción. 
    \end{UCtrayectoriaA}