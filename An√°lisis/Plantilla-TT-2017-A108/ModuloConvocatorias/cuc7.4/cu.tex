\begin{UseCase}{CUAC-7.4}{Gestionar convocatoria de movilidad}{
		Permite al \cdtRef{actor:CIEUpis}{Responsable UPIS} realizar las operaciones pertinentes para tener el control de las convocatorias de móvilidad. Para poder realizar dichas operaciones el usuario debió haber ingresado al sistema con su usuario y contraseña, previamiente. \\
		El alumno podrá consultar los procesos de movilidad por periodo escolar con la información gestionada mediante este caso de uso.
			
		}
		\UCitem{Versión}{1.0}
		\UCccsection{Administración de Requerimientos}
		\UCitem{Autor}{Ivo Sebastián Sam Álvarez-Tostado}
		\UCccitem{Evaluador}{Ulises Velez Saldaña}
		\UCitem{Operación}{Consulta}
		\UCccitem{Prioridad}{Alta}
		\UCccitem{Complejidad}{Baja}
		\UCccitem{Volatilidad}{Baja}
		\UCccitem{Madurez}{Alta}
		\UCitem{Estatus}{Por revisar}
		\UCitem{Fecha del último estatus}{11 de Mayo del 2018}
			
		\UCsection{Atributos}
		\UCitem{Actor}{
			\begin{UClist} 
				\UCli \cdtRef{actor:CIEUpis}{Responsable UPIS}
			\end{UClist}
		}
		\UCitem{Propósito}{Proporcionar un medio por el cual se pueda realizar las operaciones de registrar, editar, eliminar y consultar, de las convocatorias de movilidad.}
		\UCitem{Entradas}{Ninguna
%			\begin{UClist}
%				\UCli Correo electrónico del actor
%				\UCli Contraseña del actor
%			\end{UClist}
		}
		\UCitem{Salidas}{
			\begin{UClist}
				\UCli Se muestra una tabla con las convocatorias de la sección internacional.
				\UCli Se muestra una tabla con las convocatorias de la sección nacional.
				\UCli Periodo y año de la convocatoria
			\end{UClist}
	
 }
		\UCitem{Precondiciones}{El actor deberá haber iniciado sesión en el sistema.}
		\UCitem{Postcondiciones}{
			\begin{UClist}
				\UCli Se podrá agregar, editar o eliminar convocatorias de movilidad o de cursos.
				\UCli Los alumnos podrán consultar las convocatorias de movilidad registradas.
			\end{UClist}	
	}
		\UCitem{Reglas de negocio}{Ninguna}
		\UCitem{Errores}{No aplica}
		\UCitem{Tipo}{Primario.}
	\end{UseCase}
	
	\begin{UCtrayectoria}
		\UCpaso[\UCactor] Solicita gestionar las convocatorias de movilidad tocando el icono \botMovilidad de la pantalla \cdtIdRef{IUPA}{Pantalla Principal Administrador}.
		\UCpaso [\UCsist] Obtiene las convocatorias de movilidad registradas. \label{CUAC-7.4:Obtener} \refTray{A}
		\UCpaso [\UCsist] Muestra la pantalla \cdtIdRef{IUAC-7.4}{Gestión de convocatorias de movilidad} con las convocatorias de movilidad seccionadas por tipo de convocatoria y ordenadas de manera cronológica ascendente.
		\UCpaso [\UCactor]  \label{CUAC-7.4:Agregar} Gestiona las convocatorias mediante el uso del botón \cdtButton{Registrar convocatoria}. \refTray{B}
		
	\end{UCtrayectoria}

	\begin{UCtrayectoriaA}{A}{No existen convocatorias de movilidad registradas en el sistema.}
	\UCpaso[\UCsist] Construye la tabla de la pantalla  \cdtIdRef{IUAC-7.4}{Gestionar convocatorias de movilidad} con los datos vacíos.
	\UCpaso[] Continua en el paso \ref{CUAC-7.4:Obtener} de la trayectoria principal.
\end{UCtrayectoriaA}
	
	\begin{UCtrayectoriaA}[Fin del caso de uso]{B}{Cuando el actor requiere volver a la pantalla anterior.}
		\UCpaso[\UCactor] Solicita regresar a la pantalla anterior tocando el botón \cdtButton{Atrás}.
		
		\UCpaso [\UCsist] Regrasa a la pantalla anterior.
	\end{UCtrayectoriaA}
	
	\subsection{Puntos de extensión}
	
	\UCExtensionPoint
	{El actor requiere agregar una nueva convocatoria de movilidad}
	{ Paso \ref{CUAC-7.4:Agregar} de la trayectoria principal}
	{\cdtIdRef{CUAC-7.5}{Registrar convocatoria de movilidad}}
	
	\UCExtensionPoint
	{El actor requiere editar una convocatoria de movilidad}
	{ Paso \ref{CUAC-7.4:Obtener} de la trayectoria principal}
	{\cdtIdRef{CUAC-7.5}{Registrar convocatoria de movilidad}}
	