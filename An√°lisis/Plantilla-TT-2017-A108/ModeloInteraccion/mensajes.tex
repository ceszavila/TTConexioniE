%\DONE{} 
	En esta sección se describen los mensajes utilizados en el sistema. Los mensajes se refieren a todos
	aquellos avisos que el sistema muestra al actor a través de la pantalla debido a diversas
	razones, por ejemplo: informar acerca de algún fallo en el sistema o para notificar acerca de alguna operación importante sobre la información.

%===========================================================removiendo los puntos generados
\subsection{Parámetros comunes}
    Cuando un mensaje es recurrente se parametrizan sus elementos, por ejemplo los mensajes: ``Aún no existen registros de salones en el sistema.'', ``Aún no existen registros de profesores en el sistema.'', 
    ``Aún no existen registros de unidades de aprendizaje en el sistema.'', tienen una estructura similar 
    por lo que para definir el mensaje se utilizan parámentros, con el objetivo de que el mensaje sea genérico y  
    pueda utilizarse en todos los casos que se considere necesario.\\

\subsection{Mensajes a través de la pantalla}

%===========================  MSG1 ==================================
\begin{mensaje}{MSG1}{Operación realizada exitosamente}{Confirmación}
    \item[Ubicación:] Formulario. Se muestra en la parte superior del formulario.
    \item[Estatus:] Terminado
    \item[Objetivo:] Notificar al actor que la acción solicitada fue realizada exitosamente.
    \item[Redacción:] DETERMINADO ENTIDAD VALOR ha sido OPERACIÓN exitosamente.
    \item[Parámetros:] El mensaje se muestra con base en los siguientes parámetros:
    \begin{Citemize}
	\item DETERMINADO ENTIDAD: Es un artículo determinado más el nombre de la entidad sobre la cual se realizó la acción.
	\item VALOR: Es el valor asignado al atributo de la entidad, generalmente es el nombre o la clave.
	\item OPERACIÓN: Es la acción que el actor solicitó realizar.
    \end{Citemize}
    \item[Ejemplo:] El salón ha sido registrado exitosamente.
\end{mensaje}

%%============================== MSG2 =================================
%\begin{mensaje}{MSG2}{No existe información registrada por el momento}{Notificación}
%    \item[Ubicación:] Formulario. Se muestra dentro de la tabla de gestión.
%    \item[Estatus:] Terminado
%    \item[Objetivo:] Notificar al actor que aún no existe información registrada en el sistema.
%    \item[Redacción:] Aún no existen registros de ENTIDAD en el sistema.
%    \item[Parámetros:] El mensaje se muestra con base en los siguientes parámetros:
%    \begin{Citemize}
%	\item ENTIDAD: Especifica la entidad sobre la que se está realizando la consulta.
%    \end{Citemize}
%    \item[Ejemplo:] Aún no existen registros de {\em escuelas} en el sistema.
%\end{mensaje}

%%============================== MSG3 =================================
%\begin{mensaje}{MSG3}{Superficies del predio}{Notificación}
%    \item[Ubicación:] Formulario. Se muestra en la parte superior del formulario.
%    \item[Estatus:] Terminado
%    \item[Objetivo:] Indicar al usuario que la \cdtRef{escuela:superficieConstruida}{superficie total construida} debe ser menor o igual a la \cdtRef{escuela:superficieTotal}{superficie total del predio}.
%    \item[Redacción:] La superficie total construida debe ser menor o igual a la superficie total del predio.
%\end{mensaje}
%
%%============================== MSG4 =================================
%\begin{mensaje}{MSG4}{No se encontró información sustantiva}{Error}
%    \item[Ubicación:] Formulario. Se muestra en la parte superior del formulario.
%    \item[Estatus:] Terminado
%    \item[Objetivo:] Informar al actor que no se puede ejecutar la operación debido a que el sistema no tiene información base.
%    \item[Redacción:] Error, no se encontró información registrada en DETERMINADO-1 ENTIDAD-1, ..., DETERMINADO-N ENTIDAD-N. Favor de contactar al administrador del sistema.
%    \item[Parámetros:] El mensaje se muestra con base en los siguientes parámetros:
%    \begin{Citemize}
%	\item DETERMINADO ENTIDAD: Es un artículo determinado más el(los) nombre(s) de la(s) entidad(es) que no tienen información.
%    \end{Citemize}
%    \item[Ejemplo:] Error, no se encontró información registrada en {\em el catálogo de escuelas}. Favor de contactar al administrador del sistema.
%\end{mensaje}

%============================== MSG2 =================================
\begin{mensaje}{MSG2}{Falta un dato requerido para efectuar la operación solicitada}{Error}
    \item[Ubicación:] Campo. Se muestra debajo del campo.
    \item[Estatus:] Terminado
    \item[Objetivo:] Indicar al actor la omisión de algún dato requerido para realizar la operación solicitada.
    \item[Redacción:] Este campo es obligatorio.
\end{mensaje}

%============================== MSG3 =================================
\begin{mensaje}{MSG3}{Formato incorrecto}{Error}
    \item[Ubicación:] Campo. Se muestra debajo del campo.
    \item[Estatus:] Terminado
    \item[Objetivo:] Indicar al actor que el dato ingresado en alguno de los campos del formulario no cumple con el tipo de dato definido en el diccionario de datos.
    \item[Redacción:] El dato ingresado es incorrecto, favor de ingresar un dato válido.
\end{mensaje}

%%============================== MSG7 =================================
%\begin{mensaje}{MSG7}{Se ha excedido la longitud máxima del campo}{Error}
%    \item[Ubicación:] Campo. Se muestra debajo del campo.
%    \item[Estatus:] Terminado
%    \item[Objetivo:] Indicar al actor que el dato ingresado en alguno de los campos del formulario rebasa la longitud especificada en el diccionario de datos.
%    \item[Redacción:] El dato ingresado ha excedido la longitud máxima permitida. La longitud debe debe ser menor o igual a TAMAÑO caracteres.  
%    \item[Parámetros:] El mensaje se muestra con base en los siguientes parámetros:
%    \begin{Citemize}
%	\item TAMAÑO: Especifica la longitud del atributo definido en el diccionario de datos.
%    \end{Citemize}
%    \item[Ejemplo:] El dato ingresado ha excedido la longitud máxima permitida. La longitud debe ser menor o igual a {\em 30} caracteres.
%\end{mensaje}

%%============================== MSG8 =================================
%\begin{mensaje}{MSG8}{Registro repetido}{Error}
%    \item[Ubicación:] Formulario. Se muestra en la parte superior del formulario.
%    \item[Estatus:] Terminado
%    \item[Objetivo:] Informar al actor que ya existe un registro con los mismos datos.
%    \item[Redacción:] Error, ya se OPERACIÓN INDETERMINADO ENTIDAD con el mismo ATRIBUTO, favor de verificar.
%    \item[Parámetros:] El mensaje se muestra con base en los siguientes parámetros:
%    \begin{Citemize}
%	\item INDETERMINADO ENTIDAD: Es un artículo indeterminado más el nombre de la entidad sobre la cual se desea realizar la operación.
%	\item OPERACIÓN: Es la acción que el actor solicita realizar.
%    \end{Citemize}
%    \item[Ejemplo:] Error, ya se {\em inscribió una escuela} con la misma {\em clave de centro de trabajo}, favor de verificar.
%\end{mensaje}

%%============================== MSG9 ================================
%\begin{mensaje}{MSG9}{Error en la solicitud}{Notificación}
%	\item[Ubicación:] Correo. Es el contenido de un correo electrónico.
%	\item[Estatus:] Terminado
%	\item[Objetivo:] Informar al actor que ocurrió un error en los datos de la solicitud.
%	\item[Asunto:] Error en la solicitud de registro de SAEAR
%	\item[]{\bf Redacción:}\\
%			  Estimado(a) NOMBRE\\ \\
%			  Su solicitud ha presentado un error en: Coincidencia de datos, firma, sello o validez oficial.\\ \\
%			  Verifique sus datos y vuelva a registrarse.   \\ \\
%			  Atentamente\\
%			  Programa de Acreditación de Escuelas Ambientalmente Responsables\\
%			  Secretaría del Medio Ambiente\\
%			  Gobierno del Estado de México\\
%	\item[Parámetros:] El mensaje se muestra con base en los siguientes parámetros:
%	\begin{Citemize}
%		\item NOMBRE: Es el nombre completo del \cdtRef{coordinador}{Coordinador del programa}.
%	\end{Citemize}
%\end{mensaje}

%============================== MSG4 ================================
\begin{mensaje}{MSG4}{Confirmar la eliminación de un integrante de línea de acción}{Confirmación}
    \item[Ubicación:] Pantalla emergente. Se muestra en una pantalla emergente.
    \item[Estatus:] Terminado
    \item[Objetivo:] Indicar al actor que está a punto de eliminar un integrante de línea de acción y que se necesita su aprobación para ello.
    \item[Redacción:] Se eliminará un integrante de la línea de acción \textit{LÍNEA}. ¿Está seguro de que desea continuar? Esta acción no podrá deshacerse.
    \item[Parámetros:] El mensaje se muestra con base en los siguientes parámetros:
    \begin{Citemize}
	\item LÍNEA: Es la línea de acción de la cual se elimirá el integrante.
    \end{Citemize}
    \item[Ejemplo:] ¿Desea eliminar el salón seleccionado?.
\end{mensaje}

%============================== MSG5 ================================
\begin{mensaje}{MSG5}{Login Incorrecto}{Notificación}
	\item[Ubicación:] Pantalla emergente..
	\item[Estatus:] Terminado
	\item[Objetivo:] Informar al actor que ingresó un dato incorrecto en lo campos de correo y contraseña.
	\item[Asunto:] Ingreso de administrador
	\item[Redacción:] Usuario y/o contraseña incorrecta
\end{mensaje}

%===========================  MSG6 ==================================
\begin{mensaje}{MSG6}{Elementos No Disponibles}{Error}
	\item[Canal:] Sistema
	\item[Propósito:] Notificar al actor que no existen elementos en el sistema para mostrar.
	\item[Redacción:] ¡No hay $<ELEMENTOS>$ para mostrarte!
	\item[Parámetros:] ELEMENTOS: Información que se requiere para concluir un proceso
	\item[Ejemplo:] ¡No hay registros de profesores para mostrarte!
	
	
\end{mensaje}


%%============================== MSG12 ================================
%\begin{mensaje}{MSG12}{Entidad no encontrada}{Error}
%    \item[Ubicación:] Formulario. Se muestra en la parte superior del formulario.
%    \item[Estatus:] Terminado
%    \item[Objetivo:] Indicar al usuario que no existe una entidad asociada al atributo que ha ingresado.
%    \item[Redacción:] No existe INDETERMINADO ENTIDAD asociada a DETERMINADO ATRIBUTO ingresado. Favor de verificar. 
%    \item[Parámetros:] El mensaje se muestra con base en los siguientes parámetros:
%    \begin{Citemize}
%	    \item INDETERMINADO ENTIDAD: Es un artículo indeterminado más el nombre de la entidad sobre la cual se desea realizar la operación.
%	    \item DETERMINADO ATRIBUTO: Es un artículo determinado más el nombre del atributo de la entidad.
%    \end{Citemize}
%    \item[Ejemplo:] No existe {\em una cuenta} asociada a {\em el nombre de usuario} ingresado. Favor de verificar.
%    
%\end{mensaje}

%%============================== MSG13 ================================
%\begin{mensaje}{MSG13}{Error en formato de archivo}{Error} 
%	\item[Ubicación:] Formulario. Se muestra en la parte superior del formulario.
%	\item[Estatus:] Terminado
%	\item[Objetivo:] Indicar al actor que no se puede realizar la operación debido a que el archivo no cumple con el formato solicitado con base base en la regla de negocio \cdtIdRef{RN-N3}{Archivos permitidos en el sistema}.
%	\item[Redacción:] No se puede cargar el archivo. El formato del archivo debe ser \textbf{PDF} y tamaño máximo de \textbf{2 MB}.
%\end{mensaje}

%%============================== MSG14 ================================
%
%\begin{mensaje}{MSG14}{Error en formato de clave de centro de trabajo}{Error}
%	\item[Ubicación:] Formulario. Se muestra en la parte superior del formulario.
%	\item[Estatus:] Terminado
%	\item[Objetivo:] Indicar al actor que la clave de centro de trabajo ingresada no cumple con el formato especificado.
%	\item[Redacción:] La clave de centro de trabajo ingresada es incorrecta, favor de ingresar una clave con el siguiente formato: 
%	    número de dos dígitos, tres letras, número de cuatro dígitos y una letra. Por ejemplo 15DPR2497K.
%\end{mensaje}

%%============================== MSG15 ================================
%
%\begin{mensaje}{MSG15}{Acudir a la DGAIR en caso de inconsitencias en la información}{Notificación}
%	\item[Ubicación:] Pantalla emergente. Se muestra en una pantalla emergente.
%	\item[Estatus:] Terminado
%	\item[Objetivo:] Indicar al actor que si la información mostrada de la escuela es inconsistente, deberá acudir a la DGAIR.
%	\item[Redacción:] Si la información mostrada de la escuela cuenta con alguna inconsitencia, favor de referirse a la DGAIR 
%	      (Dirección General de Acreditación, Incorporación y Revalidación) para actualizar su información.
%\end{mensaje}
%
%%============================== MSG16 ================================
%\begin{mensaje}{MSG16}{Error en formato de correo electrónico}{Error}
%	\item[Ubicación:] Formulario. Se muestra en la parte superior del formulario.
%	\item[Estatus:] Terminado
%	\item[Objetivo:] Indicar al actor que el correo ingresado no cumple con el formato especificado.
%	\item[Redacción:] El correo electrónico ingresado es incorrecto, favor de introducir uno válido con el siguiente formato: correo@servidor.dominio
%\end{mensaje}
%
%%============================== MSG17 ================================
%\begin{mensaje}{MSG17}{Requisitos de inscripción}{Notificación}
%	\item[Ubicación:] Pantalla emergente. Se muestra en una pantalla emergente.
%	\item[Estatus:] Terminado
%	\item[Objetivo:] Indicar al actor los requisitos para inscribir su escuela.
%	\item[Redacción:] Para inscribir la escuela en el programa se requiere lo siguiente:
%	\begin{Citemize}
%		\item La escuela debe estar registrada ante la DGAIR (Dirección General de Acreditación,
%		      Incorporación y Revalidación).
%		\item Contar con el documento que avale el nombramiento del director en formato PDF.
%		\item Contar con la carta compromiso debidamente firmada y sellada en formato PDF.
%	\end{Citemize}
% \end{mensaje}
% 
%%============================== MSG18 ================================
%\begin{mensaje}{MSG18}{Error en la región}{Notificación}
%	\item[Ubicación:] Formulario. Se muestra en la parte superior del formulario.
%	\item[Estatus:] Terminado
%	\item[Objetivo:] Indicar al actor que la región seleccionada no está asociada con el municipio en el que se encuentra la escuela.
%	\item[Redacción:] La región seleccionada no está asociada con el municipio en el que se encuentra la escuela. Para actualizar
%		  su información, favor de referirse a la de DGAIR (Dirección General de Acreditación, Incorporación y Revalidación).
%\end{mensaje}
%
%%============================== MSG19 ================================
%\begin{mensaje}{MSG19}{Comité completo}{Notificación}
%    \item[Ubicación:] Formulario. Se muestra en la parte inferior del formulario, este mensaje es fijo.
%    \item[Estatus:] Terminado
%    \item[Objetivo:] Indicar al actor que ha registrado el máximo número de integrantes del comité.
%    \item[Redacción:] Ha registrado el máximo número de integrantes del comité.
%\end{mensaje}
%
%%============================== MSG20 ================================ PENDIENTE
%\begin{mensaje}{MSG20}{Registrar al menos un alumno}{Notificación}
%    \item[Ubicación:] Pantalla emergente. Se muestra en una pantalla emergente.
%    \item[Estatus:] Terminado
%    \item[Objetivo:] Indicar al actor que no es posible registrar al integrante en la línea de acción seleccionada debido a que esta
%	  requiere que el rol del integrante sea ``Alumno''.		      
%    \item[Redacción:] No es posible registrar al integrante en la línea de acción seleccionada debido a que esta requiere de al menos un integrante
%	  con rol de ``Alumno''.
%\end{mensaje}
%
%%============================== MSG21 ================================
%\begin{mensaje}{MSG21}{Confirmación de activación de cuenta}{Notificación}
%    \item[Ubicación:] Pantalla emergente. Se muestra en una pantalla emergente.
%    \item[Estatus:] Terminado
%    \item[Objetivo:] Informar al usuario que se ha activado su cuenta y que se le ha enviado al correo electrónico la información que necesita para iniciar sesión.
%    \item[Redacción:] Gracias por activar su cuenta. Se ha enviado a su correo electrónico la información necesaria para iniciar sesión.
%\end{mensaje}
%
%%============================== MSG22 ================================
%\begin{mensaje}{MSG22}{Nombre de usuario y/o contraseña incorrecto}{Error}
%    \item[Ubicación:] Formulario. Se muestra en la parte superior del formulario.
%    \item[Estatus:] Terminado
%    \item[Objetivo:] Indicar al actor que el nombre de usuario y/o contraseña son incorrectos.
%    \item[Redacción:] El nombre de usuario y/o contraseña son incorrectos. Favor de verificarlo.
%\end{mensaje}
%%============================== MSG23 ================================
%\begin{mensaje}{MSG23}{Integrante de línea de acción repetido}{Error}
%    \item[Ubicación:] Formulario. Se muestra en la parte superior del formulario.
%    \item[Estatus:] Terminado
%    \item[Objetivo:] Informar al actor que no puede registrar a este integrante debido a que ya pertenece al comité.
%    \item[Redacción:] No es posible registrar a este integrante debido a que ya es parte del comité.
%\end{mensaje}
%%============================== MSG24 ================================
%\begin{mensaje}{MSG24}{Verificación de correo electrónico}{Notificación}
%	\item[Ubicación:] Correo. Es el contenido de un correo electrónico.
%	\item[Estatus:] Terminado
%	\item[Objetivo:] Informar al actor que debe verificar su correo electrónico para completar el registro de su escuela en el Programa de Acreditación de Escuelas Ambientalmente Responsables.
%	\item[Asunto:] Verificación de correo electrónico de SAEAR
%	\item[]{\bf Redacción:}\\
%			  Estimado(a) NOMBRE\\ \\
%			  Para completar el registro de su escuela en el Programa de Acreditación de Escuelas Ambientalmente Responsables, verifique su correo electrónico mediante el siguiente enlace:\\ \\
%			  https://portal2.edomex.gob.mx/sma/paear/verificacion   \\ \\
%			  Cuenta con 5 días naturales para ratificar su correo, en caso de no verificarlo su registro será eliminado.\\ \\
%			  Atentamente\\
%			  Programa de Acreditación de Escuelas Ambientalmente Responsables\\
%			  Secretaría del Medio Ambiente\\
%			  Gobierno del Estado de México\\
%	\item[Parámetros:] El mensaje se muestra con base en los siguientes parámetros:
%	\begin{Citemize}
%		\item NOMBRE: Es el nombre completo del \cdtRef{coordinador}{Coordinador del programa}.
%	\end{Citemize}
%\end{mensaje}
%%============================== MSG25 ================================
%\begin{mensaje}{MSG25}{Envío de usuario y contraseña}{Notificación}
%    \item[Ubicación:] Correo. Es el contenido de un correo electrónico.
%    \item[Estatus:] Terminado
%    \item[Objetivo:] Informar al actor cuáles son los datos para iniciar sesión en el sistema.
%    \item[Asunto:] Información de inicio de sesión de SAEAR
%    \item[]{\bf Redacción:}\\
%	  Correo electrónico verificado exitosamente.\\ \\
%	  Usted ha completado el prerregistro. Los siguientes son sus datos de ingreso al sistema:\\ \\
%	  Nombre de usuario: USUARIO   \\
%	  Contraseña: CONTRASEÑA \\ \\
%	  Puede iniciar sesión en el siguuiente enlace:\\ \\
%	  https://portal2.edomex.gob.mx/sma/paear/sesion \\ \\
%	  Atentamente\\
%	  Programa de Acreditación de Escuelas Ambientalmente Responsables\\
%	  Secretaría del Medio Ambiente\\
%	  Gobierno del Estado de México\\
%    \item[Parámetros:] El mensaje se muestra con base en los siguientes parámetros:
%    \begin{Citemize}
%	    \item USUARIO: Es el nombre de usuario para ingresar al sistema.
%	    \item CONTRASEÑA: Es la clave para autenticarse en el sistema.
%    \end{Citemize}
%    \item[]{\bf Ejemplo:}\\
%	  Correo electrónico verificado exitosamente.\\ \\
%	  Usted ha completado el prerregistro. Los siguientes son sus datos de ingreso al sistema:\\ \\
%	  Nombre de usuario: {\em 15PPR35050}\\
%	  Contraseña: {\em $.C1V3K_+$}\\ \\
%	  Puede iniciar sesión en el siguuiente enlace:\\ \\
%	  https://portal2.edomex.gob.mx/sma/paear/sesion \\ \\
%	  Atentamente\\
%	  Programa de Acreditación de Escuelas Ambientalmente Responsables\\
%	  Secretaría del Medio Ambiente\\
%	  Gobierno del Estado de México\\
%\end{mensaje}
%%============================== MSG26 ================================
%\begin{mensaje}{MSG26}{El enlace para activación de cuenta ya no es válido}{Error}
%    \item[Ubicación:] Pantalla emergente. Se muestra en una pantalla emergente.
%    \item[Estatus:] Terminado
%    \item[Objetivo:] Informar al actor que no puede registrar a este integrante debido a que ya pertenece al comité.
%    \item[Redacción:] El enlace de activación no es válido debido a que no está vigente o ya ha sido activado.
%\end{mensaje}
%%============================== MSG27 ================================
%\begin{mensaje}{MSG27}{Cuenta no activada}{Error}
%    \item[Ubicación:] Formulario. Se muestra en la parte superior del formulario.
%    \item[Estatus:] Terminado
%    \item[Objetivo:] Informar al actor que su cuenta no está activada.
%    \item[Redacción:] Su cuenta no está activada.
%\end{mensaje}
%
%%============================== MSG28 =================================
%\begin{mensaje}{MSG28}{Operación no permitida por estado de la entidad}{Error}
%\item[Ubicación:] Formulario. Se muestra en la parte superior del formulario.
%\item[Estatus:] Terminado
%\item[Objetivo:] Indicar al actor que el estado de la entidad no permite realizar la operación.
% 	\item[Redacción:] La operación OPERACIÓN no se puede realizar sobre DETERMINADO ENTIDAD ya que no se encuentra en estado LISTA DE ESTADOS.
% 	\item[Parámetros:] El mensaje se muestra con base en los siguientes parámetros:
% 	\begin{Citemize}
%  		\item OPERACIÓN: Nombre de la acción que se desea realizar.
%		\item LISTA DE ESTADOS: Es la lista de los estatus separados por comillas.
%		\item DETERMINADO ENTIDAD: Es un artículo determinado más el nombre de la entidad sobre la cual se realiza la acción.
% 	\end{Citemize}
% 	\item[Ejemplo:] La operación {\em registrar información base para indicadores de agua} no se puede realizar sobre {\em la escuela} ya que no se encuentra en estado {\em Inscrita}.
%\end{mensaje}
%%============================== MSG29 =================================
%\begin{mensaje}{MSG29}{Registrar acciones}{Notificación}
%\item[Ubicación:] Pantalla emergente. Se muestra en una pantalla emergente.
%\item[Estatus:] Terminado
%\item[Objetivo:] Indicar al actor que es necesario registrar acciones asociadas a la meta para que esta se pueda considerar completa.
%\item[Redacción:] Recuerde que se deben registrar acciones asociadas a esta meta para que se considere completa.
%\end{mensaje}
%%============================== MSG30 =================================
%\begin{mensaje}{MSG30}{Confirmar la modificación de un registro}{Notificación}
%\item[Ubicación:] Pantalla emergente. Se muestra en una pantalla emergente.
%\item[Estatus:] Terminado
%\item[Objetivo:] Indicar al actor que está a punto de modificar un registro y se necesita su aprobación para ello.
%\item[Redacción:] Se modificará INDETERMINADO ENTIDAD. ¿Está seguro que desea continuar? Al modificar esta información se perderá la información previa.
%\item[Parámetros:] El mensaje se muestra con base en los siguientes parámetros:
%	\begin{Citemize}
%		\item INDETERMINADO ENTIDAD: Es un artículo indeterminado más el nombre de la entidad sobre la cual se realiza la acción.
%	\end{Citemize}
%	\item[Ejemplo:] Se modificará {\em la información base para indicadores}. ¿Está seguro que desea continuar? Al modificar esta información se perderá la información previa.
%\end{mensaje}
%%============================== MSG31 =================================
%\begin{mensaje}{MSG31}{Confirmar eliminación}{Confirmación}
%    \item[Ubicación:] Pantalla emergente. Se muestra en una pantalla emergente.
%    \item[Estatus:] Terminado
%    \item[Objetivo:] Indicar al actor que está a punto de eliminar el elemento seleccionado y que se necesita su aprobación para ello.
%    \item[Redacción:] Se eliminará el elemento seleccionado. ¿Está seguro de que desea continuar? Esta acción no podrá deshacerse.
%\end{mensaje}
%%============================== MSG32 =================================
%\begin{mensaje}{MSG32}{Registro exitoso de avance de meta}{Notificación}
%\item[Ubicación:] Pantalla emergente. Se muestra en una pantalla emergente.
%\item[Estatus:] Terminado
%\item[Objetivo:] Indicar al actor que el avance de la meta se ha registrado exitosamente y que es necesario registrar los avances de las acciones asociadas a esta.
%\item[Redacción:] El avance de la meta ha sido registrado exitosamente, recuerde que deben registrar avances de las acciones asociadas a esta meta.
%\end{mensaje}
%%============================== MSG33 =================================
%\begin{mensaje}{MSG33}{Unicidad de objetivos por línea de acción}{Notificación}
%\item[Ubicación:] Formulario. Se muestra en la parte superior del formulario.
%\item[Estatus:] Terminado
%\item[Objetivo:] Indicar al actor que no es posible registrar el objetivo debido a que la línea de acción seleccionada ya tiene un objetivo asociado.
%\item[Redacción:] Ya existe un objetivo asociado a la línea de acción seleccionada, favor de verificar.
%\end{mensaje}
%%============================== MSG34 =================================
%\begin{mensaje}{MSG34}{Confirmación de envío de información}{Notificación}
%\item[Ubicación:] Pantalla emergente. Se muestra en una pantalla emergente.
%\item[Estatus:] Terminado
%\item[Objetivo:] Indicar al actor que está a punto de enviar información y que se necesita su aprobación para ello.
%\item[Redacción:] ¿Está seguro de que desea enviar la información? Una vez que acepte, ya no podrá realizar cambios en ella.
%\end{mensaje}
%%============================== MSG35 =================================
%\begin{mensaje}{MSG35}{Registrar residuos sólidos}{Notificación}
%\item[Ubicación:] Formulario. Se muestra en la sección ``Información de los residuos sólidos'' debajo de la leyenda ``Administrar residuos sólidos''.
%\item[Estatus:] Terminado
%\item[Objetivo:] Indicar al actor que para registrar la cantidad de residuos sólidos debe utilizar el botón ``Registrar''.
%\item[Redacción:] Registre la cantidad de residuos sólidos a OPERACIÓN, mediante el botón ``Registrar''.
%\item[Parámetros:] El mensaje se muestra con base en los siguientes parámetros:
%\begin{Citemize}
%	\item OPERACIÓN: Es la operación que se realizará con los residuos sólidos.
%\end{Citemize}
%\item[]{\bf Ejemplo:}\\ 
%		Registre la cantidad de residuos sólidos a {\em reducir}, mediante el botón ``Registrar''.\\
%		Registre la cantidad de residuos sólidos a {\em reciclar}, mediante el botón ``Registrar''.
%\end{mensaje}
%%============================== MSG34 =================================
%\begin{mensaje}{MSG36}{Reducción de la generación o reciclaje de residuos}{Error}
%\item[Ubicación:] Formulario. Se muestra en la parte superior del formulario.
%\item[Estatus:] Terminado
%\item[Objetivo:] Indicar al actor que es necesario registrar al menos un residuo sólido para concluir el registro.
%\item[Redacción:] Es necesario que registre al menos un residuo sólido para completar el registro de la meta.
%\end{mensaje}
%
%%============================== MSG37 =================================
%\begin{mensaje}{MSG37}{Falló el envío de la información base}{Error}
%\item[Ubicación:] Formulario. Se muestra en la parte superior del formulario.
%\item[Estatus:] Terminado
%\item[Objetivo:] Indicar al actor que no puede enviar la información base porque no se ha completado el registro de ésta.
%\item[Redacción:] Es necesario que complete el registro de información base por cada una de las líneas de acción para enviar la información.
%\end{mensaje}
%%============================== MSG38 =================================
%\begin{mensaje}{MSG38}{Número de mejoras a realizar}{Error}
%\item[Ubicación:] Formulario. Es la etiqueta de uno de los campos.
%\item[Estatus:] Terminado
%\item[Objetivo:] Indicar al actor la información que debe de ingresar en el campo referente a la mejora seleccionada.
%\item[Redacción:] Indique el número de MEJORAS a realizar.
%\item[Parámetros:] El mensaje se muestra con base en los siguientes parámetros:
%\begin{Citemize}
%	\item MEJORAS: Es la mejora en plural que el actor seleccionó como ``tipo de mejora''.
%\end{Citemize}
%\item[]{\bf Ejemplo:}\\ 
%		Indique el número de {\em construcciones} a realizar.\\
%		Indique el número de {\em limpiezas} a realizar.
%\end{mensaje}
%%============================== MSG39 =================================
%\begin{mensaje}{MSG39}{La información ya ha sido enviada}{Error}
%\item[Ubicación:] Formulario. Se muestra en la parte superior del formulario.
%\item[Estatus:] Terminado
%\item[Objetivo:] Indicar al actor que no puede enviar la información base para indicadores porque ya ha sido enviada.
%\item[Redacción:] DETERMINADO ENTIDAD ya ha sido enviada anteriormente.
%\item[Parámetros:] El mensaje se muestra con base en los siguientes parámetros:
%\begin{Citemize}
%	\item DETERMINADO ENTIDAD: Es un artículo determinado más el nombre de la entidad sobre la cual se desea realizar la acción.
%\end{Citemize}
%\item[Ejemplo:] {\em La información base para indicadores} ya ha sido enviada anteriormente.
%\end{mensaje}
%%============================== MSG40 =================================
%\begin{mensaje}{MSG40}{Falló el envío del plan de acción}{Error}
%\item[Ubicación:] Pantalla emergente. Se muestra en una pantalla emergente.
%\item[Estatus:] Terminado
%\item[Objetivo:] Indicar al actor que no puede enviar el plan de acción porque es necesario registrar al menos un objetivo, una meta y una acción.
%\item[Redacción:] Es necesario que registre al menos un objetivo, una meta y una acción.
%\end{mensaje}
%%============================== MSG41 =================================
%\begin{mensaje}{MSG41}{Acción fuera del periodo}{Error}
%\item[Ubicación:] Formulario. Se muestra en la parte superior del formulario.
%\item[Estatus:] Terminado
%\item[Objetivo:] Indicar al actor que no puede realizar la acción debido a que esta se encuentra fuera del periodo definido por la SMAGEM.
%    \item[Redacción:] DETERMINADO ACCIÓN no puede realizarse debido a que la fecha actual se encuentra fuera del periodo definido por la SMAGEM para realizar la acción.
%    \item[Parámetros:] El mensaje se muestra con base en los siguientes parámetros:
%    \begin{Citemize}
%	\item DETERMINADO : Es un artículo determinado.
%	\item OPERACIÓN: Es la acción que el actor solicitó realizar.
%    \end{Citemize}
%    \item[Ejemplo:] {\em El registro de información base} no puede realizarse debido a que la fecha actual se encuentra fuera del periodo definido por la SMAGEM para realizar la acción.
%\end{mensaje}
%%============================== MSG42 =================================
%\begin{mensaje}{MSG42}{No puede calcularse el indicador}{Notificación}
%\item[Ubicación:] Formulario. Se muestra dentro de la tabla de gestión.
%\item[Estatus:] Terminado
%\item[Objetivo:] Indicar al actor que el valor del indicador aún no puede calcularse debido a que no existe la información necesaria.
%    \item[Redacción:] Este valor aún no puede calcularse debido a que no se cuenta con la información necesaria.
%\end{mensaje}
%%============================== MSG43 =================================
%\begin{mensaje}{MSG43}{Falló el envío del seguimiento del plan de acción}{Error}
%\item[Ubicación:] Pantalla emergente. Se muestra en una pantalla emergente.
%\item[Estatus:] Terminado
%\item[Objetivo:] Indicar al actor que no puede enviar el seguimiento del plan de acción porque es necesario registrar al menos el avance de una meta y de una acción.
%\item[Redacción:] Es necesario que registre al menos el avance de una meta y de una acción.
%\end{mensaje}