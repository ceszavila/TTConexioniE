\section{Estudio de Factibilidad}

El estudio de factibilidad de cierta manera es un proceso de aproximaciones sucesivas, donde se define el problema por resolver. Para ello se parte de supuestos, pronósticos y estimaciones, por lo que el grado de preparación de la información y su confiabilidad depende de la profundidad con que se realicen tanto los estudios técnicos, como los económicos, financieros y de mercado, y otros que se requieran. En cada etapa deben precisarse todos aquellos aspectos y variables que puedan mejorar el proyecto, o sea optimizarlo. Puede suceder que del resultado del trabajo pudiera aconsejarse una revisión del proyecto original, que se postergue su iniciación considerando el momento óptimo de inicio e incluso lo anterior no debe servir de excusa para no evaluar proyectos. Por el contrario, con la preparación y evaluación será posible la reducción de la incertidumbre que provocarían las variaciones de los factores. \cite{16} \\

Existen tres puntos fundamentales a analizar para realizar el estudio de factibilidad, los cuales se irán desglosando en los siguientes párrafos:

\subsection{Económica}

	\begin{Citemize} 
	%	\hspace{1pt}{}
		\item Tiempo de análisis\\
			El análisis se hace por iterciones durante todo el tiempo del Trabajo Terminal, consideramos en total 18 semanas de análisis con una dedicación de 4 horas al día y cada semana constando de 5 días.\\
			
		\item Costo de estudio del sistema\\
			Tomando los datos anteriores y con base en el salario promedio para un analista de sistema en la Ciudad de México, siendo este de 15,358.00 M.N. mensuales. \cite{17} Nos da un costo de 87 M.N. por hora.\\
			Si tomamos los datos anteriores tenemos 360 horas trabajadas en el estudio del sistema por 87 pesos la hora, tiene un resultado total de 31,320.00 M.N. como costo del estudio del sistema.\\
			
		\item Costo del tiempo del personal\\
			En el desarrollo contamos con 26 semanas para la programación del proyecto. Tomando en cuenta el salario promedio de un recién egresado en Sistemas Computacionales que es de 79.00 M.N. por hora y se le dedicaron 4 horas al día. \cite{18} Tenemos que el costo total del tiempo de personal es de 41,080 M.N.\\
			
		\item Costo estimado de los equipos\\
		Se requerirán dos computadoras con las siguientes características:
			\begin{itemize}
				\item Procesador Intel Core i5 dual core de séptima generación de 2.3 GHz
				\item Turbo Boost de hasta 3.6 GHz
				\item 8 GB de memoria LPDDR3 de 2133 MHz
				\item Almacenamiento SSD de 128 GB
				\item Intel Iris Plus Graphics 640
				\item Dos puertos Thunderbolt 3
			\end{itemize}
		El costo por computadora es de 29,999.00 M.N y por las dos requeridas son 59,998 M.N.
		
		\item Costo del software\\
		El costo de la licencia anual de Apple para ser parte del programa de desarrollador es de 99 dólares. Al tipo de cambio actual son 1,802 M.N.\\
		
		El total de costos es de 134,200 M.N.
		
	\end{Citemize}
\subsection{Tecnológica}
	\begin{itemize}
		\item ¿Mejorará el sistema actual?\\
		En realidad no hay un sistema actual para comparar los módulos implementados y los propuestos en trabajo a futuro. Por lo que consideramos que ayudará a los alumnos de manera significativa a conocer más a fondo la Escuela
		\item ¿Existe la tecnología para satisfacer las necesidades del usuario?\\
		Existen diferentes medios de difusión que se pueden utilizar para satisfacer esta necesidad. En la actualidad hay una aplicación de consulta para el Politécnico. Ésta no cubre totalmente las necesidades del alumno como el trabajo que se propone en este documento.
	\end{itemize}

\subsection{Operativa}
	\begin{itemize}
	\item ¿El sistema operará luego de instalarse?\\
	Una vez terminanda, la aplicación estará disponible en la App Store de Apple. Todo usuario con sistema operativo iOS tendrá la posibilidad de descargarlo en cuanto esté disponible.
	\item ¿El sistema es necesario?\\
	La apicación satisface la necesidad de conocer la ubicación del salón o edificio al que desea dirigirse. Por parte de la información sobre trámites, certificaciones, etc., no hay una necesidad por satisfacer pues éstos ya cuentan con difusión, aunque sea complicada la forma de obtener esa información. 
	\item ¿Existe el recurso humano para operarlo?\\
	Consideramos que los usuarios administradores que tendrán acceso a ofrecer información para la aplicación se encuentran cubiertos y son profesores que se encuentran dentro de las escuelas del IPN.
\end{itemize}

