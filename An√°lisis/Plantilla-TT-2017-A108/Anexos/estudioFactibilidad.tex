\section{Estudio de Factibilidad}

El estudio de factibilidad de cierta manera es un proceso de aproximaciones sucesivas, donde se define el problema por resolver. Para ello se parte de supuestos, pronósticos y estimaciones, por lo que el grado de preparación de la información y su confiabilidad depende de la profundidad con que se realicen tanto los estudios técnicos, como los económicos, financieros y de mercado, y otros que se requieran. En cada etapa deben precisarse todos aquellos aspectos y variables que puedan mejorar el proyecto, o sea optimizarlo. Puede suceder que del resultado del trabajo pudiera aconsejarse una revisión del proyecto original, que se postergue su iniciación considerando el momento óptimo de inicio e incluso lo anterior no debe servir de excusa para no evaluar proyectos. Por el contrario, con la preparación y evaluación será posible la reducción de la incertidumbre que provocarían las variaciones de los factores. \cite{16} \\

Existen tres puntos fundamentales a analizar para realizar el estudio de factibilidad, los cuales se irán desglosando en los siguientes párrafos:

\subsection{Económica}

\subsection{Tecnológica}

\subsection{Operativa}
