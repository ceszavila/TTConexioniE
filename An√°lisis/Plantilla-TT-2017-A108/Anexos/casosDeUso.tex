En este capítulo se definen los casos de uso que se desarrollaron para esta etapa del trabajo terminal y comprenden los módulos de salones y profesores. De la misma manera se desarrolla y describe el estudio de factibilidad de la aplicación.
\newpage
\section{Casos de Uso}
%===========================================================
\subsection{Modelo de comportamiento del módulo:  Salones \label{chp:modeloComportamientoSalones}}
En este capítulo se describen los casos de uso referentes a la consulta de espacios con la finalidad de brindarles la ubicación de áreas en la Escuela Superior de Cómputo a los alumnos y visitantes. \bigskip

\begin{objetivos}[Elementos de un caso de uso]
	\item {\bf Resumen:} Descripción textual del caso de uso.
	\item {\bf Actores:} Lista de los actores que intervienen en el caso de uso.
	\item {\bf Propósito:} Una breve descripción del objetivo que busca el actor al ejecutar el caso de uso.
	\item {\bf Entradas:} Lista de los datos de entrada requeridos durante la ejecución del caso de uso.
	\item {\bf Salidas:} Lista de los datos de salida que presenta el sistema durante la ejecución del caso de uso.
	\item {\bf Precondiciones:} Descripción de las operaciones o condiciones que se deben cumplir previamente para que el caso de uso pueda ejecutarse correctamente.
	\item {\bf Postcondiciones:} Lista de los cambios que ocurrirán en el sistema después de la ejecución del caso de uso y de las consecuencias en el sistema.
	\item {\bf Reglas de negocio:} Lista de las reglas que describen, limitan o controlan algún aspecto del negocio del caso de uso.
	\item {\bf Errores:} Lista de los posibles errores que pueden surgir durante la ejecución del caso de uso.
	\item {\bf Trayectorias:} Secuencia de los pasos que ejecutará el caso de uso.
\end{objetivos}

\cfinput{ModuloSalones/cu3/cu}
\cfinput{ModuloSalones/cu2/cu}
\cfinput{ModuloSalones/cu1/cu}

%===========================================================
\subsection{Modelo de comportamiento del módulo:  Profesores \label{chp:modeloComportamientoProfesores}}

En este capítulo se describen los casos de uso referentes a la de información de los profesores que conforman la plantilla docente de la Escuela Superior de Cómputo. \bigskip

\begin{objetivos}[Elementos de un caso de uso]
	\item {\bf Resumen:} Descripción textual del caso de uso.
	\item {\bf Actores:} Lista de los actores que intervienen en el caso de uso.
	\item {\bf Propósito:} Una breve descripción del objetivo que busca el actor al ejecutar el caso de uso.
	\item {\bf Entradas:} Lista de los datos de entrada requeridos durante la ejecución del caso de uso.
	\item {\bf Salidas:} Lista de los datos de salida que presenta el sistema durante la ejecución del caso de uso.
	\item {\bf Precondiciones:} Descripción de las operaciones o condiciones que se deben cumplir previamente para que el caso de uso pueda ejecutarse correctamente.
	\item {\bf Postcondiciones:} Lista de los cambios que ocurrirán en el sistema después de la ejecución del caso de uso y de las consecuencias en el sistema.
	\item {\bf Reglas de negocio:} Lista de las reglas que describen, limitan o controlan algún aspecto del negocio del caso de uso.
	\item {\bf Errores:} Lista de los posibles errores que pueden surgir durante la ejecución del caso de uso.
	\item {\bf Trayectorias:} Secuencia de los pasos que ejecutará el caso de uso.
\end{objetivos}
\cfinput{ModuloProfesores/cu1/cu}
\cfinput{ModuloProfesores/cu2/cu}
%===========================================================
\subsection{Modelo de comportamiento del módulo: Unidades de Aprendizaje \label{chp:modeloComportamientoUnidadesDeAprendizaje}}

En este capítulo se describen los casos de uso referentes a la consulta  de información de las unidades de aprendizaje que se encuentran disponible por semestre en la Escuela Superior de Cómputo. \bigskip

\begin{objetivos}[Elementos de un caso de uso]
	\item {\bf Resumen:} Descripción textual del caso de uso.
	\item {\bf Actores:} Lista de los actores que intervienen en el caso de uso.
	\item {\bf Propósito:} Una breve descripción del objetivo que busca el actor al ejecutar el caso de uso.
	\item {\bf Entradas:} Lista de los datos de entrada requeridos durante la ejecución del caso de uso.
	\item {\bf Salidas:} Lista de los datos de salida que presenta el sistema durante la ejecución del caso de uso.
	\item {\bf Precondiciones:} Descripción de las operaciones o condiciones que se deben cumplir previamente para que el caso de uso pueda ejecutarse correctamente.
	\item {\bf Postcondiciones:} Lista de los cambios que ocurrirán en el sistema después de la ejecución del caso de uso y de las consecuencias en el sistema.
	\item {\bf Reglas de negocio:} Lista de las reglas que describen, limitan o controlan algún aspecto del negocio del caso de uso.
	\item {\bf Errores:} Lista de los posibles errores que pueden surgir durante la ejecución del caso de uso.
	\item {\bf Trayectorias:} Secuencia de los pasos que ejecutará el caso de uso.
\end{objetivos}
\cfinput{ModuloUnidades/cuu5_4/cu}
\cfinput{ModuloUnidades/cuu5_4_1/cu}


\subsection{Modelo de comportamiento del módulo de administrador: Unidades de Aprendizaje \label{chp:modeloComportamientoUnidadesDeAprendizajeAdministrador}}

En este capítulo se describen los casos de uso referentes al registro, modificación y eliminación de información de las unidades de aprendizaje que se encuentran disponible por semestre en la Escuela Superior de Cómputo. \bigskip

\begin{objetivos}[Elementos de un caso de uso]
	\item {\bf Resumen:} Descripción textual del caso de uso.
	\item {\bf Actores:} Lista de los actores que intervienen en el caso de uso.
	\item {\bf Propósito:} Una breve descripción del objetivo que busca el actor al ejecutar el caso de uso.
	\item {\bf Entradas:} Lista de los datos de entrada requeridos durante la ejecución del caso de uso.
	\item {\bf Salidas:} Lista de los datos de salida que presenta el sistema durante la ejecución del caso de uso.
	\item {\bf Precondiciones:} Descripción de las operaciones o condiciones que se deben cumplir previamente para que el caso de uso pueda ejecutarse correctamente.
	\item {\bf Postcondiciones:} Lista de los cambios que ocurrirán en el sistema después de la ejecución del caso de uso y de las consecuencias en el sistema.
	\item {\bf Reglas de negocio:} Lista de las reglas que describen, limitan o controlan algún aspecto del negocio del caso de uso.
	\item {\bf Errores:} Lista de los posibles errores que pueden surgir durante la ejecución del caso de uso.
	\item {\bf Trayectorias:} Secuencia de los pasos que ejecutará el caso de uso.
\end{objetivos}
\cfinput{ModuloUnidades/cuau5/cu}
\cfinput{ModuloUnidades/cuau5_1/cu}
\cfinput{ModuloUnidades/cuau5_2/cu}
\cfinput{ModuloUnidades/cuau5_3/cu}

\subsection{Modelo de comportamiento del módulo de administrador: Profesores \label{chp:modeloComportamientoProfesoresAdminnistrador}}

En este capítulo se describen los casos de uso referentes al registro, modificación y eliminación de información de los profesores de la Escuela Superior de Cómputo.. \bigskip

\begin{objetivos}[Elementos de un caso de uso]
	\item {\bf Resumen:} Descripción textual del caso de uso.
	\item {\bf Actores:} Lista de los actores que intervienen en el caso de uso.
	\item {\bf Propósito:} Una breve descripción del objetivo que busca el actor al ejecutar el caso de uso.
	\item {\bf Entradas:} Lista de los datos de entrada requeridos durante la ejecución del caso de uso.
	\item {\bf Salidas:} Lista de los datos de salida que presenta el sistema durante la ejecución del caso de uso.
	\item {\bf Precondiciones:} Descripción de las operaciones o condiciones que se deben cumplir previamente para que el caso de uso pueda ejecutarse correctamente.
	\item {\bf Postcondiciones:} Lista de los cambios que ocurrirán en el sistema después de la ejecución del caso de uso y de las consecuencias en el sistema.
	\item {\bf Reglas de negocio:} Lista de las reglas que describen, limitan o controlan algún aspecto del negocio del caso de uso.
	\item {\bf Errores:} Lista de los posibles errores que pueden surgir durante la ejecución del caso de uso.
	\item {\bf Trayectorias:} Secuencia de los pasos que ejecutará el caso de uso.
\end{objetivos}

\cfinput{ModuloProfesores/cuap4/cu}
\cfinput{ModuloProfesores/cuap4_1/cu}
\cfinput{ModuloProfesores/cuap4_2/cu}
\cfinput{ModuloProfesores/cuap4_3/cu}

\section{Modelo de Interacción del Módulo Salones}
\cfinput{ModuloSalones/cu1/ui}
\cfinput{ModuloSalones/cu1/uiA}
\cfinput{ModuloSalones/cu2/ui}
\cfinput{ModuloSalones/cu2/uiA}
\cfinput{ModuloSalones/cu3/ui}
\section{Modelo de Interacción del Módulo Profesores}
\cfinput{ModuloProfesores/cu1/ui}
\cfinput{ModuloProfesores/cu2/ui}
\section{Modelo de Interacción del Módulo Unidades de Aprendizaje}
\cfinput{ModuloUnidades/cuu5_4/ui}
\cfinput{ModuloUnidades/cuu5_4_1/ui}



\section{Modelo de Interacción del Administrador del Módulo Salones}
\cfinput{ModuloUnidades/cuau5/uiA}
\cfinput{ModuloUnidades/cuau5/ui}
\cfinput{ModuloUnidades/cuau5_1/ui}
\cfinput{ModuloUnidades/cuau5_2/ui}

\section{Modelo de Interacción del Administrador del Módulo Profesores}
\cfinput{ModuloProfesores/cuap4/ui }
\cfinput{ModuloProfesores/cuap4_1/ui }
