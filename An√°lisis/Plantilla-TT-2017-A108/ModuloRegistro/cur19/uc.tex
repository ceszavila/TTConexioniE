\begin{UseCase}{CUR 19}{Administrar solicitudes de inscripción}
	{
		Este caso de uso tiene como objetivo mostrar al director del programa todas las escuelas que han solicitado su inscripción al programa y han enviado su información para ser revisada, el actor podrá acceder a la opción de aprobar solicitud para determinar si se acepta o se rechaza alguna solicitud.

    }
    \UCitem{Versión}{1.0}
    \UCccsection{Administración de Requerimientos}
    \UCitem{Autor}{Angélica Madrid Jiménez}
    \UCccitem{Evaluador}{José David Ortega Pacheco}
    \UCitem{Operación}{Consulta}
    \UCccitem{Prioridad}{Alta}
    \UCccitem{Complejidad}{Baja}
    \UCccitem{Volatilidad}{Baja}
    \UCccitem{Madurez}{Alta}
    \UCitem{Estatus}{Terminado}
    \UCitem{Fecha del último estatus}{24 de octubre de 2014}
    
%% Copie y pegue este bloque tantas veces como revisiones tenga el caso de uso.
%% Esta sección la debe llenar solo el Revisor
% %--------------------------------------------------------
% 	\UCccsection{Revisión Versión XX} % Anote la versión que se revisó.
% 	% FECHA: Anote la fecha en que se terminó la revisión.
% 	\UCccitem{Fecha}{Fecha en que se termino la revisión} 
% 	% EVALUADOR: Coloque el nombre completo de quien realizó la revisión.
% 	\UCccitem{Evaluador}{Nombre de quien revisó}
% 	% RESULTADO: Coloque la palabra que mas se apegue al tipo de acción que el analista debe realizar.
% 	\UCccitem{Resultado}{Corregir, Desechar, Rehacer todo, terminar.}
% 	% OBSERVACIONES: Liste los cambios que debe realizar el Analista.
% 	\UCccitem{Observaciones}{
% 		\begin{UClist}
% 			% PC: Petición de Cambio, describa el trabajo a realizar, si es posible indique la causa de la PC. Opcionalmente especifique la fecha en que considera razonable que se deba terminar la PC. No olvide que la numeración no se debe reiniciar en una segunda o tercera revisión.
% 			\RCitem{PC1}{\TODO{Descripción del pendiente}}{Fecha de entrega}
% 			\RCitem{PC2}{\TODO{Descripción del pendiente}}{Fecha de entrega}
% 			\RCitem{PC3}{\TODO{Descripción del pendiente}}{Fecha de entrega}
% 		\end{UClist}	
% 	}
% %--------------------------------------------------------

%--------------------------------------------------------
	\UCccsection{Revisión Versión 0.1} % Anote la versión que se revisó.
	% FECHA: Anote la fecha en que se terminó la revisión.
	\UCccitem{Fecha}{28 de Octubre del 2014} 
	% EVALUADOR: Coloque el nombre completo de quien realizó la revisión.
	\UCccitem{Evaluador}{Victor Lozano Ortega}
	% RESULTADO: Coloque la palabra que mas se apegue al tipo de acción que el analista debe realizar.
	\UCccitem{Resultado}{Corregir}
	% OBSERVACIONES: Liste los cambios que debe realizar el Analista.
	\UCccitem{Observaciones}{
		\begin{UClist}
			% PC: Petición de Cambio, describa el trabajo a realizar, si es posible indique la causa de la PC. Opcionalmente especifique la fecha en que considera razonable que se deba terminar la PC. No olvide que la numeración no se debe reiniciar en una segunda o tercera revisión.
			\RCitem{PC1}{\TODO{En el resumen quitar la liga al actor, solo debe tener la liga en la sección de actores}}{Fecha de entrega}
			\RCitem{PC2}{\TODO{En el resumen, ``que envían su información para ser revisada'' yo pondría ``que han enviado su información...''}}{Fecha de entrega}
			\RCitem{PC3}{\TODO{Agregar punto final a la sección de entradas}}{Fecha de entrega}
			\RCitem{PC4}{\TODO{Actores: revisar la liga, manda a otro actor que no es el referenciado}}{Fecha de entrega}
			\RCitem{PC5}{\TODO{En las salidas, separar ``Centro de trabajo,Nombre''}}{Fecha de entrega}
			\RCitem{PC6}{\TODO{Al mensaje en las salidas, indicar en que pantalla se muestra}}{Fecha de entrega}
			\RCitem{PC7}{\TODO{Punto final a las pre y postcondiciones}}{Fecha de entrega}
			\RCitem{PC8}{\TODO{Paso 2 de la trayectoria principal, agregar referencia al estado}}{Fecha de entrega}
			\RCitem{PC9}{\TODO{Revisar los puntos finales, faltan en secciones de la descripción del caso de uso y en la trayectoria}}{Fecha de entrega}
			\RCitem{PC10}{\TODO{En el objetivo de la pantalla, agregar la referencia al estado}}{Fecha de entrega}
			\RCitem{PC11}{\TODO{En el caso de uso, faltan los puntos de extensión, solo es uno al CUR20 pero hay que agregarle}}{Fecha de entrega}
			\RCitem{PC12}{\TODO{En la pantalla, revisar la referencia al actor, esta rota la liga}}{Fecha de entrega}
			\RCitem{PC13}{\TODO{En la pantalla, falta la sección de mensajes}}{Fecha de entrega}
			\RCitem{PC14}{\TODO{Revisar las ligas de las referencias, en las trayectorias refieren al cur1}}{Fecha de entrega}
		\end{UClist}	
	}
%--------------------------------------------------------

    \UCsection{Atributos}
    \UCitem{Actores}{\cdtRef{actor:usuarioSMAGEM}{Director del programa}}
    \UCitem{Propósito}{Administrar las solicitudes de inscripción de escuelas a través de una tabla de resultados con la opción de aprobar dichas solicitudes.}
    \UCitem{Entradas}{
	Ninguna.
	}
    \UCitem{Salidas}{
		 \begin{UClist}
	    \UCli {Tabla que muestra los siguientes datos: }{\cdtRef{escuela:cct}{Clave de centro de trabajo}, \cdtRef{escuela:nombreEscuela}{Nombre de la escuela} y \cdtRef{escuela:municipio}{Municipio}}.
	    \UCli \cdtIdRef{MSG2}{No existe información registrada por el momento}: Se muestra el mensaje en la pantalla \cdtIdRef{IUR 19}{Administrar solicitudes de inscripción} cuando no existe alguna solicitud de inscripción en el sistema.
	 \end{UClist}
    }
    \UCitem{Precondiciones}{
    	Ninguna.
    }
    \UCitem{Postcondiciones}{
	\begin{UClist}
        \UCli {\bf Interna:} Se podrá aprobar la solicitud de inscripción de una escuela por medio del caso de uso. \cdtIdRef{CUR 20}{Aprobar solicitud de inscripción}.
	\end{UClist}
    }
   \UCitem{Tipo}{Primario.}
	\UCitem{Fuente}{
	    \begin{UClist}
        \UCli Minuta de la reunión \cdtIdRef{M-3TR}{Toma de requerimientos}.
	    \end{UClist}
}
\end{UseCase}

 \begin{UCtrayectoria}
    \UCpaso[\UCactor] Solicita administrar las solicitudes de inscripción de escuelas seleccionando la opción ``Administrar solicitudes de inscripción'' del menú \cdtIdRef{MN1}{Menú del Director del programa}. 
    \UCpaso[\UCsist] Busca los datos de todas las escuelas que se encuentren en estado \cdtRef{estado:aprobar}{por aprobar} en el sistema. \refTray{A}
    \UCpaso[\UCsist] Muestra la información de las escuelas en la pantalla \cdtIdRef{IUR 19}{Administrar solicitudes de inscripción}
    \UCpaso[\UCactor] Administra a las escuelas a través del botón \botOk  \label{cur19:Acciones}
 \end{UCtrayectoria}
 
  \begin{UCtrayectoriaA}[Fin de caso de uso] {A}{No hay registros de escuelas para mostrar}
    \UCpaso[\UCsist] Muestra el mensaje \cdtIdRef{MSG2}{No existe información registrada por el momento} en la pantalla \cdtIdRef{IUR 19}{Administrar solicitudes de inscripción} indicando que aún no hay escuelas en estado \cdtRef{estado:aprobar}{por aprobar}.
 \end{UCtrayectoriaA}
 
 \subsection{Puntos de extensión}

		\UCExtensionPoint
		{El actor desea aprobar la solicitud de inscripción de una escuela}
		{ Paso \ref{cur19:Acciones} de la trayectoria principal}
		{\cdtIdRef{CUR 20}{Aprobar solicitud de inscripción}}