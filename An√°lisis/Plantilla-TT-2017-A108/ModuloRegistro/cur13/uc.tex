\begin{UseCase}{CUR 13}{Administrar integrantes de líneas de acción}
	{
	  Este caso de uso permite administrar a los integrantes de las líneas de acción mediante acciones de registro, modificación, visualizar y eliminar a los mismos. 
	   %Este caso de uso tiene como objetivo mostrar al actor los integrantes de las líneas de acción registrados en el sistema, desde el cual podrá acceder a las opciones  para registrar un integrante de línea de acción, modificar o visualizar la información de un integrante previamente registrado, así como eliminar un integrante del sistema.
	}

	\UCitem{Versión}{1.0}
	\UCccsection{Administración de Requerimientos}
	\UCitem{Autor}{Angélica Madrid Jiménez}
	\UCccitem{Evaluador}{José David Ortega Pacheco}
	\UCitem{Operación}{Administración}
	\UCccitem{Prioridad}{Alta}
	\UCccitem{Complejidad}{Baja}
	\UCccitem{Volatilidad}{Baja}
	\UCccitem{Madurez}{Alta}
    \UCitem{Estatus}{Terminado}
    \UCitem{Fecha del último estatus}{5 de Noviembre del 2014}

%% Copie y pegue este bloque tantas veces como revisiones tenga el caso de uso.
%% Esta sección la debe llenar solo el Revisor
% %--------------------------------------------------------
 	\UCccsection{Revisión Versión XX} % Anote la versión que se revisó.
% 	% FECHA: Anote la fecha en que se terminó la revisión.
 	\UCccitem{Fecha}{24 de octubre de 2014} 
% 	% EVALUADOR: Coloque el nombre completo de quien realizó la revisión.
 	\UCccitem{Evaluador}{José David Ortega Pacheco}
% 	% RESULTADO: Coloque la palabra que mas se apegue al tipo de acción que el analista debe realizar.
 	\UCccitem{Resultado}{Corregir, y terminar.}
% 	% OBSERVACIONES: Liste los cambios que debe realizar el Analista.
 	\UCccitem{Observaciones}{
 		\begin{UClist}
% 			% PC: Petición de Cambio, describa el trabajo a realizar, si es posible indique la causa de la PC. Opcionalmente especifique la fecha en que considera razonable que se deba terminar la PC. No olvide que la numeración no se debe reiniciar en una segunda o tercera revisión.
 			\RCitem{PC1}{\TOCHK{Analizar tres escenario: 1. Cuando no hay registros, 2. cuando no se tienen 12 integrantes de líneas de acción y cuando se tienen los 12 integrantes de líneas de acción }}{24 de octubre de 2014}
 			\RCitem{PC3}{\TOCHK{Las imágenes que describen los tres escenarios son IUR13, IUR13-1 e IUR13-2, las cuales ya están en la carpeta images }}{24 de octubre de 2014}
 			\RCitem{PC2}{\TOCHK{En los tres escenarios descritos anteriormente, describir cuando se habilita y se deshabilita el botón de registro }}{24 de octubre de 2014}

 		\end{UClist}		
 	}
% %--------------------------------------------------------

	\UCsection{Atributos}
	\UCitem{Actor(es)}{\cdtRef{actor:usuarioEscuela}{Coordinador del programa}}
	\UCitem{Propósito}{Permitir la administración de los integrantes de las líneas de acción, teniendo la opción de registrarlos, eliminarlos, modificarlos o visualizarlos.}
	\UCitem{Entradas}{Ninguna.}
	\UCitem{Salidas}{
	  \begin{UClist}
	    \UCli \ioTabla{\cdtRef{persona:nombre}{Nombre(s)},\cdtRef{persona:primerApellido}{ Primer apellido},\cdtRef{persona:segundoApellido}{ Segundo apellido},\cdtRef{integrante:lineaAccion}{ Línea de acción} y \cdtRef{integrante:rol}{Rol}}
	    \UCli \cdtIdRef{MSG2}{No existe información registrada por el momento}: Se muestra en la pantalla \cdtIdRef{IUR 13}{Administrar integrantes de líneas de acción} cuando no existe al menos un integrante registrado para alguna línea de acción.
	 \end{UClist}
	}
	% PRECONDICIÓN: Son sentencias intemporales y afirmativas que declaran lo que DEBE ser siempre verdadero antes de iniciar el escenario en el caso de uso. Las precondiciones no son probadas dentro del caso de uso, son condiciones que se asumen verdaderas. Una precondición puede implicar un escenario de otro Caso de Uso que se ha completado satisfactoriamente, como por ejemplo la ``autenticación'', o más general el ``cajero se identifica y se autentica''. Craig Larman ``Use Case Model: Writing Requirements in Context''. También pueden ser escenarios ajenos al sistema que el Actor debe contemplar durante la operación pero de las que el sistema no es consciente, por ejemplo: ``El alumno debe presentar su credencial vigente'', o ``contar con el expediente físico'' ``El vehículo a asegurar debe estar en buen estado''.
	% Especifique las precondiciones indicando si son internas (escenarios provenientes de otro caso de uso) o externas, referenciando para las internas el CU correspondiente y, en caso de que aplique, la Regla de negocio que se está Reforzando con esta precondición.
	\UCitem{Precondiciones}{
		\begin{UClist}
			\UCli {\bf Interna:} Que el actor haya iniciado sesión en el sistema.
			\UCli {\bf Interna:} La información escolar, el coordinador del programa y el responsable del programa deben estar previamente registrados en el sistema.
			\UCli {\bf Interna:} Que la escuela se encuentre en estado \cdtRef{estado:inscrita}{inscrita}.
			\UCli {\bf Interna:} Que el periodo de registro de escuelas se encuentre vigente.
		\end{UClist}
	}
	
	% POSTCONDICIONES: Son sentencias expresadas de manera intemporal y afirmativamente que exponen las garantías de exito o lo que DEBE ser verdadero cuando se completa exitosamente el caso de uso, sea a través de su escenario principal o a través de un flujo alternativo. La garantía debe cumplir las necesidades de todos los stakeholders.
	% Las postcondiciones en conjunto deben reflejar la condición de término del Caso de Uso y alcanzar el propósito planteado por el actor. También describe los cambios en la información y comportamiento del sistema. Indique los cambios que ocurrirán tanto dentro (Internas) como fuera (Externas) del sistema, referenciando los CU afectados por las Internas.
	\UCitem{Postcondiciones}{Ninguna.}
	\UCitem{Reglas de negocio}{Ninguna.}
	% ERRORES: Especifique los casos en los que no se podrá terminar satisfactoriamente el Caso de Uso. Contemple todos los catálogos o listas que deben tener almenos un dato para que se puedan seleccionar dentro de las pantallas asociadas al Caso de Uso.
	% Especifique: La descripción del error (condición), el comportamiento del sistema, y la forma en que el usuario se dará cuenta del error.
	\UCitem{Errores}{
	
	   \UCli \cdtIdRef{MSG28}{Operación no permitida por estado de la entidad}: Se muestra sobre la pantalla \cdtIdRef{IUR 13}{Administrar integrantes de lineas de acción} indicando al actor que no se puede administrar a los integrantes de línea de acción debido al estado en que se encuentra la escuela.
		\UCli	\cdtIdRef{MSG41}{Acción fuera del periodo}: Se muestra sobre la pantalla \cdtIdRef{IUR 5}{Administrar información escolar} para indicarle al actor que no puede administrar a los integrantes de línea de acción debido a que la fecha actual se encuentra fuera del periodo definido por la SMAGEM para realizar el registro de escuelas.	
	
	}

	\UCitem{Tipo}{Primario.}

		\UCitem{Fuente}{
	    \begin{UClist}
        \UCli Minuta de la reunión \cdtIdRef{M-3TR}{Toma de requerimientos}.
	    \end{UClist}
}

 \end{UseCase}

 \begin{UCtrayectoria}
      \UCpaso[\UCactor] Solicita administrar a los integrantes de las líneas de acción seleccionando la opción ``Información general'' del menú \cdtIdRef{MN2}{Menú del Coordinador del programa} y posteriormente la opción ``Integrantes de líneas de acción''. 
    \UCpaso[\UCsist] Verifica que la escuela se encuentre en estado ``Inscrita''.  \refTray{A}
    \UCpaso[\UCsist] Verifica que la fecha actual se encuentre dentro del periodo definido por la SMAGEM para administrar la información escolar. \refTray{B}
    \UCpaso[\UCsist] Busca los datos de los integrantes del comité registrados en el sistema. \refTray{C}
    \UCpaso[\UCsist] Muestra un listado en la pantalla \cdtIdRef{IUR 13}{Administrar integrantes de lineas de acción} de los integrantes de las líneas de acción ordenados por línea de acción de manera alfabética.
    \UCpaso[\UCactor] Administra a los integrantes de líneas de acción por medio de los botones \cdtButton{Registrar integrante}, \botEdit, \botV, \botKo  \label{cur13:Acciones}
 \end{UCtrayectoria}

%ALTERNAS 
 
  \begin{UCtrayectoriaA}[Fin del caso de uso]{A}{La escuela no se encuentra en el estado ``Inscrita''}
    \UCpaso[\UCsist] Muestra el mensaje \cdtIdRef{MSG28}{Operación no permitida por estado de la entidad} en la pantalla \cdtIdRef{IUR 13}{Administrar integrantes de líneas de acción} indicando al actor que no puede administrar a los integrantes de líneas de acción debido a que la escuela no se encuentra en estado ``Inscrita''.
 \end{UCtrayectoriaA}

 \begin{UCtrayectoriaA}[Fin del caso de uso]{B}{La fecha actual se encuentra fuera del periodo definido por la SMAGEM para el registro de escuelas}
    \UCpaso[\UCsist] Muestra el mensaje \cdtIdRef{MSG41}{Acción fuera del periodo} en la pantalla \cdtIdRef{IUR 13}{Administrar integrantes de líneas de acción} indicando al actor que no puede administrar a los integrantes de línea de acción debido a que la fecha actual se encuentra fuera del periodo definido por la SMAGEM para realizar la acción.
 \end{UCtrayectoriaA}
 
 \begin{UCtrayectoriaA}[Fin de caso de uso]{C}{El sistema no tiene datos registrados}
    \UCpaso[\UCsist] Muestra el mensaje \cdtIdRef{MSG2}{No existe información registrada por el momento} en la pantalla \cdtIdRef{IUR 13.1}{Administrar integrantes de líneas de acción} cuando no existe al menos un integrante registrado para alguna de las lineas de acción.
    \UCpaso[\UCactor] Registra a un integrante de línea de acción por medio del botón \cdtButton{Registrar integrante}.
 \end{UCtrayectoriaA}

\subsection{Puntos de extensión}

\UCExtensionPoint
{El actor desea agregar un nuevo integrante}
{ Paso \ref{cur13:Acciones} de la trayectoria principal}
{\cdtIdRef{CUR 14}{Registrar integrante de línea de acción}}

\UCExtensionPoint
{El actor desea editar la información de un integrante}
{ Paso \ref{cur13:Acciones} de la trayectoria principal}
{\cdtIdRef{CUR 15}{Modificar integrante de línea de acción}}

\UCExtensionPoint
{El actor desea eliminar un integrante del comité}
{ Paso \ref{cur13:Acciones} de la trayectoria principal}
{\cdtIdRef{CUR 16}{Eliminar integrante de línea de acción}}
