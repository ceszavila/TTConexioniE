\subsection{IUR 13 Administrar integrantes de líneas de acción}

\subsubsection{Objetivo}

% Explicar el objetivo para el que se construyo la interfaz, generalmente es la descripción de la actividad a desarrollar, como Seleccionar grupos para inscribir materias de un alumno, controlar el acceso al sistema mediante la solicitud de un login y password de los usuarios, etc.
	
	En esta pantalla el actor \cdtRef{actor:usuarioEscuela}{Coordinador del programa} puede conocer a los integrantes de líneas de acción registrados y sirve como punto de acceso para registrarlos en caso de ser necesario, así como modificarlos, visualizarlos o eliminarlos.

\subsubsection{Diseño}

% Presente la figura de la interfaz y explique paso a paso ``a manera de manual de usuario'' como se debe utilizar la interfaz. No olvide detallar en la redacción los datos de entradas y salidas. Explique como utilizar cada botón y control de la pantalla, para que sirven y lo que hacen. Si el Botón lleva a otra pantalla, solo indique la pantalla y explique lo que pasará cuando se cierre dicha pantalla (la explicación sobre el funcionamiento de la otra pantalla estará en su archivo correspondiente).

	En la figura~\ref{IUR 13} se muestra el caso en el que no hay registro de algún integrante de línea de acción en el sistema. Para este caso únicamente se permite el registro de integrantes por medio del botón \cdtButton{Registrar integrante}.\\
	
	\IUfig[.7]{pantallas/registro/IUR13-1}{IUR 13}{Administrar integrantes de líneas de acción}

	En la figura~\ref{IUR 13.1} se muestra la pantalla ``Administrar integrantes de líneas de acción'' por medio de la cual se podrán administrar los integrantes de las líneas de acción a través de una tabla de resultados en la cual se muestran el nombre o nombres de los integrantes, el primer apellido, el segundo apellido, línea de acción y rol.\\
	
	\IUfig[.8]{pantallas/registro/IUR13}{IUR 13.1}{Administrar integrantes de líneas de acción (Con registros)}
	
	En la figura~\ref{IUR 13.2} se muestra el caso en que se ha registrado el número máximo de integrantes del comité, el botón \cdtButton{Registrar integrante} queda deshabilitado y en su lugar aparece la leyenda ``Ya se han registrado a todos los integrantes de las líneas de acción''.

	\IUfig[.9]{pantallas/registro/IUR13-2}{IUR 13.2}{Administrar integrantes de líneas de acción (Comité completado)}


\subsubsection{Comandos}
\begin{itemize}
	\item \cdtButton{Registrar}: Permite al actor registrar a un integrante de línea de acción en caso de no existir o no haber completado el comité, dirige a la pantalla \cdtIdRef{IUR 14}{Registrar integrante de línea de acción}
	\item \botEdit[Modificar integrante]: Permite modificar un registro, dirige a la pantalla \cdtIdRef{IUR 15}{Modificar integrante de línea de acción} 
	\item \botV[Visualizar integrante]: Permite visualizar un integrante de línea de acción, dirige a la pantalla \cdtIdRef{IUR 17}{Visualizar integrante de línea de acción}
	\item \botKo[Eliminar integrante]: Permite eliminar a un integrante de línea de acción.
\end{itemize}

\subsubsection{Mensajes}

	
\begin{description}
	\item[\cdtIdRef{MSG2}{No existe información registrada por el momento}:] Se muestra en la pantalla \cdtIdRef{IUR 13}{Administrar integrantes de líneas de acción} cuando no existe al menos un integrante registrado para alguna de las lineas de acción.
\end{description}
