\begin{UseCase}{CUR 6}{Completar información escolar}
    {
      Este caso de uso permite al actor completar la información escolar. Una vez que los datos solicitados han sido ingresados, el sistema valida y registra la información.
    }
    \UCitem{Versión}{1.0}
    \UCccsection{Administración de Requerimientos}
    \UCitem{Autor}{Victor Lozano Ortega}
    \UCccitem{Evaluador}{José David Ortega Pacheco}
    \UCitem{Operación}{Registrar}
    \UCccitem{Prioridad}{Alta}
    \UCccitem{Complejidad}{Baja}
    \UCccitem{Volatilidad}{Baja}
    \UCccitem{Madurez}{Alta}
    \UCitem{Estatus}{Terminado}
    \UCitem{Fecha del último estatus}{5 de Noviembre del 2014}

%% Copie y pegue este bloque tantas veces como revisiones tenga el caso de uso.
%% Esta sección la debe llenar solo el Revisor
% %--------------------------------------------------------
%   \UCccsection{Revisión Versión XX} % Anote la versión que se revisó.
%   % FECHA: Anote la fecha en que se terminó la revisión.
%   \UCccitem{Fecha}{Fecha en que se termino la revisión} 
%   % EVALUADOR: Coloque el nombre completo de quien realizó la revisión.
%   \UCccitem{Evaluador}{Nombre de quien revisó}
%   % RESULTADO: Coloque la palabra que mas se apegue al tipo de acción que el analista debe realizar.
%   \UCccitem{Resultado}{Corregir, Desechar, Rehacer todo, terminar.}
%   % OBSERVACIONES: Liste los cambios que debe realizar el Analista.
%   \UCccitem{Observaciones}{
%     \begin{UClist}
%       % PC: Petición de Cambio, describa el trabajo a realizar, si es posible indique la causa de la PC. Opcionalmente especifique la fecha en que considera razonable que se deba terminar la PC. No olvide que la numeración no se debe reiniciar en una segunda o tercera revisión.
%       \RCitem{PC1}{\TOCHK{En Resumen, ESTE CASO DE USO PERMITE COMPLETAR LA INFORMACIÓN .... del pendiente}}{Fecha de entrega}
%       \RCitem{PC2}{\TOCHK{En Propósito: Comletar el registro de la información escolar}}{Fecha de entrega}
%       \RCitem{PC3}{\TOCHK{ACTUALIZAR LA PANTALLA CON LOS NUEVOS CAMBIOS}}{Fecha de entrega}
%       \RCitem{PC3}{\TOCHK{Salidas: revisar ortografía: Número con acento }}{Fecha de entrega}
%       \RCitem{PC3}{\TOCHK{Precondiciones: Que la escuela se encuentre registrada }}{Fecha de entrega}
%       \RCitem{PC3}{\TOCHK{Se quita el mensaje de NO SE ENCONTRÓ INFORMACIÓN SUSTANTIVA }}{Fecha de entrega}
%       \RCitem{PC3}{\TOCHK{Tipo: Secundario }}{Fecha de entrega}
%       \RCitem{PC3}{\TOCHK{Falta campo de Fuente }}{Fecha de entrega}       
%       \RCitem{PC3}{\DONE{Agregar un Verifica por cada Trayectoria alterna  }}{Fecha de entrega}       
%       \RCitem{PC3}{\TOCHK{En el paso 8 de la trayectoria principal, Completa el registro de la información escolar en el sistema }}{Fecha de entrega}              
%       \RCitem{PC3}{\TOCHK{En el paso 9 de la trayectoria, Muestra el mensaje cuando se completa el registro de la información escolar exitosamente }}{Fecha de entrega}              
%       \RCitem{PC3}{\TOCHK{Acomodar la imagen de la pantalla Completar información escolar  }}{Fecha de entrega}              
%       \RCitem{PC3}{\TOCHK{Objetivo: Esta pantalla permie al actor completar el registro de la información escolar en el sistema  }}{Fecha de entrega}                     
%       \RCitem{PC3}{\TOCHK{Diseño: en la cual el actor deberá ingresar los datos del sistema para completar el registro de la escuela  }}{Fecha de entrega}              
%        \RCitem{PC3}{\TOCHK{Diseño: Hasta CONFIRMAR ENVIO DE INFORMACIÓN  }}{Fecha de entrega}              
%        \RCitem{PC3}{\TOCHK{Revisar el comando de Aceptar de la pantalla, porque no permite registrar una nueva escuela, es completar el registro de la información escolar}}{Fecha de entrega}              
%       \RCitem{PC3}{\TOCHK{Revisar el comando de CANCELAR, en términos de completar el registro de la información escolar }}{Fecha de entrega}                      
%       \RCitem{PC3}{\TOCHK{Revisar los mensajes, poner en donde se van a mostrar cada uno }}{Fecha de entrega}        
%       \RCitem{PC3}{\TOCHK{Quitar el mensaje de INFORMACIÓNO SUSTANTIVA  }}{Fecha de entrega}               
%       \RCitem{PC3}{\TOCHK{Verificar ligas y nombres de en donde se van a mostrar los mensajes NO ES SOLICITAR REGISTRO DE ESCUELA  }}{Fecha de entrega}               
       
%       \RCitem{PC3}{\TOCHK{ Tipo: es secundario, entonces poner en todos los secundarios de que punto extienden o quitarlo de aquí  }}{Fecha de entrega}                      
       
%     \end{UClist}    
%   }
% %--------------------------------------------------------

    \UCsection{Atributos}
    \UCitem{Actor}{\cdtRef{actor:usuarioEscuela}{Coordinador del programa}}
    \UCitem{Propósito}{Completar el registro de la información escolar.}
    \UCitem{Entradas}{
      \begin{UClist} 
          \UCli \cdtRef{escuela:region}{Región}. \ioSeleccionar.
          \UCli \cdtRef{escuela:numeroDias}{Número de días que la escuela labora al año}. \ioEscribir.
          \UCli \cdtRef{escuela:totalHabitantes}{Total de habitantes de la localidad}.  \ioEscribir.
          \UCli \cdtRef{escuela:superficieTotal}{Superficie total del predio}. \ioEscribir.
          \UCli \cdtRef{escuela:superficieConstruida}{Superficie total construida}. \ioEscribir.
          \UCli \cdtRef{comunidad:docentesF}{Docentes femeninos}. \ioEscribir.
          \UCli \cdtRef{comunidad:docentesM}{Docentes masculinos}. \ioEscribir.
          \UCli \cdtRef{comunidad:adminF}{Personal administrativo femenino}. \ioEscribir.
          \UCli \cdtRef{comunidad:adminM}{Personal administrativo masculino}. \ioEscribir.
          \UCli \cdtRef{comunidad:alumnosF}{Alumnos femeninos}. \ioEscribir.
          \UCli \cdtRef{comunidad:alumnosM}{Alumnos masculinos}. \ioEscribir.
          \UCli \cdtRef{comunidad:limpiezaF}{Personal de limpieza y mantenimiento femenino}. \ioEscribir.
          \UCli \cdtRef{comunidad:limpiezaM}{Personal de limpieza y mantenimiento masculino}. \ioEscribir.
%           \UCli \cdtRef{comunidad:apoyoF}{Personal de mantenimiento femenino}. \ioEscribir.
%           \UCli \cdtRef{comunidad:apoyoM}{Personal de mantenimiento masculino}. \ioEscribir.
          \UCli \cdtRef{comunidad:apoyoF}{Personal de apoyo femenino}. \ioEscribir.
          \UCli \cdtRef{comunidad:apoyoM}{Personal de apoyo masculino}. \ioEscribir.
          \UCli \cdtRef{comunidad:visitantesF}{Visitantes femeninos (promedio diario)}. \ioEscribir.
          \UCli \cdtRef{comunidad:visitantesM}{Visitantes masculinos (promedio diario)}. \ioEscribir.
      \end{UClist}
      }
    \UCitem{Salidas}{
      \begin{UClist} 
	  \UCli \cdtRef{escuela:cct}{Clave de centro de trabajo}. \ioObtener.
	  \UCli \cdtRef{escuela:nombreEscuela}{Nombre de la escuela}. \ioObtener.
	  \UCli \cdtRef{escuela:cicloEscolar}{Ciclo escolar}. \ioObtener.
	  \UCli \cdtRef{escuela:nivelEscolar}{Nivel escolar}. \ioObtener.
	  \UCli \cdtRef{escuela:turno}{Turno}. \ioObtener.
	  \UCli \cdtRef{escuela:calle}{Calle}. \ioObtener.
	  \UCli \cdtRef{escuela:numero}{Número}. \ioObtener.
	  \UCli \cdtRef{escuela:localidad}{Localidad}. \ioObtener.
	  \UCli \cdtRef{escuela:municipio}{Municipio}. \ioObtener.
	  \UCli \cdtRef{escuela:codigoPostal}{Código postal}. \ioObtener.
	  \UCli \cdtRef{escuela:ambito}{Ámbito}. \ioObtener.
	  \UCli \cdtRef{escuela:control}{Control}. \ioObtener.
	  \UCli \cdtRef{escuela:servicio}{Servicio}. \ioObtener.
	  \UCli \cdtIdRef{MSG1}{Operación realizada exitosamente}: Se muestra en la pantalla \cdtIdRef{IUR 5}{Administrar información escolar} cuando se completa el registro de la información escolar exitosamente. 
      \end{UClist}
      }
    \UCitem{Precondiciones}{
	\begin{UClist}
		\UCli {\bf Interna:} Que la escuela se encuentre en estado \cdtRef{estado:inscrita}{inscrita}.
		\UCli {\bf Interna:} Que el periodo de registro de escuelas se encuentre vigente.
	\end{UClist}
    }
    \UCitem{Postcondiciones}{
	\begin{UClist}
        \UCli {\bf Interna:} El actor podrá administrar los datos de la escuela.
	\end{UClist}
    }
    \UCitem{Reglas de negocio}{
    	\begin{UClist}
            \UCli \cdtIdRef{RN-S1}{Información correcta}: Verifica que la información introducida sea correcta.
	\end{UClist}
    }
    \UCitem{Errores}{
      \begin{UClist}
      \UCli \cdtIdRef{MSG4}{No se encontró información sustantiva}: Se muestra en la pantalla \cdtIdRef{IUR 6}{Completar información escolar} cuando no hay información referente a la escuela.
      \UCli \cdtIdRef{MSG5}{Falta un dato requerido para efectuar la operación solicitada}: Se muestra en la pantalla \cdtIdRef{IUR 6}{Completar información escolar} cuando no se ha ingresado un dato marcado como obligatorio.
      \UCli \cdtIdRef{MSG6}{Formato incorrecto}: Se muestra en la pantalla \cdtIdRef{IUR 6}{Completar información escolar} especificando el dato cuyo valor no cumple con el tipo de dato definido en el diccionario de datos.
      \UCli \cdtIdRef{MSG7}{Se ha excedido la longitud máxima del campo}: Se muestra en la pantalla \cdtIdRef{IUR 6}{Completar información escolar} cuando el actor proporciona un dato que excede la longitud máxima.
      \UCli \cdtIdRef{MSG18}{Error en la región}: Se muestra en la pantalla \cdtIdRef{IUR 6}{Completar información escolar} para notificar al actor que la escuela no se encuentra en un municipio asociado a la región seleccionada.
		\UCli \cdtIdRef{MSG3}{Superficies del predio}: Se muestra en la pantalla \cdtIdRef{IUR 6}{Completar información escolar} para notificar al actor que la superficie construida que ha ingresado supera la superficie total del predio.
     \UCli \cdtIdRef{MSG28}{Operación no permitida por estado de la entidad}: Se muestra sobre la pantalla \cdtIdRef{IUR 5}{Administrar información escolar} indicando al actor que no se puede completar la información escolar debido al estado en que se encuentra la escuela.
		\UCli	\cdtIdRef{MSG41}{Acción fuera del periodo}: Se muestra sobre la pantalla \cdtIdRef{IUR 5}{Administrar información escolar} para indicarle al actor que no puede completar la información escolar debido a que la fecha actual se encuentra fuera del periodo definido por la SMAGEM para realizar el registro de escuelas.
      \end{UClist}
    }
    \UCitem{Tipo}{Secundario, extiende del caso de uso \cdtIdRef{CUR 5}{Administrar información escolar}.}
    \UCitem{Fuente}{
      \begin{UClist}
        \UCli Minuta de la reunión \cdtIdRef{M-3TR}{Toma de requerimientos}.
      \end{UClist}
      }
      \end{UseCase}

 \begin{UCtrayectoria}
    \UCpaso[\UCactor] Solicita completar los datos de la escuela mediante el botón \cdtButton{Completar información} de la pantalla \cdtIdRef{IUR 5}{Administrar información escolar}.
    
    \UCpaso[\UCsist] Verifica que la escuela se encuentre es estado ``Inscrita''. \refTray{A}
    \UCpaso[\UCsist] Verifica que la fecha actual se encuentre dentro del periodo definido por parte de la SMAGEM para el registro de escuelas. \refTray{B}
    
    \UCpaso[\UCsist] Busca la información del catálogo de regiones en el sistema. \refTray{C}
    \UCpaso[\UCsist] Muestra la pantalla \cdtIdRef{IUR 6}{Completar información escolar} en la cual se completa el registro de la escuela.
    \UCpaso[\UCactor] Ingresa los datos de la escuela en la pantalla \cdtIdRef{IUR 6}{Completar información escolar}. \label{cur6:Acciones}
    \UCpaso[\UCactor] Solicita registrar los datos de la escuela oprimiendo el botón \cdtButton{Aceptar} en la pantalla \cdtIdRef{IUR 6}{Completar información escolar}. \refTray{D}

    \UCpaso[\UCsist] Verifica que la escuela se encuentre es estado ``Inscrita''. \refTray{A}
    \UCpaso[\UCsist] Verifica que la fecha actual se encuentre dentro del periodo definido por parte de la SMAGEM para el registro de escuelas. \refTray{B}

    \UCpaso[\UCsist] Verifica que los datos proporcionados por el actor sean correctos como lo indica la regla de negocio \cdtIdRef{RN-S1}{Información correcta}. \refTray{E} \refTray{F}  \refTray{G} \refTray{H}
    \UCpaso[\UCsist] Verifica que la superficie construida no supere la superficie total del predio como lo indica la regla de negocio \cdtIdRef{RN-N6}{Superficies del predio}. \refTray{I}
%     \UCpaso[\UCsist] Muestra el mensaje \cdtIdRef{MSG3}{Confirmación de envío de información} como \cdtRef{fig:pantallaEmergente}{pantalla emergente} para que el actor confirme el envío de la información.
%     \UCpaso[\UCactor] Confirma el envío de información oprimiendo el botón \cdtButton{Aceptar} de la \cdtRef{fig:pantallaEmergente}{pantalla emergente} \refTray{E}
    \UCpaso[\UCsist] Completa el registro de la información escolar en el sistema.\label{cur6:termina}
    \UCpaso[\UCsist] Muestra el mensaje \cdtIdRef{MSG1}{Operación realizada exitosamente} en la pantalla \cdtIdRef{IUR 5}{Administrar información escolar} cuando se completa el registro de la información escolar exitosamente. 
 \end{UCtrayectoria}

	%Trayectorias alternas ------

 \begin{UCtrayectoriaA}[Fin del caso de uso]{A}{La escuela no se encuentra en el estado ``Inscrita''}
    \UCpaso[\UCsist] Muestra el mensaje \cdtIdRef{MSG28}{Operación no permitida por estado de la entidad} en la pantalla \cdtIdRef{IUR 5}{Administrar información escolar} indicando al actor que no puede completar la información escolar debido a que la escuela no se encuentra en estado ``Inscrita''.
 \end{UCtrayectoriaA}

 \begin{UCtrayectoriaA}[Fin del caso de uso]{B}{La fecha actual se encuentra fuera del periodo definido por la SMAGEM para el registro de escuelas}
    \UCpaso[\UCsist] Muestra el mensaje \cdtIdRef{MSG41}{Acción fuera del periodo} en la pantalla \cdtIdRef{IUR 5}{Administrar información escolar} indicando al actor que no puede completar la información escolar debido a que la fecha actual se encuentra fuera del periodo definido por la SMAGEM para realizar la acción.
 \end{UCtrayectoriaA}

  \begin{UCtrayectoriaA}[Fin de caso de uso]{C}{No existe información en el catálogo de regiones.}
    \UCpaso[\UCsist] Muestra el mensaje \cdtIdRef{MSG4}{No se encontró información sustantiva} en la pantalla \cdtIdRef{IUR 6}{Completar información escolar} indicando que no hay información en el catálogo de regiones en el sistema.
  \end{UCtrayectoriaA}

 \begin{UCtrayectoriaA}[Fin de caso de uso]{D}{El actor desea cancelar la operación.}
    \UCpaso[\UCactor] Solicita cancelar la operación oprimiendo el botón \cdtButton{Cancelar} en la pantalla \cdtIdRef{IUR 6}{Completar información escolar}.
    \UCpaso[\UCsist] Regresa a la pantalla \cdtIdRef{IUR 5}{Administrar información escolar}.
 \end{UCtrayectoriaA}

 \begin{UCtrayectoriaA}{E}{El actor no ingresó un dato marcado como requerido.}
    \UCpaso[\UCsist] Muestra el mensaje \cdtIdRef{MSG5}{Falta un dato requerido para efectuar la operación solicitada} en la pantalla \cdtIdRef{IUR 6}{Completar información escolar} indicando que no se puede realizar el registro de la escuela debido a la falta de información requerida.
   \UCpaso[] Continúa en el paso \ref{cur6:Acciones} de la trayectoria principal.
 \end{UCtrayectoriaA}

 \begin{UCtrayectoriaA}{F}{El actor proporciona un dato que excede la longitud máxima.}
    \UCpaso[\UCsist] Muestra el mensaje \cdtIdRef{MSG7}{Se ha excedido la longitud máxima del campo} en la pantalla \cdtIdRef{IUR 6}{Completar información escolar} especificando el dato cuya longitud excede el tamaño máximo permitido.
   \UCpaso[] Continúa en el paso \ref{cur6:Acciones} de la trayectoria principal.
 \end{UCtrayectoriaA}

   \begin{UCtrayectoriaA}{G}{El actor ingresó un tipo de dato incorrecto.}
    \UCpaso[\UCsist] Muestra el mensaje \cdtIdRef{MSG6}{Formato incorrecto} y señala el campo que presenta el dato inválido en la 
    pantalla \cdtIdRef{IUR 6}{Completar información escolar} para indicar que se ha ingresado un tipo de dato inválido.
    \UCpaso[] Continúa con el paso \ref{cur6:Acciones} de la trayectoria principal.
 \end{UCtrayectoriaA}
 
   \begin{UCtrayectoriaA}{H}{La región no está asociada al municipio donde se encuentra la escuela.}
    \UCpaso[\UCsist] Muestra el mensaje \cdtIdRef{MSG18}{Error en la región} en la pantalla \cdtIdRef{IUR 6}{Completar información escolar} para notificar al actor que la escuela no se encuentra en un municipio asociado a la región seleccionada.
    \UCpaso[] Continúa con el paso \ref{cur6:termina} de la trayectoria principal.
 \end{UCtrayectoriaA}
 
 \begin{UCtrayectoriaA}{I}{El actor ingresó una superficie construida que supera la superficie total del predio.}
  \UCpaso[\UCsist] Muestra el mensaje \cdtIdRef{MSG3}{Superficies del predio} en la pantalla \cdtIdRef{IUR 6}{Completar información escolar} para notificar al actor que la superficie construida que ha ingresado supera la superficie total del predio.
    \UCpaso[] Continúa con el paso \ref{cur6:Acciones} de la trayectoria principal.
 \end{UCtrayectoriaA}

 
%  \begin{UCtrayectoriaA}{E}{El actor desea cancelar el envío de información.}
%     \UCpaso[\UCactor] Solicita cancelar el envío de la información oprimiendo el botón \cdtButton{Cancelar} de la pantalla emergente.
%     \UCpaso[] Continúa en el paso de \ref{cur6:Acciones} la trayectoria principal.
%  \end{UCtrayectoriaA}

