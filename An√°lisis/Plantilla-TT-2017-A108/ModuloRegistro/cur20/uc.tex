\begin{UseCase}{CUR 20}{Aprobar solicitud de inscripción}
    {
	La solicitud de inscripción de la escuela está en condiciones de ser aprobada una vez que la información referente a esta ha sido revisada verificando la carta compromiso y el nombramiento del director. Este caso de uso permite al actor director del programa aprobar o rechazar la solicitud de la escuela después de revisar su información.
    }
    \UCitem{Versión}{1.0}
    \UCccsection{Administración de Requerimientos}
    \UCitem{Autor}{Angélica Madrid Jiménez}
    \UCccitem{Evaluador}{José David Ortega Pacheco}
    \UCitem{Operación}{Registro}
    \UCccitem{Prioridad}{Alta}
    \UCccitem{Complejidad}{Baja}
    \UCccitem{Volatilidad}{Baja}
    \UCccitem{Madurez}{Alta}
    \UCitem{Estatus}{Terminado}
    \UCitem{Fecha del último estatus}{24 de Octubre del 2014}

%% Copie y pegue este bloque tantas veces como revisiones tenga el caso de uso.
%% Esta sección la debe llenar solo el Revisor
% %--------------------------------------------------------
% 	\UCccsection{Revisión Versión XX} % Anote la versión que se revisó.
% 	% FECHA: Anote la fecha en que se terminó la revisión.
% 	\UCccitem{Fecha}{Fecha en que se termino la revisión} 
% 	% EVALUADOR: Coloque el nombre completo de quien realizó la revisión.
% 	\UCccitem{Evaluador}{Nombre de quien revisó}
% 	% RESULTADO: Coloque la palabra que mas se apegue al tipo de acción que el analista debe realizar.
% 	\UCccitem{Resultado}{Corregir, Desechar, Rehacer todo, terminar.}
% 	% OBSERVACIONES: Liste los cambios que debe realizar el Analista.
% 	\UCccitem{Observaciones}{
% 		\begin{UClist}
% 			% PC: Petición de Cambio, describa el trabajo a realizar, si es posible indique la causa de la PC. Opcionalmente especifique la fecha en que considera razonable que se deba terminar la PC. No olvide que la numeración no se debe reiniciar en una segunda o tercera revisión.
% 			\RCitem{PC1}{\TODO{Descripción del pendiente}}{Fecha de entrega}
% 			\RCitem{PC2}{\TODO{Descripción del pendiente}}{Fecha de entrega}
% 			\RCitem{PC3}{\TODO{Descripción del pendiente}}{Fecha de entrega}
% 		\end{UClist}		
% 	}
% %--------------------------------------------------------

% %--------------------------------------------------------
	\UCccsection{Revisión Versión 0.1} % Anote la versión que se revisó.
	% FECHA: Anote la fecha en que se terminó la revisión.
	\UCccitem{Fecha}{28 de Octubre del 2014} 
	% EVALUADOR: Coloque el nombre completo de quien realizó la revisión.
	\UCccitem{Evaluador}{Victor Lozano Ortega}
	% RESULTADO: Coloque la palabra que mas se apegue al tipo de acción que el analista debe realizar.
	\UCccitem{Resultado}{Corregir.}
	% OBSERVACIONES: Liste los cambios que debe realizar el Analista.
	\UCccitem{Observaciones}{
		\begin{UClist}
			% PC: Petición de Cambio, describa el trabajo a realizar, si es posible indique la causa de la PC. Opcionalmente especifique la fecha en que considera razonable que se deba terminar la PC. No olvide que la numeración no se debe reiniciar en una segunda o tercera revisión.
% 			\RCitem{PC1}{\TODO{En el resumen quitar la referencia al actor, la referencia va en la sección de actores}}{Fecha de entrega}
% 			\RCitem{PC2}{\TODO{``Escuela'' no debe ir en mayúscula su primer letra}}{Fecha de entrega}
% 			\RCitem{PC3}{\TODO{Lo mismo para el actor, no todas las entidades deben llevar letra capital}}{Fecha de entrega}
% 			\RCitem{PC4}{\TODO{En el resumen, este caso de uso permite aprobar o rechazar...}}{Fecha de entrega}
% 			\RCitem{PC5}{\TODO{Propósito ``Aprobar de una Escuela...'' sugiero replantear el propósito}}{Fecha de entrega}
% 			\RCitem{PC6}{\TODO{Agregar punto final al propósito}}{Fecha de entrega}
% 			\RCitem{PC7}{\TODO{En el mensaje de las salidas, la inscripción ha sido ``aprobada'' exitosamente y no lleva la leyenda ``Lo obtiene el sistema''}}{Fecha de entrega}
% 			\RCitem{PC8}{\TODO{La precondición deberia ser que la escuela se encuentre en estado ``Por aprobar'', su información de contacto no sería una precondición}}{Fecha de entrega}
% 			\RCitem{PC9}{\TODO{La postcondición no aplica pues forma parte del caso de uso el evento en que se aprueba a la escuela, falta la postcondición externa en que la escuela ya podrá hacer sus tramites u operaciones para cumplir con el programa}}{Fecha de entrega}
% 			\RCitem{PC10}{\TODO{Errores: falta un error para cuando el actor no ha seleccionado una opción (Datos requeridos)}}{Fecha de entrega}
% 			\RCitem{PC11}{\TODO{En el paso 1 de la trayectoria principal, indicar con que botón solicita realizar la acción}}{Fecha de entrega}
% 			\RCitem{PC12}{\TODO{Paso dos, indicar donde lo busca, lo busca en el sistema}}{Fecha de entrega}
% 			\RCitem{PC13}{\TODO{Paso 3, se podrá hacer la aprobación de la ``escuela'', pienso que no puede ser ``de la misma'' pues en este paso no se ha hecho referencia a la escuela}}{Fecha de entrega}
% 			\RCitem{PC14}{\TODO{El paso 4 no lleva una trayectoria alternativa, la trayectoria alternativa A va en el paso 2}}{Fecha de entrega}
% 			\RCitem{PC15}{\TODO{Paso 5, hay 2 botones, deberia ser unicamente el boton Aceptar, indicar que la opción seleccionada es ``Aprobar solicitud''}}{Fecha de entrega}
% 			\RCitem{PC16}{\TODO{La trayectoria A es fin de caso de uso}}{Fecha de entrega}
% 			\RCitem{PC17}{\TODO{La trayectoria A, cambiar la condición por ``El actor selecciono la opción de rechazar solicitud''}}{Fecha de entrega}
% 			\RCitem{PC18}{\TODO{Quitar el paso 1 de la trayectoria A, dicho paso es la condición}}{Fecha de entrega}
% 			\RCitem{PC19}{\TODO{Trayectoria A, falta un paso en que se elimina el registro de la escuela del sistema}}{Fecha de entrega}
% 			\RCitem{PC20}{\TODO{Trayectoria B es fin de caso de uso}}{Fecha de entrega}
% 			\RCitem{PC21}{\TODO{Objetivo de la pantalla, permite aceptar o rechazar...}}{Fecha de entrega}
% 			\RCitem{PC22}{\TODO{Quitar mayuscula a actor en el objetivo de la pantalla}}{Fecha de entrega}
% 			\RCitem{PC23}{\TODO{En la pantalla, al final del diseño, el mensaje se puede mostrar para ambos casos, aprobado o rechazado}}{Fecha de entrega}
% 			\RCitem{PC24}{\TODO{Comandos, a quien permiten realizar la accion los botones de aceptar y cancelar, falta especificar que al actor, no solo se hace la accion de aprobacion, tambien puede ser de rechazo}}{Fecha de entrega}
% 			\RCitem{PC25}{\TODO{El mensaje se puede mostrar para ambos casos, inscrita (Aprobada) o rechazada.}}{Fecha de entrega}
			\RCitem{PC26}{\DONE{Las opciones de rechazar y aprobar son radiobotones, no botones, quitar el rectángulo}}{06/11}
			\RCitem{PC26}{\DONE{Agregar la creación de la cuenta en estado de inactiva}}{06/11}
		\end{UClist}		
	}
%--------------------------------------------------------

    \UCsection{Atributos}
    \UCitem{Actor}{\cdtRef{actor:usuarioSMAGEM}{Director del programa}}
    \UCitem{Propósito}{Aprobar la solicitud de inscripción de una escuela después de revisar su información asociada.}
    \UCitem{Entradas}{De la sección de {\bf{ Aprobar solicitud de inscripción}} \UCli Aprobar solicitud: Se selecciona una opción de las disponibles.}
    \UCitem{Salidas}{
	\begin{UClist} 
      \UCli \cdtRef{escuela:cct}{Clave del centro de trabajo}. {\ioObtener}.
      \UCli \cdtRef{escuela:nombreEscuela}{Nombre de la escuela}. {\ioObtener}.
      \UCli \cdtRef{persona:nombre}{Nombre(s)}. {\ioObtener}.
      \UCli \cdtRef{persona:primerApellido}{Primer apellido del director}. {\ioObtener}.
      \UCli \cdtRef{persona:segundoApellido}{Segundo apellido del director}. {\ioObtener}.
      \UCli \cdtRef{coordinador:nombramiento}{Nombramiento}. {\ioObtener}.
      \UCli \cdtRef{coordinador:cartaCompromiso}{Carta compromiso}. {\ioObtener}.
      \UCli {\cdtIdRef{MSG1}{Operación realizada exitosamente}}: Se muestra en la pantalla {\cdtIdRef{IUR 19}{Administrar solicitudes de inscripción}} para indicar al actor que la solicitud de inscripción de la Escuela ha sido aprobado exitosamente.
      
	\end{UClist}
    }
    \UCitem{Precondiciones}{
	\begin{UClist}
	    \UCli {\bf Interna:} Que el estado de la escuela sea \cdtRef{estado:aprobar}{por aprobar}.
	\end{UClist}
    }
    \UCitem{Postcondiciones}{
	\begin{UClist}
        \UCli {\bf Externa:} La escuela podrá continuar con su proceso de registro de responsable y comité.
	\UCli {\bf Interna:} La escuela se encontrará en  estado \cdtRef{estado:inscrita}{inscrita}.
	\UCli {\bf Interna:} Se creará una cuenta asociada a la escuela en estado \cdtRef{estado:inactiva}{inactiva}.
	\end{UClist}
    }
    \UCitem{Reglas de negocio}{
      \begin{UClist}
            \UCli \cdtIdRef{RN-S1}{Información correcta}: Verifica que la información introducida sea correcta.
      \end{UClist}
    }
    \UCitem{Errores}{\UCli \cdtIdRef{MSG5}{Falta un dato requerido para efectuar la operación solicitada}: Se muestra en la pantalla \cdtIdRef{IUR 20}{Aprobar solicitud de inscripción} cuando no se haya proporcionado un dato requerido.
    }
    \UCitem{Tipo}{Primario}
	\UCitem{Fuente}{
	    \begin{UClist}
        \UCli Minuta de la reunión \cdtIdRef{M-3TR}{Toma de requerimientos}.
	    \end{UClist}
}
    \end{UseCase}

 \begin{UCtrayectoria}
    \UCpaso[\UCactor] Solicita aprobar la solicitud de inscripción de una escuela en la pantalla \cdtIdRef{IUR 19}{Administrar solicitudes de inscripción} mediante el botón \botOk.
    \UCpaso[\UCsist] Busca la información de la escuela que se quiere aprobar en el sistema. 
    \UCpaso[\UCsist] Muestra la pantalla \cdtIdRef{IUR 20}{Aprobar solicitud de inscripción} por medio de la cual se podrá hacer la aprobación de la escuela.
    \UCpaso[\UCactor] Verifica los datos de la información de la escuela en la pantalla \cdtIdRef{IUR 20}{Aprobar solicitud de inscripción}.
    \UCpaso[\UCactor] Solicita aprobar la inscripción de la escuela seleccionando la opción ``Aceptar solicitud'' de la pantalla \cdtIdRef{IUR 20}{Aprobar la solicitud de inscripción}. \refTray{A}
    \UCpaso[\UCactor] Solicita enviar la aprobación oprimiendo el botón \cdtButton{Aceptar} de la pantalla \cdtIdRef{IUR 20}{Aprobar solicitud de inscripción}. \refTray{B}
    \UCpaso[\UCsist] Cambia el estado de la escuela a \cdtRef{estado:inscrita}{inscrita}. 
    \UCpaso[\UCsist] Crea una cuenta asociada a la escuela en estado \cdtRef{estado:inactiva}{inactiva}.
    \UCpaso[\UCsist] Genera y registra el token de activación.
    \UCpaso[\UCsist] Envía el correo para la verificación del correo de contacto del coordinador del programa.
    \UCpaso[\UCsist] Muestra  el mensaje \cdtIdRef{MSG1}{Operación realizada exitosamente} en la pantalla \cdtIdRef{IUR 19}{Administrar solicitudes de inscripción}.
 \end{UCtrayectoria}


 \begin{UCtrayectoriaA}[Fin de caso de uso]{A}{El actor rechaza la solicitud de inscripción seleccionando la opción ``Rechazar solicitud'' de la pantalla \cdtIdRef{IUR 20}{Aprobar la solicitud de inscripción}.}
   \UCpaso[\UCactor] Solicita enviar el rechazo oprimiendo el botón \cdtButton{Aceptar} de la pantalla \cdtIdRef{IUR 20}{Aprobar solicitud de inscripción}. \refTray{B}
    \UCpaso[\UCsist] Envía el correo para la notificación del rechazo de la solicitud.
    \UCpaso[\UCsist] Muestra  el mensaje \cdtIdRef{MSG1}{Operación realizada exitosamente} en la pantalla \cdtIdRef{IUR 19}{Administrar solicitudes de inscripción}.
 \end{UCtrayectoriaA}
 
 \begin{UCtrayectoriaA}[Fin de caso de uso]{B}{El actor desea cancelar el envío de información.}
    \UCpaso[\UCactor] Solicita cancelar la operación oprimiendo el botón de \cdtButton{Cancelar} de la pantalla \cdtIdRef{IUR 20}{Aprobar solicitud de inscripción}.
    \UCpaso[\UCsist] Muestra la pantalla \cdtIdRef{IUR 19}{Administrar solicitudes de inscripción}
 \end{UCtrayectoriaA}
