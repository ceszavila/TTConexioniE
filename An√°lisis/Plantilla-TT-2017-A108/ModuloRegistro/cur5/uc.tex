\begin{UseCase}{CUR 5}{Administrar información escolar}
	{
	    Este caso de uso permite administrar la información de la escuela que ha sido registrada. Con esto, el actor podrá acceder a las operaciones para modificar, visualizar y completar la información escolar.
	}
    \UCitem{Versión}{1.0}
    \UCccsection{Administración de Requerimientos}
    \UCitem{Autor}{Victor Lozano Ortega}
    \UCccitem{Evaluador}{José David Ortega Pacheco}
    \UCitem{Operación}{Administración}
    \UCccitem{Prioridad}{Alta}
    \UCccitem{Complejidad}{Baja}
    \UCccitem{Volatilidad}{Baja}
    \UCccitem{Madurez}{Alta}
    \UCitem{Estatus}{Terminado}
    \UCitem{Fecha del último estatus}{5 de Noviembre del 2014}

%% Copie y pegue este bloque tantas veces como revisiones tenga el caso de uso.
%% Esta sección la debe llenar solo el Revisor
% %--------------------------------------------------------
% 	\UCccsection{Revisión Versión XX} % Anote la versión que se revisó.
% 	\UCccitem{Fecha}{Fecha en que se termino la revisión} 
% 	\UCccitem{Evaluador}{Nombre de quien revisó}
% 	\UCccitem{Resultado}{Corregir, Desechar, Rehacer todo, terminar.}
% 	\UCccitem{Observaciones}{
% 		\begin{UClist}
% 			% PC: Petición de Cambio, describa el trabajo a realizar, si es posible indique la causa de la PC. Opcionalmente especifique la fecha en que considera razonable que se deba terminar la PC. No olvide que la numeración no se debe reiniciar en una segunda o tercera revisión.
% 			\RCitem{PC1}{\TOCHK{En Resumen, este caso permite la administración de la escuela que ha sido registrada. Con esto el actor ...}}{21 de octubre 2014}
% 			\RCitem{PC2}{\TOCHK{Condición interna: que el usuario haya iniciado sesión }}{Fecha de entrega}
% 			\RCitem{PC3}{\TOCHK{Cambiar Registrar información de la escuela por Completar Información de la escuela}}{Fecha de entrega}

%			\RCitem{PC2}{\TOCHK{Revisar liga de Modificar Información Escolar en los puntos de Extensión }}{Fecha de 				entrega}
%			\RCitem{PC2}{\TOCHK{Revisar liga de Registrar Información Escolar en los puntos de Extensión }}{Fecha de
% 			 entrega}
% 			\RCitem{PC2}{\TOCHK{Revisar liga de Visualizar Información Escolar en los puntos de Extensión }}{Fecha de entrega, todas llevan a casos de uso de Integrantes de líneas de acción} 			
% 			\RCitem{PC2}{\TOCHK{Revisar liga de Registrar Información Escolar en los puntos de Extensión }}{Fecha de entrega}
 			
 			
% 			\RCitem{PC2}{\TOCHK{Resumen: que permite administrar a la ...}}{Fecha de entrega} 			
% 			\RCitem{PC2}{\TOCHK{Salidas: quitar la referencia a la entidad Escuela}}{Fecha de entrega} 			 	
% 			\RCitem{PC2}{\TODO{Precondicionnes: Quitar la referencia al caso de uso CUR 3}}{Fecha de entrega} 		
% 			\RCitem{PC2}{\TODO{Postcondiciones: Quitar las referencias a los caso de uso CUR 6, 7, 8, unicamente mencionar las operaciones}}{Fecha de entrega} 		 				 	 					
% 			\RCitem{PC2}{\TODO{Precondicionnes: Quitar la referencia al caso de uso CUR 3}}{Fecha de entrega} 		 			
% 		\end{UClist}		
% 	}
% %--------------------------------------------------------

	\UCsection{Atributos}
	\UCitem{Actor(es)}{\cdtRef{actor:usuarioEscuela}{Coordinador del programa}}
	\UCitem{Propósito}{Administrar la información de la escuela registrada en el sistema.}
	\UCitem{Entradas}{Ninguna.}
	\UCitem{Salidas}{
		\begin{UClist} 
	       \UCli Tabla que muestra \cdtRef{escuela:cct}{Clave de centro de trabajo}, \cdtRef{escuela:nombreEscuela}{Nombre de la escuela} y \cdtRef{escuela:turno}{Turno} de la escuela.
		\end{UClist}
	}
	\UCitem{Precondiciones}{
		\begin{UClist}
			\UCli {\bf Interna:} Que el actor haya iniciado sesión en el sistema.
			\UCli {\bf Interna:} Que exista una escuela registrada por parte del actor mediante el caso de uso \cdtIdRef{CUR 3}{Solicitar preinscripción}.
			\UCli {\bf Interna:} Que la escuela se encuentre en estado \cdtRef{estado:inscrita}{inscrita}.
			\UCli {\bf Interna:} Que el periodo de registro de escuelas se encuentre vigente.
		\end{UClist}
	}
	\UCitem{Postcondiciones}{
		\begin{UClist}
			\UCli {\bf Interna:} El actor podrá modificar la información de la escuela registrada, mediante el caso de uso \cdtIdRef{CUR 7}{Modificar información escolar}.
			\UCli {\bf Interna:} El actor podrá completar la información de la escuela, mediante el caso de uso \cdtIdRef{CUR 6}{Completar información escolar}.
			\UCli {\bf Interna:} El actor podrá visualizar la información de las escuela registrada, mediante el caso de uso \cdtIdRef{CUR 8}{Visualizar informacion escolar}.
		\end{UClist}
	}
	\UCitem{Errores}{
	
	   \UCli \cdtIdRef{MSG28}{Operación no permitida por estado de la entidad}: Se muestra sobre la pantalla \cdtIdRef{IUR 5}{Administrar información escolar} indicando al actor que no se puede administrar la información escolar debido al estado en que se encuentra la escuela.
		\UCli	\cdtIdRef{MSG41}{Acción fuera del periodo}: Se muestra sobre la pantalla \cdtIdRef{IUR 5}{Administrar información escolar} para indicarle al actor que no puede administrar la información escolar debido a que la fecha actual se encuentra fuera del periodo definido por la SMAGEM para realizar el registro de escuelas.
			
	}
	\UCitem{Tipo}{Primario.}
	\UCitem{Fuente}{
		\begin{UClist}
	    \UCli Minuta de la reunión \cdtIdRef{M-3TR}{Toma de requerimientos}.
		\end{UClist}
	}
 \end{UseCase}

 \begin{UCtrayectoria}
    \UCpaso[\UCactor] Solicita administrar la información escolar seleccionando la opción ``Información general'' del menú \cdtIdRef{MN2}{Menú del Coordinador del programa} y posteriormente la opción ``Información escolar''.
    \UCpaso[\UCsist] Verifica que la escuela se encuentre en estado ``Inscrita''.  \refTray{A}   
    \UCpaso[\UCsist] Verifica que la fecha actual se encuentre dentro del periodo definido por la SMAGEM para administrar la información escolar. \refTray{B}    
    \UCpaso[\UCsist] Busca la información de la escuela registrada en el sistema.
    \UCpaso[\UCsist] Muestra la información de la escuela registrada en la pantalla \cdtIdRef{IUR 5}{Administrar información escolar}.
    \UCpaso[\UCactor] Administra la información escolar a través de los botones: \cdtButton{Completar información}, \botEdit, \botV \label{cur5:Acciones}.
 \end{UCtrayectoria}

%ALTERNAS 
 \begin{UCtrayectoriaA}[Fin del caso de uso]{A}{La escuela no se encuentra en el estado ``Inscrita''}
    \UCpaso[\UCsist] Muestra el mensaje \cdtIdRef{MSG28}{Operación no permitida por estado de la entidad} en la pantalla \cdtIdRef{IUR 5}{Administrar información escolar} indicando al actor que no puede completar la información escolar debido a que la escuela no se encuentra en estado ``Inscrita''.
 \end{UCtrayectoriaA}

 \begin{UCtrayectoriaA}[Fin del caso de uso]{B}{La fecha actual se encuentra fuera del periodo definido por la SMAGEM para el registro de escuelas}
    \UCpaso[\UCsist] Muestra el mensaje \cdtIdRef{MSG41}{Acción fuera del periodo} en la pantalla \cdtIdRef{IUR 5}{Administrar información escolar} indicando al actor que no puede completar la información escolar debido a que la fecha actual se encuentra fuera del periodo definido por la SMAGEM para realizar la acción.
 \end{UCtrayectoriaA} 
 
\subsection{Puntos de extensión}

\UCExtensionPoint
{El actor requiere modificar la información de la escuela}
{ Paso \ref{cur5:Acciones} de la trayectoria principal}
{\cdtIdRef{CUR 7}{Modificar información escolar}}

\UCExtensionPoint
{El actor requiere completar la información de la escuela}
{ Paso \ref{cur5:Acciones} de la trayectoria principal}
{\cdtIdRef{CUR 6}{Completar información escolar}}
 
\UCExtensionPoint
{El actor requiere visualizar la información de la escuela}
{ Paso \ref{cur5:Acciones} de la trayectoria principal}
{\cdtIdRef{CUR 8}{Visualizar información escolar}}
