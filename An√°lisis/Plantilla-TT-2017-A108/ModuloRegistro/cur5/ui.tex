\subsection{IUR 5 Administrar información escolar}

\subsubsection{Objetivo}
	
     Esta pantalla permite al actor \cdtRef{actor:usuarioEscuela}{Coordinador del programa} visualizar la escuela registrada desde su cuenta, la información de la misma, y sirve como punto de acceso para las acciones de modificar, visualizar o completar la información escolar.

\subsubsection{Diseño}

    En la figura~\ref{IUR 5} se muestra la pantalla ``Administrar información escolar'', por medio de la cual se podrá administrar a la escuela registrada a través de una tabla de resultados. El actor tendrá la facultad de completar la información de la escuela. 
    Los comandos \botEdit[Modificar información escolar] y \botV[Visualizar información escolar] se mantienen deshabilitadas hasta que se haya completado en su totalidad la información de la escuela.\\

    En la figura~\ref{IUR 5.1} se muestra la pantalla ``Administrar información escolar'', por medio de la cual el actor tendrá la facultad de modificar o visualizar la información de la escuela. \\

    \IUfig[.9]{pantallas/registro/IUR5}{IUR 5}{Administrar información escolar}

    \IUfig[.9]{pantallas/registro/IUR5_1}{IUR 5.1}{Administrar información escolar (Información completa)}

    
\subsubsection{Comandos}
    \begin{itemize}
    \item \cdtButton{Completar información}: Se utiliza para completar la información escolar, dirige a la pantalla \cdtIdRef{IUR 6}{Registrar información escolar}.
	\item \botEdit[Modificar información escolar]: Se utiliza para modificar la información de la escuela, dirige a la pantalla \cdtIdRef{IUR 7}{Modificar información escolar}.
    \item \botV[Visualizar información escolar]: Se utiliza para visualizar la información de la escuela, dirige a la pantalla \cdtIdRef{IUR 8}{Visualizar información escolar}.
    \end{itemize}

 