\begin{UseCase}{CUR 8}{Visualizar información escolar}
    {
    Este caso de uso permite visualizar la información de la escuela previamente registrada.
	%El registro de una escuela podrá ser visualizado por medio de este caso de uso el cual tiene como finalidad mostrar la información registrada de la escuela, asi como la información estadística.
    }
    \UCitem{Versión}{1.0}
    \UCccsection{Administración de Requerimientos}
    \UCitem{Autor}{Victor Lozano Ortega}
    \UCccitem{Evaluador}{José David Ortega Pacheco}
    \UCitem{Operación}{Consultar}
    \UCccitem{Prioridad}{Alta}
    \UCccitem{Complejidad}{Baja}
    \UCccitem{Volatilidad}{Baja}
    \UCccitem{Madurez}{Alta}
    \UCitem{Estatus}{Terminado}
    \UCitem{Fecha del último estatus}{5 de Noviembre del 2014}

%% Copie y pegue este bloque tantas veces como revisiones tenga el caso de uso.
%% Esta sección la debe llenar solo el Revisor
% %--------------------------------------------------------
% 	\UCccsection{Revisión Versión XX} % Anote la versión que se revisó.
% 	% FECHA: Anote la fecha en que se terminó la revisión.
% 	\UCccitem{Fecha}{Fecha en que se termino la revisión} 
% 	% EVALUADOR: Coloque el nombre completo de quien realizó la revisión.
% 	\UCccitem{Evaluador}{Nombre de quien revisó}
% 	% RESULTADO: Coloque la palabra que mas se apegue al tipo de acción que el analista debe realizar.
% 	\UCccitem{Resultado}{Corregir, Desechar, Rehacer todo, terminar.}
% 	% OBSERVACIONES: Liste los cambios que debe realizar el Analista.
% 	\UCccitem{Observaciones}{
% 		\begin{UClist}
% 			% PC: Petición de Cambio, describa el trabajo a realizar, si es posible indique la causa de la PC. Opcionalmente especifique la fecha en que considera razonable que se deba terminar la PC. No olvide que la numeración no se debe reiniciar en una segunda o tercera revisión.
% 			\RCitem{PC1}{\TOCHK{Quitar la división de las filas para la salida}}{21 de octubre de 2014}
% 			\RCitem{PC1}{\TOCHK{Tipo: Secundario}}{21 de octubre de 2014} 			
% 			\RCitem{PC1}{\TOCHK{Pantalla: Modificar la pantalla acorde con las modificaciones}}{21 de octubre de 2014} 			 			
% 			\RCitem{PC2}{\TODO{Descripción del pendiente}}{Fecha de entrega}
% 			\RCitem{PC3}{\TODO{Descripción del pendiente}}{Fecha de entrega}
% 		\end{UClist}		
% 	}
% %--------------------------------------------------------

    \UCsection{Atributos}
    \UCitem{Actor}{\cdtRef{actor:usuarioEscuela}{Coordinador del programa}}
    \UCitem{Propósito}{Visualizar toda la información registrada de la escuela.}
    \UCitem{Entradas}{Ninguna.}
    \UCitem{Salidas}{
      \begin{UClist} 
      \UCli \cdtRef{escuela:cct}{Clave de centro de trabajo}. {\ioObtener}.
      \UCli \cdtRef{escuela:nombreEscuela}{Nombre de la escuela}. {\ioObtener}.
      \UCli \cdtRef{escuela:cicloEscolar}{Ciclo escolar}. {\ioObtener{}}.
      \UCli \cdtRef{escuela:nivelEscolar}{Nivel escolar}. {\ioObtener{}}.
      \UCli \cdtRef{escuela:turno}{Turno}. {\ioObtener{}}.
      \UCli \cdtRef{escuela:calle}{Calle}. {\ioObtener}.
      \UCli \cdtRef{escuela:numero}{Número}. {\ioObtener}.
      \UCli \cdtRef{escuela:localidad}{Localidad}. {\ioObtener{}}.
      \UCli \cdtRef{escuela:municipio}{Municipio}. {\ioObtener{}}.
      \UCli \cdtRef{escuela:codigoPostal}{Código postal}. {\ioObtener{}}.
      \UCli \cdtRef{escuela:ambito}{Ámbito}. {\ioObtener{}}.
      \UCli \cdtRef{escuela:control}{Control}. {\ioObtener{}}.
      \UCli \cdtRef{escuela:servicio}{Servicio}. {\ioObtener{}}.
      \UCli \cdtRef{escuela:region}{Región}.  {\ioObtener{}}.
      \UCli \cdtRef{escuela:numeroDias}{Número de días que la escuela labora al año}. \ioObtener.
      \UCli \cdtRef{escuela:totalHabitantes}{Total de habitantes de la localidad}. \ioObtener.
      \UCli \cdtRef{escuela:superficieTotal}{Superficie total del predio}. \ioObtener.	  
      \UCli \cdtRef{escuela:superficieConstruida}{Superficie total constrída}. \ioObtener.      
      \UCli \cdtRef{comunidad:docentesF}{Docentes femeninos}. \ioObtener.
      \UCli \cdtRef{comunidad:docentesM}{Docentes masculinos}. \ioObtener.
      \UCli \cdtRef{comunidad:adminF}{Personal administrativo femenino}. \ioObtener.
      \UCli \cdtRef{comunidad:adminM}{Personal administrativo masculino}. \ioObtener.
      \UCli \cdtRef{comunidad:alumnosF}{Alumnos femeninos}. \ioObtener.
      \UCli \cdtRef{comunidad:alumnosM}{Alumnos masculinos}. \ioObtener.
%           \UCli \cdtRef{comunidad:limpiezaF}{Personal de limpieza femenina}. \ioObtener.
%           \UCli \cdtRef{comunidad:limpiezaM}{Personal de limpieza masculino}. \ioObtener.
          \UCli \cdtRef{comunidad:apoyoF}{Personal de limpieza  y mantenimiento femenino}. \ioObtener.
          \UCli \cdtRef{comunidad:apoyoM}{Personal de limpieza  y mantenimiento masculino}. \ioObtener.
%           \UCli \cdtRef{comunidad:apoyoF}{Personal de apoyo femenino}. \ioObtener.
%           \UCli \cdtRef{comunidad:apoyoM}{Personal de apoyo masculino}. \ioObtener.
          \UCli \cdtRef{comunidad:visitantesF}{Visitantes femeninos  (promedio diario)}. \ioObtener.
          \UCli \cdtRef{comunidad:visitantesM}{Visitantes masculinos  (promedio diario)}. \ioObtener.
      \end{UClist}
      }
    \UCitem{Precondiciones}{
	\begin{UClist}
        	\UCli {\bf Interna:} Que la escuela se encuentre registrada.
        	\UCli {\bf Interna:} Que se haya completado la información escolar.
        	\UCli {\bf Interna:} Que la escuela se encuentre en estado \cdtRef{estado:inscrita}{inscrita}.	    
			\UCli {\bf Interna:} Que el periodo de registro de escuelas se encuentre vigente.
	\end{UClist}
    }
    \UCitem{Postcondiciones}{Ninguna.}
    \UCitem{Reglas de negocio}{Ninguna.}
    \UCitem{Errores}{

     \UCli \cdtIdRef{MSG28}{Operación no permitida por estado de la entidad}: Se muestra sobre la pantalla \cdtIdRef{IUR 5}{Administrar información escolar} indicando al actor que no se puede visualizar la información escolar debido al estado en que se encuentra la escuela.
		\UCli	\cdtIdRef{MSG41}{Acción fuera del periodo}: Se muestra sobre la pantalla \cdtIdRef{IUR 5}{Administrar información escolar} para indicarle al actor que no puede visualizar la información escolar debido a que la fecha actual se encuentra fuera del periodo definido por la SMAGEM para realizar el registro de escuelas.    
    
    }
    \UCitem{Tipo}{Secundario, extiende del caso de uso \cdtIdRef{CUR 5}{Administrar información escolar}.}
\UCitem{Fuente}{
    \begin{UClist}
      \UCli Minuta de la reunión \cdtIdRef{M-3TR}{Toma de requerimientos}.
    \end{UClist}
  }
      \end{UseCase}

 \begin{UCtrayectoria}
    \UCpaso[\UCactor] Solicita visualizar la información de una escuela oprimiendo el botón \botV del registro referente a la escuela, en la pantalla \cdtIdRef{IUR 5}{Administrar información escolar}.
    
    \UCpaso[\UCsist] Verifica que la escuela se encuentre es estado ``Inscrita''. \refTray{A}
    \UCpaso[\UCsist] Verifica que la fecha actual se encuentre dentro del periodo definido por parte de la SMAGEM para el registro de escuelas. \refTray{B}
    
    \UCpaso[\UCsist] Busca los datos de la escuela registrados en el sistema.
    \UCpaso[\UCsist] Muestra la pantalla \cdtIdRef{IUR 8}{Visualizar información escolar} con la información de la escuela.
    \UCpaso[\UCactor] Solicita concluir la visualización oprimiendo el botón \cdtButton{Aceptar} de la pantalla \cdtIdRef{IUR 8}{Visualizar información escolar}.
    \UCpaso[\UCsist] Muestra la pantalla \cdtIdRef{IUR 5}{Administrar información escolar}.
 \end{UCtrayectoria}

%%%%%%%%%

 \begin{UCtrayectoriaA}[Fin del caso de uso]{A}{La escuela no se encuentra en el estado ``Inscrita''}
    \UCpaso[\UCsist] Muestra el mensaje \cdtIdRef{MSG28}{Operación no permitida por estado de la entidad} en la pantalla \cdtIdRef{IUR 5}{Administrar información escolar} indicando al actor que no puede visualizar la información escolar debido a que la escuela no se encuentra en estado ``Inscrita''.
 \end{UCtrayectoriaA}

 \begin{UCtrayectoriaA}[Fin del caso de uso]{B}{La fecha actual se encuentra fuera del periodo definido por la SMAGEM para el registro de escuelas}
    \UCpaso[\UCsist] Muestra el mensaje \cdtIdRef{MSG41}{Acción fuera del periodo} en la pantalla \cdtIdRef{IUR 5}{Administrar información escolar} indicando al actor que no puede visualizar la información escolar debido a que la fecha actual se encuentra fuera del periodo definido por la SMAGEM para realizar la acción.
 \end{UCtrayectoriaA}

