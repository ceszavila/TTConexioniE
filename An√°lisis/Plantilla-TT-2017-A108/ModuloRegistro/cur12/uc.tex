%!TEX encoding = UTF-8 Unicode

% ESTA SECCION LA DEBE LLENAR SOLO EL ANALISTA.
% ID: Asegurese de que el ID del Caso de uso sea único.
% Nombre: Aseurese de que esté escrito de la forma: VERBO + SUSTANTIVO + ALGO
\begin{UseCase}{CUR 12}{Visualizar responsable del programa}
    {
	Después de registrar  al responsable del programa el actor puede consultar su información en forma detallada a través de este caso de uso.
    }
   
	\UCitem{Versión}{1.0}
	\UCccsection{Administración de Requerimientos}
	\UCitem{Autor}{Angélica Madrid Jiménez}
	\UCccitem{Evaluador}{José David Ortega Pacheco}
	\UCitem{Operación}{Consulta}
	\UCccitem{Prioridad}{Alta}
	\UCccitem{Complejidad}{Media}
	\UCccitem{Volatilidad}{Baja}
	\UCccitem{Madurez}{Alta}
    \UCitem{Estatus}{Terminado}
    \UCitem{Fecha del último estatus}{5 de Noviembre del 2014}


    
%% Copie y pegue este bloque tantas veces como revisiones tenga el caso de uso.
%% Esta sección la debe llenar solo el Revisor
% %--------------------------------------------------------
% 	\UCccsection{Revisión Versión XX} % Anote la versión que se revisó.
% 	% FECHA: Anote la fecha en que se terminó la revisión.
% 	\UCccitem{Fecha}{Fecha en que se termino la revisión} 
% 	% EVALUADOR: Coloque el nombre completo de quien realizó la revisión.
% 	\UCccitem{Evaluador}{Nombre de quien revisó}
% 	% RESULTADO: Coloque la palabra que mas se apegue al tipo de acción que el analista debe realizar.
% 	\UCccitem{Resultado}{Corregir, Desechar, Rehacer todo, terminar.}
% 	% OBSERVACIONES: Liste los cambios que debe realizar el Analista.
% 	\UCccitem{Observaciones}{
% 		\begin{UClist}
% 			% PC: Petición de Cambio, describa el trabajo a realizar, si es posible indique la causa de la PC. Opcionalmente especifique la fecha en que considera razonable que se deba terminar la PC. No olvide que la numeración no se debe reiniciar en una segunda o tercera revisión.
% 			\RCitem{PC1}{\TODO{Precondiciones: Que el usuario haya iniciado sesión}}{Fecha de entrega}
% 			\RCitem{PC2}{\TOCHK{Tipo: Secundario, de donde extiende}}{Fecha de entrega}
% 			\RCitem{PC3}{\TODO{Errores. Agregar mensaje de error de o se encontró información registrada }}{Fecha de entrega}
%		\end{UClist}		
% 	}
% %--------------------------------------------------------

    \UCsection{Atributos}
    % ACTOR(ES): Liste los nombres de los actores separados por comas y correctamente referenciados, cambie la etiqueta a Actor o Actores, según sea el caso.
 \UCitem{Actor(es)}{\cdtRef{actor:usuarioEscuela}{Coordinador de programa}}
	\UCitem{Propósito}{Consultar información específica del responsable de programa}
	\UCitem{Entradas}{Ninguna.}
	\UCitem{Salidas}{
	  \begin{UClist} 
		\UCli \cdtRef{persona:nombre}{Nombre(s)}: \ioObtener.
			\UCli \cdtRef{persona:primerApellido}{Primer apellido} \ioObtener.
			\UCli \cdtRef{persona:segundoApellido}{Segundo apellido} \ioObtener.
			\UCli \cdtRef{responsable:cveEmpleado}{Clave de empleado} \ioObtener.
			\UCli \cdtRef{responsable:puesto}{Puesto}: \ioObtener.
			\UCli \cdtRef{persona:correo}{Correo electrónico} \ioObtener.
			\UCli \cdtRef{empleado:telefono}{Teléfono}: \ioObtener.
			\UCli \cdtRef{empleado:extension}{Extensión}: \ioObtener.
		       
		\end{UClist}
	}
	\UCitem{Precondiciones}{
		\begin{UClist}
% 			\UCli {\bf Interna:} Que el actor Coordinador del programa haya iniciado sesión.
			\UCli {\bf Interna:} Que exista un responsable del programa asociado a la escuela registrado en el sistema.
			\UCli {\bf Interna:} Que la escuela se encuentre en estado \cdtRef{estado:inscrita}{inscrita}.	    
			\UCli {\bf Interna:} Que el periodo de registro de escuelas se encuentre vigente.
		\end{UClist}
	}
	\UCitem{Postcondiciones}{Ninguna.	}
	\UCitem{Reglas de negocio}{Ninguna.}
	
	\UCitem{Errores}{
	
     \UCli \cdtIdRef{MSG28}{Operación no permitida por estado de la entidad}: Se muestra sobre la pantalla \cdtIdRef{IUR 5}{Administrar información escolar} indicando al actor que no se puede visualizar al responsable del programa debido al estado en que se encuentra la escuela.
		\UCli	\cdtIdRef{MSG41}{Acción fuera del periodo}: Se muestra sobre la pantalla \cdtIdRef{IUR 5}{Administrar información escolar} para indicarle al actor que no puede visualizar al responsable del programa debido a que la fecha actual se encuentra fuera del periodo definido por la SMAGEM para realizar el registro de escuelas.    	
	
	}

	\UCitem{Tipo}{Secundario, extiende del caso de uso \cdtIdRef{CUR 9}{Administrar responsable del programa}.}

	\UCitem{Fuente}{
    \begin{UClist}
      \UCli Minuta de la reunión \cdtIdRef{M-3TR}{Toma de requerimientos}.
    \end{UClist}
  }

\end{UseCase}

 \begin{UCtrayectoria}
    \UCpaso[\UCactor] Solicita consultar al responsable del programa oprimiendo el botón \botV de la pantalla \cdtIdRef{IUR 9}{Administrar responsable del programa}.
        
    \UCpaso[\UCsist] Verifica que la escuela se encuentre es estado ``Inscrita''. \refTray{A}
    \UCpaso[\UCsist] Verifica que la fecha actual se encuentre dentro del periodo definido por parte de la SMAGEM para el registro de escuelas. \refTray{B}    
    
    \UCpaso[\UCsist] Busca los datos del responsable del programa asociado a la escuela registrados en el sistema. 
    \UCpaso[\UCsist] Muestra los datos del responsable del programa en la pantalla \cdtIdRef{IUR 12}{Visualizar responsable del programa}.
    \UCpaso[\UCactor] Consulta la información del responsable del programa.
    \UCpaso[\UCactor] Concluye la consulta de la información del responsable del programa oprimiendo el botón \cdtButton{Aceptar} de la pantalla \cdtIdRef{IUR 12}{Visualizar responsable del programa}.
%      \UCpaso[\UCsist] Muestra la pantalla \cdtIdRef{IUR 9}{Administrar responsable del programa}.

 \end{UCtrayectoria}

%%%%%%%%%%%%%

 \begin{UCtrayectoriaA}[Fin del caso de uso]{A}{La escuela no se encuentra en el estado ``Inscrita''}
    \UCpaso[\UCsist] Muestra el mensaje \cdtIdRef{MSG28}{Operación no permitida por estado de la entidad} en la pantalla \cdtIdRef{IUR 5}{Administrar información escolar} indicando al actor que no puede visualizar al responsable del programa debido a que la escuela no se encuentra en estado ``Inscrita''.
 \end{UCtrayectoriaA}

 \begin{UCtrayectoriaA}[Fin del caso de uso]{B}{La fecha actual se encuentra fuera del periodo definido por la SMAGEM para el registro de escuelas}
    \UCpaso[\UCsist] Muestra el mensaje \cdtIdRef{MSG41}{Acción fuera del periodo} en la pantalla \cdtIdRef{IUR 5}{Administrar información escolar} indicando al actor que no puede visualizar al responsable del programa debido a que la fecha actual se encuentra fuera del periodo definido por la SMAGEM para realizar la acción.
 \end{UCtrayectoriaA} 
