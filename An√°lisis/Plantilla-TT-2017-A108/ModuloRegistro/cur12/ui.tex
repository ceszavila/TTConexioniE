\subsection{IUR 12 Visualizar responsable del programa}

\subsubsection{Objetivo}

% Explicar el objetivo para el que se construyo la interfaz, generalmente es la descripción de la actividad a desarrollar, como Seleccionar grupos para inscribir materias de un alumno, controlar el acceso al sistema mediante la solicitud de un login y password de los usuarios, etc.
	
    	Esta pantalla permite al actor \cdtRef{actor:usuarioEscuela}{Coordinador del programa} hacer una consulta detallada de la información del responsable del programa que ha sido registrada en el sistema.

\subsubsection{Diseño}

% Presente la figura de la interfaz y explique paso a paso ``a manera de manual de usuario'' como se debe utilizar la interfaz. No olvide detallar en la redacción los datos de entradas y salidas. Explique como utilizar cada botón y control de la pantalla, para que sirven y lo que hacen. Si el Botón lleva a otra pantalla, solo indique la pantalla y explique lo que pasará cuando se cierre dicha pantalla (la explicación sobre el funcionamiento de la otra pantalla estará en su archivo correspondiente).

	En la figura~\ref{IUR 12} se muestra la pantalla ``Visualizar responsable del programa'', en la cual se muestran los datos completos del responsable del programa. 

	\IUfig[.5]{pantallas/registro/IUR12}{IUR 12}{Visualizar responsable del programa}


\subsubsection{Comandos}
    \begin{itemize}
	\item \cdtButton{Aceptar}: Permite al actor concluir la consulta a la información del responsable del programa, dirige a la pantalla \cdtIdRef{IUR 9}{Administrar responsable del programa}.
    \end{itemize}

