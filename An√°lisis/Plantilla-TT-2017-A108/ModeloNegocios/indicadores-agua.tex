% \begin{indicador}{ID}{Nombre del indicador}
% 	{Fórmula}
% 	{
% 		Descripción del indicador.
% 	} 	
% 	\INitem{Variable 1}{Unidad 1}{Descripción de variable 1}
% 	\INitem{Variable 2}{Unidad 2}{Descripción de variable 2}
% \end{indicador}
\section{Indicadores ambientales y de sustentabilidad}

  Los indicadores ambientales y de sustentabilidad son estadísticas que representan o resumen un aspecto significativo del estado del ambiente, 
  la sustentabilidad de los recursos naturales y su relación con las actividades humanas.

\subsection{Agua}

Estos indicadores permitirán medir, monitorear, evaluar y comunicar el impacto del Programa de Acreditación de Escuelas Ambientalmente Responsables que se instrumentará en el Estado de México.\\

En esta sección se muestran los indicadores para la línea de acción ``Agua'' propuestos para el programa.
%------------------------------------------------------------------------------------------------
\begin{indicador}{IA 1}{Consumo anual de agua}
	{$CA_{anual} = CA_{p_1} + CA_{p_2} + ... + CA_{p_n}$} 
	{
		Se refiere al volumen de agua consumida en un año en la escuela.
	} 	
	\INitem{$\cdtRef{gls:consumoAnual}{CA_{anual}}$}{$\frac{m^3}{a\tilde{n}o}$}{\cdtRef{gls:consumoAnual}{Consumo anual de agua}.}
	\INitem{$\cdtRef{periodo-consumo:consumo}{CA_{p_1}}$}{$\frac{m^3}{periodo}$}{\cdtRef{periodo-consumo:consumo}{Consumo} de agua en el primer periodo (mensual, bimestral, etc.).}
	\INitem{$\cdtRef{periodo-consumo:consumo}{CA_{p_2}}$}{$\frac{m^3}{periodo}$}{\cdtRef{periodo-consumo:consumo}{Consumo} de agua en el segundo periodo (mensual, bimestral, etc.).}
	\INitem{$\cdtRef{periodo-consumo:consumo}{CA_{p_n}}$}{$\frac{m^3}{periodo}$}{\cdtRef{periodo-consumo:consumo}{Consumo} de agua en el último periodo (mensual, bimestral, etc.).}
\end{indicador}
%------------------------------------------------------------------------------------------------
\begin{indicador}{IA 2}{Consumo anual de agua por persona}
	{$CA_{anual-persona} = \frac{CA_{anual}}{NP}$}
	{
		La información obtenida es una interpolación del consumo personal de agua, a partir del indicador anterior. Con este dato se podrían generar metas personales para la optimización del consumo del líquido.
	} 	
	\INitem{$\cdtRef{gls:consumoAnualPersona}{CA_{anual-persona}}$}{$\frac{m^3}{persona-a\tilde{n}o}$}{\cdtRef{gls:consumoAnualPersona}{Consumo anual de agua por persona}.}
	\INitem{$\cdtRef{gls:consumoAnual}{CA_{anual}}$}{$\frac{m^3}{a\tilde{n}o}$}{\cdtRef{gls:consumoAnual}{Consumo anual de agua}, calculado con el indicador \cdtIdRef{IA 1}{Consumo anual de agua}.}
	\INitem{$NP$}{$personas$}{Total de personas en la escuela, es la suma de \cdtRef{comunidad:docentesF}{docentes femeninos}, \cdtRef{comunidad:docentesM}{docentes masculinos},
				\cdtRef{comunidad:adminF}{personal administrativo femenino}, \cdtRef{comunidad:adminM}{personal administrativo masculino}, 
				\cdtRef{comunidad:alumnosF}{alumnos femeninos}, \cdtRef{comunidad:alumnosM}{alumnos masculinos},
				\cdtRef{comunidad:limpiezaF}{personal de limpieza y mantenimiento femenino}, \cdtRef{comunidad:limpiezaM}{personal de limpieza y mantenimiento masculino}, 
				\cdtRef{comunidad:apoyoF}{personal de apoyo femenino}, \cdtRef{comunidad:apoyoM}{personal de apoyo masculino}, 
				\cdtRef{comunidad:visitantesF}{visitantes femeninos (promedio diario)} y \cdtRef{comunidad:visitantesM}{visitantes masculinos (promedio diario)}.}
\end{indicador}
%------------------------------------------------------------------------------------------------
\begin{indicador}{IA 3}{Disminución anual en consumo de agua}
	{$DA_{CAgua} = CA_{A\tilde{n}o 2} - CA_{A\tilde{n}o 1}$}
	{
		La reducción de $m^3$ de agua consumida en un año, es una forma de medir cómo las acciones del programa están funcionando. Los resultados negativos indican que hubo una reducción.
	} 	
	\INitem{$\cdtRef{gls:disminucionAnual}{DA_{CAgua}}$}{$\frac{m^3}{a\tilde{n}o}$}{\cdtRef{gls:disminucionAnual}{Disminución anual en consumo de agua}.}
	\INitem{$\cdtRef{gls:consumoAnual}{CA_{A\tilde{n}o 2}}$}{$\frac{m^3}{a\tilde{n}o}$}{\cdtRef{gls:consumoAnual}{Consumo anual de agua} en el segundo año, calculado con el indicador \cdtIdRef{IA 1}{Consumo anual de agua}.}
	\INitem{$\cdtRef{gls:consumoAnual}{CA_{A\tilde{n}o 1}}$}{$\frac{m^3}{a\tilde{n}o}$}{\cdtRef{gls:consumoAnual}{Consumo anual de agua} en el primer año, calculado con el indicador \cdtIdRef{IA 1}{Consumo anual de agua}.}
\end{indicador}
%------------------------------------------------------------------------------------------------
\begin{indicador}{IA 4}{Disminución de emisiones a la atmósfera por reducción en consumo de agua por año}
	{$DE_{CO_2} = DA_{CAgua} * ICE * FEEE$}
	{
		Al disminuir el consumo de agua, se estima que hay una disminución en el bombeo de esta, asociado al consumo se ha calculado un valor de emisiones de $CO_2$  en un periodo de tiempo. 
		Esta medida puede asociarse a indicadores de combate a los efectos del cambio climático. Los resultados negativos indican que hubo una reducción.
	} 	
	\INitem{$\cdtRef{gls:disminucionEmisiones}{DE_{CO_2}}$}{$\frac{ton CO_2}{a\tilde{n}o}$}{\cdtRef{gls:disminucionEmisiones}{Disminución de emisiones de $CO_2$}.}
	\INitem{$\cdtRef{gls:disminucionAnual}{DA_{CAgua}}$}{$\frac{m^3}{a\tilde{n}o}$}{\cdtRef{gls:disminucionAnual}{Disminución anual en consumo de agua}, calculada con el indicador \cdtIdRef{IA 3}{Disminución anual en consumo de agua}.}
	\INitem{$\cdtRef{gls:ice}{ICE}$}{$\frac{kWh}{m^3}$}{\cdtRef{gls:ice}{Índice de consumo energético}, su valor es igual a $1.32 \frac{kWh}{m^3}$.}
	\INitem{$\cdtRef{gls:feee}{FEEE}$}{$\frac{kWh}{ton CO_2}$}{\cdtRef{gls:feee}{Factor de emisión de energía eléctrica}, su valor es igual a $0.000667 \frac{kWh}{ton CO_2}$.}
\end{indicador}
%------------------------------------------------------------------------------------------------