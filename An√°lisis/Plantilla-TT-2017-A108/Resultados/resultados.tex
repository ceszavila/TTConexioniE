\section{Resultados}

En este capítulo se muestran los avances obtenidos, los cuales comprenden los siguientes módulos:\\

\begin{UClist} 
	\UCli Salones: En él, los actores son capaces de ubicar los edificios de la Escuela y obtener su información. De la misma manera se pueden ubicar los salones desde la pantalla de asignación de salones.
	\UCli Profesores: En este módulo, se puede obtener la lista de profesores registrados en la aplicación, así como el detalle de la información de contacto y ubicación.
\end{UClist} 

\subsection{Módulo de Salones}

En esta sección se puede observar el avance en la navegación y usabilidad de la aplicación como se muestra a continuación:

%pantallas de la aplicación y si explicación

\subsection{Módulo de Profesores}

En esta sección se puede observar el avance en la navegación y usabilidad de la aplicación como se muestra a continuación:

%pantallas de la aplicación y si explicación

\section{Trabajo a Futuro}

Los módulos que están propuestos como trabajo a futuro son los sieguientes: \\

\begin{UClist} 
	\UCli Material de apoyo: 
	Este módulo comprenderá el material que profesores deseen subir de su misma autoriá o las ligas de repositorios gratuitos como ayuda en el material de las unidades de aprendizaje.\\
	
	\UCli Trámites: 
	Este módulo comprenderá la información necesaria para realizar algunos trámites dentro de la escuela como son: el servicio social y la movilidad estudiantil.\\
	
	\UCli Cursos y certificaciones: 
	Este módulo facilitará la información y publicación de convocatorias con la finalidad de aumentar el número de alumnos que se inscriben a dichas actividades extracurriculares.\\
\end{UClist} 

Además de los módulos mencionados, se realizará la programación del Backend con los Servicios Web necesarios para el correcto funcionamiento de la aplicación y se complementará el presente documento con el trabajo realizado hasta finalizar con el cronograma y lo propuesto como Trabajo Terminal.

\section{Conclusiones}

El presente Trabajo Terminal tuvo como objetivo proponer e implementar un prototipo de aplicación que permita a la comunidad consultar los diferentes espacios con los que cuenta la Escuela Superior de Cómputo. Esto se logró con el uso de tecnologías disponibles para su consulta, tales como Mapbox que aunque no fue una tecnología utilizada fue de grana ayuda para comprender el modelado de los polígonos con el fin de dibujar los edificios. La API de Google Maps como consulta también y diversos ejemplos de aplicaciones para el entorno de desarrollo de Apple.\\

Con este trabajo damos un paso importante en la especialización que ambos integrantes queremos realizar en el desarrollo de aplicaciones móviles para dispositivos con sistema operativo iOS. De la misma manera satisfacemos parcialmente el gran interés por la ingeniería de software, sabiendo a donde dirigir nuestro futuro profesional.