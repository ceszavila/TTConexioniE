En este capítulo definimos el objetivo general de la aplicación propuesta como solución a las problemáticas identificadas, los objetivos particulares que proponemos para resolverlas y el ánalisis de requerimientos que son necesarios para el desarrollo de la aplicación.

\section{Objetivo General}

	Desarrollar una aplicación móvil, que permita la consulta de horario, la ubicación y contacto del personal docente, la consulta del listado de asignaturas por grupo, además de brindar las noticias académicas de la Escuela Superior de Cómputo para apoyar a los alumnos con mejor comunicación en información académica.
	
	\section{Objetivos Particulares}	
	
	
	%=============================================
	
	\begin{UClist}
		\UCli Identificar las áreas de la Escuela con el fin de mostrar su información en la aplicación.
		\UCli Ayudar al \cdtRef{actor:CIEAlumno}{Alumno} de nuevo ingreso a ubicar sus salones mediante los grupos asignados en los primeros días de clase.
		\UCli Ayudar al \cdtRef{actor:CIEAlumno}{Alumno} a ubicar salones, salas o cubículos dentro de la ESCOM.
		\UCli Ofrecer información sobre los cursos, certificaciones y trámites que se pueden realizar en la ESCOM.
		\UCli Ofrecer información básica sobre las unidades de aprendizaje que se imparten en la Escuela.
		\UCli Ofrecer un espacio para material de apoyo de la autoría de los profesores o repositorios gratuitos con la finalidad de tener información sobre materiales de apoyo para las clases.
	\end{UClist}
	
	\section{Requerimientos}
	\subsection{Requerimientos funcionales}
	
	Los requerimientos funcionales para un sistema explican lo que el sistema debe hacer. Los requerimientos dependen del tipo de software en desarrollo, de los usuarios y del enfoque general que se tiene al escribir los requerimientos. Al ser los requerimientos del usuario, los requerimientos funcionales se describen por lo general de forma abstracta que entiendan los usuarios del sistema. \cite{15}
	
	\cfinput{reqfun}
	
	\subsection{Requerimientos no funcionales}
	
	Los requerimientos no funcionales, como su nombre lo dice, son requerimientos que no se vinculan directamente con las funciones específicas que el sistema proporciona. Pueden relacionarse con propiedades emergentes del sistema, como fiabilidad, tiempos de respuesta y capacidad de almacenamiento. De forma alternativa, pueden definir restricciones sobre la implementación del sistema, como las capacidades de los dispositivos I/O a las representaciones de datos usados en las interfaces con otros sistemas. \cite{15}
	
		\cfinput{reqnofun}