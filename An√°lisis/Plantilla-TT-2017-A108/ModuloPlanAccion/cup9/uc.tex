\begin{UseCase}{CUP 9}{Modificar acción}
	{
		Este caso de uso permite al actor modificar una \cdtRef{gls:accion}{acción} cuando existen errores o actualizaciones en la información. 
		El actor podrá visualizar la información previamente registrada e ingresar la nueva información, cuando los datos solicitados para la acción
		sean ingresados serán validados y se realizará la actualización de la información.
	}
	\UCitem{Versión}{1.0}
	\UCccsection{Administración de Requerimientos}	
	\UCitem{Autor}{	Sergio Ramírez Camacho}	
	\UCccitem{Evaluador}{}
	\UCitem{Operación}{Modificación}
	\UCccitem{Prioridad}{Alta}
	\UCccitem{Complejidad}{Media}
	\UCccitem{Volatilidad}{Media}
	\UCccitem{Madurez}{Media}
	\UCitem{Estatus}{Terminado} 
	\UCitem{Fecha del último estatus}{27 de noviembre de 2014}

%% Copie y pegue este bloque tantas veces como revisiones tenga el caso de uso.
%% Esta sección la debe llenar solo el Revisor
% %--------------------------------------------------------
 	\UCccsection{Revisión versión 0.1} 
 	\UCccitem{Fecha}{1-Dic}
 	\UCccitem{Evaluador}{Nayeli Vega}
 	\UCccitem{Resultado}{Corregir}
 	\UCccitem{Observaciones}{
 		\begin{UClist}
 			\RCitem{PC1}{\DONE{Trayectoria principal: Paso 1. El botón para solicitar la operación está mal}}{}
 			\RCitem{PC2}{\DONE{Trayectoria principal: Paso 1. No existe trayectoria principal}}{}
 			\RCitem{PC3}{\DONE{Trayectoria principal: Paso 5. La referencia a la pantalla está mal}}{}
 			\RCitem{PC3}{\DONE{Trayectoria D: La referencia a la pantalla está mal, es a la 9}}{} 
 			\RCitem{PC3}{\DONE{Trayectoria E: La referencia a la pantalla está mal, es a la 9}}{}  			
 			\RCitem{PC3}{\DONE{Trayectoria F: La referencia a la pantalla está mal, es a la 9}}{}  			 			
 		\end{UClist}		
 	}
% %--------------------------------------------------------

	\UCsection{Atributos}
	\UCitem{Actor}{\cdtRef{actor:usuarioEscuela}{Coordinador del programa}}
	\UCitem{Propósito}{Modificar una acción cuando se detectan errores o actualizaciones en la información.}
	\UCitem{Entradas}{
	  \begin{UClist}
	   \UCli De la sección ``Información de la acción':
	      \begin{Citemize}
		\item \cdtRef{accion:descripcion}{Descripción}: \ioEscribir.
		\item \cdtRef{accion:recursos}{Recursos materiales y financieros}: \ioEscribir.
		\item \cdtRef{accion:cuantificable}{Es cuantificable}: \ioRadioBoton. 
		\item \cdtRef{accion:valorAlcanzar}{Valor a alcanzar}: \ioEscribir.
		\item \cdtRef{accion:unidad}{Unidad}: \ioEscribir. 
	      \end{Citemize}
	   
	  \end{UClist}
	}	
	
	\UCitem{Salidas}{
		\begin{UClist} 
		\UCli \cdtRef{accion:descripcion}{Descripción}: \ioObtener.
		\UCli \cdtRef{accion:recursos}{Recursos materiales y financieros}: \ioObtener.
		\UCli \cdtRef{accion:cuantificable}{Es cuantificable}: \ioObtener. 
		\UCli \cdtRef{accion:valorAlcanzar}{Valor a alcanzar}: \ioObtener.
		\UCli \cdtRef{accion:unidad}{Unidad}: \ioObtener. 
		\UCli \cdtIdRef{MSG1}{Operación realizada exitosamente}: Se muestra en la pantalla \cdtIdRef{IUP 7}{Administrar acciones} cuando la acción se ha registrado correctamente.
	\end{UClist}
	}
	\UCitem{Precondiciones}{
		\UCli {\bf Interna:} Que la escuela se encuentre en estado \cdtRef{estado:planEdicion}{Plan de acción en edición}.
		\UCli {\bf Interna:} Que el periodo de registro de plan de acción se encuentre vigente.
	}
	
	\UCitem{Postcondiciones}{
		\begin{UClist}
 			\UCli {\bf Interna:} Se modificará la acción en el sistema.
		\end{UClist}
	}
	
	\UCitem{Reglas de negocio}{
		\begin{UClist}
			\UCli \cdtIdRef{RN-S1}{Información correcta}: Verifica que la información ingresada por el actor sea correcta.
		\end{UClist}
	}
	
\UCitem{Errores}{
		\begin{UClist}	
			\UCli \cdtIdRef{MSG5}{Falta un dato requerido para efectuar la operación solicitada}: Se muestra en la pantalla \cdtIdRef{IUP 9}{Modificar acción} cuando el actor no ingresó un dato requerido para realizar la operación.
			\UCli \cdtIdRef{MSG6}{Formato incorrecto}: Se muestra en la pantalla \cdtIdRef{IUP 9}{Modificar acción} cuando el formato de alguno de los datos ingresados es incorrecto.
 			\UCli \cdtIdRef{MSG7}{Se ha excedido la longitud máxima del campo}: Se muestra en la pantalla \cdtIdRef{IUP 9}{Modificar acción} cuando el actor escribió un dato que excede el tamaño especificado por el sistema.
		\UCli \cdtIdRef{MSG28}{Operación no permitida por estado de la entidad}: Se muestra en la pantalla en que se encuentre navegando el actor debido al estado en que se encuentra la escuela.
\UCli \cdtIdRef{MSG41}{Acción fuera del periodo}: Se muestra en la pantalla en que se encuentre navegando el actor indicando que la fecha no se encuentra dentro del periodo de registro de plan de acción.
		\end{UClist}
	}
	\UCitem{Tipo}{Secundario, extiende del caso de uso \cdtIdRef{CUP 7}{Administrar acciones}}

% 	\UCitem{Fuente}{
% 		\begin{UClist}
% 			\UCli 
% 		\end{UClist}
% 	}

 \end{UseCase}
 
 
\begin{UCtrayectoria}
	\UCpaso[\UCactor] Solicita modificar una acción oprimiendo el botón \botEdit en la pantalla \cdtIdRef{IUP 7}{Administrar acciones}. \refTray{A}

	\UCpaso[\UCsist] Verifica que la escuela se encuentre en  estado ``Plan de acción en edición''. \refTray{B}.
    \UCpaso[\UCsist] Verifica que la fecha actual se encuentre dentro del periodo definido por la SMAGEM para el registro del plan de acción. \refTray{C}.
	
	\UCpaso[\UCsist] Busca la información de la acción.
	\UCpaso[\UCsist] Muestra la información de la acción en la pantalla \cdtIdRef{IUP 9}{Modificar acción}.
	\UCpaso[\UCactor] Ingresa los datos nuevos de la acción en la pantalla \cdtIdRef{IUP 9}{Modificar acción}.\label{cup9:ingresaDatos} 
	\UCpaso[\UCactor] Selecciona que la acción ingresada es cuantificable en la pantalla \cdtIdRef{IUP 9}{Modificar acción}. \refTray{D} \label{cup9:selecciona}
	\UCpaso[\UCsist] Muesta la pantalla \cdtIdRef{IUP 9.1}{Modificar acción cuantificable} por medio de la cual se realizará la modificación de la acción cuantificable.
	\UCpaso[\UCactor] Ingresa los nuevos nuevos datos para el  ``Valor a alcanzar''  y la``Unidad''.
	\UCpaso[\UCactor] Solicita la modificación de la acción oprimiendo el botón \cdtButton{Aceptar} de la pantalla \cdtIdRef{IUP 9}{Modificar acción}. \refTray{E} \label{cup9:solicita}
	\UCpaso[\UCsist] Verifica que la escuela se encuentre en  estado ``Plan de acción en edición''. \refTray{B}.	
	\UCpaso[\UCsist] Verifica que los datos ingresados por el actor sean correctos como lo indica la regla de negocio \cdtIdRef{RN-S1}{Información correcta}. \refTray{F} \refTray{G} \refTray{H}
	\UCpaso[\UCsist] Registra las modificaciones de la acción en el sistema. 
	\UCpaso[\UCsist] Muestra el mensaje \cdtIdRef{MSG1}{Operación realizada exitosamente} en la pantalla \cdtIdRef{IUP 7}{Administrar acciones} indicando que se ha realizado la modificación de la acción satisfactoriamente.
\end{UCtrayectoria}

\begin{UCtrayectoriaA}[Fin del caso de uso]{A}{El actor desea regresar.}
	\UCpaso[\UCactor] Solicita regresar oprimiendo el botón \cdtButton{Regresar} de la pantalla \cdtIdRef{IUP 7}{Administrar acciones}.
	\UCpaso[\UCsist] Muestra la pantalla  \cdtIdRef{IUP 5}{Administrar metas}.
\end{UCtrayectoriaA} 

\begin{UCtrayectoriaA}[Fin del caso de uso]{B}{La escuela no se encuentra en el estado ``Plan de acción en edición''.}
    \UCpaso[\UCsist] Muestra el mensaje \cdtIdRef{MSG28}{Operación no permitida por estado de la entidad} en la pantalla en que se encuentre navegando el actor indicando que no puede administrar los objetivos del plan de acción debido a que la escuela no se encuentra en el estado ``Plan de acción en edición''. 
 \end{UCtrayectoriaA}
 
   \begin{UCtrayectoriaA}[Fin del caso de uso]{C}{La fecha actual se encuentra fuera del periodo definido por la SMAGEM para el registro del plan de acción}
    \UCpaso[\UCsist] Muestra el mensaje \cdtIdRef{MSG41}{Acción fuera del periodo} en la pantalla en que se encuentre navegando el actor indicando que no puede administrar los objetivos del plan de acción debido a que la fecha actual se encuentra fuera del periodo definido por la SMAGEM para el registro del plan de acción.
 \end{UCtrayectoriaA}


\begin{UCtrayectoriaA}{D}{El actor selecciona que la acción no es cuantificable.}
	\UCpaso[] Continúa con el paso \ref{cup9:solicita} de la trayectoria principal.
\end{UCtrayectoriaA} 


\begin{UCtrayectoriaA}[Fin del caso de uso]{E}{El actor desea cancelar la operación.}
	\UCpaso[\UCactor] Solicita cancelar la operación oprimiendo el botón \cdtButton{Cancelar} de la pantalla \cdtIdRef{IUP 9}{Modificar acción}.
	\UCpaso[\UCsist] Muestra la pantalla  \cdtIdRef{IUP 7}{Administrar acciones}.
\end{UCtrayectoriaA} 

\begin{UCtrayectoriaA}{F}{El actor no proporcionó alguno de los datos requeridos.}
	\UCpaso[\UCsist] Muestra el mensaje \cdtIdRef{MSG5}{Falta un dato requerido para efectuar la operación solicitada} en la pantalla  \cdtIdRef{IUP 9}{Modificar acción},
	indicando que no se puede modificar la acción debido a la omisión de un dato requerido y señala los campos faltantes.
	\UCpaso[] Continúa en el paso \ref{cup9:ingresaDatos} de la trayectoria principal.
\end{UCtrayectoriaA}
 
\begin{UCtrayectoriaA}{G}{El actor proporcionó un dato que excede la longitud máxima.}
	\UCpaso[\UCsist] Muestra el mensaje \cdtIdRef{MSG7}{Se ha excedido la longitud máxima del campo} en la pantalla \cdtIdRef{IUP 9}{Modificar acción},
	indicando que no se puede modificar la acción debido a que se ha excedido la longitud máxima permitida de algún dato y señala los campos que excedieron la longitud.
	\UCpaso[] Continúa en el paso \ref{cup9:ingresaDatos} de la trayectoria principal.
\end{UCtrayectoriaA}

\begin{UCtrayectoriaA}{H}{El actor ingresó un dato en formato incorrecto.}
	\UCpaso[\UCsist] Muestra el mensaje \cdtIdRef{MSG6}{Formato incorrecto} en la pantalla \cdtIdRef{IUP 9}{Modificar acción},
	indicando que no se puede modificar la acción debido a que se ingresó un dato con formato incorrecto y señala los campos incorrectos.
        \UCpaso[] Continúa en el paso \ref{cup9:ingresaDatos} de la trayectoria principal. 
  \end{UCtrayectoriaA}
 