%!TEX encoding = UTF-8 Unicode

\begin{UseCase}{CUPRS 3}{Administrar residuos sólidos del plan de acción}
    {
	Los residuos sólidos del plan de acción son aquellos materiales que se planean desechar una vez que ha finalizado su vida útil, son desechos procedentes de materiales utilizados en la fabricación, transformación o utilización de bienes de consumo. 
	Este caso de uso permite al actor administrar los residuos sólidos que planea reducir o reclicar, además sirve como punto de acceso para registrar y eliminar registros de residuos sólidos del \cdtRef{gls:planAccion}{plan de acción}.
    }
    
    \UCitem{Versión}{1.0}
    \UCccsection{Administración de Requerimientos}
    \UCitem{Autor}{Sergio Ramírez Camacho}
    \UCccitem{Evaluador}{}
    \UCitem{Operación}{Administración}
    \UCccitem{Prioridad}{Media}
    \UCccitem{Complejidad}{Media}
    \UCccitem{Volatilidad}{Alta}
    \UCccitem{Madurez}{Media}
    \UCitem{Estatus}{Terminado}
    \UCitem{Fecha del último estatus}{1 de diciembre del 2014}
    
%% Copie y pegue este bloque tantas veces como revisiones tenga el caso de uso.
%% Esta sección la debe llenar solo el Revisor
% %--------------------------------------------------------
 	\UCccsection{Revisión Versión 0.1} % Anote la versión que se revisó.
% 	% FECHA: Anote la fecha en que se terminó la revisión.
 	\UCccitem{Fecha}{4-Dic} 
% 	% EVALUADOR: Coloque el nombre completo de quien realizó la revisión.
 	\UCccitem{Evaluador}{Nayeli Vega}
% 	% RESULTADO: Coloque la palabra que mas se apegue al tipo de acción que el analista debe realizar.
 	\UCccitem{Resultado}{Corregir}
% 	% OBSERVACIONES: Liste los cambios que debe realizar el Analista.
 	\UCccitem{Observaciones}{
 		\begin{UClist}
% 			% PC: Petición de Cambio, describa el trabajo a realizar, si es posible indique la causa de la PC. Opcionalmente especifique la fecha en que considera razonable que se deba terminar la PC. No olvide que la numeración no se debe reiniciar en una segunda o tercera revisión.
 			\RCitem{PC1}{\TOCHK{Salidas: Tipo y Total semanal, las rotas}}{Fecha de entrega}
 			\RCitem{PC2}{\TOCHK{Salidas: IUPRS 3 Liga rota}}{Fecha de entrega}
 			\RCitem{PC3}{\TOCHK{Precondiciones: Enfoque, liga rota}}{Fecha de entrega}
 		\end{UClist}		
 	}
% %--------------------------------------------------------

    \UCsection{Atributos}
    \UCitem{Actor}{\cdtRef{actor:usuarioEscuela}{Coordinador del programa}}
    \UCitem{Propósito}{Administrar el registro y la eliminación de los residuos sólidos a reciclar o a disminuir.}
    \UCitem{Entradas}{
	\begin{UClist}
	    \UCli Ninguna
	\end{UClist}
    }
    \UCitem{Salidas}{
	\begin{UClist} 
	    \UCli \cdtRef{residuoSolidoMeta}{Residuo sólido del plan de acción}: \ioTabla{el \cdtRef{gls:tipoDeResiduo}{Tipo}, el \cdtRef{gls:residuo}{Residuo} y el \cdtRef{residuoSolidoMeta:cantidad}{Total semanal (kg/semana)}}{que estén en el sistema}.
	    \UCli \cdtIdRef{MSG2}{No existe información registrada por el momento}: Se muestra en la pantalla \cdtIdRef{IUPRS 1}{Registrar meta de residuos sólidos} o \cdtIdRef{IUPRS 2}{Modificar meta de residuos sólidos} indicando al actor que no existen registros de residuos sólidos del plan de acción en el sistema por el momento.
	\end{UClist}
    }

    \UCitem{Precondiciones}{
	\begin{UClist}
\UCli {\bf Interna:} Que la escuela se encuentre en estado \cdtRef{estado:planEdicion}{Plan de acción en edición}.
			\UCli {\bf Interna:} Que el periodo de registro de plan de acción se encuentre vigente.	
	    \UCli Que el \cdtRef{gls:enfoqueMeta}{enfoque de la meta} que se está registrando sea ``Reducción o reciclaje de residuos''.
	\end{UClist}
    }
    
    \UCitem{Postcondiciones}{
	\begin{UClist}
	    \UCli {\bf Interna:} Se podrá registrar un residuo sólido del plan de acción a través del caso de uso \cdtIdRef{CUPR 1}{Registrar residuo sólido}.
	    \UCli {\bf Interna:} Se podrá eliminar un residuo sólido del plan de acción a través del caso de uso \cdtIdRef{CUPR 2}{Eliminar residuo sólido}.
	\end{UClist}
    }
    
    \UCitem{Reglas de negocio}{
    	\begin{UClist}
	    \UCli Ninguna
	\end{UClist}
    }
    
    \UCitem{Errores}{
	\begin{UClist}
\UCli \cdtIdRef{MSG28}{Operación no permitida por estado de la entidad}: Se muestra en la pantalla en que se encuentre navegando el actor debido al estado en que se encuentra la escuela.
\UCli \cdtIdRef{MSG41}{Acción fuera del periodo}: Se muestra en la pantalla en que se encuentre navegando el actor indicando que la fecha no se encuentra dentro del periodo de registro de plan de acción.
	\end{UClist}
    }

    \UCitem{Tipo}{Secundario, extiende del caso de uso \cdtIdRef{CUPRS 1}{Registrar meta de residuos sólidos} o del caso de uso \cdtIdRef{CUPRS 2}{Modificar meta de residuos sólidos}.}


%    \UCitem{Fuente}{
%	\begin{UClist}
%	    \UCli Minuta de la reunión \cdtIdRef{M-17RT}{Reunión de trabajo}.
%	\end{UClist}
 %   }
\end{UseCase}

 \begin{UCtrayectoria}
 \UCpaso[\UCsist] Verifica que la escuela se encuentre en  estado ``Plan de acción en edición''. \refTray{A}.
    \UCpaso[\UCsist] Verifica que la fecha actual se encuentre dentro del periodo definido por la SMAGEM para el registro del plan de acción. \refTray{B}.
    \UCpaso[\UCsist] Busca la información de los residuos sólidos del plan de acción registrados en el sistema. \refTray{C}.
    \UCpaso[\UCsist] Muestra la tabla de residuos sólidos del plan de acción en la pantalla \cdtIdRef{IUPRS 1}{Registrar meta de residuos sólidos} o en la pantalla \cdtIdRef{IUPRS 2}{Modificar meta de residuos sólidos}.
    \UCpaso[\UCactor] Administra los residuos sólidos del plan de acción a través de los botones \cdtButton{Registrar} y \botKo  . \label{cuprs3:Registrar}
 \end{UCtrayectoria}
 
\begin{UCtrayectoriaA}[Fin del caso de uso]{A}{La escuela no se encuentra en el estado ``Plan de acción en edición''.}
    \UCpaso[\UCsist] Muestra el mensaje \cdtIdRef{MSG28}{Operación no permitida por estado de la entidad} en la pantalla en que se encuentre navegando el actor indicando que no puede administrar los objetivos del plan de acción debido a que la escuela no se encuentra en el estado ``Plan de acción en edición''. 
 \end{UCtrayectoriaA}
 
   \begin{UCtrayectoriaA}[Fin del caso de uso]{B}{La fecha actual se encuentra fuera del periodo definido por la SMAGEM para el registro del plan de acción}
    \UCpaso[\UCsist] Muestra el mensaje \cdtIdRef{MSG41}{Acción fuera del periodo} en la pantalla en que se encuentre navegando el actor indicando que no puede administrar los objetivos del plan de acción debido a que la fecha actual se encuentra fuera del periodo definido por la SMAGEM para el registro del plan de acción.
 \end{UCtrayectoriaA}
 
 
  \begin{UCtrayectoriaA}[Fin del caso de uso]{C}{No hay registros de residuos sólidos del plan de acción para mostrar.}
    \UCpaso[\UCsist] Muestra el mensaje \cdtIdRef{MSG2}{No existe información registrada por el momento} en la tabla de residuos sólidos del plan de acción de la pantalla \cdtIdRef{IUPRS 1}{Registrar meta de residuos sólidos} o de la pantalla \cdtIdRef{IUPRS 2}{Modificar meta de residuos sólidos} 
		     indicando al actor que aún no hay residuos sólidos del plan de acción registrados. 
 \end{UCtrayectoriaA}


\subsection{Puntos de extensión}

\UCExtensionPoint
{El actor desea registrar un residuo sólido del plan de acción}
{ Paso \ref{cuprs3:Registrar} de la trayectoria principal}
{\cdtIdRef{CUPR 1}{Registrar residuo sólido}}

\UCExtensionPoint
{El actor desea eliminar un residuo sólido del plan de acción}
{ Paso \ref{cuprs3:Registrar} de la trayectoria principal}
{\cdtIdRef{CUPR 2}{Eliminar residuo sólido}}
