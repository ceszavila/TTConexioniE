%!TEX encoding = UTF-8 Unicode

\begin{UseCase}{CUP 11}{Enviar plan de acción}
	{
		El plan de acción es el conjunto de actividades que la escuela propone para mejorar su situación actual en cada línea de acción.
		Cuando el actor registre cada objetivo, meta y acción del plan de acción puede enviar la información a la SMAGEM por medio de este caso de uso.
		Una vez aceptado el envío de la información, el actor no podrá realizar cambios en el plan de acción.
	}
	%\UCitem{\DONEUC}{Edición}
	\UCitem{Versión}{1.0}
	\UCccsection{Administración de Requerimientos}	
	\UCitem{Autor}{Natalia Giselle Hernández Sánchez}	
	\UCccitem{Evaluador}{}
	\UCitem{Operación}{Administración}
	\UCccitem{Prioridad}{Alta}
	\UCccitem{Complejidad}{Media}
	\UCccitem{Volatilidad}{Alta}
	\UCccitem{Madurez}{Baja}
	\UCitem{Estatus}{Terminado}
	\UCitem{Fecha del último estatus}{04 de diciembre de 2014}
%% Copie y pegue este bloque tantas veces como revisiones tenga el caso de uso.
%% Esta sección la debe llenar solo el Revisor

%% Copie y pegue este bloque tantas veces como revisiones tenga el caso de uso.
%% Esta sección la debe llenar solo el Revisor
% %--------------------------------------------------------
 	\UCccsection{Revisión Versión 0.1} % Anote la versión que se revisó.
% 	% FECHA: Anote la fecha en que se terminó la revisión.
 	\UCccitem{Fecha}{4-Dic} 
% 	% EVALUADOR: Coloque el nombre completo de quien realizó la revisión.
 	\UCccitem{Evaluador}{Nayeli Vega}
% 	% RESULTADO: Coloque la palabra que mas se apegue al tipo de acción que el analista debe realizar.
 	\UCccitem{Resultado}{Corregir}
% 	% OBSERVACIONES: Liste los cambios que debe realizar el Analista.
 	\UCccitem{Observaciones}{
 		\begin{UClist}
% 			% PC: Petición de Cambio, describa el trabajo a realizar, si es posible indique la causa de la PC. Opcionalmente especifique la fecha en que considera razonable que se deba terminar la PC. No olvide que la numeración no se debe reiniciar en una segunda o tercera revisión.
 			\RCitem{PC1}{\TOCHK{Resumen: Yo sugiero, Una vez aceptado en envío de la información, ...}}{Fecha de entrega}
 			\RCitem{PC2}{\TOCHK{Reglas de negocio: La primera regla no tiene el identificador}}{Fecha de entrega}

% 			\RCitem{PC3}{\TODO{Descripción del pendiente}}{Fecha de entrega}
 		\end{UClist}		
 	}
% %--------------------------------------------------------


	\UCsection{Atributos}
	\UCitem{Actor}{\cdtRef{actor:usuarioEscuela}{Coordinador del programa}}
	\UCitem{Propósito}{
		Enviar el plan de acción para que sea revisado por el \cdtRef{actor:usuarioSMAGEM}{Director del programa} en la SMAGEM.
	}
	\UCitem{Entradas}{
		Ninguna
	}
	\UCitem{Salidas}{
		\begin{UClist}
			\UCli \cdtIdRef{MSG34}{Confirmación de envío de información}: Se muestra sobre la pantalla en que esté navegando el actor para indicarle que una vez enviada la información no podrá modificarla posteriormente.
			\UCli \cdtIdRef{MSG1}{Operación realizada exitosamente}: Se muestra sobre la pantalla en que esté navegando el actor cuando el plan de acción se ha enviado correctamente.
		\end{UClist}
	}

	\UCitem{Precondiciones}{
		\begin{UClist}
			\UCli {\bf Interna:} Que la escuela se encuentre en estado \cdtRef{estado:planEdicion}{Plan de acción en edición}.
			\UCli {\bf Interna:} Que el periodo de registro de plan de acción se encuentre vigente.
			\UCli {\bf Interna:} Que la escuela haya registrado al menos un objetivo, una meta y una acción.
		\end{UClist}
	}
	
	\UCitem{Postcondiciones}{
		\begin{UClist}
			\UCli {\bf Interna:} El estado de la escuela será \cdtRef{estado:planAprobar}{Plan de acción por aprobar}.
		\end{UClist}
	}

	\UCitem{Reglas de \hspace{1 cm} negocio}{
		\begin{UClist}
			\UCli \cdtIdRef{RN-N15}{Restricción de envío del plan de acción}: Indica las condiciones para que el actor pueda enviar el plan de acción.
		\end{UClist}
	}

	\UCitem{Errores}{
		\begin{UClist}
			\UCli \cdtIdRef{MSG28}{Operación no permitida por estado de la entidad}: Se muestra sobre la pantalla en que esté navegando el actor para indicarle que no se puede enviar el plan de acción debido al estado en que se encuentra.
			\UCli \cdtIdRef{MSG39}{La información ya ha sido enviada}: Se muestra sobre la pantalla en que esté navegando el actor para indicarle que no puede enviar el plan de acción debido a que ya ha sido enviada.
			\UCli \cdtIdRef{MSG40}{Falló el envío del plan de acción}: Se muestra sobre la pantalla en que esté navegando el actor para indicarle que no puede enviar el plan de acción debido a que es necesario que registre al menos un objetivo.
			\UCli \cdtIdRef{MSG41}{Acción fuera del periodo}: Se muestra en la pantalla en que se encuentre navegando el actor indicando que la fecha no se encuentra dentro del periodo de registro de plan de acción.
		\end{UClist}
	}

	\UCitem{Tipo}{
		Primario
	}

% 	\UCitem{Fuente}{
% 		\begin{UClist}
% 			\UCli %Minuta de la reunión \cdtIdRef{M-15TR}{Toma de Requerimientos}.
% 		\end{UClist}
% 	}
	
\end{UseCase}
%-------------------------------------------------------%trayectoria Principal-----------------------------------------------
  \begin{UCtrayectoria}
    \UCpaso[\UCactor] Solicita enviar el plan de acción, seleccionando en el menú \cdtIdRef{MN2}{Menú del Coordinador del programa} la opción ``Plan de acción'' y posteriormente la opción ``Enviar plan de acción''. 
	\UCpaso[\UCsist] Verifica que la escuela se encuentre en  estado ``Plan de acción en edición''. \refTray{A}
    \UCpaso[\UCsist] Verifica que la fecha actual se encuentre dentro del periodo definido por la SMAGEM para el registro del plan de acción. \refTray{B}
    \UCpaso[\UCsist] Verifica que exista al menos un objetivo registrado. \refTray{C}
    \UCpaso[\UCsist] Muestra el mensaje \cdtIdRef{MSG34}{Confirmación de envío de información} en una pantalla emergente.
    \UCpaso[\UCactor] Solicita confirmar el envío de la información oprimiendo el botón \cdtButton{Aceptar} en una pantalla emergente. \refTray{D} \refTray{E}
	\UCpaso[\UCsist] Verifica que la escuela se encuentre en  estado ``Plan de acción en edición''. \refTray{A}
    \UCpaso[\UCsist] Envía la información para que sea revisada por el Director del programa.
    \UCpaso[\UCsist] Cambia el estado de el plan de acción a ``Plan de acción por aprobar ''.
    \UCpaso[\UCsist] Muestra el mensaje \cdtIdRef{MSG1}{Operación realizada exitosamente} sobre la pantalla en que se encuentre navegando el actor para indicar que el plan de acción ha sido enviado a revisión exitosamente.
 \end{UCtrayectoria}
 
\begin{UCtrayectoriaA}[Fin del caso de uso]{A}{La escuela no se encuentra en el estado ``Plan de acción en edición''.}
    \UCpaso[\UCsist] Muestra el mensaje \cdtIdRef{MSG28}{Operación no permitida por estado de la entidad} en la pantalla en que se encuentre navegando el actor indicando que no puede enviar el plan de acción debido a que la escuela no se encuentra en el estado ``Plan de acción en edición''. 
 \end{UCtrayectoriaA}
 
   \begin{UCtrayectoriaA}[Fin del caso de uso]{B}{La fecha actual se encuentra fuera del periodo definido por la SMAGEM para el registro del plan de acción.}
    \UCpaso[\UCsist] Muestra el mensaje \cdtIdRef{MSG41}{Acción fuera del periodo} en la pantalla en que se encuentre navegando el actor indicando que no puede enviar el plan de acción debido a que la fecha actual se encuentra fuera del periodo definido por la SMAGEM para el registro del plan de acción.
 \end{UCtrayectoriaA}
 
  \begin{UCtrayectoriaA}[Fin del caso de uso]{C}{No existe al menos un objetivo, una meta y una acción registrada.}
    \UCpaso[\UCsist] Muestra el mensaje \cdtIdRef{MSG40}{Falló el envío del plan de acción} sobre la pantalla en que se encuentre navegando el actor  indicando que no puede enviar el plan de acción debido a que falta registrar al menos un objetivo, una meta y una acción. 
 \end{UCtrayectoriaA}
 
     \begin{UCtrayectoriaA}[Fin del caso de uso]{D}{El actor desea cancelar el envío de información.}
    \UCpaso[\UCactor] Solicita cancelar la operación oprimiendo el botón \cdtButton{Cancelar} en la pantalla emergente.
    \end{UCtrayectoriaA}
    
     \begin{UCtrayectoriaA}[Fin del caso de uso]{E}{La información ya ha sido enviada.}
    \UCpaso[\UCsist] Muestra el mensaje \cdtIdRef{MSG39}{La información ya ha sido enviada} sobre la pantalla en que se encuentre navegando el actor indicando que no puede enviar la información debido a que esta ya ha sido enviada anteriormente. 
    \end{UCtrayectoriaA}