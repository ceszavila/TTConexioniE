\begin{UseCase}{CUP 8}{Registrar acción}
	{
		Este caso de uso permite al actor registrar una \cdtRef{gls:accion}{acción} para una meta determinada, esta acción
		deberá llevarse a cabo para poder alcanzar esta meta, cuando los datos solicitados para la acción sean ingresados serán validados y registrados.
		
	}
	\UCitem{Versión}{1.0}
	\UCccsection{Administración de Requerimientos}	
	\UCitem{Autor}{Sergio Ramírez Camacho}	
	\UCccitem{Evaluador}{}
	\UCitem{Operación}{Registro}
	\UCccitem{Prioridad}{Alta}
	\UCccitem{Complejidad}{Media}
	\UCccitem{Volatilidad}{Media}
	\UCccitem{Madurez}{Media}
	\UCitem{Estatus}{Terminado} 
	\UCitem{Fecha del último estatus}{27 de noviembre de 2014}

%% Copie y pegue este bloque tantas veces como revisiones tenga el caso de uso.
%% Esta sección la debe llenar solo el Revisor
% %--------------------------------------------------------
 	\UCccsection{Revisión Versión 1.0} % Anote la versión que se revisó.
 	\UCccitem{Fecha}{1-Dic} 
 	\UCccitem{Evaluador}{Nayeli Vega}
 	\UCccitem{Resultado}{Corregir}
 	\UCccitem{Observaciones}{
 		\begin{UClist}
% 			% PC: Petición de Cambio, describa el trabajo a realizar, si es posible indique la causa de la PC. Opcionalmente especifique la fecha en que considera razonable que se deba terminar la PC. No olvide que la numeración no se debe reiniciar en una segunda o tercera revisión.
 			\RCitem{PC1}{\DONE{Resumen: Liga rota en acción}}{Fecha de entrega}
 			\RCitem{PC2}{\DONE{Resumen: Registrar una acción asociada a quien}}{Fecha de entrega}
 			\RCitem{PC3}{\DONE{Entradas: La UNIDAD NO es un catálogo, esto en la sección de la entidad Acción}}{Fecha de entrega}
 			\RCitem{PC4}{\DONE{Interfaz: En objetivo, liga rota a meta}}{Fecha de entrega} 			
 		\end{UClist}		
 	}
% %--------------------------------------------------------


	\UCsection{Atributos}
	\UCitem{Actor}{\cdtRef{actor:usuarioEscuela}{Coordinador del programa}}
	\UCitem{Propósito}{Registrar una acción en el sistema para una alguna \cdtRef{gls:meta}{meta}.}

	\UCitem{Entradas}{
	  \begin{UClist}
	   \UCli De la sección ``Información de la acción':
	      \begin{Citemize}
		\item \cdtRef{accion:descripcion}{Descripción}: \ioEscribir.
		\item \cdtRef{accion:recursos}{Recursos materiales y financieros}: \ioEscribir.
		\item \cdtRef{accion:cuantificable}{Es cuantificable}: \ioRadioBoton. 
		\item \cdtRef{accion:valorAlcanzar}{Valor a alcanzar}: \ioEscribir.
		\item \cdtRef{accion:unidad}{Unidad}: \ioEscribir. 
	      \end{Citemize}
	   
	  \end{UClist}
	}	
	
	\UCitem{Salidas}{
		\begin{UClist} 
			\UCli \cdtIdRef{MSG1}{Operación realizada exitosamente}: Se muestra en la pantalla \cdtIdRef{IUP 7}{Administrar acciones} cuando la acción se ha registrado correctamente.
		\end{UClist}
	}
	\UCitem{Precondiciones}{
		\UCli {\bf Interna:} Que la escuela se encuentre en estado \cdtRef{estado:planEdicion}{Plan de acción en edición}.
		\UCli {\bf Interna:} Que el periodo de registro de plan de acción se encuentre vigente.
	}
	
	\UCitem{Postcondiciones}{
		\begin{UClist}
 			\UCli {\bf Interna:} Se registrará una nueva acción asociada a una meta en el sistema.
 			\UCli {\bf Interna:} Se podrá modificar la acción registrada a través del caso de uso \cdtIdRef{CUP 9}{Modificar acción}
 			\UCli {\bf Interna:} Se podrá eliminar la acción registrada a través del caso de uso \cdtIdRef{CUP 10}{Eliminar acción}
		\end{UClist}
	}
	
	\UCitem{Reglas de negocio}{
		\begin{UClist}
			\UCli \cdtIdRef{RN-S1}{Información correcta}: Verifica que la información ingresada por el actor sea correcta.
		\end{UClist}
	}
	
	\UCitem{Errores}{
		\begin{UClist}	
			\UCli \cdtIdRef{MSG5}{Falta un dato requerido para efectuar la operación solicitada}: Se muestra en la pantalla \cdtIdRef{IUP 8}{Registrar acción} cuando el actor no ingresó un dato requerido para realizar la operación.
			\UCli \cdtIdRef{MSG6}{Formato incorrecto}: Se muestra en la pantalla \cdtIdRef{IUP 8}{Registrar acción} cuando el formato de alguno de los datos ingresados es incorrecto.
 			\UCli \cdtIdRef{MSG7}{Se ha excedido la longitud máxima del campo}: Se muestra en la pantalla \cdtIdRef{IUP 8}{Registrar acción} cuando el actor escribió un dato que excede el tamaño especificado por el sistema.
			\UCli \cdtIdRef{MSG28}{Operación no permitida por estado de la entidad}: Se muestra en la pantalla en que se encuentre navegando el actor debido al estado en que se encuentra la escuela.
			\UCli \cdtIdRef{MSG41}{Acción fuera del periodo}: Se muestra en la pantalla en que se encuentre navegando el actor indicando que la fecha no se encuentra dentro del periodo de registro de plan de acción. 			
		\end{UClist}
	}

	\UCitem{Tipo}{Secundario, extiende del caso de uso \cdtIdRef{CUP 7}{Administrar acciones}}
% 	\UCitem{Fuente}{
% 		\begin{UClist}
% 			\UCli 
% 		\end{UClist}
% 	}

\end{UseCase}
 
 
\begin{UCtrayectoria}
	\UCpaso[\UCactor] Solicita registrar una acción oprimiendo el botón \cdtButton{Registrar} en la pantalla \cdtIdRef{IUP 7}{Administrar acciones}. \refTray{A}
	\UCpaso[\UCsist] Verifica que la escuela se encuentre en  estado ``Plan de acción en edición''. \refTray{B}.
    \UCpaso[\UCsist] Verifica que la fecha actual se encuentre dentro del periodo definido por la SMAGEM para el registro del plan de acción. \refTray{C}.
	\UCpaso[\UCsist] Muestra la pantalla \cdtIdRef{IUP 8}{Registrar acción} por medio de la cual se realizará el registro de la acción.
	\UCpaso[\UCactor] Ingresa los datos correspondientes a la acción en la pantalla \cdtIdRef{IUP 8}{Registrar acción}. \label{cup8:ingresaDatos}
	\UCpaso[\UCactor] Selecciona que la acción ingresada es cuantificable en la pantalla \cdtIdRef{IUP 8}{Registrar acción}. \refTray{D} \label{cup8:selecciona}
	\UCpaso[\UCsist] Muesta la pantalla \cdtIdRef{IUP 8.1}{Registrar acción cuantificable} por medio de la cual se realizará el registro de la acción cuantificable.
	\UCpaso[\UCactor] Ingresa el ``Valor a alcanzar''  y la``Unidad''.
	\UCpaso[\UCactor] Solicita registrar la acción oprimiendo el botón \cdtButton{Aceptar} de la pantalla \cdtIdRef{IUP 8}{Registrar acción}. \refTray{E} \label{cup8:registrar}
	\UCpaso[\UCsist] Verifica que la escuela se encuentre en  estado ``Plan de acción en edición''. \refTray{B}.
	\UCpaso[\UCsist] Verifica que los datos ingresados por el actor sean correctos como lo indica la regla de negocio \cdtIdRef{RN-S1}{Información correcta}. \refTray{F} \refTray{G} \refTray{H} 
	\UCpaso[\UCsist] Registra la nueva acción en el sistema.
	\UCpaso[\UCsist] Muestra el mensaje \cdtIdRef{MSG1}{Operación realizada exitosamente} en la pantalla \cdtIdRef{IUP 7}{Administrar acciones} indicando que se ha realizado el registro de la acción satisfactoriamente.
\end{UCtrayectoria}



\begin{UCtrayectoriaA}[Fin del caso de uso]{A}{El actor desea regresar.}
	\UCpaso[\UCactor] Solicita regresar oprimiendo el botón \cdtButton{Regresar} de la pantalla \cdtIdRef{IUP 7}{Administrar acciones}.
	\UCpaso[\UCsist] Muestra la pantalla  \cdtIdRef{IUP 5}{Administrar metas}.
\end{UCtrayectoriaA} 

\begin{UCtrayectoriaA}[Fin del caso de uso]{B}{La escuela no se encuentra en el estado ``Plan de acción en edición''.}
    \UCpaso[\UCsist] Muestra el mensaje \cdtIdRef{MSG28}{Operación no permitida por estado de la entidad} en la pantalla en que se encuentre navegando el actor indicando que no puede administrar los objetivos del plan de acción debido a que la escuela no se encuentra en el estado ``Plan de acción en edición''. 
 \end{UCtrayectoriaA}
 
   \begin{UCtrayectoriaA}[Fin del caso de uso]{C}{La fecha actual se encuentra fuera del periodo definido por la SMAGEM para el registro del plan de acción}
    \UCpaso[\UCsist] Muestra el mensaje \cdtIdRef{MSG41}{Acción fuera del periodo} en la pantalla en que se encuentre navegando el actor indicando que no puede administrar los objetivos del plan de acción debido a que la fecha actual se encuentra fuera del periodo definido por la SMAGEM para el registro del plan de acción.
 \end{UCtrayectoriaA}
 

\begin{UCtrayectoriaA}{D}{El actor selecciona que la acción no es cuantificable.}
	\UCpaso[] Continúa con el paso \ref{cup8:registrar} de la trayectoria principal.
\end{UCtrayectoriaA} 

   
\begin{UCtrayectoriaA}[Fin del caso de uso]{E}{El actor desea cancelar la operación.}
	\UCpaso[\UCactor] Solicita cancelar la operación oprimiendo el botón \cdtButton{Cancelar} de la pantalla \cdtIdRef{IUP 8}{Registrar acción}.
	\UCpaso[\UCsist] Muestra la pantalla  \cdtIdRef{IUP 7}{Administrar acciones}.
\end{UCtrayectoriaA} 

\begin{UCtrayectoriaA}{F}{El actor no proporcionó alguno de los datos requeridos.}
	\UCpaso[\UCsist] Muestra el mensaje \cdtIdRef{MSG5}{Falta un dato requerido para efectuar la operación solicitada} en la pantalla  \cdtIdRef{IUP 8}{Registrar acción},
	indicando que no se puede registrar la acción debido a la omisión de un dato requerido y señala los campos faltantes.
	\UCpaso[] Continúa en el paso \ref{cup8:ingresaDatos} de la trayectoria principal.
\end{UCtrayectoriaA}
 
\begin{UCtrayectoriaA}{G}{El actor proporcionó un dato que excede la longitud máxima.}
	\UCpaso[\UCsist] Muestra el mensaje \cdtIdRef{MSG7}{Se ha excedido la longitud máxima del campo} en la pantalla \cdtIdRef{IUP 8}{Registrar acción},
	indicando que no se puede registrar la acción debido a que se ha excedido la longitud máxima permitida de algún dato y señala los campos que excedieron la longitud.
	\UCpaso[] Continúa en el paso \ref{cup8:ingresaDatos} de la trayectoria principal.
\end{UCtrayectoriaA}

\begin{UCtrayectoriaA}{H}{El actor ingresó un dato en formato incorrecto.}
	\UCpaso[\UCsist] Muestra el mensaje \cdtIdRef{MSG6}{Formato incorrecto} en la pantalla \cdtIdRef{IUP 8}{Registrar acción},
	indicando que no se puede registrar la acción debido a que se ingresó un dato con formato incorrecto y señala los campos incorrectos.
        \UCpaso[] Continúa en el paso \ref{cup8:ingresaDatos} de la trayectoria principal. 
  \end{UCtrayectoriaA}