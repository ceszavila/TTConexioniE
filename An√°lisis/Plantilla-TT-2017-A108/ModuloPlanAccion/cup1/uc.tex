%!TEX encoding = UTF-8 Unicode

\begin{UseCase}{CUP 1}{Administrar objetivos}
	{
		
		Este caso de uso tiene como objetivo mostrar al actor todos los \cdtRef{gls:objetivo}{objetivos}
		registrados en el sistema, el actor podrá acceder a diversas operaciones como registrar, modificar y
		eliminar objetivos, así como administrar las metas asociadas a estos.
	}
	%\UCitem{\DONEUC}{Edición}
	\UCitem{Versión}{1.0}
	\UCccsection{Administración de Requerimientos}	
	\UCitem{Autor}{Natalia Giselle Hernández Sánchez}	
	\UCccitem{Evaluador}{}
	\UCitem{Operación}{Administración}
	\UCccitem{Prioridad}{Alta}
	\UCccitem{Complejidad}{Media}
	\UCccitem{Volatilidad}{Alta}
	\UCccitem{Madurez}{Baja}
	\UCitem{Estatus}{Terminado}
	\UCitem{Fecha del último estatus}{24 de noviembre de 2014}
%% Copie y pegue este bloque tantas veces como revisiones tenga el caso de uso.
%% Esta sección la debe llenar solo el Revisor
% %--------------------------------------------------------
 	\UCccsection{Revisión Versión 1} % Anote la versión que se revisó.
 	\UCccitem{Fecha}{3-Dic} 
 	\UCccitem{Evaluador}{Nayeli Vega}
 	\UCccitem{Resultado}{Corregir}
 	\UCccitem{Observaciones}{
 		\begin{UClist}
% 			% PC: Petición de Cambio, describa el trabajo a realizar, si es posible indique la causa de la PC. Opcionalmente especifique la fecha en que considera razonable que se deba terminar la PC. No olvide que la numeración no se debe reiniciar en una segunda o tercera revisión.
 			\RCitem{PC1}{\TOCHK{Precondiciones: Revisar redacción, Que el estado del plan de acción SEA Edición}}{Fecha de entrega}
 			\RCitem{PC1}{\DONE{Cambiar botón de flecha}}{Fecha de entrega}
 			\RCitem{PC2}{\DONE{Interfaz: Cambiar botón de flecha}}{Fecha de entrega}
 			\RCitem{PC3}{\DONE{Interfaz: Comandos: Liga rota de Eliminar obetivo}}{Fecha de entrega} 			
 		\end{UClist}		
 	}
% %--------------------------------------------------------

	\UCsection{Atributos}
	\UCitem{Actor}{\cdtRef{actor:usuarioEscuela}{Coordinador del programa}}
	\UCitem{Propósito}{
		Administrar los objetivos registrados en el sistema a través de una tabla de resultados.
	}
	\UCitem{Entradas}{
		Ninguna
	}
	\UCitem{Salidas}{
		\begin{UClist}
 			\UCli \cdtRef{objetivo}{Objetivo}: \ioTabla{ la \cdtRef{gls:lineaAccion}{Línea de acción} y el \cdtRef{objetivo:objetivoGeneral}{Objetivo general}}{de objetivos}.
			\UCli \cdtIdRef{MSG2}{No existe información registrada por el momento}: Se muestra en la pantalla \cdtIdRef{IUP 1}{Administrar objetivos} cuando no existen objetivos registrados.
		\end{UClist}
	}

	\UCitem{Precondiciones}{
		\begin{UClist}
			\UCli {\bf Interna:} Que la escuela se encuentre en estado \cdtRef{estado:planEdicion}{Plan de acción en edición}.
			\UCli {\bf Interna:} Que el periodo de registro de plan de acción se encuentre vigente
		\end{UClist}
		
	}
	
	\UCitem{Postcondiciones}{
		\begin{UClist}
			\UCli {\bf Interna:} Se podrá registrar un objetivo por medio del caso de uso \cdtIdRef{CUP 2}{Registrar objetivo}.
			\UCli {\bf Interna:} Se podrá modificar un objetivo por medio del caso de uso \cdtIdRef{CUP 3}{Modificar objetivo}.
			\UCli {\bf Interna:} Se podrá eliminar un objetivo por medio del caso de uso \cdtIdRef{CUP 4}{Eliminar objetivo}.
			\UCli {\bf Interna:} Se podrán administrar las metas de un objetivo por medio del caso de uso \cdtIdRef{CUP 5}{Administrar metas}.
		\end{UClist}
	}

	\UCitem{Reglas de \hspace{1 cm} negocio}{
		Ninguna
	}

	\UCitem{Errores}{
		\begin{UClist}	
			\UCli \cdtIdRef{MSG28}{Operación no permitida por estado de la entidad}: Se muestra en la pantalla en que se encuentre navegando el actor debido al estado en que se encuentra la escuela.
			
			\UCli \cdtIdRef{MSG41}{Acción fuera del periodo}: Se muestra en la pantalla en que se encuentre navegando el actor indicando que la fecha no se encuentra dentro del periodo de registro de plan de acción
		\end{UClist}
	}

	\UCitem{Tipo}{
		Primario
	}

% 	\UCitem{Fuente}{
% 		\begin{UClist}
% 			\UCli %Minuta de la reunión \cdtIdRef{M-15TR}{Toma de Requerimientos}.
% 		\end{UClist}
% 	}
	
\end{UseCase}
%-------------------------------------------------------%trayectoria Principal-----------------------------------------------
 \begin{UCtrayectoria}
    \UCpaso[\UCactor] Solicita administrar los objetivos, seleccionando en el menú \cdtIdRef{MN2}{Menú del Coordinador del programa} la opción ``Objetivos''.
    \UCpaso[\UCsist] Verifica que la escuela se encuentre en  estado ``Plan de acción en edición''. \refTray{A}.
    \UCpaso[\UCsist] Verifica que la fecha actual se encuentre dentro del periodo definido por la SMAGEM para el registro del plan de acción. \refTray{B}.
    \UCpaso[\UCsist] Busca la información de los objetivos registrados en el sistema. \refTray{C}
    \UCpaso[\UCsist] Muestra la información de los objetivos en la pantalla \cdtIdRef{IUP 1}{Administrar objetivos}.
    \UCpaso[\UCactor] Administra los objetivos a través de los botones: \cdtButton{Registrar}, \botAdm, \botEdit y \botKo. \label{cup1:Mostrar}
 \end{UCtrayectoria}
 
   \begin{UCtrayectoriaA}[Fin del caso de uso]{A}{La escuela no se encuentra en el estado ``Plan de acción en edición''.}
    \UCpaso[\UCsist] Muestra el mensaje \cdtIdRef{MSG28}{Operación no permitida por estado de la entidad} en la pantalla en que se encuentre navegando el actor indicando que no puede administrar los objetivos del plan de acción debido a que la escuela no se encuentra en el estado ``Plan de acción en edición''. 
 \end{UCtrayectoriaA}
 
   \begin{UCtrayectoriaA}[Fin del caso de uso]{B}{La fecha actual se encuentra fuera del periodo definido por la SMAGEM para el registro del plan de acción}
    \UCpaso[\UCsist] Muestra el mensaje \cdtIdRef{MSG41}{Acción fuera del periodo} en la pantalla en que se encuentre navegando el actor indicando que no puede administrar los objetivos del plan de acción debido a que la fecha actual se encuentra fuera del periodo definido por la SMAGEM para el registro del plan de acción.
 \end{UCtrayectoriaA}
 
\begin{UCtrayectoriaA}[Fin del caso de uso]{C}{No hay registros de objetivos para mostrar.}
    \UCpaso[\UCsist] Muestra el mensaje \cdtIdRef{MSG2}{No existe información registrada por el momento} en pantalla \cdtIdRef{IUP 1}{Administrar objetivos}
    indicando que aún no hay objetivos.
 \end{UCtrayectoriaA}
 

\subsection{Puntos de extensión}


\UCExtensionPoint{El actor requiere registrar un objetivo}
      {Paso \ref{cup1:Mostrar}}
      {\cdtIdRef{CUP 2}{Registrar objetivo}}
      
\UCExtensionPoint{El actor require modificar un objetivo}
      {Paso \ref{cup1:Mostrar}}
      {\cdtIdRef{CUP 3}{Modificar objetivo}}

\UCExtensionPoint{El actor require eliminar un objetivo}
      {Paso \ref{cup1:Mostrar}}
      {\cdtIdRef{CUP 4}{Eliminar objetivo}}
      
\UCExtensionPoint{El actor require administrar las metas de un objetivo}
      {Paso \ref{cup1:Mostrar}}
      {\cdtIdRef{CUP 5}{Administrar metas}}

