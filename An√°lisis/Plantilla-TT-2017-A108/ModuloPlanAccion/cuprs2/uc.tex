\begin{UseCase}{CUPRS 2}{Modificar meta de residuos sólidos}
	{
		Este caso de uso permite al actor modificar una \cdtRef{gls:meta}{meta} de la línea de acción ``Residuos sólidos'' cuando existen errores o actualizaciones en la información. 
		El actor podrá visualizar la información previamente registrada e ingresar la nueva información, cuando los datos solicitados para la meta
		sean ingresados serán validados y se realizará la actualización de la información.
	}
	\UCitem{Versión}{1.0}
	\UCccsection{Administración de Requerimientos}	
	\UCitem{Autor}{Natalia Giselle Hernández Sánchez}
	\UCccitem{Evaluador}{}
	\UCitem{Operación}{Modificación}
	\UCccitem{Prioridad}{Alta}
	\UCccitem{Complejidad}{Media}
	\UCccitem{Volatilidad}{Media}
	\UCccitem{Madurez}{Media}
	\UCitem{Estatus}{Terminado} 
	\UCitem{Fecha del último estatus}{01 de diciembre de 2014}

%% Copie y pegue este bloque tantas veces como revisiones tenga el caso de uso.
%% Esta sección la debe llenar solo el Revisor
% %--------------------------------------------------------
 	\UCccsection{Versión 0.1} 
 	\UCccitem{Fecha}{3-Dic}
 	\UCccitem{Evaluador}{Nayeli Vega}
 	\UCccitem{Resultado}{Corregir}
 	\UCccitem{Observaciones}{
 		\begin{UClist}
 			\RCitem{PC1}{\DONE{Entradas: Liga rota en Unidad}}{}
 			\RCitem{PC2}{\DONE{Entradas: Liga rota en Nueva unidad}}{} 			 			 			
 			\RCitem{PC3}{\DONE{Salidas: Liga rota en Unidad}}{}
 			\RCitem{PC4}{\DONE{Salidas: Liga rota en Nueva unidad}}{} 			 			 			 			
 			\RCitem{PC5}{\DONE{Salidas: Mensaje 35 Se debe mostrar en la pantalla IUPRS2, está mal hecha la referencia}}{} 			 			 			 			 			
			\RCitem{PC6}{\DONE{Trayectoria principal: Paso 7. La referencia a la pantalla está mal hecha, no corresponde}}{} 			 			 			 			 			
 			\RCitem{PC7}{\DONE{Trayectoria F: La liga para el retorno a la trayectoria principal no llega ahí, llega a la trayectoria G pero del caso de uso IUPRS1}}{} 			 			 			 						
 			\RCitem{PC7}{\DONE{Trayectoria G: La liga para el retorno a la trayectoria principal no llega ahí, llega a la trayectoria G pero del caso de uso IUPRS1}}{} 			 			 			 						 		
			\RCitem{PC7}{\DONE{Trayectoria I: La liga para el retorno a la trayectoria principal no llega ahí, llega a la trayectoria G pero del caso de uso IUPRS1}}{}  			
 			\RCitem{PC3}{\TOCHK{Trayectoria B: Revisar redacción de la condición, falta un DE en LA META, también en el caso de uso anterior en el mismo punto :( no me había dado cuenta}}{}			
 			\RCitem{PC9}{\DONE{Interfaz: Liga rota a CUPRS3}}{Fecha de entrega} 			 			 			
 			\RCitem{PC10}{\DONE{Interfaz: En la sección de mensajes no está el mensaje 36}}{Fecha de entrega} 		
 		\end{UClist}		
 	}
% %--------------------------------------------------------


	\UCsection{Atributos}
	\UCitem{Actor}{\cdtRef{actor:usuarioEscuela}{Coordinador del programa}}
	\UCitem{Propósito}{Modificar una meta de la línea de acción ``Residuos sólidos'' cuando se detectan errores o actualizaciones en la información.}
	\UCitem{Entradas}{
	  \begin{UClist}
	    \UCli De la sección ``Información de la meta'':
		\begin{Citemize}
		  \item \cdtRef{meta:problematica}{Problemática}: \ioEscribir.
		  \item \cdtRef{meta:meta}{Meta}: \ioEscribir.
		  \item \cdtRef{meta:fechaInicio}{Fecha de inicio de ejecución de la meta}: \ioCalendario.
		  \item \cdtRef{meta:fechaTermino}{Fecha de término de ejecución de la meta}: \ioCalendario.
		\end{Citemize}
	    \UCli De la sección ``Información de los residuos sólidos'':
		\begin{Citemize}
		  \item \cdtRef{reduccionReciclaje}{Reducción o reciclaje de residuos}: \ioRadioBoton. %PENDIENTE
		\end{Citemize}
	    \UCli De la sección ``Información de la capacitación y/o sensibilización'':
		\begin{Citemize}
		  \item \cdtRef{capacitacionSensibilizacion}{Enfocada a capacitación y/o sensibilización}: \ioRadioBoton.
		  \item \cdtRef{capacitacionSensibilizacion:numActividades}{Número de actividades de capacitación y/o sensibilización a realizar}: \ioEscribir.
		  \item \cdtRef{capacitacionSensibilizacion:numAlumnos}{Alumnos}: \ioEscribir.
		  \item \cdtRef{capacitacionSensibilizacion:numDocenteAdministrativo}{Personal docente y administrativo}: \ioEscribir.
		  \item \cdtRef{capacitacionSensibilizacion:numDirectivos}{Personal directivo}: \ioEscribir.
		  \item \cdtRef{capacitacionSensibilizacion:numPadres}{Padres de familia}: \ioEscribir.
		\end{Citemize}
	      \UCli De la sección ``Cuantificar meta'':
		\begin{Citemize}
		  \item \cdtRef{meta:valorAlcanzar}{Valor a alcanzar}: \ioEscribir.
		  \item \cdtRef{meta:nombreValor}{Descripción del valor a alcanzar}: \ioEscribir.
		  \item \cdtRef{unidad}{Unidad}: \ioSeleccionar.
		  \item \cdtRef{unidad:unidad}{Nueva unidad}: \ioEscribir.
		\end{Citemize}
	  \end{UClist}
	}
	
	\UCitem{Salidas}{
	    \begin{UClist} 
	    \UCli \cdtRef{gls:periodoPlanAccion}{Periodo para ejecutar el plan de acción}: \ioObtener.
	    \UCli De la sección ``Información de la meta'':
	      \begin{Citemize}
		\item \cdtRef{meta:problematica}{Problemática}: \ioObtener.
		\item \cdtRef{meta:meta}{Meta}: \ioObtener.
		\item \cdtRef{meta:fechaInicio}{Fecha de inicio de ejecución de la meta}: \ioObtener.
		\item \cdtRef{meta:fechaTermino}{Fecha de término de ejecución de la meta}: \ioObtener.
	      \end{Citemize}
	   \UCli De la sección ``Información de los residuos sólidos'':
	      \begin{Citemize}
		\item \cdtRef{reduccionReciclaje}{Reducción o reciclaje de residuos}: \ioObtener. %PENDIENTE
	      \end{Citemize}
	   \UCli De la sección ``Información de la capacitación y/o sensibilización'':
	      \begin{Citemize}
		\item \cdtRef{capacitacionSensibilizacion}{Enfocada a capacitación y/o sensibilización}: \ioObtener.
		\item \cdtRef{capacitacionSensibilizacion:numActividades}{Número de actividades de capacitación y/o sensibilización a realizar}: \ioObtener.
		\item \cdtRef{capacitacionSensibilizacion:numAlumnos}{Alumnos}: \ioObtener.
		\item \cdtRef{capacitacionSensibilizacion:numDocenteAdministrativo}{Personal docente y administrativo}: \ioObtener.
		\item \cdtRef{capacitacionSensibilizacion:numDirectivos}{Personal directivo}: \ioObtener.
		\item \cdtRef{capacitacionSensibilizacion:numPadres}{Padres de familia}: \ioObtener.
	      \end{Citemize}
	    \UCli De la sección ``Cuantificar meta'':
	      \begin{Citemize}
		\item \cdtRef{meta:valorAlcanzar}{Valor a alcanzar}: \ioObtener.
		\item \cdtRef{meta:nombreValor}{Descripción del valor a alcanzar}: \ioObtener.
		\item \cdtRef{unidad}{Unidad}: \ioObtener.
		\item \cdtRef{unidad:unidad}{Nueva unidad}: \ioObtener.
	      \end{Citemize}
	    \UCli \cdtIdRef{MSG1}{Operación realizada exitosamente}: Se muestra en la pantalla \cdtIdRef{IUP 5}{Administrar metas} cuando la meta se ha modificado correctamente.
	    \UCli \cdtIdRef{MSG35}{Registrar residuos sólidos}: Se muestra en la pantalla \cdtIdRef{IUPRS 2}{Modificar meta de residuos sólidos} cuando el actor seleccione que la meta está enfocada a la reducción de la generación o reciclaje 
			de residuos indicando la operación de la que se trata.
	    \end{UClist}
	}
	\UCitem{Precondiciones}{
		\begin{UClist}
\UCli {\bf Interna:} Que la escuela se encuentre en estado \cdtRef{estado:planEdicion}{Plan de acción en edición}.
			\UCli {\bf Interna:} Que el periodo de registro de plan de acción se encuentre vigente.		
			\UCli {\bf Interna:} Que exista información referente a la unidad.
		\end{UClist}
	}
	
	\UCitem{Postcondiciones}{
		\begin{UClist}
 			\UCli {\bf Interna:} Se modificará la meta en el sistema.
 			\UCli {\bf Interna:} Se registrará una unidad con estado ``Por aprobar''.
		\end{UClist}
	}
	
	\UCitem{Reglas de negocio}{
		\begin{UClist}
			\UCli \cdtIdRef{RN-S1}{Información correcta}: Verifica que la información ingresada por el actor sea correcta.
			\UCli \cdtIdRef{RN-N14}{Reducción de la generación o reciclaje de residuos}: Verifica que se registre al menos un residuo sólido cuando la meta está enfocada a la reducción de la generación o reciclaje de residuos.
		\end{UClist}
	}
	
	\UCitem{Errores}{
		\begin{UClist}
			\UCli \cdtIdRef{MSG4}{No se encontró información sustantiva}: Se muestra en la pantalla \cdtIdRef{IUP 5}{Administrar metas} cuando hace falta información referente a la unidad.
			\UCli \cdtIdRef{MSG5}{Falta un dato requerido para efectuar la operación solicitada}: Se muestra en la pantalla \cdtIdRef{IUPRS 2}{Modificar meta de residuos sólidos} cuando el actor no ingresó un dato requerido para realizar la operación.
			\UCli \cdtIdRef{MSG6}{Formato incorrecto}: Se muestra en la pantalla \cdtIdRef{IUPRS 2}{Modificar meta de residuos sólidos} cuando el formato de alguno de los datos ingresados es incorrecto.
 			\UCli \cdtIdRef{MSG7}{Se ha excedido la longitud máxima del campo}: Se muestra en la pantalla \cdtIdRef{IUPRS 2}{Modificar meta de residuos sólidos} cuando el actor escribió un dato que excede el tamaño especificado por el sistema.
\UCli \cdtIdRef{MSG28}{Operación no permitida por estado de la entidad}: Se muestra en la pantalla en que se encuentre navegando el actor debido al estado en que se encuentra la escuela. 			
 			\UCli \cdtIdRef{MSG36}{Reducción de la generación o reciclaje de residuos}: Se muestra en la pantalla \cdtIdRef{IUPRS 2}{Modificar meta de residuos sólidos} cuando hace falta registrar un residuo.
\UCli \cdtIdRef{MSG41}{Acción fuera del periodo}: Se muestra en la pantalla en que se encuentre navegando el actor indicando que la fecha no se encuentra dentro del periodo de registro de plan de acción. 			
		\end{UClist}
	}
	\UCitem{Tipo}{Secundario, extiende del caso de uso \cdtIdRef{CUP 5}{Administrar metas}}

	\UCitem{Fuente}{
		\begin{UClist}
			\UCli 
		\end{UClist}
	}

 \end{UseCase}
 
 
\begin{UCtrayectoria}
	\UCpaso[\UCactor] Solicita modificar una meta oprimiendo el botón \botEdit del registro que desea modificar en la pantalla \cdtIdRef{IUP 5}{Administrar metas}. 
	\UCpaso[\UCsist] Verifica que la escuela se encuentre en  estado ``Plan de acción en edición''. \refTray{A}.
    \UCpaso[\UCsist] Verifica que la fecha actual se encuentre dentro del periodo definido por la SMAGEM para el registro del plan de acción. \refTray{B}.
	
	\UCpaso[\UCsist] Busca el periodo de ejecución del plan de acción.
	\UCpaso[\UCsist] Verifica que exista información referente a la unidad. \refTray{C}
	\UCpaso[\UCsist] Muestra la pantalla \cdtIdRef{IUPRS 2}{Modificar meta de residuos sólidos} por medio de la cual se realizará la modificación de la meta.
	\UCpaso[\UCactor] Modifica los datos correspondientes a la sección ``Información de la meta'' de la pantalla \cdtIdRef{IUPRS 2}{Modificar meta de residuos sólidos}. \label{cuprs2:ingresaDatos}
	\UCpaso[\UCactor] Selecciona que el \cdtRef{gls:enfoqueMeta}{enfoque de la meta} es ``Reducción de la generación o reciclaje de residuos'' en la pantalla \cdtIdRef{IUPRS 2}{Modificar meta de residuos sólidos}. \refTray{B}
	\UCpaso[\UCsist] Muestra los campos referentes a la reducción de la generación o reciclaje de residuos en la pantalla \cdtIdRef{IUPRS 2.1}{Modificar meta de residuos sólidos: Reducción de la generación o reciclaje de residuos}.
	\UCpaso[\UCsist] Muestra el mensaje \cdtIdRef{MSG35}{Registrar residuos sólidos} indicando si se trata de la reducción o reciclaje de residuos.
	\UCpaso[\UCactor] Continúa con el caso de uso \cdtIdRef{CUPRS 3}{Administrar residuos sólidos del plan de acción}.
	\UCpaso[\UCactor] Solicita modificar la meta oprimiendo el botón \cdtButton{Aceptar} de la pantalla \cdtIdRef{IUPRS 2}{Modificar meta de residuos sólidos}. \refTray{G}
	\UCpaso[\UCsist] Verifica que los datos ingresados por el actor sean correctos como lo indica la regla de negocio \cdtIdRef{RN-S1}{Información correcta}. \refTray{H} \refTray{I} \refTray{J}
	\UCpaso[\UCsist] Verifica que el usuario haya ingresado al menos un residuo sólido como lo indica la regla de negocio \cdtIdRef{RN-N14}{Reducción de la generación o reciclaje de residuos}. \refTray{K}
	\UCpaso[\UCsist] Registra la modificación de meta en el sistema con enfoque ``Reducción de la generación o reciclaje de residuos''.
	\UCpaso[\UCsist] Muestra el mensaje \cdtIdRef{MSG1}{Operación realizada exitosamente} en la pantalla \cdtIdRef{IUP 5}{Administrar metas} indicando que se ha realizado la modificación de la meta satisfactoriamente. \label{cuprs2:muestraMensaje}
\end{UCtrayectoria}


\begin{UCtrayectoriaA}[Fin del caso de uso]{A}{La escuela no se encuentra en el estado ``Plan de acción en edición''.}
    \UCpaso[\UCsist] Muestra el mensaje \cdtIdRef{MSG28}{Operación no permitida por estado de la entidad} en la pantalla en que se encuentre navegando el actor indicando que no puede administrar los objetivos del plan de acción debido a que la escuela no se encuentra en el estado ``Plan de acción en edición''. 
 \end{UCtrayectoriaA}
 
   \begin{UCtrayectoriaA}[Fin del caso de uso]{B}{La fecha actual se encuentra fuera del periodo definido por la SMAGEM para el registro del plan de acción}
    \UCpaso[\UCsist] Muestra el mensaje \cdtIdRef{MSG41}{Acción fuera del periodo} en la pantalla en que se encuentre navegando el actor indicando que no puede administrar los objetivos del plan de acción debido a que la fecha actual se encuentra fuera del periodo definido por la SMAGEM para el registro del plan de acción.
 \end{UCtrayectoriaA}

 
\begin{UCtrayectoriaA}[Fin del caso de uso]{C}{No existe información referente a la unidad.}
	\UCpaso[\UCsist] Muestra el mensaje \cdtIdRef{MSG4}{No se encontró información sustantiva} en la pantalla \cdtIdRef{IUP 5}{Administrar metas} indicando que la operación no puede continuar debido a la falta información necesaria para el sistema.
\end{UCtrayectoriaA}

\begin{UCtrayectoriaA}{D}{El actor selecciona que el enfoque de la meta no es ``Reducción de la generación o reciclaje de residuos''.}
	\UCpaso[\UCactor] Selecciona que el \cdtRef{gls:enfoqueMeta}{enfoque de la meta} no es ``Reducción de la generación o reciclaje de residuos''. 
	\UCpaso[\UCsist] Muestra la sección ``Información de la capacitación y/o sensibilización'' en la pantalla \cdtIdRef{IUPRS 2.2}{Modificar meta de residuos sólidos: Capacitación y/o sensibilización}.
	\UCpaso[\UCactor] Selecciona que el \cdtRef{gls:enfoqueMeta}{enfoque de la meta} es ``Capacitación y/o sensibilización''. \refTray{E}
	\UCpaso[\UCactor] Modifica los datos correspondientes a la sección ``Información de la capacitación y/o sensibilización'' de la pantalla \cdtIdRef{IUPRS 2.2}{Modificar meta de residuos sólidos: Capacitación y/o sensibilización}.
	\UCpaso[\UCactor] Solicita modificar la meta oprimiendo el botón \cdtButton{Aceptar} de la pantalla \cdtIdRef{IUPRS 2}{Modificar meta de residuos sólidos}. \refTray{G} 
	\UCpaso[\UCsist] Verifica que los datos ingresados por el actor sean correctos como lo indica la regla de negocio \cdtIdRef{RN-S1}{Información correcta}. \refTray{H} \refTray{I} \refTray{J}
	\UCpaso[\UCsist] Registra la modificación de meta en el sistema con enfoque ``Capacitación y/o sensibilización''.
	\UCpaso[] Continúa en el paso \ref{cuprs2:muestraMensaje} de la trayectoria principal.
\end{UCtrayectoriaA}

\begin{UCtrayectoriaA}{E}{El actor selecciona que el enfoque de la meta no es ``Capacitación y/o sensibilización''.}
	\UCpaso[\UCactor] Selecciona que \cdtRef{gls:enfoqueMeta}{enfoque de la meta} no es ``Capacitación y/o sensibilización''. 
	\UCpaso[\UCsist] Muestra la sección ``Cuantificar meta'' en la pantalla \cdtIdRef{IUPRS 2.3}{Modificar meta de residuos sólidos: Cuantificar meta}.
	\UCpaso[\UCactor] Modifica el valor y el nombre del valor a alcanzar en la sección ``Cuantificar meta'' de la pantalla \cdtIdRef{IUPRS 2.3}{Modificar meta de residuos sólidos: Cuantificar meta}.
	\UCpaso[\UCactor] Selecciona la unidad del valor a alcanzar en la sección ``Cuantificar meta'' de la pantalla \cdtIdRef{IUPRS 2.3}{Modificar meta de residuos sólidos: Cuantificar meta}. \refTray{F}
	\UCpaso[\UCactor] Solicita modificar la meta oprimiendo el botón \cdtButton{Aceptar} de la pantalla \cdtIdRef{IUPRS 2}{Modificar meta de residuos sólidos}. \refTray{G} 
	\UCpaso[\UCsist] Verifica que los datos ingresados por el actor sean correctos como lo indica la regla de negocio \cdtIdRef{RN-S1}{Información correcta}. \refTray{H} \refTray{I} \refTray{J}
	\UCpaso[\UCsist] Registra la modificación de meta en el sistema con enfoque ``General''.
	\UCpaso[] Continúa en el paso \ref{cuprs2:muestraMensaje} de la trayectoria principal.
\end{UCtrayectoriaA}

\begin{UCtrayectoriaA}{F}{El actor selecciona la opción ``Otra'' como unidad del valor a alcanzar.}
	\UCpaso[\UCactor] Selecciona ``Otra'' como la unidad del valor a alcanzar en la sección ``Cuantificar meta'' de la pantalla \cdtIdRef{IUPRS 2.3}{Modificar meta de residuos sólidos: Cuantificar meta}.
	\UCpaso[\UCsist] Muestra la pantalla  \cdtIdRef{IUPRS 2}{Modificar meta de residuos sólidos} con el campo ``Nueva unidad''.
	\UCpaso[\UCactor] Modifica la nueva unidad del valor a alcanzar.
	\UCpaso[\UCactor] Solicita modificar la meta oprimiendo el botón \cdtButton{Aceptar} de la pantalla \cdtIdRef{IUPRS 2}{Modificar meta de residuos sólidos}. \refTray{G} 
	\UCpaso[\UCsist] Verifica que los datos ingresados por el actor sean correctos como lo indica la regla de negocio \cdtIdRef{RN-S1}{Información correcta}. \refTray{H} \refTray{I} \refTray{J}
	\UCpaso[\UCsist] Modifica la nueva unidad en el sistema con estado ``Por aprobar''.
	\UCpaso[\UCsist] Registra la modificación de meta en el sistema con enfoque ``General''.
	\UCpaso[] Continúa en el paso \ref{cuprs2:muestraMensaje} de la trayectoria principal.
\end{UCtrayectoriaA} 

\begin{UCtrayectoriaA}[Fin del caso de uso]{G}{El actor desea cancelar la operación.}
	\UCpaso[\UCactor] Solicita cancelar la operación oprimiendo el botón \cdtButton{Cancelar} de la pantalla \cdtIdRef{IUPRS 2}{Modificar meta de residuos sólidos}.
	\UCpaso[\UCsist] Muestra la pantalla  \cdtIdRef{IUP 5}{Administrar metas}.
\end{UCtrayectoriaA} 

\begin{UCtrayectoriaA}{H}{El actor no proporcionó alguno de los datos requeridos.}
	\UCpaso[\UCsist] Muestra el mensaje \cdtIdRef{MSG5}{Falta un dato requerido para efectuar la operación solicitada} en la pantalla \cdtIdRef{IUPRS 2}{Modificar meta de residuos sólidos},
	indicando que no se puede modificar la meta debido a la omisión de un dato requerido y señala los campos faltantes.
	\UCpaso[] Continúa en el paso \ref{cuprs2:ingresaDatos} de la trayectoria principal.
\end{UCtrayectoriaA}

\begin{UCtrayectoriaA}{I}{El actor proporcionó un dato que es incorrecto.}
	\UCpaso[\UCsist] Muestra el mensaje \cdtIdRef{MSG6}{Formato incorrecto} en la pantalla \cdtIdRef{IUPRS 2}{Modificar meta de residuos sólidos}
	cuando se ha ingresado un dato con formato inválido y señala los campos inválidos.
	\UCpaso[] Continúa en el paso \ref{cuprs2:ingresaDatos} de la trayectoria principal.
\end{UCtrayectoriaA}
 
\begin{UCtrayectoriaA}{J}{El actor proporcionó un dato que excede la longitud máxima.}
	\UCpaso[\UCsist] Muestra el mensaje \cdtIdRef{MSG7}{Se ha excedido la longitud máxima del campo} en la pantalla \cdtIdRef{IUPRS 2}{Modificar meta de residuos sólidos}
	cuando se ha excedido la longitud máxima permitida y señala los campos que excedieron la longitud.
	\UCpaso[] Continúa en el paso \ref{cuprs2:ingresaDatos} de la trayectoria principal.
\end{UCtrayectoriaA}

\begin{UCtrayectoriaA}{K}{El actor no registró ningún residuo sólido.}
	\UCpaso[\UCsist] Muestra el mensaje \cdtIdRef{MSG36}{Reducción de la generación o reciclaje de residuos} en la pantalla \cdtIdRef{IUPRS 2}{Modificar meta de residuos sólidos},
	indicando que no se puede modificar la meta debido a que no se ha registrado ningún residuo.
	\UCpaso[] Continúa en el paso \ref{cuprs2:ingresaDatos} de la trayectoria principal.
\end{UCtrayectoriaA}