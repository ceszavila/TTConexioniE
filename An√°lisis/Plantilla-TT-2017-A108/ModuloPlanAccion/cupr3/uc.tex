\begin{UseCase}{CUPR 3}{Eliminar residuo sólido del plan de acción}
	{   
	    Después de registrar un residuo sólido del plan de acción, el actor puede haber detectado algún error en la información ingresada o puede ocurrir que esta
	    ya no sea requerida, en ese caso el actor podrá eliminar el residuo sólido del plan de acción a través de este caso de uso.
	}
	
	\UCitem{Versión}{1.0}
	\UCccsection{Administración de Requerimientos}
	\UCitem{Autor}{Sergio Ramírez Camacho}
	\UCccitem{Evaluador}{}
	\UCitem{Operación}{Eliminación}
	\UCccitem{Prioridad}{Media}
	\UCccitem{Terminado}{}
	\UCccitem{Evaluado}{}
	\UCccitem{Complejidad}{Media}
	\UCccitem{Volatilidad}{Media}
	\UCccitem{Madurez}{Media}
	\UCitem{Estatus}{Terminado}
	\UCitem{Fecha del último estatus}{2 de diciembre de 2014}


%% Esta sección la debe llenar solo el Revisor
% %--------------------------------------------------------
 	\UCccsection{Revisión Versión 0.1} % Anote la versión que se revisó.
% 	% FECHA: Anote la fecha en que se terminó la revisión.
 	\UCccitem{Fecha}{3-Dic} 
% 	% EVALUADOR: Coloque el nombre completo de quien realizó la revisión.
 	\UCccitem{Evaluador}{Nayeli Vega}
% 	% RESULTADO: Coloque la palabra que mas se apegue al tipo de acción que el analista debe realizar.
 	\UCccitem{Resultado}{Revisado}
% 	% OBSERVACIONES: Liste los cambios que debe realizar el Analista.
 	\UCccitem{Observaciones}{
 		\begin{UClist}
% 			% PC: Petición de Cambio, describa el trabajo a realizar, si es posible indique la causa de la PC. Opcionalmente especifique la fecha en que considera razonable que se deba terminar la PC. No olvide que la numeración no se debe reiniciar en una segunda o tercera revisión.

% 			\RCitem{PC2}{\TODO{Trayectoria principal: Paso 2. Liga rota en referencia a imagen}}{Fecha de entrega}
 %			\RCitem{PC3}{\TODO{Trayectoria principal: Paso 3. Liga rota en referencia a imagen}}{Fecha de entrega} 	
 %			\RCitem{PC4}{\TODO{Trayectoria alternativa: Liga rota en referencia a imagen}}{Fecha de entrega} 					
 		\end{UClist}		
 	}
% %--------------------------------------------------------



	\UCsection{Atributos}
	\UCitem{Actor}{\cdtRef{actor:usuarioEscuela}{Coordinador del programa}}
	\UCitem{Propósito}{Eliminar un residuo sólido del plan de acción que fue registrado por error o que después de una revisión se determinó que es innecesario.}
	\UCitem{Entradas}{
		Ninguna
		
	}
	\UCitem{Salidas}{
		\begin{UClist} 
			\UCli \cdtIdRef{MSG31}{Confirmar eliminación}: Se muestra en la pantalla \cdtIdRef{IUPRS 1}{Registrar meta de residuos sólidos} o en la pantalla \cdtIdRef{IUPRS 2}{Modificar meta de residuos sólidos} para que el actor confirme la eliminación del residuo sólido.
		\end{UClist}
	}

	\UCitem{Precondiciones}{
\UCli {\bf Interna:} Que la escuela se encuentre en estado \cdtRef{estado:planEdicion}{Plan de acción en edición}.
			\UCli {\bf Interna:} Que el periodo de registro de plan de acción se encuentre vigente.
	}
	
	\UCitem{Postcondiciones}{
		\begin{UClist}
			\UCli {\bf Interna:} El registro del residuo sólido se eliminará del sistema.
		\end{UClist}
	}

	\UCitem{Reglas de \hspace{1 cm} negocio}{
		Ninguna
	}
	\UCitem{Errores}{
\UCli \cdtIdRef{MSG28}{Operación no permitida por estado de la entidad}: Se muestra en la pantalla en que se encuentre navegando el actor debido al estado en que se encuentra la escuela.
\UCli \cdtIdRef{MSG41}{Acción fuera del periodo}: Se muestra en la pantalla en que se encuentre navegando el actor indicando que la fecha no se encuentra dentro del periodo de registro de plan de acción.

	}

	\UCitem{Tipo}{Secundario, extiende del caso de uso \cdtIdRef{CUPRS 3}{Administrar residuos sólidos del plan de acción}}

	\UCitem{Fuente}{
		\begin{UClist}
			\UCli 
		\end{UClist}
	}



 \end{UseCase}

 \begin{UCtrayectoria}
\UCpaso[\UCactor] Solicita eliminar el residuo sólido del plan de acción oprimiendo el botón \botKo  del registro que desea eliminar en la pantalla \cdtIdRef{IUPRS 1}{Registrar meta de residuos sólidos} o en la pantalla \cdtIdRef{IUPRS 2}{Modificar meta de residuos sólidos}.

	\UCpaso[\UCsist] Verifica que la escuela se encuentre en  estado ``Plan de acción en edición''. \refTray{A}.
    \UCpaso[\UCsist] Verifica que la fecha actual se encuentre dentro del periodo definido por la SMAGEM para el registro del plan de acción. \refTray{B}.


\UCpaso[\UCsist] Muestra el mensaje \cdtIdRef{MSG31}{Confirmar eliminación} en la pantalla \cdtIdRef{IUPRS 1}{Registrar meta de residuos sólidos} o en la pantalla \cdtIdRef{IUPRS 1}{Registrar meta de residuos sólidos} indicando al actor que si acepta eliminar 
el elemento no podrá recuperar la información posteriormente.
\UCpaso[\UCactor] Confirma la eliminación del registro oprimiendo el botón \cdtButton{Aceptar} del mensaje emergente. \refTray{C}
\UCpaso[\UCsist] Verifica que la escuela se encuentre en  estado ``Plan de acción en edición''. \refTray{A}.
\UCpaso[\UCsist] Elimina el registro que hace referencia al residuo sólido seleccionado.
\UCpaso[\UCsist] Muestra el mensaje \cdtIdRef{MSG1}{Operación realizada exitosamente} en la pantalla  
\cdtIdRef{IUPRS 1}{Registrar meta de residuos sólidos} o en la pantalla \cdtIdRef{IUPRS 2}{Modificar meta de residuos sólidos} y actualiza la lista de residuos que se muestra en la pantalla.
 \end{UCtrayectoria}
 
\begin{UCtrayectoriaA}[Fin del caso de uso]{A}{La escuela no se encuentra en el estado ``Plan de acción en edición''.}
    \UCpaso[\UCsist] Muestra el mensaje \cdtIdRef{MSG28}{Operación no permitida por estado de la entidad} en la pantalla en que se encuentre navegando el actor indicando que no puede administrar los objetivos del plan de acción debido a que la escuela no se encuentra en el estado ``Plan de acción en edición''. 
 \end{UCtrayectoriaA}
 
   \begin{UCtrayectoriaA}[Fin del caso de uso]{B}{La fecha actual se encuentra fuera del periodo definido por la SMAGEM para el registro del plan de acción}
    \UCpaso[\UCsist] Muestra el mensaje \cdtIdRef{MSG41}{Acción fuera del periodo} en la pantalla en que se encuentre navegando el actor indicando que no puede administrar los objetivos del plan de acción debido a que la fecha actual se encuentra fuera del periodo definido por la SMAGEM para el registro del plan de acción.
 \end{UCtrayectoriaA} 
 
\begin{UCtrayectoriaA}[Fin del caso de uso]{C}{El actor cancela la operación.}
	\UCpaso[\UCactor] Solicita cancelar la operación oprimiendo el botón \cdtButton{Cancelar} del mensaje emergente.
	\UCpaso[\UCsist] Muestra la pantalla \cdtIdRef{IUPRS 1}{Registrar meta de residuos sólidos} o la pantalla \cdtIdRef{IUPRS 2}{Modificar meta de residuos sólidos}.
\end{UCtrayectoriaA}
  
