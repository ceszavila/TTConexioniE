\subsection{IUP 7 Administrar acciones}

\subsubsection{Objetivo}
	
	En esta pantalla el \cdtRef{actor:usuarioEscuela}{Coordinador del programa} puede conocer las acciones asociadas a una meta registradas en el sistema,
	además sirve como punto de acceso para registrar, modificar y eliminar \cdtRef{gls:accion}{acciones}.
	

\subsubsection{Diseño}

    En la figura ~\ref{IUP 7} se muestra la pantalla ``Administrar acciones'', por medio de la cual 
    se podrán administrar las acciones a través de una tabla de resultados.
    El actor podrá registrar, modificar y eliminar una acción utilizando los botones \cdtButton{Registrar}, \botEdit y \botKo. \\
    
    \IUfig[.9]{pantallas/planAccion/cup7/iup7.png}{IUP 7}{Administrar acciones}


\subsubsection{Comandos}
\begin{itemize}
	\item \cdtButton{Registrar}: Permite al actor registrar una acción, dirige a la pantalla \cdtIdRef{IUP 8}{Registrar acción}.
	\item \cdtButton{Regresar}: Dirige a la pantalla \cdtIdRef{IUP 5}{Administrar acciones}.
	\item \botEdit[Modificar acción]: Permite al actor modificar una acción, dirige a la pantalla \cdtIdRef{IUP 9}{Modificar acción}.
	\item \botOk[Eliminar acción]: Permite al actor eliminar una acción, muestra el mensaje \cdtIdRef{MSG31}{Confirmar eliminación}.
	
\end{itemize}

\subsubsection{Mensajes}

	
\begin{description}
	\item[\cdtIdRef{MSG2}{No existe información registrada por el momento}:] Se muestra en la pantalla \cdtIdRef{IUP 7}{Administrar acciones} cuando no existen acciones registradas.
	\item[\cdtIdRef{MSG28}{Operación no permitida por estado de la entidad}:] Se muestra en la pantalla en que se encuentre navegando el actor debido al estado en que se encuentra la escuela.	
	\item[\UCli \cdtIdRef{MSG41}{Acción fuera del periodo}:] Se muestra en la pantalla en que se encuentre navegando el actor indicando que la fecha no se encuentra dentro del periodo de registro de plan de acción.	
\end{description}
