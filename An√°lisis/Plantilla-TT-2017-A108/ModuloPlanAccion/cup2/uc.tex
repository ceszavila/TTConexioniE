\begin{UseCase}{CUP 2}{Registrar objetivo}
	{
		Este caso de uso permite al actor solicitar el registro de un \cdtRef{gls:objetivo}{objetivo}.
		El actor podrá ingresar la información que describa el objetivo, así como seleccionar la línea de acción que corresponde,
		cuando los datos solicitados para el objetivo sean ingresados serán
		validados y registrados.
		
	}
	\UCitem{Versión}{1.0}
	\UCccsection{Administración de Requerimientos}	
	\UCitem{Autor}{Natalia Giselle Hernández Sánchez}	
	\UCccitem{Evaluador}{}
	\UCitem{Operación}{Registro}
	\UCccitem{Prioridad}{Alta}
	\UCccitem{Complejidad}{Media}
	\UCccitem{Volatilidad}{Media}
	\UCccitem{Madurez}{Media}
	\UCitem{Estatus}{Terminado} 
	\UCitem{Fecha del último estatus}{24 de noviembre de 2014}

%% Copie y pegue este bloque tantas veces como revisiones tenga el caso de uso.
%% Esta sección la debe llenar solo el Revisor
% %--------------------------------------------------------
 	\UCccsection{Revisión Versión 1} % Anote la versión que se revisó.
 	\UCccitem{Fecha}{1-Dic} 
 	\UCccitem{Evaluador}{Nayeli Vega}
 	\UCccitem{Resultado}{Corregir}
 	\UCccitem{Observaciones}{
 		\begin{UClist}
 			\RCitem{PC1}{\DONE{Resumen: Dar una descripción mas detallada del objetivo, Creo que aqui quedaría mejor el texto que tiene Modificar objetivo y al reves}}{}
 			\RCitem{PC2}{\DONE{Salida: Liga rota en periodo}}{}
 			\RCitem{PC3}{\DONE{Interfaz: Diseño: sugiero que diga ... La línea de acción a la que corresponde ... }}{} 			
% 			\RCitem{PC3}{\TODO{}}{}
 		\end{UClist}		
 	}
% %--------------------------------------------------------
% %--------------------------------------------------------

	\UCsection{Atributos}
	\UCitem{Actor}{\cdtRef{actor:usuarioEscuela}{Coordinador del programa}}
	\UCitem{Propósito}{Registrar un objetivo en el sistema para cada línea de acción.}

	\UCitem{Entradas}{
	  \begin{UClist}
	   \UCli De la sección ``Información del objetivo'':
	      \begin{Citemize}
		\item \cdtRef{gls:lineaAccion}{Línea de acción}: \ioSeleccionar. %PENDIENTE
		\item \cdtRef{objetivo:objetivoGeneral}{Objetivo general}: \ioEscribir.
	      \end{Citemize}
	   
	  \end{UClist}
	}	
	
	\UCitem{Salidas}{
		\begin{UClist} 
			\UCli \cdtRef{gls:periodoPlanAccion}{Periodo para ejecutar el plan de acción}: \ioObtener.
			\UCli \cdtIdRef{MSG1}{Operación realizada exitosamente}: Se muestra en la pantalla \cdtIdRef{IUP 1}{Administrar objetivos} cuando el objetivo se ha registrado correctamente.
		\end{UClist}
	}
	\UCitem{Precondiciones}{
		\begin{UClist}
			\UCli {\bf Interna:} Que exista información referente a la línea de acción.
			\UCli {\bf Interna:} Que la escuela se encuentre en estado \cdtRef{estado:planEdicion}{Plan de acción en edición}.
			\UCli {\bf Interna:} Que el periodo de registro de plan de acción se encuentre vigente.
		\end{UClist}
	}
	
	\UCitem{Postcondiciones}{
		\begin{UClist}
 			\UCli {\bf Interna:} Se registrará un nuevo objetivo en el sistema.
		\end{UClist}
	}
	
	\UCitem{Reglas de negocio}{
		\begin{UClist}
			\UCli \cdtIdRef{RN-S1}{Información correcta}: Verifica que la información ingresada por el actor sea correcta.
			\UCli \cdtIdRef{RN-N9}{Unicidad de objetivos por línea de acción}: Verifica que exista solamente un objetivo por línea de acción.
		\end{UClist}
	}
	
	\UCitem{Errores}{
		\begin{UClist}	
		        \UCli \cdtIdRef{MSG4}{No se encontró información sustantiva}: Se muestra en la pantalla \cdtIdRef{IUP 1}{Administrar objetivos} cuando hace falta información referente a las línea de acción.
			\UCli \cdtIdRef{MSG5}{Falta un dato requerido para efectuar la operación solicitada}: Se muestra en la pantalla \cdtIdRef{IUP 2}{Registrar objetivo} cuando el actor no ingresó un dato requerido para realizar la operación.
 			\UCli \cdtIdRef{MSG7}{Se ha excedido la longitud máxima del campo}: Se muestra en la pantalla \cdtIdRef{IUP 2}{Registrar objetivo} cuando el actor escribió un dato que excede el tamaño especificado por el sistema.
			\UCli \cdtIdRef{MSG28}{Operación no permitida por estado de la entidad}: Se muestra en la pantalla en que se encuentre navegando el actor debido al estado en que se encuentra la escuela. 			
 			\UCli \cdtIdRef{MSG33}{Unicidad de objetivos por línea de acción}: Se muestra en la pantalla \cdtIdRef{IUP 2}{Registrar objetivo} cuando el actor seleccionó una línea de acción que ya tiene asociado un objetivo.
			\UCli \cdtIdRef{MSG41}{Acción fuera del periodo}: Se muestra en la pantalla en que se encuentre navegando el actor indicando que la fecha no se encuentra dentro del periodo de registro de plan de acción.
		\end{UClist}
	}

	\UCitem{Tipo}{Secundario, extiende del caso de uso \cdtIdRef{CUP 1}{Administrar objetivos}}
% 	\UCitem{Fuente}{
% 		\begin{UClist}
% 			\UCli 
% 		\end{UClist}
% 	}

\end{UseCase}
 
 
\begin{UCtrayectoria}
	\UCpaso[\UCactor] Solicita registrar un objetivo oprimiendo el botón \cdtButton{Registrar} en la pantalla \cdtIdRef{IUP 1}{Administrar objetivos}.
	%	
	\UCpaso[\UCsist] Verifica que la escuela se encuentre en  estado ``Plan de acción en edición''. \refTray{A}.
    \UCpaso[\UCsist] Verifica que la fecha actual se encuentre dentro del periodo definido por la SMAGEM para el registro del plan de acción. \refTray{B}.
    %
	\UCpaso[\UCsist] Verifica que exista información referente a la línea de acción. \refTray{C}
	\UCpaso[\UCsist] Busca las líneas de acción disponibles para la lista desplegable ``línea de acción'' con base en la regla de negocio \cdtIdRef{RN-N9}{Unicidad de objetivos por línea de acción}.
	\UCpaso[\UCsist] Muestra la pantalla \cdtIdRef{IUP 2}{Registrar objetivo} por medio de la cual se realizará el registro del objetivo.
	\UCpaso[\UCactor] Ingresa los datos correspondientes al objetivo en la pantalla \cdtIdRef{IUP 2}{Registrar objetivo}. \label{cup2:ingresaDatos}
	\UCpaso[\UCactor] Solicita registrar el objetivo oprimiendo el botón \cdtButton{Aceptar} de la pantalla \cdtIdRef{IUP 2}{Registrar objetivo}. \refTray{D}
	\UCpaso[\UCsist] Verifica que la escuela se encuentre en  estado ``Plan de acción en edición''. \refTray{A}.
	\UCpaso[\UCsist] Verifica que los datos ingresados por el actor sean correctos como lo indica la regla de negocio \cdtIdRef{RN-S1}{Información correcta}. \refTray{E} \refTray{F}
	\UCpaso[\UCsist] Verifica que no exista un objetivo asociado a la línea de acción seleccionada, con base en la regla de negocio \cdtIdRef{RN-N9}{Unicidad de objetivos por línea de acción}. \refTray{G} %PENDIENTE
	\UCpaso[\UCsist] Registra el nuevo objetivo en el sistema.
	\UCpaso[\UCsist] Muestra el mensaje \cdtIdRef{MSG1}{Operación realizada exitosamente} en la pantalla \cdtIdRef{IUP 1}{Administrar objetivos} indicando que se ha realizado el registro del objetivo satisfactoriamente.
\end{UCtrayectoria}
   \begin{UCtrayectoriaA}[Fin del caso de uso]{A}{La escuela no se encuentra en el estado ``Plan de acción en edición''.}
    \UCpaso[\UCsist] Muestra el mensaje \cdtIdRef{MSG28}{Operación no permitida por estado de la entidad} en la pantalla en que se encuentre navegando el actor indicando que no puede administrar los objetivos del plan de acción debido a que la escuela no se encuentra en el estado ``Plan de acción en edición''. 
 \end{UCtrayectoriaA}
 
   \begin{UCtrayectoriaA}[Fin del caso de uso]{B}{La fecha actual se encuentra fuera del periodo definido por la SMAGEM para el registro del plan de acción}
    \UCpaso[\UCsist] Muestra el mensaje \cdtIdRef{MSG41}{Acción fuera del periodo} en la pantalla en que se encuentre navegando el actor indicando que no puede administrar los objetivos del plan de acción debido a que la fecha actual se encuentra fuera del periodo definido por la SMAGEM para el registro del plan de acción.
 \end{UCtrayectoriaA}
 
\begin{UCtrayectoriaA}[Fin del caso de uso]{C}{No existe información referente a la línea de acción.}
	\UCpaso[\UCsist] Muestra el mensaje  \cdtIdRef{MSG4}{No se encontró información sustantiva} en la pantalla \cdtIdRef{IUP 1}{Administrar objetivos} indicando que la operación no puede continuar debido a la falta información necesaria para el sistema.
\end{UCtrayectoriaA}
  
\begin{UCtrayectoriaA}[Fin del caso de uso]{D}{El actor desea cancelar la operación.}
	\UCpaso[\UCactor] Solicita cancelar la operación oprimiendo el botón \cdtButton{Cancelar} de la pantalla \cdtIdRef{IUP 2}{Registrar objetivo}.
	\UCpaso[\UCsist] Muestra la pantalla  \cdtIdRef{IUP 1}{Administrar objetivos}.
\end{UCtrayectoriaA} 

\begin{UCtrayectoriaA}{E}{El actor no proporcionó alguno de los datos requeridos.}
	\UCpaso[\UCsist] Muestra el mensaje \cdtIdRef{MSG5}{Falta un dato requerido para efectuar la operación solicitada} en la pantalla  \cdtIdRef{IUP 2}{Registrar objetivo},
	indicando que no se puede registrar el objetivo debido a la omisión de un dato requerido y señala los campos faltantes.
	\UCpaso[] Continúa en el paso \ref{cup2:ingresaDatos} de la trayectoria principal.
\end{UCtrayectoriaA}
 
\begin{UCtrayectoriaA}{F}{El actor proporcionó un dato que excede la longitud máxima.}
	\UCpaso[\UCsist] Muestra el mensaje \cdtIdRef{MSG7}{Se ha excedido la longitud máxima del campo} en la pantalla \cdtIdRef{IUP 2}{Registrar objetivo}
	cuando se ha excedido la longitud máxima permitida y señala los campos que excedieron la longitud.
	\UCpaso[] Continúa en el paso \ref{cup2:ingresaDatos} de la trayectoria principal.
\end{UCtrayectoriaA}

\begin{UCtrayectoriaA}{G}{Existe un objetivo asociado a la línea de acción seleccionada.}
	\UCpaso[\UCsist] Muestra el mensaje \cdtIdRef{MSG33}{Unicidad de objetivos por línea de acción} en la pantalla \cdtIdRef{IUP 2}{Registrar objetivo}
	cuando se ha seleccionado una línea de acción que ya tiene asociado un objetivo.
	\UCpaso[] Continúa en el paso \ref{cup2:ingresaDatos} de la trayectoria principal.
\end{UCtrayectoriaA}