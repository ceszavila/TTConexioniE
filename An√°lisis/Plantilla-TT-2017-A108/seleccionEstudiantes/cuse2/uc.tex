
\begin{UseCase}{CUS 2}{Administrar avances de ambiente escolar}
    {
	El actor podrá ingresar a este caso de uso para registrar un avance en la meta o en las acciones correspondientes a ella. Para ello se mostrarán las metas registradas referentes al objetivo de la línea de acción seleccionado.
    }

    \UCitem{Versión}{1.0}
    \UCccsection{Administración de Requerimientos}
    \UCitem{Autor}{Francisco Javier Ponce Cruz}
    \UCccitem{Evaluador}{}
    \UCitem{Operación}{Administrar}
    \UCccitem{Prioridad}{Alta}
    \UCccitem{Complejidad}{Baja}
    \UCccitem{Volatilidad}{}
    \UCccitem{Madurez}{}
    \UCitem{Estatus}{Terminado}
    \UCitem{Fecha del último estatus}{9 diciembre 2014}
    
%% Copie y pegue este bloque tantas veces como revisiones tenga el caso de uso.
%% Esta sección la debe llenar solo el Revisor
% %--------------------------------------------------------
 	\UCccsection{Revisión Versión 1.0} % Anote la versión que se revisó.
% 	% FECHA: Anote la fecha en que se terminó la revisión.
 	\UCccitem{Fecha}{9-Dic} 
% 	% EVALUADOR: Coloque el nombre completo de quien realizó la revisión.
 	\UCccitem{Evaluador}{Nayeli Vega}
% 	% RESULTADO: Coloque la palabra que mas se apegue al tipo de acción que el analista debe realizar.
 	\UCccitem{Resultado}{Corregir}
% 	% OBSERVACIONES: Liste los cambios que debe realizar el Analista.
 	\UCccitem{Observaciones}{
 		\begin{UClist}
% 			% PC: Petición de Cambio, describa el trabajo a realizar, si es posible indique la causa de la PC. Opcionalmente especifique la fecha en que considera razonable que se deba terminar la PC. No olvide que la numeración no se debe reiniciar en una segunda o tercera revisión.
 			\RCitem{PC1}{\TODO{Resumen: sugiero ... objetivo de la línea de acción de ambiente escolar.}}{Fecha de entrega}
 			\RCitem{PC2}{\TODO{Modificar propósito }}{Fecha de entrega} 			
			\RCitem{PC3}{\TODO{Agregar precondición de estado. Que la escuela se encuentre en estado Avance en edición }}{Fecha de entrega} 					 			
			\RCitem{PC4}{\TODO{Agregar precondición de fecha }}{Fecha de entrega} 					 				
			\RCitem{PC5}{\TODO{Agregar mensaje 41 en los errores}}{Fecha de entrega} 					 							\RCitem{PC6}{\TODO{Agregar a la trayectria las dos verificaciones, estado y fecha}}{Fecha de entrega} 	
			\RCitem{PC7}{\TODO{Agregar rutas alternativas }}{Fecha de entrega} 					 					
			\RCitem{PC8}{\TODO{Interfaz: agregar mensaje 41 }}{Fecha de entrega} 									 									
% 			\RCitem{PC2}{\TODO{Descripción del pendiente}}{Fecha de entrega}
% 			\RCitem{PC3}{\TODO{Descripción del pendiente}}{Fecha de entrega}
 		\end{UClist}		
 	}
% %--------------------------------------------------------

    \UCsection{Atributos}
    \UCitem{Actor(es)}{\cdtRef{actor:usuarioEscuela}{Coordinador del programa}}
    \UCitem{Propósito}{Visualizar y administrar las metas y acciones de ambiente escolar para acceder al registro de los avances.}
    \UCitem{Entradas}{Ninguna.}
    \UCitem{Salidas}{
    \cdtRef{meta}{Metas}: Tabla que muestra \cdtRef{meta:meta}{Meta}, \cdtRef{meta:enfoqueMeta}{Capacitación}, \cdtRef{meta:enfoqueMeta}{Mejora}, \cdtRef{meta:fechaInicio}{Fecha de inicio}, \cdtRef{meta:fechaTermino}{Fecha de término} de las metas registradas en el sistema}
    \UCitem{Precondiciones}{
	\begin{UClist}
	    \UCli {\bf Interna:} Que la escuela se encuentre en estado \cdtRef{estado:avanceEdicion}{Avance en edición}.
        \UCli {\bf Interna:} Que el periodo de registros de avances se encuentre vigente. 
	\end{UClist}
    }
    
    \UCitem{Postcondiciones}{
	\begin{UClist}
	    \UCli {\bf Interna:} Se podrá acceder a la administración de los avances de las metas y acciones mediante los casos de uso: \cdtIdRef{CUS 3}{Registrar avance de acciones de Ambiente escolar} y \cdtIdRef{CUS 4}{Registrar avance de metas de Ambiente escolar}.
	\end{UClist}
    }

    %Reglas de negocio: Especifique las reglas de negocio que utiliza este caso de uso
    \UCitem{Reglas de negocio}{Ninguna.}
    \UCitem{Errores}{
    \begin{UClist}
        \UCli \cdtIdRef{MSG28}{Operación no permitida por estado de la entidad}: Se muestra en la pantalla \cdtIdRef{IUS 1}{Administrar avances de objetivos} indicando al actor que no se puede administrar los avances de meta ya que el Plan de acción no se encuentra aprobado.
    \end{UClist}
    }
    \UCitem{Tipo}{Secudario, extiende del caso de uso \cdtIdRef{CUS 1}{Administrar avances de objetivos}.}
\end{UseCase}



 \begin{UCtrayectoria}
    \UCpaso[\UCactor] Oprime el botón \botAcciones del objetivo que desea administrar en la pantalla \cdtIdRef{IUS 1}{Administrar avances de objetivos}.
    \label{cus2:oprimeAvance}
    \UCpaso[\UCsist] Verifica que la escuela se encuentre en estado ``Avance en edición''. \refTray{A}.
    \UCpaso[\UCsist] Verifica que la fecha actual se encuentre dentro del periodo definido por la SMAGEM para registrar avances en las acciones de agua. \refTray{B}.
     \UCpaso[\UCsist] Muestra la información de las metas de ``Ambiente escolar'' en la pantalla \cdtIdRef{IUS 2}{Administrar avances de ambiente escolar}.
     \UCpaso[\UCactor] Administra los avances de meta o acciones a través de los botones \botAutoAjus y \botMetas. \label{cus2:Administrar}
    \end{UCtrayectoria}

    \begin{UCtrayectoriaA}[Fin del caso de uso]{A}{La escuela no se encuentra en un estado que permita registrar avance en las acciones}
    \UCpaso[\UCactor] Muestra el mensaje \cdtIdRef{MSG28}{Operación no permitida por estado de la entidad} en la pantalla \cdtIdRef{IUS 1}{Administrar avances de objetivos} indicando al actor que no se puede administrar los avances por el estado en que se encuentra la escuela.
    \end{UCtrayectoriaA}

    \begin{UCtrayectoriaA}[Fin del caso de uso]{B}{La fecha actual se encuentra fuera del periodo definido por la SMAGEM para el registro de avances.}
    \UCpaso[\UCsist] Muestra el mensaje \cdtIdRef{MSG41}{Acción fuera del periodo} en la pantalla \cdtIdRef{IUS 1}{Administrar avances de objetivos} indicando al actor que no puede registrar avances debido a que la fecha actual se encuentra fuera del periodo definido por la SMAGEM para realizar la acción. 
    \end{UCtrayectoriaA}

\subsection{Puntos de extensión}

\UCExtensionPoint
{El Usuario desea registrar un avance de meta}
{Paso \ref{cus2:Administrar} de la Trayectoria Principal}
{\cdtRef{CUS 3}{Registrar avance de meta de ambiente escolar}}

\UCExtensionPoint
{El Usuario desea registrar un avance de las acciones de una meta}
{Paso \ref{cus2:Administrar} de la Trayectoria Principal}
{\cdtRef{CUS 2}{Registrar avance de acciones de ambiente escolar}}