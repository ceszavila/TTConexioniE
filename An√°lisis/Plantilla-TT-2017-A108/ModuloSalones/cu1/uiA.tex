\subsection{IUSM-01  Consultar Asignación de Grupos}

\subsubsection{Objetivo}

% Explicar el objetivo para el que se construyo la interfaz, generalmente es la descripción de la actividad a desarrollar, como Seleccionar grupos para inscribir materias de un alumno, controlar el acceso al sistema mediante la solicitud de un login y password de los usuarios, etc.
	
    Esta pantalla permite al \cdtRef{actor:CIEAlumno}{Alumno} conocer la asginación de salones a grupos en los que se encuentra inscrito en el semestre en curso.
\subsubsection{Diseño}

% Presente la figura de la interfaz y explique paso a paso ``a manera de manual de usuario'' como se debe utilizar la interfaz. No olvide detallar en la redacción los datos de entradas y salidas. Explique como utilizar cada botón y control de la pantalla, para que sirven y lo que hacen. Si el Botón lleva a otra pantalla, solo indique la pantalla y explique lo que pasará cuando se cierre dicha pantalla (la explicación sobre el funcionamiento de la otra pantalla estará en su archivo correspondiente).

    En la figura~\ref{IUSM-01} se muestra la pantalla ``Consultar Asignación de Grupos'', en ella se encuntra la tabla que contiene la lista de la asignación de los grupos que fueron ofertados en el semestre en curso.
    \IUfig[.3]{/ModuloSalones/S1.PNG}{IUSM-01}{Consultar Asignación de Grupos}



\subsubsection{Comandos}
    \begin{itemize}

	\item \textbf{Fila de la tabla que contiene el nombre del grupo}, dirige al caso de uso \cdtIdRef{CUSM-02}{Consultar Nivel de Salón}
		\item Botón \cdtButton{Atrás}, dirige a la pantalla \cdtIdRef{IUPP}{Pantalla Principal} 
    \end{itemize}
