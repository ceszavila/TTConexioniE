\begin{UseCase}{CU-01}{Consultar salón}
    {
	Este caso de uso le permite al \textbf{Actor (Visitante y Alumno)} localizar un espacio dentro de la Escuela Superior de Cómputo y obtener las vistas necesarias para llegar a su destino. 
	Cuando el aspirante a nuevo ingreso es aceptado e inicia su primer periodo escolar, podrá consultar los espacios dentro de la escuela para asistir a sus primeras clases, laboratorios, conferencias etc. De la misma manera para los alumnos de semestres más avanzados y visitantes en general, podrán consultar y ubicar los espacios que no conocían para realización de trámites, búsqueda de profesores, investigadores, etc.
	
    }
    \UCitem{Versión}{1.0}
    \UCccsection{Administración de Requerimientos}
    \UCitem{Autor}{Ivo Sebastián Sam Álvarez-Tostado}
    \UCccitem{Evaluador}{José David Ortega Pacheco}
    \UCitem{Operación}{Consulta}
    \UCccitem{Prioridad}{Alta}
    \UCccitem{Complejidad}{Baja}
    \UCccitem{Volatilidad}{Baja}
    \UCccitem{Madurez}{Media}
    \UCitem{Estatus}{Por revisar}
    \UCitem{Fecha del último estatus}{15 de Octubre del 2017}


%--------------------------------------------------------
	\UCccsection{Revisión Versión 0.3} % Anote la versión que se revisó.
	% FECHA: Anote la fecha en que se terminó la revisión.
	\UCccitem{Fecha}{11-11-14} 
	% EVALUADOR: Coloque el nombre completo de quien realizó la revisión.
	\UCccitem{Evaluador}{Natalia Giselle Hernández Sánchez}
	% RESULTADO: Coloque la palabra que mas se apegue al tipo de acción que el analista debe realizar.
	\UCccitem{Resultado}{Corregir}
	% OBSERVACIONES: Liste los cambios que debe realizar el Analista.
	\UCccitem{Observaciones}{
		\begin{UClist}
			% PC: Petición de Cambio, describa el trabajo a realizar, si es posible indique la causa de la PC. Opcionalmente especifique la fecha en que considera razonable que se deba terminar la PC. No olvide que la numeración no se debe reiniciar en una segunda o tercera revisión.
			\RCitem{PC1}{\DONE{Agregar a precondiciones el estado de la cuenta}}{Fecha de entrega}
			\RCitem{PC2}{\DONE{Agregar el paso de la trayectoria de validación del estado de la cuenta}}{Fecha de entrega}
			\RCitem{PC3}{\DONE{Agregar el mensaje de cuenta no activada a la sección de errores}}{Fecha de entrega}
			\RCitem{PC4}{\DONE{Verificar las ligas a los estados}}{Fecha de entrega}
			
		\end{UClist}		
	}
%--------------------------------------------------------

	\UCsection{Atributos}
	\UCitem{Actor}{
		\begin{UClist} 
			\UCli \cdtRef{actor:usuarioEscuela}{Coordinador del programa}
			\UCli \cdtRef{actor:usuarioSMAGEM}{Director del programa}
	\end{UClist}
}
	\UCitem{Propósito}{}
	\UCitem{Entradas}{
        \begin{UClist} 
           \UCli
           \UCli
        \end{UClist}}
	\UCitem{Salidas}{Ninguna.}
	\UCitem{Precondiciones}{
		\begin{UClist}		
			\UCli {\bf Interna:} El sistema debe tener cargados los espacios de la Escuela Superior de Cómputo.
			\UCli {\bf Interna:} El sistema debe tener asociados los espacios a los grupos por periodo escolar.
		\end{UClist}
		}
	\UCitem{Postcondiciones}{
	    \begin{UClist}
		\UCli {\bf Externa:} Los alumnos podrán saber en el momento que lo necesiten, el salón que fue asignado a su grupo sin tener que buscar las hojas de asignación de salones.
   	    \end{UClist}
	}
    \UCitem{Reglas de negocio}{
    	\begin{UClist}
%            \UCli \cdtIdRef{RN-S1}{Información correcta}: Verifica que la información introducida sea correcta.
	\end{UClist}
    }
	\UCitem{Errores}{
	    \begin{UClist}
%		\UCli \cdtIdRef{MSG5}{Falta un dato requerido para efectuar la operación solicitada}: Se muestra en la pantalla \cdtIdRef{IUR 1}{Iniciar sesión} cuando el actor omitió un dato marcado como requerido.
%		\UCli \cdtIdRef{MSG22}{Nombre de usuario y/o contraseña incorrecto}: Se muestra en la pantalla \cdtIdRef{IUR 1}{Iniciar sesión} indicando que el nombre de usuario y/o contraseña son incorrectos.
%		\UCli \cdtIdRef{MSG27}{Cuenta no activada}: Se muestra en la pantalla \cdtIdRef{IUR 1}{Iniciar sesión} indicando que la cuenta no está activada.
	    \end{UClist}
	}
	\UCitem{Tipo}{Primario.}
	\UCitem{Fuente}{
	    \begin{UClist}
%        \UCli Minuta de la reunión \cdtIdRef{M-3TR}{Toma de requerimientos}.
	    \end{UClist}
	}
 \end{UseCase}

 \begin{UCtrayectoria}
    \UCpaso[\UCactor] El actor Alumno requiere consultar el salón que se le asignó al grupo al que pertenece presionando el botón \cdtButton{Asignación de salones} del menú principal \textbf{CIE-IU001}. \refTray{A} \refTray{B}
    \UCpaso[\UCsist] Obtiene las asignaciones de salones por grupos y las muestra en forma de lista ordenada ascendentemente y por turno como se muestra en la pantalla \textbf{CIE-IU002}. 
    \UCpaso[\UCactor] Ubica el grupo al que está inscrito y selecciona el número de salón para saber cómo llegar.
    \UCpaso[\UCsist] Obtiene las vistas que el actor requiere para llegar a su destino (aérea, lateral y de salón). \label{cur1:consultar}
    \UCpaso[\UCsist] Le muestra las opciones de navegación entre las vistas disponibles del salón de destino.
	\UCpaso[\UCactor] El actor presiona el botón de \cdtButton{Menú} y regresa al menú principal.
 \end{UCtrayectoria}

 \begin{UCtrayectoriaA}{A}{El actor alumno desea consultar las áreas de ESCOM.}
    \UCpaso[\UCactor] Desea conocer la ubicación de un espacio en ESCOM presionando la opción Áreas de ESCOM del menú principal \textbf{CIE-IU001}
    
   \UCpaso[\UCsist] Obtiene la vista aérea de la Escuela Superior de cómputo y los polígonos de las zonas definidas como marcadores de los distintos espacios de la escuela como se muestra en la pantalla CIE-IU003
   
   \UCpaso[\UCactor] Desea saber como llegar a un espacio específico presionando su marcador. \ref{cur1:consultar}
 \end{UCtrayectoriaA}
 

 
%\subsection{Puntos de extensión}
%
%\UCExtensionPoint
%{El actor requiere recuperar su contraseña}
%{ Paso \ref{cur1:Acciones} de la trayectoria principal}
%{\cdtIdRef{CUR 2}{Recuperar contraseña}}
%
%\UCExtensionPoint
%{El actor requiere solicitar la inscripción de su escuela al programa}
%{ Paso \ref{cur1:Acciones} de la trayectoria principal}
%{\cdtIdRef{CUR 3}{Solicitar inscripción}}
 