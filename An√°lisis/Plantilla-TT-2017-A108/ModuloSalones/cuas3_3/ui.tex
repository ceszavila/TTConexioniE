\subsection{IUAS-03.3 Eliminar salón}

\subsubsection{Objetivo}

% Explicar el objetivo para el que se construyo la interfaz, generalmente es la descripción de la actividad a desarrollar, como Seleccionar grupos para inscribir materias de un alumno, controlar el acceso al sistema mediante la solicitud de un login y password de los usuarios, etc.
	
    Esta pantalla permite al \cdtRef{actor:CIEPrefectura}{Responsable Prefectura} eliminar un salón que por alguna razón se desee derogar para no hacer uso de él en un periodo escolar futuro.
\subsubsection{Diseño}

% Presente la figura de la interfaz y explique paso a paso ``a manera de manual de usuario'' como se debe utilizar la interfaz. No olvide detallar en la redacción los datos de entradas y salidas. Explique como utilizar cada botón y control de la pantalla, para que sirven y lo que hacen. Si el Botón lleva a otra pantalla, solo indique la pantalla y explique lo que pasará cuando se cierre dicha pantalla (la explicación sobre el funcionamiento de la otra pantalla estará en su archivo correspondiente).

    En la figura~\ref{IUAS-03.3} se muestra la pantalla ``Eliminar salón“  en donde se muestra un mensaje de confirmación para la eliminación del salón.
    
    \IUfig[.3]{/ModuloSalones/IUAS3_3.png}{IUAS-03.3}{Eliminar salón}



\subsubsection{Comandos}
    \begin{itemize}

	\item Botón \cdtButton{Eliminar}, dirige a la pantalla \cdtIdRef{IUAS-03}{Gestionar salones} 
	\item Botón \cdtButton{Cancelar}, dirige a la pantalla \cdtIdRef{IUAS-03.2}{Editar salón} 
    \end{itemize}
