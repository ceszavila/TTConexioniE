\subsection{IUAP-4.1  Registrar  Profesor}

\subsubsection{Objetivo}

% Explicar el objetivo para el que se construyo la interfaz, generalmente es la descripción de la actividad a desarrollar, como Seleccionar grupos para inscribir materias de un alumno, controlar el acceso al sistema mediante la solicitud de un login y password de los usuarios, etc.
	
    Esta pantalla permite al \cdtRef{actor:CIEProfesor}{Responsable de Profesores}  registrar la información de los profesores que están impartiendo clases en la ESCOM.
\subsubsection{Diseño}

% Presente la figura de la interfaz y explique paso a paso ``a manera de manual de usuario'' como se debe utilizar la interfaz. No olvide detallar en la redacción los datos de entradas y salidas. Explique como utilizar cada botón y control de la pantalla, para que sirven y lo que hacen. Si el Botón lleva a otra pantalla, solo indique la pantalla y explique lo que pasará cuando se cierre dicha pantalla (la explicación sobre el funcionamiento de la otra pantalla estará en su archivo correspondiente).

    En la figura~\ref{IUAP-4.1} se muestra la pantalla ``Registrar Profesor'', la cual esta compuesta por los campos: Nombre, cubículo, academia, e-mail, teléfono, página web, materias, así como de la fotografía del profesor.
       \IUfig[.3]{/ModuloProfesores/IUAP4_1.png}{IUAP-4.1}{Registrar Profesor}



\subsubsection{Comandos}
    \begin{itemize}
	\item \cdtButton{Registrar}, dirige a la misma pantalla.
	\item \textbf{Nombre del profesor}, dirige a la pantalla \cdtIdRef{IUAP-4.2}{Editar Profesor}.
	\item \cdtButton{Atrás}, dirige a la pantalla \cdtIdRef{IUAP-4}{Gestionar Profesores}
    \end{itemize}
