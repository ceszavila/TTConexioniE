\begin{UseCase}{CUPM-02}{Consultar Detalle de Profesor}
    {
	Este caso de uso le permite al actor conocer la información del profesor relacionada con su actividad académica, tal como sus medios de contacto institucionales, horarios de atención a alumnos y ubicación del espacio donde labora. 
	Cuando el \cdtRef{actor:CIEAlumno}{Alumno} desea buscar a un profesor por asesorias, consultas relacionadas a los temas de interés del profesor, trámites o citas en el proceso de titulación, etc. Este caso de uso le ofrecerá una herramienta que le ayude a conocer esta información.\\
	
	Tambien podrá consultar la información no institucional del profesor, únicamente si éste lo autoriza.
	
    }
    \UCitem{Versión}{1.0}
    \UCccsection{Administración de Requerimientos}
    \UCitem{Autor}{Ivo Sebastián Sam Álvarez-Tostado}
    \UCccitem{Evaluador}{José David Ortega Pacheco}
    \UCitem{Operación}{Consulta}
    \UCccitem{Prioridad}{Alta}
    \UCccitem{Complejidad}{Baja}
    \UCccitem{Volatilidad}{Baja}
    \UCccitem{Madurez}{Media}
    \UCitem{Estatus}{Por revisar}
    \UCitem{Fecha del último estatus}{10 de Abril del 2018}


%--------------------------------------------------------
%	\UCccsection{Revisión Versión 0.3} % Anote la versión que se revisó.
%	% FECHA: Anote la fecha en que se terminó la revisión.
%	\UCccitem{Fecha}{11-11-14} 
%	% EVALUADOR: Coloque el nombre completo de quien realizó la revisión.
%	\UCccitem{Evaluador}{Natalia Giselle Hernández Sánchez}
%	% RESULTADO: Coloque la palabra que mas se apegue al tipo de acción que el analista debe realizar.
%	\UCccitem{Resultado}{Corregir}
%	% OBSERVACIONES: Liste los cambios que debe realizar el Analista.
%	\UCccitem{Observaciones}{
%		\begin{UClist}
%			% PC: Petición de Cambio, describa el trabajo a realizar, si es posible indique la causa de la PC. Opcionalmente especifique la fecha en que considera razonable que se deba terminar la PC. No olvide que la numeración no se debe reiniciar en una segunda o tercera revisión.
%			\RCitem{PC1}{\DONE{Agregar a precondiciones el estado de la cuenta}}{Fecha de entrega}
%			\RCitem{PC2}{\DONE{Agregar el paso de la trayectoria de validación del estado de la cuenta}}{Fecha de entrega}
%			\RCitem{PC3}{\DONE{Agregar el mensaje de cuenta no activada a la sección de errores}}{Fecha de entrega}
%			\RCitem{PC4}{\DONE{Verificar las ligas a los estados}}{Fecha de entrega}
%			
%		\end{UClist}		
%	}
%--------------------------------------------------------

	\UCsection{Atributos}
	\UCitem{Actor}{
		\begin{UClist} 
\UCli \cdtRef{actor:CIEAlumno}{Alumno}
	\end{UClist}
}
	\UCitem{Propósito}{Proporciona una herramienta que le permita al alumno conocer el lugar y los horarios en los que puede buscar a algún profesor sin la necesidad de buscar esa información mediante sus compañeros, otros profesores o prefectos.}
	\UCitem{Entradas}{
        \begin{UClist} 
			\UCli No Aplica
        \end{UClist}}
	\UCitem{Salidas}{
		\UCli Fotografía del profesor
		\UCli Nombre del profesor
		\UCli Nombre del espacio asignado al profesor
		\UCli Academía a la que pertenece el profesor
		\UCli Tabla que muestra los datos de: 
			\begin{itemize}
				\item Número de cubículo (en caso de contar con él)
				\item Horario de Atención
				\item E-mail
				\item Teléfono
				\item Página Web
				\item Materias Impartidas
				\item Trabajos terminales dirigidos de interés del profesor por mostrar.
			\end{itemize}
	}
	\UCitem{Precondiciones}{
		\begin{UClist}		
			\UCli {\bf Interna:} Para que la fotografía y número de teléfono aparezcan el profesor deberá aceptar los términos y condiciones de la aplicación.
			\UCli {\bf Interna:} Contar con el catálogo de profesores.
			\UCli {\bf Interna:} Contar con la información cargada por profesor del catálogo de profesores.
		\end{UClist}
		}
	\UCitem{Postcondiciones}{
	    \begin{UClist}
		\UCli {\bf Externa:} Los alumnos podrán visualizar información mas detallada de un profesor en específico.
   	    \end{UClist}
	}
    \UCitem{Reglas de negocio}{
    	\begin{UClist}
            \UCli No Aplica
	\end{UClist}
    }
	\UCitem{Errores}{
	    \begin{UClist}
		\UCli No Aplica
%		\UCli \cdtIdRef{MSG22}{Nombre de usuario y/o contraseña incorrecto}: Se muestra en la pantalla \cdtIdRef{IUR 1}{Iniciar sesión} indicando que el nombre de usuario y/o contraseña son incorrectos.
%		\UCli \cdtIdRef{MSG27}{Cuenta no activada}: Se muestra en la pantalla \cdtIdRef{IUR 1}{Iniciar sesión} indicando que la cuenta no está activada.
	    \end{UClist}
	}
	\UCitem{Tipo}{Secundario, viene del caso de uso \cdtIdRef{CUPM-01}{Consultar Profesor}.}
 \end{UseCase}

 \begin{UCtrayectoria}
    \UCpaso[\UCactor] El alumno desea obtener información detallada del profesor tocando ls celda del profesor en la pantalla \cdtIdRef{IUPM-01}{Consultar Profesor}.
    \UCpaso[\UCsist] Obtiene la información del profesor. \label{CUPM-02:detalleProfesor}
    \UCpaso[\UCsist] Construye la tabla de la pantalla \cdtIdRef{IUPM-02}{Consultar Detalle de Profesor} con la información obtenida en el paso \ref{CUPM-02:detalleProfesor}.
    \UCpaso[\UCsist] Muestra la pantalla \cdtIdRef{IUPM-02}{Consultar Detalle de Profesor} con la información obtenida.
    \UCpaso[\UCactor] Toca el botón \cdtButton{Atrás}. \footnote{La funcionalidad del botón \textbf{Atrás} es un componente precargado del entorno de desarrollo XCode y puede variar dependiendo del idioma configurado por el usuario.}
    \UCpaso[\UCsist] Muestra la pantalla \cdtIdRef{IUPM-01}{Consultar Profesor}
 \end{UCtrayectoria}


 
%\subsection{Puntos de extensión}
%
%\UCExtensionPoint
%{El actor requiere recuperar su contraseña}
%{ Paso \ref{cur1:Acciones} de la trayectoria principal}
%{\cdtIdRef{CUR 2}{Recuperar contraseña}}
%
%\UCExtensionPoint
%{El actor requiere solicitar la inscripción de su escuela al programa}
%{ Paso \ref{cur1:Acciones} de la trayectoria principal}
%{\cdtIdRef{CUR 3}{Solicitar inscripción}}
 