\begin{UseCase}{CUPM-01}{Consultar Profesor}
	{
		Este caso de uso le permite al \cdtRef{actor:CIEAlumno}{Alumno} conocer el listado de la plantilla los profesores que están impartiendo clases en la ESCOM
			
		}
		\UCitem{Versión}{1.0}
		\UCccsection{Administración de Requerimientos}
		\UCitem{Autor}{Ivo Sebastián Sam Álvarez-Tostado}
		\UCccitem{Evaluador}{José David Ortega Pacheco}
		\UCitem{Operación}{Consulta}
		\UCccitem{Prioridad}{Alta}
		\UCccitem{Complejidad}{Baja}
		\UCccitem{Volatilidad}{Baja}
		\UCccitem{Madurez}{Media}
		\UCitem{Estatus}{Por revisar}
		\UCitem{Fecha del último estatus}{15 de Octubre del 2017}
		
		
		%--------------------------------------------------------
		%	\UCccsection{Revisión Versión 0.3} % Anote la versión que se revisó.
		%	% FECHA: Anote la fecha en que se terminó la revisión.
		%	\UCccitem{Fecha}{11-11-14} 
		%	% EVALUADOR: Coloque el nombre completo de quien realizó la revisión.
		%	\UCccitem{Evaluador}{Natalia Giselle Hernández Sánchez}
		%	% RESULTADO: Coloque la palabra que mas se apegue al tipo de acción que el analista debe realizar.
		%	\UCccitem{Resultado}{Corregir}
		%	% OBSERVACIONES: Liste los cambios que debe realizar el Analista.
		%	\UCccitem{Observaciones}{
		%		\begin{UClist}
		%			% PC: Petición de Cambio, describa el trabajo a realizar, si es posible indique la causa de la PC. Opcionalmente especifique la fecha en que considera razonable que se deba terminar la PC. No olvide que la numeración no se debe reiniciar en una segunda o tercera revisión.
		%			\RCitem{PC1}{\DONE{Agregar a precondiciones el estado de la cuenta}}{Fecha de entrega}
		%			\RCitem{PC2}{\DONE{Agregar el paso de la trayectoria de validación del estado de la cuenta}}{Fecha de entrega}
		%			\RCitem{PC3}{\DONE{Agregar el mensaje de cuenta no activada a la sección de errores}}{Fecha de entrega}
		%			\RCitem{PC4}{\DONE{Verificar las ligas a los estados}}{Fecha de entrega}
		%			
		%		\end{UClist}		
		%	}
		%--------------------------------------------------------
		
		\UCsection{Atributos}
		\UCitem{Actor}{
			\begin{UClist} 
				\UCli \cdtRef{actor:CIEAlumno}{Alumno}
			\end{UClist}
		}
		\UCitem{Propósito}{Proporcionarle al actor una herramienta que le permita con facilidad encontrar a un profesor de la Escuela Superior de Cómputo.}
		\UCitem{Entradas}{No Aplica
			%        \begin{UClist} 
			%           \UCli
			%           \UCli
			%        \end{UClist}
		}
		\UCitem{Salidas}{
			\begin{UClist} 
				\UCli Fotografía del profesor
				\UCli Nombre del profesor
				\UCli Número de salón o cubículo del profesor
			\end{UClist}	
		}
		\UCitem{Precondiciones}{
			\begin{UClist}		
				\UCli {\bf Interna:} El sistema debe tener cargados los profesores de la Escuela Superior de Cómputo.
			\end{UClist}
		}
		\UCitem{Postcondiciones}{
			\begin{UClist}
				\UCli {\bf Externa:} Los alumnos que deseen obtener asesorías de algún profesor conocido por su especialidad pordrán localizarlo.
			\end{UClist}
		}
		\UCitem{Reglas de negocio}{
			\begin{UClist}
				       \UCli No Aplica
			\end{UClist}
		}
		\UCitem{Errores}{
			\begin{UClist}
						\UCli No Aplica
			\end{UClist}
		}
		\UCitem{Tipo}{Primario.}
		%	\UCitem{Fuente}{
		%	    \begin{UClist}
		%        \UCli Minuta de la reunión \cdtIdRef{M-3TR}{Toma de requerimientos}.
		%	    \end{UClist}
		%	}
	\end{UseCase}
	
	\begin{UCtrayectoria}
		\UCpaso[\UCactor] Desea conocer a los profesores que estan impartiendo clases tocando el botón \botProfesores de la pantalla \cdtIdRef{IUPP}{Pantalla Principal}
		
		\UCpaso[\UCsist] Obtiene la lista de los profesores registrados.
		
		\UCpaso[\UCsist] Muestra en la pantalla \cdtIdRef{IUPM-01}{Consultar Profesor}
		
		\UCpaso[\UCactor] Desea conocer mas detalles de un profesor tocando el nombre del profesor seleccionado. \refTray{A}
		
		\UCpaso[\UCsist] Ejecuta el caso de uso \cdtIdRef{CUPM-02}{Consultar Detalles de Profesor}
		
	\end{UCtrayectoria}
	
	
	\begin{UCtrayectoriaA}{A}{El alumno no desea conocer el nivel y salón.}
		\UCpaso[\UCactor] Toca el botón \cdtButton{Atrás} de la pantalla .
		\UCpaso[\UCsist] Muestra la pantalla \cdtIdRef{IUPP}{Pantalla Principal}.
	\end{UCtrayectoriaA}
	
	% 
	%  \begin{UCtrayectoriaA}{B}{El actor visitante desea consultar las áreas de ESCOM.}
	% 	\UCpaso[\UCactor] Desea conocer la ubicación de un espacio en ESCOM presionando la opción Áreas de ESCOM del menú principal \textbf{CIE-IU001}
	% 	
	% 	\UCpaso[\UCsist] Obtiene la vista aérea de la Escuela Superior de cómputo y los polígonos de las zonas definidas como marcadores de los distintos espacios de la escuela como se muestra en la pantalla CIE-IU003
	% 	
	% 	\UCpaso[\UCactor] Desea saber como llegar a un espacio específico presionando su marcador. \ref{cur1:consultar}
	% \end{UCtrayectoriaA}
	
	
	
	%\UCExtensionPoint
	%{El actor requiere solicitar la inscripción de su escuela al programa}
	%{ Paso \ref{cur1:Acciones} de la trayectoria principal}
	%{\cdtIdRef{CUR 3}{Solicitar inscripción}}
