\begin{UseCase}{CUPM-01}{Consultar Profesores}
	{
		Este caso de uso le permite al \cdtRef{actor:CIEAlumno}{Alumno} conocer el listado de los profesores adscritos a la Escuela Superior de Cómputo y que están impartiendo clases en ella o que se encuentren cubriendo algún puesto administrativo.
		Cuando el alumno requiera conocer la plantilla docente de la escuela con la finalidad de consultar posteriormente el medio de contacto o ubicación de algún profesor en partícular, este caso de uso le mostrará una lista ordenada de los profesores que podrá consultar. 
			
		}
		\UCitem{Versión}{1.0}
		\UCccsection{Administración de Requerimientos}
		\UCitem{Autor}{Ivo Sebastián Sam Álvarez-Tostado}
		\UCccitem{Evaluador}{José David Ortega Pacheco}
		\UCitem{Operación}{Consulta}
		\UCccitem{Prioridad}{Alta}
		\UCccitem{Complejidad}{Baja}
		\UCccitem{Volatilidad}{Baja}
		\UCccitem{Madurez}{Media}
		\UCitem{Estatus}{Por revisar}
		\UCitem{Fecha del último estatus}{10 de abril del 2018}
		
		
		%--------------------------------------------------------
		%	\UCccsection{Revisión Versión 0.3} % Anote la versión que se revisó.
		%	% FECHA: Anote la fecha en que se terminó la revisión.
		%	\UCccitem{Fecha}{11-11-14} 
		%	% EVALUADOR: Coloque el nombre completo de quien realizó la revisión.
		%	\UCccitem{Evaluador}{Natalia Giselle Hernández Sánchez}
		%	% RESULTADO: Coloque la palabra que mas se apegue al tipo de acción que el analista debe realizar.
		%	\UCccitem{Resultado}{Corregir}
		%	% OBSERVACIONES: Liste los cambios que debe realizar el Analista.
		%	\UCccitem{Observaciones}{
		%		\begin{UClist}
		%			% PC: Petición de Cambio, describa el trabajo a realizar, si es posible indique la causa de la PC. Opcionalmente especifique la fecha en que considera razonable que se deba terminar la PC. No olvide que la numeración no se debe reiniciar en una segunda o tercera revisión.
		%			\RCitem{PC1}{\DONE{Agregar a precondiciones el estado de la cuenta}}{Fecha de entrega}
		%			\RCitem{PC2}{\DONE{Agregar el paso de la trayectoria de validación del estado de la cuenta}}{Fecha de entrega}
		%			\RCitem{PC3}{\DONE{Agregar el mensaje de cuenta no activada a la sección de errores}}{Fecha de entrega}
		%			\RCitem{PC4}{\DONE{Verificar las ligas a los estados}}{Fecha de entrega}
		%			
		%		\end{UClist}		
		%	}
		%--------------------------------------------------------
		
		\UCsection{Atributos}
		\UCitem{Actor}{
			\begin{UClist} 
				\UCli \cdtRef{actor:CIEAlumno}{Alumno}
			\end{UClist}
		}
		\UCitem{Propósito}{Proporcionarle al actor una herramienta que le permita encontrar a un profesor de la Escuela Superior de Cómputo en el listado de los docentes para consultar el detalle de su información si así lo desea.}
		\UCitem{Entradas}{No Aplica
			%        \begin{UClist} 
			%           \UCli
			%           \UCli
			%        \end{UClist}
		}
		\UCitem{Salidas}{
			\begin{UClist} 
				\UCli Fotografía del profesor
				\UCli Nombre del profesor
				\UCli Número de salón o cubículo del profesor
			\end{UClist}	
		}
		\UCitem{Precondiciones}{
			\begin{UClist}		
				\UCli {\bf Interna:} El sistema debe tener cargados los profesores adcritos a la Escuela Superior de Cómputo.
			\end{UClist}
		}
		\UCitem{Postcondiciones}{
			\begin{UClist}
				\UCli {\bf Externa:} Los alumnos que deseen conocer la plantilla docente de la ESCOM podrán ver una lista de ellos.
			\end{UClist}
		}
		\UCitem{Reglas de negocio}{
			\begin{UClist}
				       \UCli No Aplica
			\end{UClist}
		}
		\UCitem{Errores}{
			\begin{UClist}
						\UCli No Aplica
			\end{UClist}
		}
		\UCitem{Tipo}{Primario.}
		%	\UCitem{Fuente}{
		%	    \begin{UClist}
		%        \UCli Minuta de la reunión \cdtIdRef{M-3TR}{Toma de requerimientos}.
		%	    \end{UClist}
		%	}
	\end{UseCase}
	
	\begin{UCtrayectoria}
		\UCpaso[\UCactor] Desea conocer a los profesores que están impartiendo clases o cuentan con algún puesto administrativo en la Escuela Superior de Cómputo tocando el botón \botProfesores de la pantalla \cdtIdRef{IUPP}{Pantalla Principal}
		
		\UCpaso[\UCsist] Obtiene la lista de los profesores registrados.
		
		\UCpaso[\UCsist] Muestra en la pantalla \cdtIdRef{IUPM-01}{Consultar Profesor} la lista de los profesores adscritos a la ESCOM de los que podrá consultar el detalle de su información si así lo desea. \label{CUPM-01:detalles}
				
%		\UCpaso[\UCactor] Desea conocer mas detalles de un profesor tocando el nombre del profesor seleccionado. \refTray{A}
%		
%		\UCpaso[\UCsist] Ejecuta el caso de uso \cdtIdRef{CUPM-02}{Consultar Detalles de Profesor}
		
	\end{UCtrayectoria}
	
	
%	\begin{UCtrayectoriaA}{A}{El alumno no desea conocer el nivel y salón.}
%		\UCpaso[\UCactor] Toca el botón \cdtButton{Atrás} de la pantalla .
%		\UCpaso[\UCsist] Muestra la pantalla \cdtIdRef{IUPP}{Pantalla Principal}.
%	\end{UCtrayectoriaA}
	
	% 
	%  \begin{UCtrayectoriaA}{B}{El actor visitante desea consultar las áreas de ESCOM.}
	% 	\UCpaso[\UCactor] Desea conocer la ubicación de un espacio en ESCOM presionando la opción Áreas de ESCOM del menú principal \textbf{CIE-IU001}
	% 	
	% 	\UCpaso[\UCsist] Obtiene la vista aérea de la Escuela Superior de cómputo y los polígonos de las zonas definidas como marcadores de los distintos espacios de la escuela como se muestra en la pantalla CIE-IU003
	% 	
	% 	\UCpaso[\UCactor] Desea saber como llegar a un espacio específico presionando su marcador. \ref{cur1:consultar}
	% \end{UCtrayectoriaA}
	
	\subsection{Puntos de extensión}
	
	\UCExtensionPoint
	{El actor requiere consultar el detalle de algún profesor en particular}
	{ Paso \ref{CUPM-01:detalles} de la trayectoria principal}
	{\cdtIdRef{CUPM-02}{Consultar Detalle de Profeso}}
