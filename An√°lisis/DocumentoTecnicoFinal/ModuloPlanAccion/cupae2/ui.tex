\subsection{IUPAE 2 Modificar meta de ambiente escolar}
                     
\subsubsection{Objetivo}

   Esta pantalla permite al actor modificar una \cdtRef{gls:meta}{meta} de la línea de acción ``Ambiente escolar'', cuando existen errores en la información o simplemente cuando esta se haya actualizado.

	
\subsubsection{Diseño}
  En la figura ~\ref{IUPAE 2} se muestra la pantalla ``Modificar meta de ambiente escolar'', en la cual se mostrará la información de la meta en modo edición para que
  de esta forma puedan ingresarse los nuevos datos.\\

  El actor deberá modificar los datos referentes a la meta, así como definir si la meta está enfocada a la mejora de espacio, en este caso 
  cuando el actor seleccione un tipo de mejora, el sistema mostrará el mensaje \cdtIdRef{MSG38}{Número de mejoras a realizar} como etiqueta del campo de número de mejoras, indicando el tipo de mejora,
  como se muestra en la figura ~\ref{IUPAE 2.1}.\\
  
  Si el actor selecciona que la meta no está enfocada a la mejora de espacio, el sistema mostrará la sección ``Información de la capacitación y/o sensibilización'',
  si el actor selecciona que la meta está enfocada a dar capacitación y/o sensibilización, el sistema mostrará los campos relacionados y
  el actor deberá ingresar la información solicitada, como se muestra en la pantalla ~\ref{IUPAE 2.2}.\\
  
  Si el actor selecciona que la meta no está enfocada a una capacitación y/o sensibilización, se mostrará la sección ``Cuantificar meta'' como se muestra en la pantalla ~\ref{IUPAE 2.3}, el actor 
  deberá ingresar los datos referentes al valor a alcanzar.\\
 
  \IUfig[.9]{pantallas/planAccion/cupae2/iupae2}{IUPAE 2}{Modificar meta de ambiente escolar}
  \IUfig[.9]{pantallas/planAccion/cupae2/iupae2_1}{IUPAE 2.1}{Modificar meta de ambiente escolar: Mejora de espacio}
  \IUfig[.9]{pantallas/planAccion/cupae2/iupae2_2}{IUPAE 2.2}{Modificar meta de ambiente escolar: Capacitación y/o sensibilización}
  \IUfig[.9]{pantallas/planAccion/cupae2/iupae2_3}{IUPAE 2.3}{Modificar meta de ambiente escolar: Cuantificar meta}
  
  El sistema buscará las unidades disponibles para colocarlas como opciones en la lista desplegable ``Unidad'', si el actor selecciona la opción ``Otra'' de la lista desplegable, el sistema mostrará
  el campo ``Nueva unidad'' para que el actor defina la nueva unidad.\\
  
  Al oprimir el botón \cdtButton{Aceptar} el sistema validará la información ingresada y señalará los campos cuyos datos no cumplan con las reglas de negocio establecidas.\\
  
  Finalmente se mostrará el mensaje \cdtIdRef{MSG1}{Operación realizada exitosamente} en la pantalla \cdtIdRef{IUP 5}{Administrar metas}, para indicar que la información de la
  meta se ha modificado correctamente.
    
\subsubsection{Comandos}
\begin{itemize}
	\item \cdtButton{Aceptar}: Permite al actor confirmar la modificación de la meta, dirige a la pantalla \cdtIdRef{IUP 5}{Administrar metas}.
	\item \cdtButton{Cancelar}: Permite al actor cancelar la modificación de la meta, dirige a la pantalla \cdtIdRef{IUP 5}{Administrar metas}.
\end{itemize}


\subsubsection{Mensajes}

\begin{description}
	\item[\cdtIdRef{MSG1}{Operación realizada exitosamente}:] Se muestra en la pantalla \cdtIdRef{IUP 5}{Administrar metas} cuando la meta se ha modificado correctamente.
	\item[\cdtIdRef{MSG4}{No se encontró información sustantiva}:] Se muestra en la pantalla \cdtIdRef{IUP 5}{Administrar metas} cuando hace falta información referente al tipo de mejora, espacio a mejorar o a la unidad.
	\item[\cdtIdRef{MSG5}{Falta un dato requerido para efectuar la operación solicitada}:] Se muestra en la pantalla \cdtIdRef{IUPAE 2}{Modificar meta de ambiente escolar} cuando el actor no ingresó un dato requerido para realizar la operación.
	\item[\cdtIdRef{MSG6}{Formato incorrecto}:] Se muestra en la pantalla \cdtIdRef{IUPAE 2}{Modificar meta de ambiente escolar} cuando el formato de alguno de los datos ingresados es incorrecto.
	\item[\cdtIdRef{MSG7}{Se ha excedido la longitud máxima del campo}:] Se muestra en la pantalla \cdtIdRef{IUPAE 2}{Modificar meta de ambiente escolar} cuando el actor escribió un dato que excede el tamaño especificado por el sistema.
		\item[\cdtIdRef{MSG28}{Operación no permitida por estado de la entidad}:] Se muestra en la pantalla en que se encuentre navegando el actor debido al estado en que se encuentra la escuela.	
	\item[\cdtIdRef{MSG38}{Número de mejoras a realizar}:] Se muestra en la pantalla \cdtIdRef{IUPAE 2}{Modificar meta de ambiente escolar} como etiqueta en el campo de número de mejoras.
	\item[\cdtIdRef{MSG41}{Acción fuera del periodo}:] Se muestra en la pantalla en que se encuentre navegando el actor indicando que la fecha no se encuentra dentro del periodo de registro de plan de acción.
\end{description}
