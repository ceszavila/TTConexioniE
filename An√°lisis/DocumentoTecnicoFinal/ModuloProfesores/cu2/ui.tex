\subsection{IUPM-02 Consultar Detalle de Profesor}

\subsubsection{Objetivo}

% Explicar el objetivo para el que se construyo la interfaz, generalmente es la descripción de la actividad a desarrollar, como Seleccionar grupos para inscribir materias de un alumno, controlar el acceso al sistema mediante la solicitud de un login y password de los usuarios, etc.
	
    Esta pantalla permite al \cdtRef{actor:CIEAlumno}{Alumno} consultar los detalles de un profesor. Esto con la finalidad de poder visualizar los horarios de atención de un profesor así como los medios de contacto que tiene.\\
    El alumno podrá visualizar el télefono y fotografía del profesor, si y solo si el profesor lo autoriza.
\subsubsection{Diseño}

% Presente la figura de la interfaz y explique paso a paso ``a manera de manual de usuario'' como se debe utilizar la interfaz. No olvide detallar en la redacción los datos de entradas y salidas. Explique como utilizar cada botón y control de la pantalla, para que sirven y lo que hacen. Si el Botón lleva a otra pantalla, solo indique la pantalla y explique lo que pasará cuando se cierre dicha pantalla (la explicación sobre el funcionamiento de la otra pantalla estará en su archivo correspondiente).

    En la figura~\ref{IUPM-02} se muestra la pantalla ``Consultar Detalle de profesor'', en ella se muestra una tabla que contiene la información del cubículo donde se encuentra el profesor, sus medios de contacto, así como las materias que ha impartido y los trabajos terminales que ha dirigido.
    \IUfig[.3]{/ModuloProfesores/P2.png}{IUPM-02}{Consultar Detalle De Profesor}



\subsubsection{Comandos}
    \begin{itemize}

	\item Botón \cdtButton{Atrás}, dirige a la pantalla \cdtIdRef{IUPM-01}{Consultar Profesor}
    \end{itemize}
