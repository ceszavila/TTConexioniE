\subsection{IUUM-5.4.1  Consultar  Detalle de Unidad de Aprendizaje}

\subsubsection{Objetivo}

% Explicar el objetivo para el que se construyo la interfaz, generalmente es la descripción de la actividad a desarrollar, como Seleccionar grupos para inscribir materias de un alumno, controlar el acceso al sistema mediante la solicitud de un login y password de los usuarios, etc.
	
    Esta pantalla permite al \cdtRef{actor:CIEAlumno}{Alumno} visualizar la información relacionada a una unidad de aprendizaje en particular, permitiendo al actor visualizar el tipo de unidad de aprendizaje, academia, nombre y programa académico de dicha unidad.
\subsubsection{Diseño}

% Presente la figura de la interfaz y explique paso a paso ``a manera de manual de usuario'' como se debe utilizar la interfaz. No olvide detallar en la redacción los datos de entradas y salidas. Explique como utilizar cada botón y control de la pantalla, para que sirven y lo que hacen. Si el Botón lleva a otra pantalla, solo indique la pantalla y explique lo que pasará cuando se cierre dicha pantalla (la explicación sobre el funcionamiento de la otra pantalla estará en su archivo correspondiente).

    En la figura \ref{IUUM-5.4.1} se muestra la pantalla ``Consultar Detalle de Unidad de Aprendizaje'', en la cual se muestra toda la información relacionada con una unidad de aprendizaje, dentro de esta pantalla aparece información como:
    \begin{itemize}
    	\item Nombre de la unidad de aprendizaje
    	\item Academia a la que pertenece la unidad de aprendizaje.
    	\item Tipo de unidad de aprendizaje.
    	\item Programa académico de la unidad de aprendizaje.
	\end{itemize} 
    \IUfig[.3]{/ModuloUnidades/IUUM5_4_1}{IUUM-5.4.1}{Consultar Detalle de Unidad de Aprendizaje}



\subsubsection{Comandos}
    \begin{itemize}
	\item \cdtButton{Atrás}, dirige a la pantalla \cdtIdRef{IUUM-5.4}{Consultar Unidades de Aprendizaje}
    \end{itemize}
