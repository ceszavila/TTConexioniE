\subsection{IUAU-5.2 Editar Unidad de Aprendizaje}

\subsubsection{Objetivo}

% Explicar el objetivo para el que se construyo la interfaz, generalmente es la descripción de la actividad a desarrollar, como Seleccionar grupos para inscribir materias de un alumno, controlar el acceso al sistema mediante la solicitud de un login y password de los usuarios, etc.
	
    Esta pantalla permite al \cdtRef{actor:CIEEstructura}{Responsable EE} modificar la información de las unidades de aprendizaje que estarán disponibles para un semestre de un ciclo escolar vigente.
\subsubsection{Diseño}

% Presente la figura de la interfaz y explique paso a paso ``a manera de manual de usuario'' como se debe utilizar la interfaz. No olvide detallar en la redacción los datos de entradas y salidas. Explique como utilizar cada botón y control de la pantalla, para que sirven y lo que hacen. Si el Botón lleva a otra pantalla, solo indique la pantalla y explique lo que pasará cuando se cierre dicha pantalla (la explicación sobre el funcionamiento de la otra pantalla estará en su archivo correspondiente).

    En la figura \ref{IUAU-5.2} se muestra la pantalla ``Editar Unidad de Aprendizaje'', la cual esta compuesta por los botones  \cdtButton{Actualizar}, \cdtButton{Eliminar} y \cdtButton{Cancelar} así como los campos de entrada denomidados: nombre, tipo, academia y programa académico.
     
    \IUfig[.3]{/ModuloUnidades/IUAU5_2}{IUAU-5.2}{Editar Unidad de Aprendizaje}



\subsubsection{Comandos}
    \begin{itemize}
	\item \cdtButton{Actualizar}, dirige a la pantalla \cdtIdRef{IUAU-5}{Gestionar Unidades de Aprendizaje}.
	\item \cdtButton{Eliminar}, dirige a la pantalla \cdtIdRef{IUAU-5}{Gestionar Unidades de Aprendizaje}..
	\item \cdtButton{Cancelar}, dirige a la pantalla \cdtIdRef{IUAU-5}{Gestionar Unidades de Aprendizaje}
    \end{itemize}
