\subsection{IUSM-03 Consultar Edificio de Salón}

\subsubsection{Objetivo}

% Explicar el objetivo para el que se construyo la interfaz, generalmente es la descripción de la actividad a desarrollar, como Seleccionar grupos para inscribir materias de un alumno, controlar el acceso al sistema mediante la solicitud de un login y password de los usuarios, etc.
	
    Esta pantalla permite al \cdtRef{actor:CIEAlumno}{Alumno} consultar los detalles del edficio seleccionado.
\subsubsection{Diseño}

% Presente la figura de la interfaz y explique paso a paso ``a manera de manual de usuario'' como se debe utilizar la interfaz. No olvide detallar en la redacción los datos de entradas y salidas. Explique como utilizar cada botón y control de la pantalla, para que sirven y lo que hacen. Si el Botón lleva a otra pantalla, solo indique la pantalla y explique lo que pasará cuando se cierre dicha pantalla (la explicación sobre el funcionamiento de la otra pantalla estará en su archivo correspondiente).

    En la figura~\ref{IUSM-03} se muestra la pantalla ``Consultar Edificio de Salón" en donde se muestra el nombre, identificador, númro de niveles,listado de servicios con el que cuenta el edificio, asi como los tipos de aulas que tiene.
    \IUfig[.3]{/ModuloSalones/S4.png}{IUSM-03}{Consultar Edificio de Salón}



\subsubsection{Comandos}
    \begin{itemize}

	\item Botón \cdtButton{Atrás}, dirige a la pantalla \cdtIdRef{IUSM-02}{Consultar Nivel de Salón} 
    \end{itemize}
