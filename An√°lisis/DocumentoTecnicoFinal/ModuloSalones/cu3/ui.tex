\subsection{IUSM-01 Consultar Asignación de Grupo}

\subsubsection{Objetivo}

% Explicar el objetivo para el que se construyo la interfaz, generalmente es la descripción de la actividad a desarrollar, como Seleccionar grupos para inscribir materias de un alumno, controlar el acceso al sistema mediante la solicitud de un login y password de los usuarios, etc.
	
    Esta pantalla permite al \cdtRef{actor:CIEAlumno}{Alumno} consultar los la asignación de grupos que se realiza por periodo escolar para ubicar los salones o laboratorios en los que tomará clase. El alumno puede utilizar la barra de búsqueda para encontrar el grupo específico.
\subsubsection{Diseño}

% Presente la figura de la interfaz y explique paso a paso ``a manera de manual de usuario'' como se debe utilizar la interfaz. No olvide detallar en la redacción los datos de entradas y salidas. Explique como utilizar cada botón y control de la pantalla, para que sirven y lo que hacen. Si el Botón lleva a otra pantalla, solo indique la pantalla y explique lo que pasará cuando se cierre dicha pantalla (la explicación sobre el funcionamiento de la otra pantalla estará en su archivo correspondiente).

    En la figura~\ref{IUSM-01} se muestra la pantalla ``Consultar asignación de grupos" en donde se muestra el nombre del grupo y el salón que fue asignado a él..
    
    \IUfig[.3]{/ModuloSalones/S1.png}{IUSM-01}{Consultar Asignación de Grupo}
    
	 \IUfig[.3]{/ModuloSalones/S1_1.png}{IUSM-01a}{Consultar Asignación de Grupo por tipo}


\subsubsection{Comandos}
    \begin{itemize}

	\item Botón \cdtButton{Atrás}, dirige a la pantalla \cdtIdRef{IUPA}{Pantalla Principal Administrador}
	\item \textbf{Barra de búesqueda}, permite localizar con mayor facilidad la asignación de espacio a grupo.
	\item \textbf{Barra de tipos de espacio}, permite agrupar las asignaciones por su tipo de espacio.
    \end{itemize}
