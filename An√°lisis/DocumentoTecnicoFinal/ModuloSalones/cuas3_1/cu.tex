\begin{UseCase}{CUAS-03.1}{Registrar salón}{
	Permite llevar a cabo el registro de un nuevo salón para la Escuela Superior de Cómputo debido a que se requiere la asignación de un espacio qué será utilizado para impartir unidades de aprendizaje, laboratorio, espacios administrativos o sala de trabajo terminal. \\
    }
    \UCitem{Versión}{1.0}
    \UCccsection{Administración de Requerimientos}
    \UCitem{Autor}{Ivo Sebastián Sam Álvarez-Tostado}
    \UCccitem{Evaluador}{José David Ortega Pacheco}
    \UCitem{Operación}{Registro}
    \UCccitem{Prioridad}{Alta}
    \UCccitem{Complejidad}{Baja}
    \UCccitem{Volatilidad}{Baja}
    \UCccitem{Madurez}{Media}
    \UCitem{Estatus}{Por revisar}
    \UCitem{Fecha del último estatus}{20 de mayo del 2018}


%--------------------------------------------------------
%	\UCccsection{Revisión Versión 0.3} % Anote la versión que se revisó.
%	% FECHA: Anote la fecha en que se terminó la revisión.
%	\UCccitem{Fecha}{11-11-14} 
%	% EVALUADOR: Coloque el nombre completo de quien realizó la revisión.
%	\UCccitem{Evaluador}{Natalia Giselle Hernández Sánchez}
%	% RESULTADO: Coloque la palabra que mas se apegue al tipo de acción que el analista debe realizar.
%	\UCccitem{Resultado}{Corregir}
%	% OBSERVACIONES: Liste los cambios que debe realizar el Analista.
%	\UCccitem{Observaciones}{
%		\begin{UClist}
%			% PC: Petición de Cambio, describa el trabajo a realizar, si es posible indique la causa de la PC. Opcionalmente especifique la fecha en que considera razonable que se deba terminar la PC. No olvide que la numeración no se debe reiniciar en una segunda o tercera revisión.
%			\RCitem{PC1}{\DONE{Agregar a precondiciones el estado de la cuenta}}{Fecha de entrega}
%			\RCitem{PC2}{\DONE{Agregar el paso de la trayectoria de validación del estado de la cuenta}}{Fecha de entrega}
%			\RCitem{PC3}{\DONE{Agregar el mensaje de cuenta no activada a la sección de errores}}{Fecha de entrega}
%			\RCitem{PC4}{\DONE{Verificar las ligas a los estados}}{Fecha de entrega}
%			
%		\end{UClist}		
%	}
%--------------------------------------------------------

	\UCsection{Atributos}
	\UCitem{Actor}{
		\begin{UClist} 
\UCli \cdtRef{actor:CIEPrefectura}{Responsable Prefectura}
	\end{UClist}
}
	\UCitem{Propósito}{Proporcionar una herramienta que permita realizar el registro de nuevos salones para la Escuela Superior de Cómputo.}
	\UCitem{Entradas}{
        \begin{UClist} 
           \UCli Nombre
           \UCli Número
           \UCli Nivel
           \UCli Edificio
           \UCli Tipo
           \UCli Descripción
        \end{UClist}}
	\UCitem{Salidas}{
		\begin{UClist}
			\UCli Longitud
			\UCli Latitud
			\UCli Mapa de la Escuela Superior de Cómputo.
		\end{UClist}	
	}
	\UCitem{Precondiciones}{{\bf Externa:} Debe existir el espacio en la Escuela Superior de Cómputo.}
	\UCitem{Postcondiciones}{
	    \begin{UClist}
		\UCli {\bf Externa:} Los alumnos podrán consultar el registro de los salones.
   	    \end{UClist}
	}
    \UCitem{Reglas de negocio}{\cdtIdRef{RN-S1}{Campos Obligatorios}}
	\UCitem{Errores}{Ninguno}
	\UCitem{Tipo}{Secundario, viene de \cdtIdRef{CUAS-03}{Gestionar salones}}
%	\UCitem{Fuente}{

 \end{UseCase}

 \begin{UCtrayectoria}
    
    \UCpaso [\UCactor] Solicta registrar un nuevo salón tocando el botón \textbf{Registrar Salón} de la pantalla \cdtIdRef{IUAS-03}{Gestionar salones}.
    
    \UCpaso Obtiene la locaclización de la Escuela Superior de Cómputo.
    
    \UCpaso Muestra la pantalla \cdtIdRef{IUAS-03.1}{Registrar salón}.
    
    \UCpaso [\UCactor] Ingresa los datos solicitados.
    
    \UCpaso [\UCactor] Solicita finalizar la operación tocando el botón \textbf{Registrar}. \refTray{A}
    
    \UCpaso Verifica que se haya ingresado los campos marcados como obligatorios, con base en la regla \cdtIdRef{RN-S1}{Datos Obligatorios}.
    
    \UCpaso Persiste la información registrada.
    
    \UCpaso Regresa a la pantalla \cdtIdRef{IUAS-03}{Gestionar salones}.
    
\end{UCtrayectoria}

\begin{UCtrayectoriaA}{A}{Cuando el actor requiere cancelar la operación.}
	
	\UCpaso [\UCactor] Solicita cancelar la operación tocando el botón \textbf{Volver}.
	
	\UCpaso Regresa a la pantalla \cdtIdRef{IUAS-03}{Gestionar salones}.
	
\end{UCtrayectoriaA}


