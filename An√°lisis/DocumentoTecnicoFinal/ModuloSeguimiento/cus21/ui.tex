\subsection{IUS 21 Actualizar información del consumo de agua}

\subsubsection{Objetivo}

      En esta pantalla el \cdtRef{actor:usuarioEscuela}{Coordinador del programa} puede actualizar la información referente al consumo de agua en el caso de que esta haya cambiado.

\subsubsection{Diseño}

    En la figura~\ref{IUS 21} se muestra la pantalla ``Actualizar información del consumo de agua'', por medio de la cual se podrá actualizar la información referente al consumo de agua en el caso de que esta haya cambiado. El actor deberá ingresar la información solicitada en la pantalla, en la cual aparecerán los datos referentes a los tipos de abastecimiento con que cuenta la escuela y el consumo e importe total.\\
    
    Si el actor selecciona la opción ``No'' en la pregunta ``¿Han cambiado los tipos de abastecimiento de agua con los que cuenta la escuela?'' el sistema únicamente mostrará los botones \cdtButton{Aceptar}, \cdtButton{Cancelar} y la pregunta referente a la existencia de recibos del consumo de agua como se muestra en la figura~\ref{IUS 21.1}. En caso contrario se mostrarán las preguntas referentes a los tipos de abastecimiento de agua como se muestra en la figura~\ref{IUS 21.2}\\
    
    Si el actor selecciona la opción ``Sin acceso al agua'' en la pregunta ``¿Con qué tipos de abastecimiento de agua cuenta la escuela?'' el sistema únicamente mostrará los botones \cdtButton{Aceptar} y \cdtButton{Cancelar} sin solicitar mas información como se muestra en la figura~\ref{IUS 21.3}.\\
    
    Si se selecciona la opción ``Si'' en la pregunta ``¿Cuenta con recibos sobre el consumo de agua en la escuela?'' se solicitará la frecuencia con la que recibe los recibos de cobro por concepto de agua como se muestra en la figura~\ref{IUS 21.4}. Si elige la opción ``Anual'' se mostrarán dos campos de texto para el ``Consumo anual'' en metros cúbicos y el ``Importe anual'' en pesos respectivamente.\\
    
    En el caso de que seleccione la opción ``Bimestral'', deberá seleccionar el bimestre al que corresponde cada uno de los recibos que registrará e ingresar el consumo de agua por bimestre en metros cúbicos y el importe bimestral en pesos. Los registros por cada uno de los recibos bimestrales se mostrarán en una tabla, el consumo e importe por cada bimestre se sumará para mostrar los totales de ambos datos al final de la tabla como se muestra en la figura~\ref{IUS 21.5}. Lo mismo ocurrirá para las opciones ``Mensual'' y ``Semestral''. \\
    
    Si se selecciona la opción ``No'' en la pregunta ``¿Cuenta con recibos sobre el consumo de agua en la escuela?'' se solicitará únicamente el consumo de agua e importe anual promedio, como se muestra en la figura~\ref{IUS 21.6}.\\
    
    Una vez que se haya ingresado toda la información solicitada para el registro de la información deberá oprimir el botón \cdtButton{Aceptar}, el sistema validará y registrará la información sólo si se han cumplido todas las reglas de negocio establecidas.\\
    
    Finalmente se mostrará el mensaje \cdtIdRef{MSG1}{Operación realizada exitosamente} en la pantalla \cdtIdRef{IUS 18}{Administrar avances de agua}, para indicar que la información referente al consumo de agua se ha actualizado exitosamente.
      
    \IUfig[.9]{pantallas/seguimiento/cus21/IUS21ActualizarInformacion.png}{IUS 21}{Actualizar información del consumo de agua}
    \IUfig[.9]{pantallas/seguimiento/cus21/IUS21ActualizarInformacion1.png}{IUS 21.1}{Actualizar información del consumo de agua: Sin cambios en los abastecimientos de agua}
    \IUfig[.9]{pantallas/seguimiento/cus21/IUS21ActualizarInformacion2.png}{IUS 21.2}{Actualizar información del consumo de agua: Cambios en los abastecimientos de agua}
    \IUfig[.9]{pantallas/seguimiento/cus21/IUS21ActualizarInformacion3.png}{IUS 21.3}{Actualizar información del consumo de agua: Sin acceso al agua}
    \IUfig[.9]{pantallas/seguimiento/cus21/IUS21ActualizarInformacion4.png}{IUS 21.4}{Actualizar información del consumo de agua: Con recibos de agua}
    \IUfig[.9]{pantallas/seguimiento/cus21/IUS21ActualizarInformacion5.png}{IUS 21.5}{Actualizar información del consumo de agua: Recibos bimestrales}
    \IUfig[.9]{pantallas/seguimiento/cus21/IUS21ActualizarInformacion6.png}{IUS 21.6}{Actualizar información del consumo de agua: Sin recibos de agua}
    


\subsubsection{Comandos}
    \begin{itemize}
	\item \cdtButton{Aceptar}: Permite al actor actualizar la información referente al consumo de agua, dirige a la pantalla \cdtIdRef{IUS 18}{Administrar avances de agua}.
	\item \cdtButton{Cancelar}: Permite al actor cancelar la información referente al consumo de agua, dirige a la pantalla \cdtIdRef{IUS 18}{Administrar avances de agua}.
    \end{itemize}

\subsubsection{Mensajes}

    \begin{description}
	  \item [\cdtIdRef{MSG1}{Operación realizada exitosamente}:] Se muestra en la pantalla \cdtIdRef{IUS 18}{Administrar avances de agua} cuando la actualización de la información referente al consumo de agua se ha realizado correctamente.
	    
	  \item [\cdtIdRef{MSG5}{Falta un dato requerido para efectuar la operación solicitada}:] Se muestra en la pantalla \cdtIdRef{IUS 21}{Actualizar información del consumo de agua} cuando no se ha ingresado un dato marcado como requerido.
	    
	  \item  [\cdtIdRef{MSG6}{Formato incorrecto}:] Se muestra en la pantalla \cdtIdRef{IUS 21}{Actualizar información del consumo de agua} cuando el tipo de dato ingresado no cumple con el tipo de dato solicitado en el campo.
	    
	  \item [\cdtIdRef{MSG7}{Se ha excedido la longitud máxima del campo}:] Se muestra en la pantalla \cdtIdRef{IUS 21}{ Actualizar información del consumo de agua} cuando se ha excedido la longitud de alguno de los campos.	    
	  
	  \item [\cdtIdRef{MSG28}{Operación no permitida por estado de la entidad}:] Se muestra en la pantalla \cdtIdRef{IUS 18}{Administrar avances de agua} indicando al actor que no se puede realizar la operación debido al estado en que se encuentra la escuela.
	    
	    \item [\cdtIdRef{MSG41}{Acción fuera del periodo}:] Se muestra en la pantalla \cdtIdRef{IUS 18}{Administrar avances de agua} para indicarle al actor que no puede realizar la operación debido a que la fecha actual se encuentra fuera del periodo definido por la SMAGEM para realizarla.
    \end{description}
