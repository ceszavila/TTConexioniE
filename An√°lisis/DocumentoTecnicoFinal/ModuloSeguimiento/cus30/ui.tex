\subsection{IUS 30 Registrar residuo sólido}

\subsubsection{Objetivo}

      En esta pantalla el \cdtRef{actor:usuarioEscuela}{Coordinador del programa} puede registrar en el sistema los residuos sólidos que se generan en la escuela.

\subsubsection{Diseño}

    En la figura~\ref{IUS 30} se muestra la pantalla ``Registrar residuo sólido'', por medio de la cual se podrán registrar nuevos residuos sólidos para actualizar la información de la línea de acción ``Residuos sólidos''. El actor deberá ingresar la información solicitada para el registro del residuo sólido en la pantalla.\\
        
    Una vez que se haya ingresado toda la información solicitada para el registro del residuo sólido deberá oprimir el botón \cdtButton{Aceptar}, el sistema validará y registrará la información sólo si se han cumplido todas las reglas de negocio establecidas.\\
    
    Finalmente se mostrará el mensaje \cdtIdRef{MSG1}{Operación realizada exitosamente} en la pantalla \cdtIdRef{IUS 29}{Actualizar información de residuos sólidos}, para indicar que la información del residuo sólido se ha registrado exitosamente.
      
    \IUfig[.9]{pantallas/InformacionBase/cuibr2/IUIBR2RegistrarResiduo.png}{IUS 30}{Registrar residuo sólido}
    



\subsubsection{Comandos}
    \begin{itemize}
	\item \cdtButton{Aceptar}: Permite al actor confirmar el registro del residuo sólido, dirige a la pantalla \cdtIdRef{IUS 29}{Actualizar información de residuos sólidos}.
	\item \cdtButton{Cancelar}: Permite al actor cancelar el registro del residuo sólido, dirige a la pantalla \cdtIdRef{IUS 29}{Actualizar información de residuos sólidos}.
    \end{itemize}

\subsubsection{Mensajes}

    \begin{description}
      
	    \item [\cdtIdRef{MSG1}{Operación realizada exitosamente}:] Se muestra en la pantalla \cdtIdRef{IUS 29}{Actualizar información de residuos sólidos} cuando el registro del residuo sólido se ha realizado correctamente.	    
	    
	    \item [\cdtIdRef{MSG4}{No se encontró información sustantiva}:] Se muestra en la pantalla \cdtIdRef{IUS 29}{Actualizar información de residuos sólidos} cuando el sistema no cuenta con información en los catálogos de origen y tipo.
	    
	    \item [\cdtIdRef{MSG5}{Falta un dato requerido para efectuar la operación solicitada}:] Se muestra en la pantalla \cdtIdRef{IUS 30}{Registrar residuo sólido} cuando no se ha ingresado un dato marcado como requerido.
	    
	     \item [\cdtIdRef{MSG6}{Formato incorrecto}:] Se muestra en la pantalla \cdtIdRef{IUS 30}{Registrar residuo sólido} cuando el tipo de dato ingresado no cumple con el tipo de dato solicitado en el campo.
	    
	    \item [\cdtIdRef{MSG7}{Se ha excedido la longitud máxima del campo}:] Se muestra en la pantalla \cdtIdRef{IUS 30}{Registrar residuo sólido} cuando se ha excedido la longitud de alguno de los campos.
	    
	    \item [\cdtIdRef{MSG28}{Operación no permitida por estado de la entidad}:] Se muestra en la pantalla \cdtIdRef{IUS 26}{Administrar avances de residuos sólidos} indicando al actor que no se puede realizar la operación debido al estado en que se encuentra la escuela.
	
	\item [\cdtIdRef{MSG41}{Acción fuera del periodo}:] Se muestra en la pantalla \cdtIdRef{IUS 26}{Administrar avances de residuos sólidos} para indicarle al actor que no puede realizar la operación debido a que la fecha actual se encuentra fuera del periodo definido por la SMAGEM para realizarla.
	    
	    
    \end{description}
