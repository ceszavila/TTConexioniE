%!TEX encoding = UTF-8 Unicode

\begin{UseCase}{CUS 13}{Registrar fauna}
    {
    Las especies animales propias de un ecosistema se conocen como fauna, este caso de uso permite al actor registrar la fauna que se encuentra ubicada en los ecosistemas y áreas verdes cercanas a la escuela, con el propésito de actualizar el inventario de fauna existente.
    }
    
    \UCitem{Versión}{1.0}
    \UCccsection{Administración de Requerimientos}
    \UCitem{Autor}{Francisco Javier Ponce Cruz}
    \UCccitem{Evaluador}{}
    \UCitem{Operación}{Registro}
    \UCccitem{Prioridad}{Media}
    \UCccitem{Complejidad}{Media}
    \UCccitem{Volatilidad}{Alta}
    \UCccitem{Madurez}{Media}
    \UCitem{Estatus}{Terminado}
    \UCitem{Fecha del último estatus}{9 diciembre 2014}
    
%% Copie y pegue este bloque tantas veces como revisiones tenga el caso de uso.
%% Esta sección la debe llenar solo el Revisor
% %--------------------------------------------------------
%   \UCccsection{Revisión Versión XX} % Anote la versión que se revisó.
%   % FECHA: Anote la fecha en que se terminó la revisión.
%   \UCccitem{Fecha}{Fecha en que se termino la revisión} 
%   % EVALUADOR: Coloque el nombre completo de quien realizó la revisión.
%   \UCccitem{Evaluador}{Nombre de quien revisó}
%   % RESULTADO: Coloque la palabra que mas se apegue al tipo de acción que el analista debe realizar.
%   \UCccitem{Resultado}{Corregir, Desechar, Rehacer todo, terminar.}
%   % OBSERVACIONES: Liste los cambios que debe realizar el Analista.
%   \UCccitem{Observaciones}{
%       \begin{UClist}
%           % PC: Petición de Cambio, describa el trabajo a realizar, si es posible indique la causa de la PC. Opcionalmente especifique la fecha en que considera razonable que se deba terminar la PC. No olvide que la numeración no se debe reiniciar en una segunda o tercera revisión.
%           \RCitem{PC1}{\TODO{Descripción del pendiente}}{Fecha de entrega}
%           \RCitem{PC2}{\TODO{Descripción del pendiente}}{Fecha de entrega}
%           \RCitem{PC3}{\TODO{Descripción del pendiente}}{Fecha de entrega}
%       \end{UClist}        
%   }
% %--------------------------------------------------------

    \UCsection{Atributos}
    \UCitem{Actor}{\cdtRef{actor:usuarioEscuela}{Coordinador del programa}}
    \UCitem{Propósito}{Registra la información referente a la fauna que se encuentra en los ecosistemas y áreas verdes cercanas a la escuela.}
    \UCitem{Entradas}{
    \begin{UClist}
       \UCli {\bf Información de la fauna:}
       \begin{itemize}
        \item \cdtRef{gls:categoriaFauna}{Categoría}: \ioSeleccionar.
        \item \cdtRef{inventarioSeguimiento:nombreComun}{Nombre común}: \ioEscribir.
        \item \cdtRef{inventarioSeguimiento:nombreCientifico}{Nombre científico}: \ioEscribir.
        \item \cdtRef{gls:endemico}{Endémico}: \ioRadioBoton.
        \item \cdtRef{gls:riesgo}{En riesgo de desaparecer de la región}: \ioRadioBoton.
        \item \cdtRef{inventarioSeguimiento:cantidad}{Cantidad}: \ioEscribir.
        \item \cdtRef{inventarioSeguimiento:ubicacion}{Ubicación}: \ioEscribir.        
       \end{itemize}
    \end{UClist}
    }

    \UCitem{Salidas}{
    \begin{UClist}
        \UCli \cdtIdRef{MSG1}{Operación realizada exitosamente:} Se muestra en la pantalla \cdtIdRef{IUS 12}{Administrar inventario de fauna} cuando el registro de la fauna se ha realizado correctamente.
    \end{UClist}
    }

    \UCitem{Precondiciones}{
    \begin{UClist}
        \UCli {\bf Interna:} Que la escuela se encuentre en estado \cdtRef{estado:avanceEdicion}{Avance en edición}.
        \UCli {\bf Interna:} Que el periodo de registro de avances se encuentre vigente. 
    \end{UClist}
    }
    
    \UCitem{Postcondiciones}{
    \begin{UClist}
        \UCli {\bf Interna:} Existe un nuevo registro de especie animal en el sistema.
        \UCli {\bf Interna:} Se podrá eliminar el registro de la fauna a través del caso de uso \cdtIdRef{CUS 14}{Eliminar fauna}.
    \end{UClist}
    }
    
    \UCitem{Reglas de negocio}{
        \begin{UClist}
        \UCli \cdtIdRef{RN-S1}{Información correcta}: Verifica que la información introducida sea correcta.
        \UCli \cdtIdRef{RN-N8}{Unicidad de nombres}: Verifica que no exista otra especie animal con el mismo nombre en el sistema.
    \end{UClist}
    }
    
    \UCitem{Errores}{
    \begin{UClist}
    
        \UCli \cdtIdRef{MSG4}{No se encontró información sustantiva}: Se muestra en la pantalla \cdtIdRef{IUS 12}{Administrar inventario de fauna} cuando el sistema no cuenta con información en el catálogo de categoría.
        
        \UCli \cdtIdRef{MSG5}{Falta un dato requerido para efectuar la operación solicitada}: Se muestra en la pantalla \cdtIdRef{IUS 13}{Registrar fauna} cuando no se ha ingresado un dato marcado como requerido.
        
         \UCli \cdtIdRef{MSG6}{Formato incorrecto}: Se muestra en la pantalla \cdtIdRef{IUS 13}{Registrar fauna} cuando el tipo de dato ingresado no cumple con el tipo de dato solicitado en el campo.
        
        \UCli \cdtIdRef{MSG7}{Se ha excedido la longitud máxima del campo}: Se muestra en la pantalla \cdtIdRef{IUS 13}{Registrar fauna} cuando se ha excedido la longitud de alguno de los campos.
        
         \UCli \cdtIdRef{MSG8}{Registro repetido}: Se muestra en la pantalla \cdtIdRef{IUS 13}{Registrar fauna} cuando el actor proporcionó un nombre científico  para la especie animal que ya se encuentra registrado en el sistema.
         
         \UCli \cdtIdRef{MSG28}{Operación no permitida por estado de la entidad}: Se muestra en la pantalla \cdtIdRef{IUS 12}{Administrar inventario de fauna} indicando al actor que no se puede registrar una especie animal debido al estado en que se encuentra la escuela.
        
        \UCli \cdtIdRef{MSG41}{Acción fuera del periodo}: Se muestra sobre la pantalla en la pantalla \cdtIdRef{IUS 12}{Administrar inventario de fauna} para indicarle al actor que no puede registrar una especie animal debido a que la fecha actual se encuentra fuera del periodo definido por la SMAGEM para realizar el registro.
        
    \end{UClist}
    }

    \UCitem{Tipo}{Secundario, extiende del caso de uso \cdtIdRef{CUIBB 3}{Administrar inventario de fauna}.}

%    \UCitem{Fuente}{
%   \begin{UClist}
%       \UCli Minuta de la reunión \cdtIdRef{M-17RT}{Reunión de trabajo}.
%   \end{UClist}
 %   }
\end{UseCase}

 \begin{UCtrayectoria}
    \UCpaso[\UCactor] Solicita registrar una especie animal oprimiendo el botón \cdtButton{Registrar} en la pantalla \cdtIdRef{IUS 12}{Administrar inventario de fauna}.
    \UCpaso[\UCsist] Verifica que la escuela se encuentre en estado ``Avance en edición''. \refTray{A}.
    \UCpaso[\UCsist] Verifica que la fecha actual se encuentre dentro del periodo definido por la SMAGEM para el registro de avances. \refTray{B}.
    \UCpaso[\UCsist] Busca la información de categorías registrada en el sistema. \refTray{C}.
    \UCpaso[\UCsist] Muestra la pantalla \cdtIdRef{IUS 13}{Registrar fauna}.
    \UCpaso[\UCactor] Ingresa los datos de la especie animal en la pantalla \cdtIdRef{IUS 13}{Registrar fauna}. \label{cus13:Registrar}
    \UCpaso[\UCactor] Solicita guardar la información de la especie animal oprimiendo el botón \cdtButton{Aceptar} en la pantalla \cdtIdRef{IUS 13}{Registrar fauna}. \refTray{D}.
    \UCpaso[\UCsist] Verifica que la escuela se encuentre en estado ``Avance en edición''. \refTray{A}.
    \UCpaso[\UCsist] Verifica que la fecha actual se encuentre dentro del periodo definido por la SMAGEM para el registro de avances. \refTray{B}.
    \UCpaso[\UCsist] Verifica que el nombre científico de la especie animal no se encuentre registrado en el sistema como se especifica en la regla de negocio \cdtIdRef{RN-N8}{Unicidad de nombres}. \refTray{E}.
    \UCpaso[\UCsist] Verifica que los datos requeridos sean proporcionados correctamente como se especifica en la regla de negocio \cdtIdRef{RN-S1}{Información correcta}. \refTray{F}. \refTray{G}. \refTray{H}.
    \UCpaso[\UCsist] Registra la información de la especie animal en el sistema.
    \UCpaso[\UCsist] Muestra el mensaje \cdtIdRef{MSG1}{Operación realizada exitosamente} en la pantalla \cdtIdRef{IUS 12}{Administrar inventario de fauna} para indicar al actor que el registro de la especie animal se ha realizado exitosamente.     
 \end{UCtrayectoria}
   \begin{UCtrayectoriaA}[Fin del caso de uso]{A}{La escuela no se encuentra en un estado que permita registrar una especie animal.}
    \UCpaso[\UCsist] Muestra el mensaje \cdtIdRef{MSG28}{Operación no permitida por estado de la entidad} en la pantalla \cdtIdRef{IUS 12}{Administrar inventario de fauna} indicando al actor que no puede registrar una especie animal debido a que la escuela no se encuentra en estado ``Avance en edición''. 
 \end{UCtrayectoriaA}

   \begin{UCtrayectoriaA}[Fin del caso de uso]{B}{La fecha actual se encuentra fuera del periodo definido por la SMAGEM para el registro de especies animales.}
    \UCpaso[\UCsist] Muestra el mensaje \cdtIdRef{MSG41}{Acción fuera del periodo} en la pantalla \cdtIdRef{IUS 12}{Administrar inventario de fauna} indicando al actor que no puede registrar una especie animal debido a que la fecha actual se encuentra fuera del periodo definido por la SMAGEM para realizar la acción. 
 \end{UCtrayectoriaA}
 
 \begin{UCtrayectoriaA}[Fin del caso de uso]{C}{No existe información en el catálogo de categoría.}
    \UCpaso[\UCsist] Muestra el mensaje \cdtIdRef{MSG4}{No se encontró información sustantiva} en la pantalla \cdtIdRef{IUS 12}{Administrar inventario de fauna} indicando al actor que no puede registrar la información de la especie animal debido a que no se cuenta con información sustantiva para el catálogo de categoría.
 \end{UCtrayectoriaA}
 
    \begin{UCtrayectoriaA}[Fin del caso de uso]{D}{El actor desea cancelar la operación.}
    \UCpaso[\UCactor] Solicita cancelar la operación oprimiendo el botón \cdtButton{Cancelar} en la pantalla \cdtIdRef{IUS 13}{Registrar fauna}.
    \UCpaso[\UCsist] Regresa a la pantalla \cdtIdRef{IUS 12}{Administrar inventario de fauna}. 
    \end{UCtrayectoriaA}
  
   \begin{UCtrayectoriaA}{E}{El actor ingresó un nombre científico de especie animal repetido.}
    \UCpaso[\UCsist] Muestra el mensaje \cdtIdRef{MSG8}{Registro repetido} en la pantalla \cdtIdRef{IUS 13}{Registrar fauna}, indicando al actor que existe una especie animal registrada con el mismo nombre científico.
    \UCpaso[] Continúa con el paso \ref{cus13:Registrar} de la trayectoria principal.
 \end{UCtrayectoriaA}
 
    \begin{UCtrayectoriaA}{F}{El actor no ingresó un dato marcado como requerido.}    
    \UCpaso[\UCsist] Muestra el mensaje \cdtIdRef{MSG5}{Falta un dato requerido para efectuar la operación solicitada} en la pantalla \cdtIdRef{IUS 13}{Registrar fauna} indicando que el registro de la especie animal no puede realizarse debido a la falta de información requerida.
    \UCpaso[] Continúa con el paso \ref{cus13:Registrar} de la trayectoria principal.     
    \end{UCtrayectoriaA}
 
    \begin{UCtrayectoriaA}{G}{El actor ingresó un tipo de dato incorrecto.}    
    \UCpaso[\UCsist] Muestra el mensaje \cdtIdRef{MSG6}{Formato incorrecto} en la pantalla \cdtIdRef{IUS 13}{Registrar fauna} indicando que el registro de la especie animal no puede realizarse debido a que la información ingresada no es correcta.
    \UCpaso[] Continúa con el paso \ref{cus13:Registrar} de la trayectoria principal.     
    \end{UCtrayectoriaA}
    
    \begin{UCtrayectoriaA}{H}{El actor ingresó un dato que excede la longitud máxima.}    
    \UCpaso[\UCsist] Muestra el mensaje \cdtIdRef{MSG7}{Se ha excedido la longitud máxima del campo} en la pantalla \cdtIdRef{IUS 13}{Registrar fauna} indicando que el registro de la especie animal no puede realizarse debido a que la longitud del campo excede la longitud máxima definida.
    \UCpaso[] Continúa con el paso \ref{cus13:Registrar} de la trayectoria principal.     
    \end{UCtrayectoriaA}
