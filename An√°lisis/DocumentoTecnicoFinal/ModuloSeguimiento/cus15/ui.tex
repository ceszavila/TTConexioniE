\subsection{IUS 15 Administrar inventario de flora}

\subsubsection{Objetivo}
En esta pantalla el \cdtRef{actor:usuarioEscuela}{Coordinador del programa} puede conocer las especies vegetales que se encuentran ubicadas en los ecosistemas y áreas verdes cercanas a la escuela, el número total de especies y el número total de especies endémicas, y actualizar la información de flora de acuerdo a las especies que se encuentran durante el seguimiento del plan de acción.

\subsubsection{Diseño}

    En la figura~\ref{IUS 15} se muestra la pantalla ``Administrar inventario de flora'', por medio de la cual se podrá acceder al registro y eliminación de registros de especies vegetales.\\
    
    El actor tendrá la facultad de consultar las especies vegetales que se encuentran en los ecosistemas y áreas verdes cercanas a la escuela, registrar nuevas especies vegetales o eliminarlas, esto a través de los botones \cdtButton{Registrar} y \botKo.\\
    
    En el caso de que no existan registros de especies vegetales en el sistema los campos de ``Total de especies'' y ``Total de especies endémicas'' aparecerán con el número ``0''. Además se mostrará el mensaje \cdtIdRef{MSG2}{No existe información registrada por el momento} indicando que no se encuentran registros de especies vegetales en el sistema.

    \IUfig[.9]{pantallas/seguimiento/cus15/ius15}{IUS 15}{Administrar inventario de flora}

\subsubsection{Comandos}
    \begin{itemize}
    \item \cdtButton{Registrar}: Permite al actor registrar especies vegetales en el sistema, dirige a la pantalla \cdtIdRef{IUS 16}{Registrar flora}.
    
    \item \botKo[Eliminar]: Permite al actor eliminar especies vegetales del sistema, dirige a la pantalla emergente \cdtIdRef{IUS 17}{Eliminar flora}.
    
    \item \cdtButton{Regresar}: Permite al actor cancelar la administración de inventario de flora, dirige a la pantalla \cdtIdRef{IUS 11}{Actualizar inventarios de flora y fauna}.
    \end{itemize}

\subsubsection{Mensajes}

    \begin{description}
    \item [\cdtIdRef{MSG2}{No existe información registrada por el momento}:] Se muestra en la pantalla \cdtIdRef{IUS 15}{Administrar inventario de flora} indicando al actor que no existen registros de especies vegetales en el sistema por el momento.
    
    \item[\cdtIdRef{MSG28}{Operación no permitida por estado de la entidad}:] Se muestra en la pantalla \cdtIdRef{IUS 11}{Actualizar inventarios de flora y fauna} indicando al actor que no puede administrar el inventario de flora debido al estado en que se encuentra la escuela.
    
    \item [\cdtIdRef{MSG41}{Acción fuera del periodo}:] Se muestra sobre la pantalla en la pantalla \cdtIdRef{IUS 11}{Actualizar inventarios de flora y fauna} para indicarle al actor que no puede administrar el inventario de flora debido a que la fecha actual se encuentra fuera del periodo definido por la SMAGEM para realizar la acción.
    \end{description}