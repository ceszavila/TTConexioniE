\subsection{IUR 17 Visualizar integrante de línea de acción}

\subsubsection{Objetivo}

Esta pantalla permite al  actor \cdtRef{actor:usuarioEscuela}{Coordinador del programa} hacer una consulta detallada de la información de cierto integrante del comité escolar.

\subsubsection{Diseño}

% Presente la figura de la interfaz y explique paso a paso ``a manera de manual de usuario'' como se debe utilizar la interfaz. No olvide detallar en la redacción los datos de entradas y salidas. Explique como utilizar cada botón y control de la pantalla, para que sirven y lo que hacen. Si el Botón lleva a otra pantalla, solo indique la pantalla y explique lo que pasará cuando se cierre dicha pantalla (la explicación sobre el funcionamiento de la otra pantalla estará en su archivo correspondiente).

	En la figura~\ref{IUR 17} se muestra la pantalla ``Visualizar integrante de línea de acción'', en la cual se muestran los datos completos de uno de los integrantes del comité escolar. 
	
	\IUfig[.9]{pantallas/registro/IUR17}{IUR 17}{Visualizar integrante de línea de acción}

	

	
	
\subsubsection{Comandos}
    \begin{itemize}
	
	\item \cdtButton{Aceptar}: Permite al actor concluir la visualización del integrante de la línea de acción. 
    \end{itemize}

