\subsection{IUR 7 Modificar información escolar}

\subsubsection{Objetivo}

   Esta pantalla permite al usuario \cdtRef{actor:usuarioEscuela}{Coordinador del programa} modificar la información de la escuela en caso de que la información sea incorrecta.

\subsubsection{Diseño}

    
    En la figura ~\ref{IUR 7} se muestra la pantalla ``Modificar información escolar'', que permite visualizar la información referente a la escuela para realizar las modificaciones. El actor deberá ingresar la información que desea modificar o actualizar.
    
    Una vez ingresada la información en la pantalla, deberá oprimir el botón \cdtButton{Aceptar}, posteriormente %se mostrará una \TODO (liga a pantalla emergente) \cdtRef{fig:pantallaEmergente}{pantalla emergente} con el mensaje \TODO (Incluir mensaje en la sección de mensajes y ligarlo bien)\cdtIdRef{MSG9}{Confirmar la modificación de un registro}, donde deberá confirmar la modificación oprimiendo el botón \cdtButton{Aceptar}. 
    el sistema validará y registrará la información sólo si se han cumplido todas las reglas de negocio establecidas.  \\
    
    Finalmente se mostrará el mensaje \cdtIdRef{MSG1}{Operación realizada exitosamente} en 
    la pantalla \cdtIdRef{IUR 5}{Administrar información escolar} para indicar que la información de la escuela
    se ha modificado correctamente. 
    
    \IUfig[.7]{pantallas/registro/IUR7}{IUR 7}{Modificar información escolar}

    
\subsubsection{Comandos}
    \begin{itemize}
   \item \cdtButton{Aceptar}: Permite al actor confirmar la modificación de la escuela, dirige a la pantalla \cdtIdRef{IUR 5}{Administrar información escolar}.
    \item \cdtButton{Cancelar}: Permite al actor cancelar la modificación de la escuela, dirige a la pantalla \cdtIdRef{IUR 5}{Administrar información escolar}.
    \end{itemize}

\subsubsection{Mensajes} 

    \begin{description}
      \item[\cdtIdRef{MSG1}{Operación realizada exitosamente}:] Se muestra en la pantalla \cdtIdRef{IUR 5}{Administrar información escolar} cuando la escuela ha sido modificada exitosamente.
	\item [\cdtIdRef{MSG3}{Superficies del predio}:] Se muestra en la pantalla \cdtIdRef{IUR 7}{Modificar información escolar} para notificar al actor que la superficie construida que ha ingresado supera la superficie total del predio.
      \item[\cdtIdRef{MSG4}{No se encontró información sustantiva}:] Se muestra en la pantalla \cdtIdRef{IUR 5}{Administrar información escolar} cuando no hay información registrada referente a la escuela.
      \item[\cdtIdRef{MSG5}{Falta un dato requerido para efectuar la operación solicitada}:] Se muestra en la pantalla \cdtIdRef{IUR 7}{Modificar información escolar} cuando no se ha ingresado un dato marcado como obligatorio.
      \item[\cdtIdRef{MSG6}{Formato incorrecto}:] Se muestra en la pantalla \cdtIdRef{IUR 7}{Modificar información escolar} cuando el tipo de dato ingresado no cumple con el tipo de dato solicitado en el campo.
      \item[\cdtIdRef{MSG7}{Se ha excedido la longitud máxima del campo}:] Se muestra en la pantalla el mensaje \cdtIdRef{IUR 7}{Modificar información escolar} cuando se ha excedido la longitud de alguno de los campos.
%       \item[\cdtIdRef{MSG9}{Confirmar la modificación de un registro}:] Se muestra en una  \cdtRef{fig:pantallaEmergente}{pantalla emergente} para que el actor confirme la modificación de la escuela.
      \item[\cdtIdRef{MSG18}{Error en la región}:] Se muestra  en la pantalla \cdtIdRef{IUR 7}{Modificar información escolar} para notificar al actor que la escuela no se encuentra en un municipio asociado a la región seleccionada.

    \end{description}
