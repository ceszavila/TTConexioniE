%!TEX encoding = UTF-8 Unicode

% ESTA SECCION LA DEBE LLENAR SOLO EL ANALISTA.
% ID: Asegurese de que el ID del Caso de uso sea único.
% Nombre: Aseurese de que esté escrito de la forma: VERBO + SUSTANTIVO + ALGO
\begin{UseCase}{CUR 18}{Visualizar coordinador del programa}
    {
	Este caso de uso permite al actor consultar los datos del coordinador del programa.
    }
    
	\UCitem{Versión}{1.0}
	\UCccsection{Administración de Requerimientos}
	\UCitem{Autor}{Angélica Madrid Jiménez}
	\UCccitem{Evaluador}{José David Ortega Pacheco}
	\UCitem{Operación}{Consulta}
	\UCccitem{Prioridad}{Alta}
	\UCccitem{Complejidad}{Media}
	\UCccitem{Volatilidad}{Baja}
	\UCccitem{Madurez}{Alta}
    \UCitem{Estatus}{Terminado}
    \UCitem{Fecha del último estatus}{5 de Noviembre del 2014}
    
    
%% Copie y pegue este bloque tantas veces como revisiones tenga el caso de uso.
%% Esta sección la debe llenar solo el Revisor
% %--------------------------------------------------------
% 	\UCccsection{Revisión Versión XX} % Anote la versión que se revisó.
% 	% FECHA: Anote la fecha en que se terminó la revisión.
% 	\UCccitem{Fecha}{Fecha en que se termino la revisión} 
% 	% EVALUADOR: Coloque el nombre completo de quien realizó la revisión.
% 	\UCccitem{Evaluador}{Nombre de quien revisó}
% 	% RESULTADO: Coloque la palabra que mas se apegue al tipo de acción que el analista debe realizar.
% 	\UCccitem{Resultado}{Corregir, Desechar, Rehacer todo, terminar.}
% 	% OBSERVACIONES: Liste los cambios que debe realizar el Analista.
% 	\UCccitem{Observaciones}{
% 		\begin{UClist}
% 			% PC: Petición de Cambio, describa el trabajo a realizar, si es posible indique la causa de la PC. Opcionalmente especifique la fecha en que considera razonable que se deba terminar la PC. No olvide que la numeración no se debe reiniciar en una segunda o tercera revisión.
% 			\RCitem{PC1}{\TODO{Descripción del pendiente}}{Fecha de entrega}
% 			\RCitem{PC2}{\TODO{Descripción del pendiente}}{Fecha de entrega}
% 			\RCitem{PC3}{\TODO{Descripción del pendiente}}{Fecha de entrega}
% 		\end{UClist}		
% 	}
% %--------------------------------------------------------

    \UCsection{Atributos}
    \UCitem{Actor(es)}{\cdtRef{actor:usuarioEscuela}{Coordinador del programa}}
  % PROPOSITO: Escriba una sentencia que defina una situación deseable por el Actor. Este mismo enunciado debe describir el ``Valor Agregado'' que se lleva el actor al ejecutar el Caso de Uso y describe también la ``Condición de Término''.
    \UCitem{Propósito}{Consultar información detallada del coordinador del programa}
    % ENTRADAS: Liste y referencíe los datos de entrada al sistema durante el CU: Nombre y forma en que se debe proporcionar el dato al Caso de uso: teclado, raton, camara, lector de barras, algun sensor, etc.
  	\UCitem{Entradas}{Ninguna.
  	}
	\UCitem{Salidas}{
	\begin{UClist}
	    \UCli \cdtRef{persona:nombre}{Nombre(s)}: \ioObtener.
	    \UCli \cdtRef{persona:primerApellido}{Primer apellido}: \ioObtener.
	    \UCli \cdtRef{persona:segundoApellido}{Segundo apellido}: \ioObtener.
	    \UCli \cdtRef{coordinador:nombramiento}{Nombramiento}: \ioObtener.
	    \UCli \cdtRef{coordinador:cartaCompromiso}{Carta compromiso}: \ioObtener.
	    \UCli \cdtRef{persona:correo}{Correo electrónico}: \ioObtener.
	    \UCli \cdtRef{empleado:telefono}{Teléfono}: \ioObtener.
	    \UCli \cdtRef{empleado:extension}{Extensión}: \ioObtener.
	\end{UClist}
    }
    % PRECONDICIÓN: Son sentencias intemporales y afirmativas que declaran lo que DEBE ser siempre verdadero antes de iniciar el escenario en el caso de uso. Las precondiciones no son probadas dentro del caso de uso, son condiciones que se asumen verdaderas. Una precondición puede implicar un escenario de otro Caso de Uso que se ha completado satisfactoriamente, como por ejemplo la ``autenticación'', o más general el ``cajero se identifica y se autentica''. Craig Larman ``Use Case Model: Writing Requirements in Context''. También pueden ser escenarios ajenos al sistema que el Actor debe contemplar durante la operación pero de las que el sistema no es consciente, por ejemplo: ``El alumno debe presentar su credencial vigente'', o ``contar con el expediente físico'' ``El vehículo a asegurar debe estar en buen estado''.
    % Especifique las precondiciones indicando si son internas (escenarios provenientes de otro caso de uso) o externas, referenciando para las internas el CU correspondiente y, en caso de que aplique, la Regla de negocio que se está Reforzando con esta precondición.
    \UCitem{Precondiciones}{
	\begin{UClist}
	   \UCli {\bf Interna:} Que se encuentre registrado en el sistema el coordinador del programa.
		\UCli {\bf Interna:} Que la escuela se encuentre en estado \cdtRef{estado:inscrita}{inscrita}.	    
		\UCli {\bf Interna:} Que el periodo de registro de escuelas se encuentre vigente.	    
	\end{UClist}
    }
    
    % POSTCONDICIONES: Son sentencias expresadas de manera intemporal y afirmativamente que exponen las garantías de exito o lo que DEBE ser verdadero cuando se completa exitosamente el caso de uso, sea a través de su escenario principal o a través de un flujo alternativo. La garantía debe cumplir las necesidades de todos los stakeholders.
    % Las postcondiciones en conjunto deben reflejar la condición de término del Caso de Uso y alcanzar el propósito planteado por el actor. También describe los cambios en la información y comportamiento del sistema. Indique los cambios que ocurrirán tanto dentro (Internas) como fuera (Externas) del sistema, referenciando los CU afectados por las Internas.
    \UCitem{Postcondiciones}{
	    Ninguna.
    }
    
    
    \UCitem{Reglas de negocio}{Ninguna.}
    \UCitem{Errores}{
    
     \UCli \cdtIdRef{MSG28}{Operación no permitida por estado de la entidad}: Se muestra sobre la pantalla \cdtIdRef{IUR 5}{Administrar información escolar} indicando al actor que no se puede visualizar al coordinador del programa debido al estado en que se encuentra la escuela.
		\UCli	\cdtIdRef{MSG41}{Acción fuera del periodo}: Se muestra sobre la pantalla \cdtIdRef{IUR 5}{Administrar información escolar} para indicarle al actor que no puede visualizar al coordinador del programa debido a que la fecha actual se encuentra fuera del periodo definido por la SMAGEM para realizar el registro de escuelas.        
    
    }
    %Reglas de negocio: Especifique las reglas de negocio que utiliza este caso de uso
%     \UCitem{Reglas de negocio}{
%     	\begin{UClist}
% 	    \UCli \cdtIdRef{RN-S1}{Información correcta}: Verifica que la información introducida sea correcta.
% 	    \UCli \cdtIdRef{RN-S4}{Unicidad de identificadores}: Verifica que no exista un predio con la misma clave en el sistema.``disponibles''.
% 	\end{UClist}
%     }
    % ERRORES: Especifique los casos en los que no se podrá terminar satisfactoriamente el Caso de Uso. Contemple todos los catálogos o listas que deben tener almenos un dato para que se puedan seleccionar dentro de las pantallas asociadas al Caso de Uso.
    % Especifique: La descripción del error (condición), el comportamiento del sistema, y la forma en que el usuario se dará cuenta del error.
%     \UCitem{Errores}{
% 	\begin{UClist}
% 	    \UCli No hay vehículos disponibles: El sistema mostrará el mensaje \cdtIdRef{MSG14}{Falta de ``Vehículos'' ``disponibles''}.
% 	    \UCli Las tareas son incompatibles de acuerdo a lo especificado en la \cdtIdRef{RN34}{Compatibilidad entre tareas de un mismo proyecto}: El sistema mostrará el mensaje \cdtIdRef{MSG20}{Tareas incompatibles}.
% 	\end{UClist}
%     }

    % TIPO: Indique si el Caso de Uso es primario o secundario. Los casos de uso son primarios cuando el Actor los puede ejecutar directamente, y secundario cuando este se ejecuta a travéz de una extensión o inclusión de otro caso de uso, en cuyo caso se debe especificar el Caso de Uso relacionado y la trayectoria debe iniciar a partir del paso en el que se extendió o Incluyo el Caso de Uso.
    \UCitem{Tipo}{Primario.}

  	\UCitem{Fuente}{
	    \begin{UClist}
        \UCli Minuta de la reunión \cdtIdRef{M-3TR}{Toma de requerimientos}.
	    \end{UClist}
}
\end{UseCase}

 % La trayectoria principal debe describir los pasos del caso correcto simple completo. Esto es:
 % Correcto: Describe el caso mas común de la ejecución correcta del Caso de uso.
 % Simple: No repite pasos ni especifica iteraciones, no abunda en errores.
 % Completo: Anque no abunda en errores especifica todas las validaciones, verificaciones, comparaciones y cálculos que debe realizar el sistema durante la trayectoria.
 % FACTORES CRITICOS:
 %-------------------
 % 1.- Cada paso del actor debe estar redactado ``en lo posible'' con dos partes:
 %   - La primera parte especifica ``lo que el Actor cree que está haciendo'' por ejemplo: ``Solicita su registro''.
 %   - La segunda especifica ``la acción exacta que realiza en el sistema''. por ejemplo ``oprime el botón \IUButton{Registrar} de la \cdtIdRef{IU4}{Datos personales}''.
 % En conjunto la redacción de los pasos se debe ver de la siguiente forma: 
 %
 %``El actor solicita su registro oprimiendo el botón \cdtButton{Registrar} de la \cdtIdRef{IU4}{Datos personales}.''
 %
 % Para los pasos del sistema aplica lo mismo solo para los pasos asociados con una Regla de Negocio o un cambio en la Interfaz de Usuario. Por ejemplo:
 % ``El sistema busca los vehículos disponibles.''
 % ``El sistema verifica que las tareas correspondan, de acuerdo a lo especificado en la \cdtIdRef{RN34}{Compatibilidad entre tareas de un mismo proyecto}''
 % ``El sistema muestra los datos personales del Alumno mediante la \cdtIdRef{IU23}{Datos del Alumno}''
 %
 % 2.- Cada paso debe comenzar con un verbo, jamás con un si, mientras, cuando, entonces, etc. utilice: Escribe, registra, calcula, verifica, muestra, lista, selecciona, asocia, filtra, busca, etc.
 %
 % 3.- Ponga especial atención en las verificaciones y detección de errores
 %
 % 4.- Describa con un paso que diga ``el sistema busca ...'' antes de mostrar una pantalla que tiene listas que se llenan desde un catálogo.
 %
 % 5.- Referencie las reglas de negocio en las validaciones que correspondan.
 %
 % 6.- Referencie la Interfaz de Usuario correspondiente cuando el sistema muestre un dato o mensaje.
 % 7.- El Caso de Uso debe iniciar con el actor justo después del paso en el que se marque el Punto de Extensión o Inclusión del caso de uso proveniente. Considere que al inicio del Caso de Uso cuando es secundario se debe especificar la(s) interfaz(es) con la que se está trabajando y debe coincidir con la Interfaz activa al momento de la Extensión o Inclusión.
  \begin{UCtrayectoria}
      \UCpaso[\UCactor] Solicita consultar la información del coordinador del programa seleccionando la opción ``Información general'' del menú \cdtIdRef{MN2}{Menú del Coordinador del programa} y posteriormente la opción ``Coordinador del programa''. 
      
    \UCpaso[\UCsist] Verifica que la escuela se encuentre es estado ``Inscrita''. \refTray{A}
    \UCpaso[\UCsist] Verifica que la fecha actual se encuentre dentro del periodo definido por parte de la SMAGEM para el registro de escuelas. \refTray{B}            
      
    \UCpaso[\UCsist] Busca los datos del coordinador del programa asociado a la escuela registrados en el sistema. 
    \UCpaso[\UCsist] Muestra los datos del coordinador del programa en la pantalla \cdtIdRef{IUR 18}{Visualizar coordinador del programa}.
     \UCpaso[\UCactor] Consulta la información del responsable del programa.
    \UCpaso[\UCactor] Concluye la consulta de la información del responsable del programa oprimiendo el botón \cdtButton{Aceptar} de la pantalla \cdtIdRef{IUR 18}{Visualizar responsable del programa}.
    %%%%Si da aceptar, a donde dirige???
 \end{UCtrayectoria}
 
 %%%%%%%%%%%%
 
  \begin{UCtrayectoriaA}[Fin del caso de uso]{A}{La escuela no se encuentra en el estado ``Inscrita''}
    \UCpaso[\UCsist] Muestra el mensaje \cdtIdRef{MSG28}{Operación no permitida por estado de la entidad} en la pantalla \cdtIdRef{IUR 5}{Administrar información escolar} indicando al actor que no puede visualizar al coordinador del programa debido a que la escuela no se encuentra en estado ``Inscrita''.
 \end{UCtrayectoriaA}

 \begin{UCtrayectoriaA}[Fin del caso de uso]{B}{La fecha actual se encuentra fuera del periodo definido por la SMAGEM para el registro de escuelas}
    \UCpaso[\UCsist] Muestra el mensaje \cdtIdRef{MSG41}{Acción fuera del periodo} en la pantalla \cdtIdRef{IUR 5}{Administrar información escolar} indicando al actor que no puede visualizar al coordinador del programa debido a que la fecha actual se encuentra fuera del periodo definido por la SMAGEM para realizar la acción.
 \end{UCtrayectoriaA}
 
 