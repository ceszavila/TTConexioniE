% %!TEX encoding = UTF-8 Unicode
% 
% % ESTA SECCION LA DEBE LLENAR SOLO EL ANALISTA.
% % ID: Asegurese de que el ID del Caso de uso sea único.
% % Nombre: Aseurese de que esté escrito de la forma: VERBO + SUSTANTIVO + ALGO
\begin{UseCase}{CUR 16}{Eliminar integrante de línea de acción}
	{
		Este caso de uso permite eliminar a un integrante de línea de acción en el caso de querer dar de baja a algún miembro del comité. 
	}
	
	\UCitem{Versión}{1.0}
	\UCccsection{Administración de Requerimientos}
	\UCitem{Autor}{Angélica Madrid Jiménez}
	\UCccitem{Evaluador}{José David Ortega Pacheco}
	\UCitem{Operación}{Eliminación}
	\UCccitem{Prioridad}{Alta}
	\UCccitem{Complejidad}{Media}
	\UCccitem{Volatilidad}{Baja}
	\UCccitem{Madurez}{Alta}
    \UCitem{Estatus}{Terminado}
    \UCitem{Fecha del último estatus}{5 de Noviembre del 2014}
	

%% Copie y pegue este bloque tantas veces como revisiones tenga el caso de uso.
%% Esta sección la debe llenar solo el Revisor
% %--------------------------------------------------------
% 	\UCccsection{Revisión Versión XX} % Anote la versión que se revisó.
% 	% FECHA: Anote la fecha en que se terminó la revisión.
% 	\UCccitem{Fecha}{Fecha en que se termino la revisión} 
% 	% EVALUADOR: Coloque el nombre completo de quien realizó la revisión.
% 	\UCccitem{Evaluador}{Nombre de quien revisó}
% 	% RESULTADO: Coloque la palabra que mas se apegue al tipo de acción que el analista debe realizar.
% 	\UCccitem{Resultado}{Corregir, Desechar, Rehacer todo, terminar.}
% 	% OBSERVACIONES: Liste los cambios que debe realizar el Analista.
% 	\UCccitem{Observaciones}{
% 		\begin{UClist}
% 			% PC: Petición de Cambio, describa el trabajo a realizar, si es posible indique la causa de la PC. Opcionalmente especifique la fecha en que considera razonable que se deba terminar la PC. No olvide que la numeración no se debe reiniciar en una segunda o tercera revisión.
% 			\RCitem{PC1}{\TODO{Descripción del pendiente}}{Fecha de entrega}
% 			\RCitem{PC2}{\TODO{Descripción del pendiente}}{Fecha de entrega}
% 			\RCitem{PC3}{\TODO{Descripción del pendiente}}{Fecha de entrega}
% 		\end{UClist}		
% 	}
% %--------------------------------------------------------

	\UCsection{Atributos}
	\UCitem{Actor(es)}{\cdtRef{actor:usuarioEscuela}{Coordinador del programa}}
	\UCitem{Propósito}{Eliminar un integrante de línea de acción que fue registrado por error o que después de una revisión se determinó que debe ser removido del comité.}
	\UCitem{Entradas}{Ninguna.}
% 	% SALIDAS: Liste y referencíe los datos de salida o resultados del sistema, Especifíque el dispositivo en donde se presentarán las salidas: pantalla, impresora, otro sistema, brazo mecánico, etc. 
	\UCitem{Salidas}{
		\begin{UClist} 
			\UCli \cdtIdRef{MSG10}{Confirmar la eliminación de un integrante de línea de acción}.
		\end{UClist}
	}
% 	% PRECONDICIÓN: Son sentencias intemporales y afirmativas que declaran lo que DEBE ser siempre verdadero antes de iniciar el escenario en el caso de uso. Las precondiciones no son probadas dentro del caso de uso, son condiciones que se asumen verdaderas. Una precondición puede implicar un escenario de otro Caso de Uso que se ha completado satisfactoriamente, como por ejemplo la ``autenticación'', o más general el ``cajero se identifica y se autentica''. Craig Larman ``Use Case Model: Writing Requirements in Context''. También pueden ser escenarios ajenos al sistema que el Actor debe contemplar durante la operación pero de las que el sistema no es consciente, por ejemplo: ``El alumno debe presentar su credencial vigente'', o ``contar con el expediente físico'' ``El vehículo a asegurar debe estar en buen estado''.
% 	% Especifique las precondiciones indicando si son internas (escenarios provenientes de otro caso de uso) o externas, referenciando para las internas el CU correspondiente y, en caso de que aplique, la Regla de negocio que se está Reforzando con esta precondición.
	\UCitem{Precondiciones}{
		\begin{UClist}
			\UCli {\bf Interna:} Que se encuentre registrado el integrante a eliminar.
		\UCli {\bf Interna:} Que la escuela se encuentre en estado \cdtRef{estado:inscrita}{inscrita}.	    
			\UCli {\bf Interna:} Que el periodo de registro de escuelas se encuentre vigente.
		\end{UClist}
	}
	
% 	% POSTCONDICIONES: Son sentencias expresadas de manera intemporal y afirmativamente que exponen las garantías de exito o lo que DEBE ser verdadero cuando se completa exitosamente el caso de uso, sea a través de su escenario principal o a través de un flujo alternativo. La garantía debe cumplir las necesidades de todos los stakeholders.
% 	% Las postcondiciones en conjunto deben reflejar la condición de término del Caso de Uso y alcanzar el propósito planteado por el actor. También describe los cambios en la información y comportamiento del sistema. Indique los cambios que ocurrirán tanto dentro (Internas) como fuera (Externas) del sistema, referenciando los CU afectados por las Internas.
	\UCitem{Postcondiciones}{
		\begin{UClist}
			\UCli {\bf Interna:} Al aceptar la eliminación del integrante, la información registrada se eliminará del sistema.
		\end{UClist}
	}
	\UCitem{Reglas de negocio}{Ninguna.}
% 	% ERRORES: Especifique los casos en los que no se podrá terminar satisfactoriamente el Caso de Uso. Contemple todos los catálogos o listas que deben tener almenos un dato para que se puedan seleccionar dentro de las pantallas asociadas al Caso de Uso.
% 	% Especifique: La descripción del error (condición), el comportamiento del sistema, y la forma en que el usuario se dará cuenta del error.
	\UCitem{Errores}{
	
     \UCli \cdtIdRef{MSG28}{Operación no permitida por estado de la entidad}: Se muestra sobre la pantalla \cdtIdRef{IUR 13}{Administrar integrantes de líneas de acción} indicando al actor que no puede eliminar al integrante de línea de acción debido al estado en que se encuentra la escuela.
		\UCli	\cdtIdRef{MSG41}{Acción fuera del periodo}: Se muestra sobre la pantalla \cdtIdRef{IUR 13}{Administrar integrantes de líneas de acción} para indicarle al actor que no puede eliminar al integrante de línea de acción debido a que la fecha actual se encuentra fuera del periodo definido por la SMAGEM para realizar el registro de escuelas.				
	
	}
	
% 
% 	% TIPO: Indique si el Caso de Uso es primario o secundario. Los casos de uso son primarios cuando el Actor los puede ejecutar directamente, y secundario cuando este se ejecuta a travéz de una extensión o inclusión de otro caso de uso, en cuyo caso se debe especificar el Caso de Uso relacionado y la trayectoria debe iniciar a partir del paso en el que se extendió o Incluyo el Caso de Uso.
	\UCitem{Tipo}{Secundario, extiende del caso de uso \cdtIdRef{CUR 13}{Administrar integrantes de líneas de acción}.}
	\UCitem{Fuente}{
	    \begin{UClist}
        \UCli Minuta de la reunión \cdtIdRef{M-3TR}{Toma de requerimientos}.
	    \end{UClist}
}
% 
% 	
 \end{UseCase}
% 
 \begin{UCtrayectoria}
    \UCpaso[\UCactor] Solicita eliminar un integrante del comité oprimiendo el botón \botKo del registro que se desea eliminar en la pantalla  \cdtIdRef{IUR 13}{Administrar integrantes de líneas de acción}

    \UCpaso[\UCsist] Verifica que la escuela se encuentre es estado ``Inscrita''. \refTray{A}
    \UCpaso[\UCsist] Verifica que la fecha actual se encuentre dentro del periodo definido por parte de la SMAGEM para el registro de escuelas. \refTray{B}    
    
%     \UCpaso[\UCsist] Busca el registro del integrante solicitado. 
    \UCpaso[\UCsist] Muestra el mensaje \cdtIdRef{MSG10}{Confirmar la eliminación de un integrante de línea de acción} como pantalla emergente indicando al actor que si acepta eliminar al integrante no podrá recuperar la información referente al mismo.
    \UCpaso[\UCactor] Confirma la eliminación del registro oprimiendo el botón \cdtButton{Aceptar} de la pantalla emergente \refTray{C}
    
    \UCpaso[\UCsist] Verifica que la escuela se encuentre es estado ``Inscrita''. \refTray{A}
    \UCpaso[\UCsist] Verifica que la fecha actual se encuentre dentro del periodo definido por parte de la SMAGEM para el registro de escuelas. \refTray{B}    
    
    \UCpaso[\UCsist] Elimina el registro que hace referencia al integrante de la línea de acción seleccionado.
    \UCpaso[\UCsist] Muestra el mensaje \cdtIdRef{MSG1}{Operación realizada exitosamente} en la pantalla \cdtIdRef{IUR 13}{Administrar integrantes de líneas de acción} especificando que los datos han sido eliminados de manera exitosa y actualiza la lista de integrantes que se muestra en la pantalla.

 \end{UCtrayectoria}


%%%%%%%%%%%

 \begin{UCtrayectoriaA}[Fin del caso de uso]{A}{La escuela no se encuentra en el estado ``Inscrita''}
    \UCpaso[\UCsist] Muestra el mensaje \cdtIdRef{MSG28}{Operación no permitida por estado de la entidad} en la pantalla \cdtIdRef{IUR 13}{Administrar integrantes de líneas de acción} indicando al actor que no puede eliminar integrantes de líneas de acción debido a que la escuela no se encuentra en estado ``Inscrita''.
 \end{UCtrayectoriaA}

 \begin{UCtrayectoriaA}[Fin del caso de uso]{B}{La fecha actual se encuentra fuera del periodo definido por la SMAGEM para el registro de escuelas}
    \UCpaso[\UCsist] Muestra el mensaje \cdtIdRef{MSG41}{Acción fuera del periodo} en la pantalla \cdtIdRef{IUR 13}{Administrar integrantes de líneas de acción} indicando al actor que no puede eliminar intregrantes de líneas de acción debido a que la fecha actual se encuentra fuera del periodo definido por la SMAGEM para realizar la acción.
 \end{UCtrayectoriaA}
  
 \begin{UCtrayectoriaA}[Fin del caso de uso]{C}{El actor cancela la operación}
    \UCpaso[\UCactor] Solicita cancelar la operación oprimiendo el botón \cdtButton{Cancelar} de la pantalla emergente.
    \UCpaso[\UCsist] Muestra la pantalla \cdtIdRef{IUR 13}{Administrar integrantes de líneas de acción}.
 \end{UCtrayectoriaA}
 
