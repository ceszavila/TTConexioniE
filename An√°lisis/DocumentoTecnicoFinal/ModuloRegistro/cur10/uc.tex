
\begin{UseCase}{CUR 10}{Registrar responsable del programa}
	{
	  Este caso de uso permite al actor solicitar el registro del responsable del programa. Una vez que los datos han sido ingresados, el sistema valida y registra la información. 
	}
	
	\UCitem{Versión}{1.0}
	\UCccsection{Administración de Requerimientos}
	\UCitem{Autor}{Angélica Madrid Jiménez}
	\UCccitem{Evaluador}{José David Ortega Pacheco}
	\UCitem{Operación}{Registro}
	\UCccitem{Prioridad}{Alta}
	\UCccitem{Complejidad}{Media}
	\UCccitem{Volatilidad}{Baja}
	\UCccitem{Madurez}{Alta}
    \UCitem{Estatus}{Terminado}
    \UCitem{Fecha del último estatus}{5 de Noviembre del 2014}

%% Copie y pegue este bloque tantas veces como revisiones tenga el caso de uso.
%% Esta sección la debe llenar solo el Revisor
% %--------------------------------------------------------
% 	\UCccsection{Revisión Versión XX} % Anote la versión que se revisó.
% 	% FECHA: Anote la fecha en que se terminó la revisión.
% 	\UCccitem{Fecha}{Fecha en que se termino la revisión} 
% 	% EVALUADOR: Coloque el nombre completo de quien realizó la revisión.
% 	\UCccitem{Evaluador}{Nombre de quien revisó}
% 	% RESULTADO: Coloque la palabra que mas se apegue al tipo de acción que el analista debe realizar.
% 	\UCccitem{Resultado}{Corregir, Desechar, Rehacer todo, terminar.}
% 	% OBSERVACIONES: Liste los cambios que debe realizar el Analista.
% 	\UCccitem{Observaciones}{
% 		\begin{UClist}
% 			% PC: Petición de Cambio, describa el trabajo a realizar, si es posible indique la causa de la PC. Opcionalmente especifique la fecha en que considera razonable que se deba terminar la PC. No olvide que la numeración no se debe reiniciar en una segunda o tercera revisión.
% 			\RCitem{PC1}{\TODO{Precondiciones: Inicio de sesión es una también}}{Fecha de entrega}
% 			\RCitem{PC2}{\TODO{Postcondiciones: Se podrá administsrar la información de responsable del programa}}{Fecha de entrega}
% 			\RCitem{PC3}{\TODO{En los Mensajes de error, comenzar con el mensaje y después la rescripción del porque se darán para homologar con el formato que tienen los demás casos de uso }}{Fecha de entrega}
%			\RCitem{PC4}{\TODO{Tipo: Secundario }}{Fecha de entrega} 			 	
%			\RCitem{PC4}{\TODO{Trayectoria principal, paso 1: Solicita REGISTRAR AL  }}{Fecha de entrega} 			
%			\RCitem{PC4}{\TODO{Trayectoria principal, realizar un paso por cada una de las reglas de negocio }}{Fecha de entrega} 			 				
%			\RCitem{PC4}{\TODO{En el paso 1 de la Alterna A. presiona el botón cancelar en la pantalla de Registrar al responsable }}{Fecha de entrega} 						
%			\RCitem{PC4}{\TODO{En el paso 1 de la Alterna A. Hacer bien la refererncia a la pantalla de visualizar }}{Fecha de entrega} 									
%			\RCitem{PC4}{\TODO{Trayectoria B. agregar el retorno de la trayectoria al paso 3 de la principal }}{Fecha de entrega} 	
%			\RCitem{PC3}{\TODO{En pantalla: Estana pantalla registrar al responsable  }}{Fecha de entrega}			
% 			\RCitem{PC3}{\TODO{En pantalla: Comandos o stá el botón de Cancelar  }}{Fecha de entrega}			 	
% 			\RCitem{PC3}{\TOCHK{Mensajes: Falta el mensaje de formato de correo para la regla de negocio RN-S2}}{Fecha de entrega}			 	 										
%			\RCitem{PC3}{\TOCHK{Trayectoria princpial, paso 5, separar una verificación por regla de negocio}}{Fecha de entrega}			 	 										
% 			\RCitem{PC3}{\TOCHK{La trayetoria alternativa E no está dentro de la trayectoria principal, agregarla al mismo paso en donde se verifica con respecto a la regla de negocio RN-S2}}{Fecha de entrega}			 	 			
% 			\RCitem{PC3}{\TOCHK{En la trayectoria principal, paso 1. Cambién la referencia a la pantalla 9.1 en lugar de la 9}}{Fecha de entrega}			 	 										
 			 			
 %		\end{UClist}		
%	}
% %--------------------------------------------------------
\UCccsection{Revisión Versión XX} % Anote la versión que se revisó.
	% FECHA: Anote la fecha en que se terminó la revisión.
	\UCccitem{Fecha}{Fecha en que se termino la revisión} 
	% EVALUADOR: Coloque el nombre completo de quien realizó la revisión.
	\UCccitem{Evaluador}{Natalia Giselle Hernández Sánchez}
	% RESULTADO: Coloque la palabra que mas se apegue al tipo de acción que el analista debe realizar.
	\UCccitem{Resultado}{Corregir}
	% OBSERVACIONES: Liste los cambios que debe realizar el Analista.
	\UCccitem{Observaciones}{
		\begin{UClist}
			% PC: Petición de Cambio, describa el trabajo a realizar, si es posible indique la causa de la PC. Opcionalmente especifique la fecha en que considera razonable que se deba terminar la PC. No olvide que la numeración no se debe reiniciar en una segunda o tercera revisión.
			\RCitem{PC1}{\TODO{La etiqueta del botón debería de ser ``Registrar'' solamente}}{Fecha de entrega}
 			\RCitem{PC2}{\TODO{No se especifica el paso donde se bloquea o desaparece el botón de registro, tampoco se especifica el comportamiento en la pantalla}}{Fecha de entrega}
 		\end{UClist}		
	}
	\UCsection{Atributos}
	\UCitem{Actor(es)}{\cdtRef{actor:usuarioEscuela}{Coordinador del programa}}
	\UCitem{Propósito}{Registrar al responsable del programa en el sistema.}
	\UCitem{Entradas}{
		\begin{UClist}
			\UCli \cdtRef{persona:nombre}{Nombre(s)}: \ioEscribir.
			\UCli \cdtRef{persona:primerApellido}{Primer apellido} \ioEscribir.
			\UCli \cdtRef{persona:segundoApellido}{Segundo apellido} \ioEscribir.
			\UCli \cdtRef{responsable:cveEmpleado}{Clave de empleado} \ioEscribir.
			\UCli \cdtRef{responsable:puesto}{Puesto}: \ioEscribir.
			\UCli \cdtRef{persona:correo}{Correo electrónico} \ioEscribir.
			\UCli \cdtRef{empleado:telefono}{Teléfono}: \ioEscribir.
			\UCli \cdtRef{empleado:extension}{Extensión}: \ioEscribir.
		\end{UClist}
	}
	\UCitem{Salidas}{
		\begin{UClist} 
			\UCli \cdtIdRef{MSG1}{Operación realizada exitosamente}: Se muestra en la pantalla \cdtIdRef{IUR 9}{Administrar responsable del programa} cuando el registro se realizó correctamente.
		\end{UClist}
	}
	\UCitem{Precondiciones}{
		\begin{UClist}
% 			\UCli {\bf Interna:} El coordinador del programa ha iniciado sesión en el sistema.
% 			\UCli {\bf Interna:} Existe un Coordinador del programa registrado en el sistema, el cual está asociado a la escuela.
			\UCli{\bf Interna: } Que no exista un responsable del programa asociado a la escuela en el sistema.
		\UCli {\bf Interna:} Que la escuela se encuentre en estado \cdtRef{estado:inscrita}{inscrita}.	    
			\UCli {\bf Interna:} Que el periodo de registro de escuelas se encuentre vigente.		
		\end{UClist}
	}
	\UCitem{Postcondiciones}{
		\begin{UClist}
			\UCli {\bf Interna:} Se podrá administrar la información del responsable del programa por medio del caso de uso \cdtIdRef{CUR 9}{Administrar responsable del programa}.	
		\end{UClist}
	}
	
	\UCitem{Reglas de negocio}{
		\begin{UClist}
			\UCli \cdtIdRef{RN-S1}{Información correcta}. Verifica que la información introducida sea correcta.
			\UCli \cdtIdRef{RN-S2}{Formato de correo electrónico}. Verifica que el correo electrónico proporcionado tenga un formato correcto de escritura.
		\end{UClist}
			
	}

	\UCitem{Errores}{	
		\begin{UClist}
			\UCli \cdtIdRef{MSG5}{Falta un dato requerido para efectuar la operación solicitada}: Se muestra en la pantalla \cdtIdRef{IUR 10}{Registrar responsable del programa} cuando no se haya proporcionado un dato requerido.
			\UCli \cdtIdRef{MSG6}{Formato incorrecto}: Se muestra en la pantalla \cdtIdRef{IUR 10}{Registrar responsable del programa} especificando el dato cuyo valor no cumple con el tipo de dato definido en el diccionario de datos
			\UCli \cdtIdRef{MSG7}{Se ha excedido la longitud máxima del campo}: Se muestra en la \cdtIdRef{IUR 10}{Registrar responsable del programa} cuando el actor proporciona un dato que excede la longitud máxima.
			\UCli \cdtIdRef{MSG16}{Error en formato de correo electrónico}: Se muestra en la pantalla \cdtIdRef{IUR 10}{Registrar responsable del programa} indicando que el correo electrónico proporcionado no cumple con un formato válido.
     \UCli \cdtIdRef{MSG28}{Operación no permitida por estado de la entidad}: Se muestra sobre la pantalla \cdtIdRef{IUR 9}{Administrar responsable del programa} indicando al actor que no puede registrar al responsable del programa debido al estado en que se encuentra la escuela.
		\UCli	\cdtIdRef{MSG41}{Acción fuera del periodo}: Se muestra sobre la pantalla \cdtIdRef{IUR 9}{Administrar responsable del programa} para indicarle al actor que no puede registrar al responsable del programa debido a que la fecha actual se encuentra fuera del periodo definido por la SMAGEM para realizar el registro de escuelas.				
		\end{UClist}
	}

	% TIPO: Indique si el Caso de Uso es primario o secundario. Los casos de uso son primarios cuando el Actor los puede ejecutar directamente, y secundario cuando este se ejecuta a travéz de una extensión o inclusión de otro caso de uso, en cuyo caso se debe especificar el Caso de Uso relacionado y la trayectoria debe iniciar a partir del paso en el que se extendió o Incluyo el Caso de Uso.
	\UCitem{Tipo}{Secundario, extiende del caso de uso \cdtIdRef{CUR 9}{Administrar responsable del programa}.}

	% PROPOSITO: Escriba una sentencia que defina una situación deseable por el Actor. Este mismo enunciado debe describir el ``Valor Agregado'' que se lleva el actor al ejecutar el Caso de Uso y describe también la ``Condición de Término''.
	
	\UCitem{Fuente}{
    \begin{UClist}
      \UCli Minuta de la reunión \cdtIdRef{M-3TR}{Toma de requerimientos}.
    \end{UClist}
  }

	
 \end{UseCase}

 % La trayectoria principal debe describir los pasos del caso correcto simple completo. Esto es:
 % Correcto: Describe el caso mas común de la ejecución correcta del Caso de uso.
 % Simple: No repite pasos ni especifica iteraciones, no abunda en errores.
 % Completo: Anque no abunda en errores especifica todas las validaciones, verificaciones, comparaciones y cálculos que debe realizar el sistema durante la trayectoria.
 % FACTORES CRITICOS:
 %-------------------
 % 1.- Cada paso del actor debe estar redactado ``en lo posible'' con dos partes:
 %   - La primera parte especifica ``lo que el Actor cree que está haciendo'' por ejemplo: ``Solicita su registro''.
 %   - La segunda especifica ``la acción exacta que realiza en el sistema''. por ejemplo ``oprime el botón \IUButton{Registrar} de la \cdtIdRef{IU4}{Datos personales}''.
 % En conjunto la redacción de los pasos se debe ver de la siguiente forma: 
 %
 %``El actor solicita su registro oprimiendo el botón \cdtButton{Registrar} de la \cdtIdRef{IU4}{Datos personales}.''
 %{RN-S1}{Información correcta}
 % Para los pasos del sistema aplica lo mismo solo para los pasos asociados con una Regla de Negocio o un cambio en la Interfaz de Usuario. Por ejemplo:
 % ``El sistema busca los vehículos disponibles.''
 % ``El sistema verifica que las tareas correspondan, de acuerdo a lo especificado en la \cdtIdRef{RN34}{Compatibilidad entre tareas de un mismo proyecto}''
 % ``El sistema muestra los datos personales del Alumno mediante la \cdtIdRef{IU23}{Datos del Alumno}''
 %
 % 2.- Cada paso debe comenzar con un verbo, jamás con un si, mientras, cuando, entonces, etc. utilice: Escribe, registra, calcula, verifica, muestra, lista, selecciona, asocia, filtra, busca, etc.
 %
 % 3.- Ponga especial atención en las verificaciones y detección de errores
 %
 % 4.- Describa con un paso que diga ``el sistema busca ...'' antes de mostrar una pantalla que tiene listas que se llenan desde un catálogo.
 %
 % 5.- Referencie las reglas de negocio en las validaciones que correspondan.
 %
 % 6.- Referencie la Interfaz de Usuario correspondiente cuando el sistema muestre un dato o mensaje.
 % 7.- El Caso de Uso debe iniciar con el actor justo después del paso en el que se marque el Punto de Extensión o Inclusión del caso de uso proveniente. Considere que al inicio del Caso de Uso cuando es secundario se debe especificar la(s) interfaz(es) con la que se está trabajando y debe coincidir con la Interfaz activa al momento de la Extensión o Inclusión.
 \begin{UCtrayectoria}
    \UCpaso[\UCactor] Solicita registrar al responsable del programa oprimiendo el botón \cdtButton{Registrar responsable} de la pantalla \cdtIdRef{IUR 9.1}{Administrar responsable del programa}.
    
    \UCpaso[\UCsist] Verifica que la escuela se encuentre es estado ``Inscrita''. \refTray{A}
    \UCpaso[\UCsist] Verifica que la fecha actual se encuentre dentro del periodo definido por parte de la SMAGEM para el registro de escuelas. \refTray{B}    

    \UCpaso[\UCsist] Muestra la pantalla \cdtIdRef{IUR 10}{Registrar responsable del programa} en la cual se hará el registro.
    \UCpaso[\UCactor] Ingresa los datos del responsable del programa en la pantalla \cdtIdRef{IUR 10}{Registrar responsable del programa}.\label{cur10:Acciones}
    \UCpaso[\UCactor] Solicita guardar la información del responsable del programa oprimiendo el botón \cdtButton{Aceptar} de la pantalla \cdtIdRef{IUR 10}{Registrar responsable del programa}.\refTray{C}

    \UCpaso[\UCsist] Verifica que la escuela se encuentre es estado ``Inscrita''. \refTray{A}
    \UCpaso[\UCsist] Verifica que la fecha actual se encuentre dentro del periodo definido por parte de la SMAGEM para el registro de escuelas. \refTray{B}    
    
    
    \UCpaso[\UCsist] Verifica que los datos ingresados proporcionen información correcta de acuerdo a la regla de negocio \cdtRef{RN-S1}{RN-S1 Información correcta} \refTray{D} \refTray{E} \refTray{F}
   \UCpaso[\UCsist]Verifica el formato del correo electrónico como lo indica la regla de negocio \cdtIdRef{RN-S2}{Formato de correo electrónico} \refTray{G} %%El usuario no ingresó todos los datos solicitados/no ingresó datos correctos
    \UCpaso[\UCsist] Registra los datos del responsable de programa en el sistema.
    \UCpaso[\UCsist] Muestra el mensaje \cdtIdRef{MSG1}{Operación realizada exitosamente} en la pantalla \cdtIdRef{IUR 9}{Administrar responsable del programa} especificando que los datos del responsable se han registrado de manera exitosa.
 \end{UCtrayectoria}

%%%%%%la primer trayectoria alterna, el actor desea cancelar la operación 

 \begin{UCtrayectoriaA}[Fin del caso de uso]{A}{La escuela no se encuentra en el estado ``Inscrita''}
    \UCpaso[\UCsist] Muestra el mensaje \cdtIdRef{MSG28}{Operación no permitida por estado de la entidad} en la pantalla \cdtIdRef{IUR 9}{Administrar responsable del programa} indicando al actor que no puede registrar al responsable del programa debido a que la escuela no se encuentra en estado ``Inscrita''.
 \end{UCtrayectoriaA}

 \begin{UCtrayectoriaA}[Fin del caso de uso]{B}{La fecha actual se encuentra fuera del periodo definido por la SMAGEM para el registro de escuelas}
    \UCpaso[\UCsist] Muestra el mensaje \cdtIdRef{MSG41}{Acción fuera del periodo} en la pantalla \cdtIdRef{IUR 9}{Administrar responsable del programa} indicando al actor que no puede registrar al responsable del programa debido a que la fecha actual se encuentra fuera del periodo definido por la SMAGEM para realizar la acción.
 \end{UCtrayectoriaA}

\begin{UCtrayectoriaA}[Fin del caso de uso]{C}{El actor desea cancelar la operación.}
    \UCpaso[\UCactor] Solicita cancelar la operación oprimiendo el botón  \cdtButton{Cancelar} de la pantalla \cdtIdRef{IUR 10}{Registrar responsable del programa}
    \UCpaso[\UCsist] Regresa a la pantalla \cdtIdRef{IUR 9}{Administrar responsable del programa}
	
 \end{UCtrayectoriaA} 
 
 \begin{UCtrayectoriaA}{D}{El actor no ingresa un dato que está marcado como requerido.}
    \UCpaso[\UCsist] Muestra el mensaje \cdtIdRef{MSG5}{Falta un dato requerido para efectuar la operación solicitada} en la pantalla \cdtIdRef{IUR 10}{Registrar responsable del programa} indicando que la operación de registro no puede realizarse debido a la falta de información requerida.
        \UCpaso Continúa en el paso \ref{cur10:Acciones} de la trayectoria principal.
	
 \end{UCtrayectoriaA}
 
  \begin{UCtrayectoriaA}{E}{El actor ingresó un dato que no cumple con el formato requerido.}
    \UCpaso[\UCsist] Muestra el mensaje \cdtIdRef{MSG6}{Formato incorrecto} y señala el campo que presenta el tipo de dato erróneo en la pantalla \cdtIdRef{IUR 10}{Registrar responsable del programa}, indicando que la operación de registro no puede realizarse debido a que la información no es correcta. 
    \UCpaso Continúa en el paso \ref{cur10:Acciones} de la trayectoria principal.
 \end{UCtrayectoriaA}

 \begin{UCtrayectoriaA}{F}{El actor ingresó un dato que no cumple con la longitud máxima requerida.}
    \UCpaso[\UCsist] Muestra el mensaje \cdtIdRef{MSG7}{Se ha excedido la longitud máxima del campo} y señala el campo que presenta el tipo de dato erróneo en la pantalla \cdtIdRef{IUR 10}{Registrar responsable del programa}, indicando que la operación de registro no puede realizarse debido a que la información no es correcta. 
    \UCpaso Continúa en el paso \ref{cur10:Acciones} de la trayectoria principal.
 \end{UCtrayectoriaA}
 
  \begin{UCtrayectoriaA}{G}{El actor no proporciona un correo válido.}
    \UCpaso[\UCsist] Muestra el mensaje \cdtIdRef{MSG16}{Error en formato de correo electrónico} en la pantalla \cdtIdRef{IUR 10}{Registrar responsable del programa} indicando que el correo electrónico proporcionado no cumple con un formato válido.
	    \UCpaso Continúa en el paso \ref{cur10:Acciones} de la trayectoria principal.
 \end{UCtrayectoriaA}