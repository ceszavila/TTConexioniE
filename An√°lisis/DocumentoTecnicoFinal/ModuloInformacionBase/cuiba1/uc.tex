%!TEX encoding = UTF-8 Unicode

\begin{UseCase}{CUIBA 1}{Administrar información base para indicadores de agua}
    {
	La información base para los indicadores de agua proporciona una visión general del estado en que se encuentra la escuela en cuestión al consumo y ahorro de agua. Este caso de uso sirve como punto de acceso para registrar o modificar la información referente a los tipos de abastecimiento de agua con que cuenta la escuela y el consumo total que hace de este recurso.
    }
    
    \UCitem{Versión}{1.0}
    \UCccsection{Administración de Requerimientos}
    \UCitem{Autor}{Jessica Stephany Becerril Delgado}
    \UCccitem{Evaluador}{}
    \UCitem{Operación}{Administración}
    \UCccitem{Prioridad}{Media}
    \UCccitem{Complejidad}{Media}
    \UCccitem{Volatilidad}{Alta}
    \UCccitem{Madurez}{Media}
    \UCitem{Estatus}{Terminado}
    \UCitem{Fecha del último estatus}{18 de Noviembre del 2014}
    
%% Copie y pegue este bloque tantas veces como revisiones tenga el caso de uso.
%% Esta sección la debe llenar solo el Revisor
% %--------------------------------------------------------
 	\UCccsection{Revisión Versión 0.1} % Anote la versión que se revisó.
% 	% FECHA: Anote la fecha en que se terminó la revisión.
 	\UCccitem{Fecha}{24-nov-2014} 
% 	% EVALUADOR: Coloque el nombre completo de quien realizó la revisión.
 	\UCccitem{Evaluador}{David Ortega Pacheco}
% 	% RESULTADO: Coloque la palabra que mas se apegue al tipo de acción que el analista debe realizar.
 	\UCccitem{Resultado}{Corregir.}
% 	% OBSERVACIONES: Liste los cambios que debe realizar el Analista.
 	\UCccitem{Observaciones}{
 		\begin{UClist}
% 			% PC: Petición de Cambio, describa el trabajo a realizar, si es posible indique la causa de la PC. Opcionalmente especifique la fecha en que considera razonable que se deba terminar la PC. No olvide que la numeración no se debe reiniciar en una segunda o tercera revisión.
 			\RCitem{PC1}{\TODO{En Resumen, colocar Punto y seguido en lugar de punto y coma en: , estecaso de uso ...}}{27-Nov-2014}
 			\RCitem{PC2}{\TODO{Postcondiciones, cambiar a: registrar o modificar, o definir que se podrá modificar una vez que hay información}}{27-Nov-2014}
 			\RCitem{PC3}{\TODO{Causa de la extensión, Cambiar a: El actor desea registrar o modificar la ...}}{27-Nov-2014}
 			\RCitem{PC4}{\TODO{Pantalla, obetivo (Primer párrafo), Cambiar a: de la información base}}{27-Nov-2014}
 			\RCitem{PC5}{\TODO{Pantalla, diseño (Segundo párrafo), Cambiar: en cuestión al por: respecto al }}{27-Nov-2014}
 			\RCitem{PC6}{\TODO{Pantalla, Diseño, juntar los dos párrafos }}{24-Nov-2014}
 			\RCitem{PC7}{\TODO{Pantalla, comandos, cambiar a: registrar o modificar la ... }}{27-Nov-2014}
 			\RCitem{PC8}{\DONE{Pantalla, comandos, mencionar bajo qué condiciones es cuando se registra y cuando es que se realiza modificación de la información base }}{27-Nov-2014}
			\RCitem{PC9}{\DONE{Pantalla, comandos, liga rota: IUBA 2 Registrar información base para indicadores de agua}}{27-Nov-2014}
			\RCitem{PC10}{\DONE{Causa de la extensión, termina con doble punto}}{24-Nov-2014}
			\RCitem{PC11}{\TOCHK{Trayectoria principal: verificar el estado de EDICIÓN}}{24-Nov-2014}

 		\end{UClist}		
 	}
% %--------------------------------------------------------

    \UCsection{Atributos}
    \UCitem{Actor}{\cdtRef{actor:usuarioEscuela}{Coordinador del programa}}
    \UCitem{Propósito}{Administrar el registro y modificación de la información base para indicadores de agua.}
    \UCitem{Entradas}{
	\begin{UClist}
	    \UCli Ninguna.
	\end{UClist}
    }
    \UCitem{Salidas}{
	\begin{UClist} 
	    \UCli Ninguna.
	\end{UClist}
    }

    \UCitem{Precondiciones}{
	\begin{UClist}
	   \UCli {\bf Interna:} Que la escuela se encuentre en estado \cdtRef{estado:infoBaseEdicion}{Información base en edición}.
	    \UCli {\bf Interna:} Que el periodo de registro de información base se encuentre vigente.
	\end{UClist}
    }
    
    \UCitem{Postcondiciones}{
	\begin{UClist}
	    \UCli {\bf Interna:} Se podrá registrar o modificar la información base para indicadores de agua a través del caso de uso \cdtIdRef{CUIBA 2}{Registrar información base para indicadores de agua}.
	\end{UClist}
    }
    
    \UCitem{Reglas de negocio}{
    	\begin{UClist}
	    \UCli Ninguna.
	\end{UClist}
    }
    
    \UCitem{Errores}{
	\begin{UClist}
	    \UCli \cdtIdRef{MSG28}{Operación no permitida por estado de la entidad}: Se muestra en la pantalla \cdtIdRef{IUIBA 1}{Administrar información base para indicadores de agua} indicando al actor que no se puede realizar la operación debido al estado en que se encuentra la escuela.
	    
	    \UCli \cdtIdRef{MSG41}{Acción fuera del periodo}: Se muestra en la pantalla \cdtIdRef{IUIBA 1}{Administrar información base para indicadores de agua} para indicarle al actor que no puede realizar la operación debido a que la fecha actual se encuentra fuera del periodo definido por la SMAGEM para realizarla.
	\end{UClist}
    }

    \UCitem{Tipo}{Primario.}

%    \UCitem{Fuente}{
%	\begin{UClist}
%	    \UCli Minuta de la reunión \cdtIdRef{M-17RT}{Reunión de trabajo}.
%	\end{UClist}
 %   }
\end{UseCase}

 \begin{UCtrayectoria}
    \UCpaso[\UCactor] Solicita administrar la información base para indicadores de agua, seleccionando en el menú \cdtIdRef{MN2}{Menú del Coordinador del programa} la opción ``Información base para indicadores'' y posteriormente la opción ``Agua''. 
    \UCpaso[\UCsist] Verifica que la escuela se encuentre en estado ``Información base por aprobar''. \refTray{A}.
    \UCpaso[\UCsist] Verifica que la fecha actual se encuentre dentro del periodo definido por la SMAGEM para realizar la operación. \refTray{B}.
    \UCpaso[\UCsist] Muestra la pantalla \cdtIdRef{IUIBA 1}{Administrar información base para indicadores de agua}.
    \UCpaso[\UCactor] Administra la información base para indicadores de agua a través del botón \botEdit. \label{cuiba1:RegistrarIA}
 \end{UCtrayectoria}
 
  \begin{UCtrayectoriaA}[Fin del caso de uso]{A}{La escuela no se encuentra en un estado que permita realizar la operación.}
    \UCpaso[\UCsist] Muestra el mensaje \cdtIdRef{MSG28}{Operación no permitida por estado de la entidad} indicando al actor que no puede realizar la operación debido a que la escuela no se encuentra en estado ``Información base en edición''. 
 \end{UCtrayectoriaA}

   \begin{UCtrayectoriaA}[Fin del caso de uso]{B}{La fecha actual se encuentra fuera del periodo definido por la SMAGEM para realizar la operación.}
    \UCpaso[\UCsist] Muestra el mensaje \cdtIdRef{MSG41}{Acción fuera del periodo} en la pantalla \cdtIdRef{IUIBA 1}{Administrar información base para indicadores de agua} indicando al actor que no puede realizar la operación debido a que la fecha actual se encuentra fuera del periodo definido por la SMAGEM para realizarla. 
 \end{UCtrayectoriaA}
 
\subsection{Puntos de extensión}

\UCExtensionPoint
{El actor desea registrar o modificar la información base para los indicadores de agua}
{ Paso \ref{cuiba1:RegistrarIA} de la trayectoria principal}
{\cdtIdRef{CUIBA 2}{Registrar información base para indicadores de agua}}
