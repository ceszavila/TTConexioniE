\begin{UseCase}{CUAC-7.6}{Editar convocatoria de movilidad}{
		Permite llevar a cabo la modificación de una convocatoria de movilidad para la Escuela Superior de Cómputo debido a que se modificó la información sobre el curso que será impartido en dicha escuela. \\
		}
		\UCitem{Versión}{1.0}
		\UCccsection{Administración de Requerimientos}
		\UCitem{Autor}{Ivo Sebastián Sam Álvarez-Tostado}
		\UCccitem{Evaluador}{Ulises Velez Saldaña}
		\UCitem{Operación}{Consulta}
		\UCccitem{Prioridad}{Alta}
		\UCccitem{Complejidad}{Baja}
		\UCccitem{Volatilidad}{Baja}
		\UCccitem{Madurez}{Alta}
		\UCitem{Estatus}{Por revisar}
		\UCitem{Fecha del último estatus}{20 de Mayo del 2018}
			
		\UCsection{Atributos}
		\UCitem{Actor}{
			\begin{UClist} 
				\UCli \cdtRef{actor:CIEUpis}{Responsable UPIS}
			\end{UClist}
		}
		\UCitem{Propósito}{Proporcionar un medio por el cual se pueda editar las convocatorias de movilidad para la Escuela Superior de Cómputo.}
		\UCitem{Entradas}{
			\begin{UClist}
				\UCli Periodo escolar para la movilidad
				\UCli Año en el que se llevará a cabo la movilidad
				\UCli URL de universidades participantes
				\UCli URL de la convocatoria
				\UCli URL de los resultados 
			\end{UClist}
		}
		\UCitem{Salidas}{
			\begin{UClist}
				\UCli Periodo escolar para la movilidad
				\UCli Año en el que se llevará a cabo la movilidad
				\UCli URL de universidades participantes
				\UCli URL de la convocatoria
				\UCli URL de los resultados 
			\end{UClist}
		}
		\UCitem{Precondiciones}{\textbf{Externa}: Se debió aprobar la modificación de la convocatoria de movilidad.}
		\UCitem{Postcondiciones}{Se podrá consultar la convocatoria del curso}
		\UCitem{Reglas de negocio}{\cdtIdRef{RN-S1}{Datos Obligatorios}}
		\UCitem{Errores}{Ninguno}
		\UCitem{Tipo}{Primario.}
	\end{UseCase}
	
	\begin{UCtrayectoria}
		
		\UCpaso [\UCactor] Solicta editar una convocatoria de cursos tocando la convocatoria que requiere modificar, en la pantalla \cdtIdRef{UIAC-7.4}{Gestionar convocatorias de movilidad}.
		
		\UCpaso Obtiene el periodo escolar y el año en que se llevará a cabo la movilidad.
		
		\UCpaso Obtiene las URL de las universidades participantes, de la convocotario, y de los resultados.
		
		\UCpaso Muestra la pantalla \cdtIdRef{UIAC-7.6}{Editar convocatoria de movilidad}.
		
		\UCpaso [\UCactor] Ingresa los datos solicitados.
		
		\UCpaso [\UCactor] Solicita finalizar la operación tocando el botón \cdtButton{Editar}. \refTray{A} \refTray{B}
		
		\UCpaso Verifica que se haya ingresado los campos marcados como obligatorios, con base en la regla \cdtIdRef{RN-S1}{Datos Obligatorios}.
		
		\UCpaso Persiste la información registrada.
		
		\UCpaso Regresa a la pantalla \cdtIdRef{UIAC-7}{Gestionar convocatorias de cursos}.
		
	\end{UCtrayectoria}

	\begin{UCtrayectoriaA}{A}{Cuando el actor requiere cancelar la operación.}
		
		\UCpaso [\UCactor] Solicita cancelar la operación tocando el botón \cdtButton{Volver}.
		
		\UCpaso Regresa a la pantalla \cdtIdRef{UIAC-7}{Gestionar convocatorias de cursos}.
		
	\end{UCtrayectoriaA}

	\begin{UCtrayectoriaA}[Fin del caso de uso]{B}{Cuando el actor requiere eliminar una convocatoria de movilidad.}
		
		\UCpaso [\UCactor] Solicita eliminar la convocotaria de movilidad tocando el botón \cdtButton{Eliminar} en la pantalla \cdtIdRef{UIAC-7.6}{Editar convocatoria de movilidad}.
		
		\UCpaso \label{CUAC-7.6:Eliminar} Ejecuta el caso de uso \cdtIdRef{CUAC-7.7}{Eliminar convocatoria de movilidad}.
		
	\end{UCtrayectoriaA}

	\subsection{Puntos de extensión}
	
	\UCExtensionPoint
	{El actor requiere eliminar una convocatoria de movilidad}
	{ Paso \ref{CUAC-7.6:Eliminar} de la trayectoria alterna \textbf{B}}
	{\cdtIdRef{CUAC-7.7}{Eliminar convocatoria de curso}}