\subsection{IUAP-4.3  Eliminar  Profesor}

\subsubsection{Objetivo}

% Explicar el objetivo para el que se construyo la interfaz, generalmente es la descripción de la actividad a desarrollar, como Seleccionar grupos para inscribir materias de un alumno, controlar el acceso al sistema mediante la solicitud de un login y password de los usuarios, etc.
	
    Esta pantalla permite al \cdtRef{actor:CIEProfesor}{Responsable de Profesores} eliminar a algún profesor que por alguna razón ya no forme parte de la plantilla docente de la ESCOM.
\subsubsection{Diseño}

% Presente la figura de la interfaz y explique paso a paso ``a manera de manual de usuario'' como se debe utilizar la interfaz. No olvide detallar en la redacción los datos de entradas y salidas. Explique como utilizar cada botón y control de la pantalla, para que sirven y lo que hacen. Si el Botón lleva a otra pantalla, solo indique la pantalla y explique lo que pasará cuando se cierre dicha pantalla (la explicación sobre el funcionamiento de la otra pantalla estará en su archivo correspondiente).

    En la figura~\ref{IUAP-4.3} se muestra la pantalla ``Eliminar Profesor“, en ella se ve el mensaje de confirmación de eliminación.
       \IUfig[.3]{/ModuloProfesores/IUAP4_3.png}{IUAP-4.3}{Eliminar Profesor}



\subsubsection{Comandos}
    \begin{itemize}
	\item \cdtButton{Eliminar}, dirige a la pantalla \cdtIdRef{IUAP-4}{Gestionar Profesores}.
	\item \cdtButton{Cancelar}, dirige a la pantalla \cdtIdRef{IUAP-4.3}{Editar Profesor}
    \end{itemize}
