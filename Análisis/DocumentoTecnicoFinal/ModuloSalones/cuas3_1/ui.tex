\subsection{IUAS-03.1 Registrar salón}

\subsubsection{Objetivo}

% Explicar el objetivo para el que se construyo la interfaz, generalmente es la descripción de la actividad a desarrollar, como Seleccionar grupos para inscribir materias de un alumno, controlar el acceso al sistema mediante la solicitud de un login y password de los usuarios, etc.
	
    Esta pantalla permite al \cdtRef{actor:CIEPrefectura}{Responsable Prefectura} registrar un nuevo salón con la finalidad de ponerlo a disposición de algún grupo al inicio de un periodo escolar.
\subsubsection{Diseño}

% Presente la figura de la interfaz y explique paso a paso ``a manera de manual de usuario'' como se debe utilizar la interfaz. No olvide detallar en la redacción los datos de entradas y salidas. Explique como utilizar cada botón y control de la pantalla, para que sirven y lo que hacen. Si el Botón lleva a otra pantalla, solo indique la pantalla y explique lo que pasará cuando se cierre dicha pantalla (la explicación sobre el funcionamiento de la otra pantalla estará en su archivo correspondiente).

    En la figura~\ref{IUAS-03.1} se muestra en la pantalla ``Registrar Salón“ un mapa interactivo con los polígonos de los edificios de la Escuela Superior de Cómputo, el nombre del nuevo salón, el grupo al que será asignado, el nivel y edificio de su asignación.
    
    \IUfig[.3]{/ModuloSalones/IUAS3_1.png}{IUAS-03.1}{Registrar salón}



\subsubsection{Comandos}
    \begin{itemize}

	\item Botón \cdtButton{Atrás}, dirige a la pantalla \cdtIdRef{IUAS-03}{Gestionar salones}
	\item Botón \cdtButton{Registrar}, guarda la información ingresada y dirige a la pantalla \cdtIdRef{IUAS-03}{Gestionar salones}
    \end{itemize}
