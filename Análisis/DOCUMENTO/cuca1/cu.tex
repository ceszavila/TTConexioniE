%CUCA1 Solicitar Activar cuenta

\begin{UseCase}{CUCA1}{Solicitar Activar cuenta}
	{		
	    
	    Para solicitar activar una cuenta, el usuario deberá proporcionar su número de boleta o RFC para que el sistema le envíe un correo con el último paso para su activación, el cual se encuentra indicado en el caso de uso \cdtIdRef{CUCA1.1}{Activar Cuenta}.\\

	}
	\UCitem{Versión}{1.0}
	\UCitem{Autor}{Liliana Flores Sánchez}
	\UCsection{Atributos}
	\UCitem{Actor}{\cdtRef{actor:Alumno}{Alumno}, \cdtRef{actor:Profesor}{Profesor} y \cdtRef{actor:Presidente de Academia}{Presidente de Academia}}
	\UCitem{Propósito}{Solicitar activar una cuenta para poder ingresar al sistema.}
	\UCitem{Entradas}{
		\begin{UClist}
			\UCli Ninguna
		\end{UClist}
	}
	\UCitem{Salidas}{
		
		\begin{UClist}
			\UCli Correo de confirmación de cuenta.
			\UCli Mensaje del estado de operación.
		\end{UClist}				
	}
	\UCitem{Precondiciones}{
		\begin{UClist}			
			\UCli El número de boleta o RFC deben estar registrados en el sistema con una cuenta inactiva.
		\end{UClist}
	}
	
	\UCitem{Postcondiciones}{
		\begin{UClist}
			\UCli Manda un correo con un link para la activación de la cuenta.
		\end{UClist}
	}
	
	\UCitem{Reglas de negocio}{
		\begin{UClist}
			\UCli \cdtIdRef{RN-S1}{Información correcta}
		\end{UClist}
	}
	
	\UCitem{Tipo}{Secundario, extiende del caso de uso \cdtIdRef{CUCA2}{Iniciar Sesión}.}
		
	
	
\end{UseCase}


%-------------------------Trayectoria principal------------------------------
\begin{UCtrayectoria}
	\UCpaso[\UCactor] Requiere activar una cuenta oprimiendo el botón \cdtButton{Activar cuenta} que aparece en la pantalla \cdtIdRef{IUCA2A}{Iniciar Sesión} del caso de uso \cdtIdRef{CUCA2}{Iniciar Sesión}.
	\UCpaso[\UCsist]  Muestra la pantalla \cdtIdRef{IUCA1A}{Buscar cuenta}. 
	\UCpaso[\UCactor] Realiza una búsqueda con base en el caso de uso \cdtIdRef{CUCA2.4}{Buscar Usuario}. \label{cuca1:ingresaDato}
%	\UCpaso[\UCactor] Oprime el botón \cdtButton{Buscar}.
%	\UCpaso[\UCsist]  Verifica que el número de boleta o RFC haya sido introducida correctamente con base en la regla de negocios \cdtIdRef{RN-S1}{Información correcta}. \refTray{A} \refTray{B} \refTray{C}
%	\UCpaso[\UCsist]  Verifica que el número de boleta o RFC exista en el sistema. \refTray{D}
	\UCpaso[\UCsist]  Verifica que el número de boleta o RFC introducido tenga una cuenta en estado \textbf{inactiva}. \refTray{E}
	\UCpaso[\UCsist]  Muestra en la pantalla \cdtIdRef{IUCA1B}{Activar cuenta} el nombre completo y el correo electrónico de quien pertenece el número de boleta o RFC. \refTray{F}
	\UCpaso[\UCsist] Muestra el mensaje \cdtIdRef{MSG18}{Correo de confirmación de cuenta} en la pantalla \cdtIdRef{IUCA1B}{Activar cuenta}.
	\UCpaso[\UCactor] Verifica que la información que aparece en la pantalla \cdtIdRef{IUCA1B}{Activar cuenta} sea correcta. \refTray{G}
	\UCpaso[\UCactor] Oprime el botón \cdtButton{Aceptar}. \refTray{I}
	\UCpaso[\UCsist]  Envía un link de confirmación de cuenta al correo electrónico que se muestra en la pantalla \cdtIdRef{IUCA1B}{Activar cuenta}. \refTray{H}
	\UCpaso[\UCsist] Muestra el mensaje \cdtIdRef{MSG1}{Operación exitosa} en la pantalla \cdtIdRef{IUCA2A}{Iniciar Sesión}.
\end{UCtrayectoria}

%-------------------------Trayectoria A------------------------------
\begin{UCtrayectoriaA}{A}{El formato de la información introducida es incorrecto.}
	\UCpaso[\UCsist] Muestra el mensaje \cdtIdRef{MSG3}{Formato incorrecto}.
	\UCpaso[] Regresa al paso \ref{cuca1:ingresaDato} de la trayectoria principal.
\end{UCtrayectoriaA}

%-------------------------Trayectoria B------------------------------
\begin{UCtrayectoriaA}{B}{No se introdujo un campo que es obligatorio.}
	\UCpaso[\UCsist] Muestra el mensaje \cdtIdRef{MSG2}{Campo obligatorio}.
	\UCpaso[] Regresa al paso \ref{cuca1:ingresaDato} de la trayectoria principal.
\end{UCtrayectoriaA}

%-------------------------Trayectoria C------------------------------
\begin{UCtrayectoriaA}{C}{Se ingresó un dato que no cumple con la longitud especificada.}
	\UCpaso[\UCsist] Muestra el mensaje \cdtIdRef{MSG4}{Longitud incorrecta}.
	\UCpaso[] Regresa al paso \ref{cuca1:ingresaDato} de la trayectoria principal.
\end{UCtrayectoriaA}

%-------------------------Trayectoria D------------------------------
\begin{UCtrayectoriaA}[Fin de caso de uso]{D}{Se ingresó un dato que no existe en el sistema.}
	\UCpaso[\UCsist] Muestra el mensaje \cdtIdRef{MSG33}{No existe información} en la pantalla \cdtIdRef{IUCA1A}{Buscar cuenta}.
\end{UCtrayectoriaA}

%-------------------------Trayectoria E------------------------------
\begin{UCtrayectoriaA}[Fin de caso de uso]{E}{La cuenta del número de boleta o RFC introducidos ya ha sido activada.}
	\UCpaso[\UCsist] Muestra el mensaje \cdtIdRef{MSG16}{Cuenta activada}.
	\UCpaso[\UCactor] Oprime el botón \cdtButton{Volver}. 
\end{UCtrayectoriaA}

%-------------------------Trayectoria F------------------------------
\begin{UCtrayectoriaA}[Fin de caso de uso]{F}{El usuario no tiene un correo registrado en el sistema.}
	\UCpaso[\UCsist] Muestra el mensaje  \cdtIdRef{MSG17}{No existe correo electrónico} en la pantalla \cdtIdRef{IUCA1C}{No existe correo}.
	\UCpaso[\UCactor] Oprime el botón \cdtButton{Volver}. 
\end{UCtrayectoriaA}

%-------------------------Trayectoria G------------------------------
\begin{UCtrayectoriaA}[Fin de caso de uso]{G}{La información que visualiza el usuario es incorrecta.}
	\UCpaso[\UCactor] Debe dirigirse al departamento de la CATT. 
\end{UCtrayectoriaA}

%-------------------------Trayectoria H------------------------------
\begin{UCtrayectoriaA}[Fin de caso de uso]{H}{No se pudo mandar el correo de confirmación.}
	\UCpaso[\UCsist] Muestra el mensaje \cdtIdRef{MSG8}{Problemas al mandar el correo de confirmación} en la pantalla \cdtIdRef{IUCA2A}{Iniciar Sesión}.
\end{UCtrayectoriaA}

%-------------------------Trayectoria I------------------------------
\begin{UCtrayectoriaA}[Fin de caso de uso]{I}{El usuario desea cancelar la operación.}
	\UCpaso[\UCactor] Oprime el botón \cdtButton{Volver}.
	\UCpaso[\UCsist] Muestra la pantalla \cdtIdRef{IUCA2A}{Iniciar Sesión}.
\end{UCtrayectoriaA}