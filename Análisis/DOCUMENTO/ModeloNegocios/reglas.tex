\begin{flushleft}
	
\end{flushleft}%%http://code.tutsplus.com/tutorials/8-regular-expressions-you-should-know--net-6149
%Estatus: Edición, Pendiente de Corrección, Corregido, Revisado y Liberado
\section{Reglas de negocio}

\subsection{Reglas derivadas del sistema}


%------------------------------------------------------------------------------------------------------------------

\begin{BusinessRule}{RN-S1}{Información correcta}
	\BRitem{Versión}{1.0}
	\BRitem{Autor}{Liliana Flores Sánchez}
	\BRitem{Estatus}{Edición}
	\BRitem{Descripción}{
		Para considerar que la información es correcta se deben cumplir las siguientes condiciones:
		\begin{itemize}
			\item Aquellos datos marcados como obligatorios no se deben omitir.
			\item Todos los datos proporcionados deben pertenecer al \cdtRef{gls:tipoDato}{tipo de dato} y respetar el formato especificado en el modelo de información.
			\item Cada dato proporcionado debe cumplir con la longitud definida por la base de datos y el modelo de información.
		\end{itemize}
	}
\end{BusinessRule}

%-----------------------------------------------

\begin{BusinessRule}{RN-S2}{Archivo del protocolo}
	\BRitem{Versión}{1.0}
	\BRitem{Autor}{Liliana Flores Sánchez}
	\BRitem{Estatus}{Edición}
	\BRitem{Descripción}{El formato del protocolo debe cumplir con los siguientes requisitos:
		\begin{itemize}
			\item Debe tener un formato en PDF.
			\item El tamaño del archivo no debe ser mayor a 5 MB. 
	\end{itemize}
}
\end{BusinessRule}

%-----------------------------------------------

\begin{BusinessRule}{RN-S3}{Formato de correo}
	\BRitem{Versión}{1.0}
	\BRitem{Autor}{Liliana Floresz Sánchez}
	\BRitem{Estatus}{Edición}
	\BRitem{Descripción}{Una URL válida tiene que cumplir con la expresión regular: \\
		\^{}([a-z0-9\_\textbackslash{}.-]+)@([\textbackslash{}da-z\textbackslash{}.-]+)\textbackslash{}.([a-z\textbackslash{}.]\{2,6\})\textdollar{}
		
	}
\end{BusinessRule}

%-----------------------------------------------

\begin{BusinessRule}{RN-S4}{Formato de número de boleta}
	\BRitem{Versión}{1.0}
	\BRitem{Autor}{Liliana Floresz Sánchez}
	\BRitem{Estatus}{Edición}
	\BRitem{Descripción}{Un número de boleta válida tiene que cumplir con la expresión regular: \\
		\^{}([a-z0-9\_\textbackslash{}.-]+)@([\textbackslash{}da-z\textbackslash{}.-]+)\textbackslash{}.([a-z\textbackslash{}.]\{2,6\})\textdollar{}
		
		
		%20xx63xxxx
		
		
	}
\end{BusinessRule}

%-----------------------------------------------

\begin{BusinessRule}{RN-S5}{Formato de RFC}
	\BRitem{Versión}{1.0}
	\BRitem{Autor}{Liliana Floresz Sánchez}
	\BRitem{Estatus}{Edición}
	\BRitem{Descripción}{Un RFC debe de cumplir con la expresión regular: \\
		\^{}([a-z0-9\_\textbackslash{}.-]+)@([\textbackslash{}da-z\textbackslash{}.-]+)\textbackslash{}.([a-z\textbackslash{}.]\{2,6\})\textdollar{}
		
		%\^([A-Z,Ñ,&]{3,4}([0-9]{2})(0[1-9]|1[0-2])(0[1-9]|1[0-9]|2[0-9]|3[0-1])[A-Z|\d]{3})$/
		
	}
\end{BusinessRule}

%-----------------------------------------------

\begin{BusinessRule}{RN-S6}{Archivo de información}
	\BRitem{Versión}{1.0}
	\BRitem{Autor}{Liliana Flores Sánchez}
	\BRitem{Estatus}{Edición}
	\BRitem{Descripción}{El formato del archivo que contiene la información personal de los alumnos debe cumplir con los siguientes requisitos:
		\begin{itemize}
			\item Debe tener un formato en XLSX, XLSM o XLSB.
			\item El tamaño del archivo no debe ser mayor a 1 MB. 
		\end{itemize}
	}
\end{BusinessRule}

\subsection{Reglas derivadas del negocio}

%-----------------------------------------------

\begin{BusinessRule}{RN-N1}{Operaciones disponibles Protocolo}
	\BRitem{Versión}{1.0}
	\BRitem{Autor}{Liliana Flores Sánchez}
	\BRitem{Estatus}{Edición}
	\BRitem{Descripción}{Las operaciones disponibles para realizar con el protocolo dependen del estado en que se encuentre:
		\begin{itemize}
			\item \textbf{Registro terminado:} Visualizar información general, integrantes, directores y descargar documento del protocolo.
			\item \textbf{Registro pendiente:} Cargar archivo del protocolo al sistema, agregar y editar información general, integrantes y directores del protocolo. 
		\end{itemize}
	}
\end{BusinessRule}

%-----------------------------------------------

\begin{BusinessRule}{RN-N2}{Palabra Clave Correcta}
	\BRitem{Versión}{1.0}
	\BRitem{Autor}{Liliana Flores Sánchez}
	\BRitem{Estatus}{Edición}
	\BRitem{Descripción}{Las palabras clave deben ser acorde al documento de palabras clave posicionado al pie de página. La sección de palabras clave debe tener como mínimo 2 palabras clave y como máximo 4 palabras clave.
	}
\end{BusinessRule}

%-----------------------------------------------
\begin{BusinessRule}{RN-N3}{Datos de sesión}
	\BRitem{Versión}{1.0}
	\BRitem{Autor}{Liliana Flores Sánchez}
	\BRitem{Estatus}{Edición}
	\BRitem{Descripción}{Los usuarios iniciarán sesión de la siguiente manera:
		\begin{itemize}
			\item El personal de la ESCOM y los directores externos ingresarán su RFC a 10 dígitos y su contraseña que se ingresó al momento de activar su cuenta.
			\item Un alumno ingresará su número de boleta y su contraseña que se ingresó al momento de activar su cuenta.
		\end{itemize}
	}
\end{BusinessRule}	

%-----------------------------------------------

\begin{BusinessRule}{RN-N4}{Operaciones disponibles del Alumno}
	\BRitem{Versión}{1.0}
	\BRitem{Autor}{Liliana Flores Sánchez}
	\BRitem{Estatus}{Edición}
	\BRitem{Descripción}{Las operaciones disponibles que puede realizar un alumno son:
		\begin{itemize}
			\item Puede agregar a otro alumno a su protocolo.
			\item Puede agregar a un director a su protocolo.
		\end{itemize}
	}
\end{BusinessRule}

%-----------------------------------------------
\begin{BusinessRule}{RN-N5}{Número de boleta para iniciar sesión obligatorio}
	\BRitem{Versión}{1.0}
	\BRitem{Autor}{Liliana Floresz Sánchez}
	\BRitem{Estatus}{Edición}
	\BRitem{Descripción}{No se puede activar un usuario alumno sin un número de boleta único.}
\end{BusinessRule}

%-----------------------------------------------
\begin{BusinessRule}{RN-N6}{RFC para iniciar sesión obligatorio}
	\BRitem{Versión}{1.0}
	\BRitem{Autor}{Liliana Floresz Sánchez}
	\BRitem{Estatus}{Edición}
	\BRitem{Descripción}{No se puede activar un usuario Profesor, Director, Director Externo, Presidente de Academia, Administrador CATT o Jefe CATT sin un RFC único.}
\end{BusinessRule}



%-----------------------------------------------
\begin{BusinessRule}{RN-N7}{Unicidad de correos}
	\BRitem{Versión}{1.0}
	\BRitem{Autor}{Liliana Flores Sánchez}
	\BRitem{Estatus}{Edición}
	\BRitem{Descripción}{Un correo electrónico no se puede duplicar en todo el sistema, ni registrar más de una ocasión.}
\end{BusinessRule}


%-----------------------------------------------
\begin{BusinessRule}{RN-N8}{Notificaciones al usuario}
	\BRitem{Versión}{1.0}
	\BRitem{Autor}{Liliana Flores Sánchez}
	\BRitem{Estatus}{Edición}
	\BRitem{Descripción}{Las notificaciones que se tengan que enviar al usuario, se enviarán al correo electrónico con el que fue activada su cuenta.}
\end{BusinessRule}

%-----------------------------------------------
\begin{BusinessRule}{RN-N9}{Unicidad de RFC}
	\BRitem{Versión}{1.0}
	\BRitem{Autor}{Liliana Flores Sánchez}
	\BRitem{Estatus}{Edición}
	\BRitem{Descripción}{Un RFC no se puede duplicar en todo el sistema, ni registrar en más de una ocasión.}
\end{BusinessRule}

%-----------------------------------------------
\begin{BusinessRule}{RN-N10}{Eliminar alumno del protocolo}
	\BRitem{Versión}{1.0}
	\BRitem{Autor}{Liliana Flores Sánchez}
	\BRitem{Estatus}{Edición}
	\BRitem{Descripción}{Un alumno no podrá ser eliminado por un alumno si ya ha aceptado participar en el protocolo al que fue invitado.}
\end{BusinessRule}
	
%-----------------------------------------------
\begin{BusinessRule}{RN-N11}{Eliminar director del protocolo}
	\BRitem{Versión}{1.0}
	\BRitem{Autor}{Liliana Flores Sánchez}
	\BRitem{Estatus}{Edición}
	\BRitem{Descripción}{Un director no podrá ser eliminado por un alumno si ya ha aceptado participar en el protocolo al que fue invitado.}
\end{BusinessRule}

%-----------------------------------------------
\begin{BusinessRule}{RN-N12}{Visualizar protocolos por validar}
	\BRitem{Versión}{1.0}
	\BRitem{Autor}{Liliana Flores Sánchez}
	\BRitem{Estatus}{Edición}
	\BRitem{Descripción}{Un Gestor TT podrá visualizar los protocolos que han sido pre-registrados con base en lo siguiente:
		\begin{itemize}
			\item Los protocolos deben pertenecer al periodo en curso.
			\item La opción para visualizar los protocolos pre-registrados debe mostrarse en el periodo de recepción de protocolos.
		\end{itemize}	
}
\end{BusinessRule}

%-----------------------------------------------
\begin{BusinessRule}{RN-N13}{Protocolo validado}
	\BRitem{Versión}{1.0}
	\BRitem{Autor}{Ricardo Ruiz Maldonado}
	\BRitem{Estatus}{Edición}
	\BRitem{Descripción}{Un protocolo se encuentra en el estado de validado cuando cumple con lo siguiente:
		\begin{itemize}
			\item El protocolo debe pertenecer al periodo en curso.
			\item El formato del protocolo ha sido validado por un Gestor TT.
			\item El registro del protocolo ha sido finalizado por un Gestor TT.
		\end{itemize}	
	}
\end{BusinessRule}

%-----------------------------------------------
\begin{BusinessRule}{RN-N14}{Invitación Pendiente}
	\BRitem{Versión}{1.0}
	\BRitem{Autor}{Liliana Flores Sánchez}
	\BRitem{Estatus}{Edición}
	\BRitem{Descripción}{Un protocolo se encuentra en el estado de \textbf{Invitación pendiente} cuando algún integrante agregado al protocolo no ha dado respuesta a la invitación que le fue enviada.
	}
\end{BusinessRule}

%-----------------------------------------------
\begin{BusinessRule}{RN-N15}{Protocolo Pre-registrado}
	\BRitem{Versión}{1.0}
	\BRitem{Autor}{Liliana Flores Sánchez}
	\BRitem{Estatus}{Edición}
	\BRitem{Descripción}{Un protocolo se encuentra en el estado de \textbf{Pre-registrado} cuando cumple con lo siguiente: 
		\begin{itemize}
			\item Se haya registrado la información base del protocolo.
			\item Todos los integrantes han dado respuesta a la invitación que les fue enviada.
		\end{itemize} 
	}
\end{BusinessRule}

%-----------------------------------------------
\begin{BusinessRule}{RN-N16}{Protocolo en estado \textit{Enviado a Validación}}
	\BRitem{Versión}{1.0}
	\BRitem{Autor}{Liliana Flores Sánchez}
	\BRitem{Estatus}{Edición}
	\BRitem{Descripción}{Un protocolo estará en estado de \textbf{Enviado a validación} cuando cumple con lo siguiente: 
		\begin{itemize}
			\item Todos los integrantes han dado respuesta a la invitación que les fue enviada.
			\item Haya terminado el registro del protocolo.
		\end{itemize}
Un protocolo también puede estar en este estado cuando se han concluido las correcciones que el Gestor TT le envió a los alumnos integrantes del protocolo.

	}
\end{BusinessRule}

%-----------------------------------------------
\begin{BusinessRule}{RN-N17}{Protocolo con correcciones}
	\BRitem{Versión}{1.0}
	\BRitem{Autor}{Liliana Flores Sánchez}
	\BRitem{Estatus}{Edición}
	\BRitem{Descripción}{Un protocolo estará en estado \textbf{Con Correcciones} cuando un Gestor TT haya revisado el protocolo y le haya hecho observaciones para que sean corregidas.
	}
\end{BusinessRule}

%-----------------------------------------------
\begin{BusinessRule}{RN-N18}{Protocolo pre-validado}
	\BRitem{Versión}{1.0}
	\BRitem{Autor}{Liliana Flores Sánchez}
	\BRitem{Estatus}{Edición}
	\BRitem{Descripción}{Un protocolo estará en estado \textbf{Pre-Validado} cuando un Gestor TT haya revisado el protocolo y no le haya hecho ninguna observación.
	}
\end{BusinessRule}

%-----------------------------------------------
\begin{BusinessRule}{RN-N19}{Protocolo en estado \textit{Sin documento escaneado}}
	\BRitem{Versión}{1.0}
	\BRitem{Autor}{Liliana Flores Sánchez}
	\BRitem{Estatus}{Edición}
	\BRitem{Descripción}{Un protocolo estará en estado \textbf{Sin Documento Escaneado} cuando los Gestores TT no han hecho la actualización del documento en el sistema.
	}
\end{BusinessRule}

%-----------------------------------------------
\begin{BusinessRule}{RN-N20}{Protocolo en estado \textit{Espera de Sinodales}}
	\BRitem{Versión}{1.0}
	\BRitem{Autor}{Liliana Flores Sánchez}
	\BRitem{Estatus}{Edición}
	\BRitem{Descripción}{Un protocolo estará en estado \textbf{Espera de Sinodales} cuando los Gestores TT han actualizado el documento en el sistema.
	}
\end{BusinessRule}

%-----------------------------------------------
\begin{BusinessRule}{RN-N21}{Protocolo en estado \textit{Evaluación}}
	\BRitem{Versión}{1.0}
	\BRitem{Autor}{Liliana Flores Sánchez}
	\BRitem{Estatus}{Edición}
	\BRitem{Descripción}{Un protocolo estará en estado \textbf{Espera de Sinodales} cuando le hayan asignado sinodales.
	}
\end{BusinessRule}

%-----------------------------------------------
\begin{BusinessRule}{RN-N22}{Protocolo en estado \textit{Reestructuración}}
	\BRitem{Versión}{1.0}
	\BRitem{Autor}{Liliana Flores Sánchez}
	\BRitem{Estatus}{Edición}
	\BRitem{Descripción}{Un protocolo estará en estado \textbf{Re-estructuración} cuando algún sinodal le mande correcciones al protocolo.
	}
\end{BusinessRule}

%-----------------------------------------------
\begin{BusinessRule}{RN-N23}{Protocolo en estado \textit{Rechazado}}
	\BRitem{Versión}{1.0}
	\BRitem{Autor}{Liliana Flores Sánchez}
	\BRitem{Estatus}{Edición}
	\BRitem{Descripción}{Un protocolo estará en estado \textbf{Rechazado} cuando los sinodales asignados no están de acuerdo con el protocolo. registrado
	}
\end{BusinessRule}

%-----------------------------------------------
\begin{BusinessRule}{RN-N23}{Protocolo en estado \textit{Aprobado}}
	\BRitem{Versión}{1.0}
	\BRitem{Autor}{Liliana Flores Sánchez}
	\BRitem{Estatus}{Edición}
	\BRitem{Descripción}{Un protocolo estará en estado \textbf{Aprobado} cuando se cumple lo siguiente:
		\begin{itemize}
			\item Los sinodales no le ha hecho correcciones.
			\item Los sinodales están de acuerdo con el protocolo.
		\end{itemize}
	}
\end{BusinessRule}

%-----------------------------------------------
\begin{BusinessRule}{RN-N24}{Orden de visualización de Trabajos Terminales}
	\BRitem{Versión}{1.0}
	\BRitem{Autor}{Liliana Flores Sánchez}
	\BRitem{Estatus}{Edición}
	\BRitem{Descripción}{Un Trabajo Terminal se mostrará del periodo escolar más reciente al periodo escolar más antiguo.
	}
\end{BusinessRule}

%-----------------------------------------------
\begin{BusinessRule}{RN-N25}{Orden de visualización de Protocolos en la CATT}
	\BRitem{Versión}{1.0}
	\BRitem{Autor}{Liliana Flores Sánchez}
	\BRitem{Estatus}{Edición}
	\BRitem{Descripción}{Un Protocolo se mostrará por id de Protocolo, esto quiere decir que se mostrará del primer protocolo que fue aceptado en la CATT hasta el último, en el periodo escolar actual.
	}
\end{BusinessRule}

%-----------------------------------------------
\begin{BusinessRule}{RN-N26}{Asignar Sinodal al Protocolo}
	\BRitem{Versión}{1.0}
	\BRitem{Autor}{Liliana Flores Sánchez}
	\BRitem{Estatus}{Edición}
	\BRitem{Descripción}{La asignación de un Sinodal debe ser de la siguiente manera:
		\begin{itemize}
			\item El sistema verificará si tiene menos o el número exacto de Sinodales que le fueron asignados a su academia.
			\item Los Directores del Protocolo no pueden fungir como Sinodales del mismo.
			\item No se puede asignar el mismo Sinodal más de una vez en el mismo Protocolo.
		\end{itemize}
	}
\end{BusinessRule}

%-----------------------------------------------
\begin{BusinessRule}{RN-N27}{Sinodales asignados}
	\BRitem{Versión}{1.0}
	\BRitem{Autor}{Liliana Flores Sánchez}
	\BRitem{Estatus}{Edición}
	\BRitem{Descripción}{Un Protocolo debe tener tres Sinodales asignados.
	}
\end{BusinessRule}

%-----------------------------------------------
\begin{BusinessRule}{RN-N28}{Orden de visualización de Protocolos para el Jefe de Departamento}
	\BRitem{Versión}{1.0}
	\BRitem{Autor}{Liliana Flores Sánchez}
	\BRitem{Estatus}{Edición}
	\BRitem{Descripción}{La lista de los Protocolos que se mostrará al Jefe de Departamento debe comenzar por aquellos Protocolos a los que no se les han asignado Sinodales, precedidos de los Protocolos que parcialmente tienen Sinodales asignados y finalmente se mostrarán los protocolos que ya tienen todos los sinodales asignados.\\
	}
\end{BusinessRule}
