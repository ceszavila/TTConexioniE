\section{Resultados}

En este capítulo se muestran los avances obtenidos, los cuales comprenden los siguientes módulos:\\

\begin{UClist} 
	\UCli Salones: En él, los actores son capaces de ubicar los edificios de la Escuela y obtener su información. De la misma manera se pueden ubicar los salones desde la pantalla de asignación de salones.
	\UCli Profesores: En este módulo, se puede obtener la lista de profesores registrados en la aplicación, así como el detalle de la información de contacto y ubicación.
\end{UClist} 

\subsection{Módulo de Salones}

En esta sección se puede observar el avance en la navegación y usabilidad de la aplicación como se muestra a continuación:

%pantallas de la aplicación y si explicación

\subsection{Módulo de Profesores}

En esta sección se puede observar el avance en la navegación y usabilidad de la aplicación como se muestra a continuación:

%pantallas de la aplicación y si explicación

\section{Trabajo a Futuro}

Los módulos que están propuestos como trabajo a futuro son los sieguientes: \\

\begin{UClist} 
	\UCli Material de apoyo: 
	Este módulo comprenderá el material que profesores deseen subir de su misma autoriá o las ligas de repositorios gratuitos como ayuda en el material de las unidades de aprendizaje.\\
	
	\UCli Trámites: 
	Este módulo comprenderá la información necesaria para realizar algunos trámites dentro de la escuela como son: el servicio social y la movilidad estudiantil.\\
	
	\UCli Cursos y certificaciones: 
	Este módulo facilitará la información y publicación de convocatorias con la finalidad de aumentar el número de alumnos que se inscriben a dichas actividades extracurriculares.\\
\end{UClist} 

\section{Conclusiones}