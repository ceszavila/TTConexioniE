\section{Conclusiones}

El presente Trabajo Terminal tuvo como objetivo proponer e implementar un prototipo de aplicación que permita a la comunidad consultar los diferentes espacios con los que cuenta la Escuela Superior de Cómputo. Esto se logró con el uso de tecnologías disponibles para su consulta, tales como Mapbox que aunque no fue una tecnología utilizada fue de gran a ayuda para comprender el modelado de los polígonos con el fin de dibujar los edificios. La API de Google Maps como consulta también y diversos ejemplos de aplicaciones para el entorno de desarrollo de Apple.\\

Existen herramientas externas a las utilizadas por Apple que fueron de gran ayuda en la realización de este trabajo terminal. Un ejemplo de estas herramientas fue la utilización de \textbf{Pods} importados para mejorar la vista y funcionalidad de la aplicación. Librerías como \textbf{SVProgressHud} para la transición de carga, \textbf{Chameleon} para la utilización de colores, entre otras.

Con este trabajo damos un paso importante en la especialización que ambos integrantes queremos realizar en el desarrollo de aplicaciones móviles para dispositivos con sistema operativo iOS. De la misma manera satisfacemos parcialmente el gran interés por la ingeniería de software, sabiendo a donde dirigir nuestro futuro profesional.

\section{Resultados}

En este capítulo se muestran los resultados obtenidos, los cuales comprenden los siguientes módulos:\\

\begin{UClist} 
	\UCli Salones: En él, los actores son capaces de ubicar los edificios de la Escuela y obtener su información. De la misma manera se pueden ubicar los salones desde la pantalla de asignación de salones.
	\UCli Profesores: En este módulo, se puede obtener la lista de profesores registrados en la aplicación, así como el detalle de la información de contacto y ubicación.
	\UCli Unidades de Aprendizaje: Este módulo se compone de la lista de unidades de aprendizaje que se imparten o forman parte del plan de estudio de Ingeniería en Sistemas Computacionales en la Escuela Superior de Cómputo.
	\UCli Convocatorias de cursos: En este módulo se puede realizar la gestión de cursos que se ofertan en la ESCOM y que pueden ser difundidos por la misma escuela o algún centro dentro del Instituto que se relacione con la carrera de Ing. en Sistemas Computacionales.
	\UCli Convocarorias de movilidad:  En él, se muestran las dos convocatorias que se aperturan cada periodo escolar para realizar una movilidad acdémica ya sea dentro del país o en el extranjero.
\end{UClist} 

%\subsection{Módulo de Salones}
%
%En esta sección se puede observar el avance en la navegación y usabilidad de la aplicación como se muestra a continuación:
%
%%pantallas de la aplicación y si explicación
%
%\subsection{Módulo de Profesores}
%
%En esta sección se puede observar el avance en la navegación y usabilidad de la aplicación como se muestra a continuación:

%pantallas de la aplicación y si explicación

\section{Trabajo a Futuro}

Actualmente se está desarrollando el proyecto \textbf{CALMECAC} que será un sistema que integre todos los demás subsistemas con los que trabaja el Instituto y del que, como trabajadores del análisis y desarrollo del mismo, se pretende continuar con la generación de actualizaciones y mejoras al Trabajo Terminal, incluso la implementación de módulos no pretendidos en el alcance con la finalidad de ofrecerle a los alumnos la herramienta que integre los trámites y noticias qu le permitan obtener todo lo necesario durante su permanencia estudiantil en la Escuela Superior de Cómputo y exista la posibilidad de extenderse a todas las escuelas y centros de investigación del Instituto Politécnico Nacional. \\

Existe actualmente un Trabajo Terminal que busca implementar algunos de los módulos aquí propuestos y otros no tomados en cuenta dentro de nuestro alcance para la plataforma de desarrollo y sistema operativo Android. Se pretende colaborar con este equipo para retroalimentar los trabajos y llegar a ofrecer una herramienta que cubra con el mayor porcentaje posible de usuarios finales. \\

\section{Pruebas}

\cfinput{pruebas}