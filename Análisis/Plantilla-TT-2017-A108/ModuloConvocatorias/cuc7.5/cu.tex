\begin{UseCase}{CUAC-7.5}{Registrar convocatoria de movilidad}{
		Permite llevar a cabo el registro de una nueva convocatoria de movilidad. Las convocatorias de movilidad se agregan cada vez que una nueva es aprovada en el instituto y estas son por año.
			
		}
		\UCitem{Versión}{1.0}
		\UCccsection{Administración de Requerimientos}
		\UCitem{Autor}{Ivo Sebastián Sam Álvarez-Tostado}
		\UCccitem{Evaluador}{Ulises Velez Saldaña}
		\UCitem{Operación}{Consulta}
		\UCccitem{Prioridad}{Alta}
		\UCccitem{Complejidad}{Baja}
		\UCccitem{Volatilidad}{Baja}
		\UCccitem{Madurez}{Alta}
		\UCitem{Estatus}{Por revisar}
		\UCitem{Fecha del último estatus}{11 de Mayo del 2018}
			
		\UCsection{Atributos}
		\UCitem{Actor}{ \textbf{Responsable UPIS}}
		\UCitem{Propósito}{Proporcionar un medio por el cual se pueda realizar el registro de nuevas convocatorias de movilidad.}
		\UCitem{Entradas}{
			\begin{UClist}
				\UCli Año de la convocatoria
				\UCli URL de las convocatorias
			\end{UClist}
		}
		\UCitem{Salidas}{	
			\begin{UClist}
				\UCli Año de la convocatoria
				\UCli URL de las convocatorias
			\end{UClist}
		}
		\UCitem{Precondiciones}{\textbf{Manual: }La convocatoria debió haber sido aprobada en el instituto.}
		\UCitem{Postcondiciones}{Se podrá consultar la nueva convocatoria de movilidad.}
		\UCitem{Reglas de negocio}{Ninguna}
		\UCitem{Errores}{\UCerr{Uno}{Cuando no se logra llevar a cabo la operación de manera correcta,}{se muestra el mensaje de error termina el caso de uso.}}
		\UCitem{Tipo}{Primario.}
	\end{UseCase}
	
	\begin{UCtrayectoria}
		\UCpaso [\UCactor] Solicita agregar una nueva convocatoria de movilidad tocando el botón \cdtButton{Agregar} de la pantalla \cdtIdRef{IUC7c}{Gestión de convocatorias}. 
		\UCpaso [\UCsist] Muestra la pantalla \cdtIdRef{IUC7.5}{Agregar convocatoria de movilidad}.
		\UCpaso [\UCactor] Ingresa la información solicitada.
		\UCpaso [\UCactor]  Solicita finalizar el registro de la convocatoria tocando el botón \cdtButton{Guardar}
		\UCpaso [\UCsist] Persiste la información registrada. \refErr{Uno}
		\UCpaso [\UCsist] Regresa a la pantalla \cdtIdRef{IUC7c}{Gestión de convocatorias} con la nueva convocatoria registrada.
	\end{UCtrayectoria}
	
	\begin{UCtrayectoriaA}[Fin del caso de uso]{A}{Cuando el actor requiere volver a la pantalla anterior.}
		\UCpaso[\UCactor] Solicita regresar a la pantalla anterior tocando el botón \cdtButton{Atrás}.
		
		\UCpaso [\UCsist] Regrasa a la pantalla anterior.
	\end{UCtrayectoriaA}
	